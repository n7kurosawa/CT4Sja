
%%%%%%%% Chapter %%%%%%%%

\printindex


\bibliographystyle{amsalpha}
\begin{thebibliography}{SGWB}\rr

\bibitem [Ati]{Ati} Atiyah, M. (1989) ``Topological quantum field theories.'' \textit{Publications Math\'{e}matiques de l'IHÉS} 68 (68), pp. 175--186.

\bibitem [Axl]{Axl} Axler, S. (1997) \textit{Linear algebra done right}. Springer. 

\bibitem [Awo]{Awo} S. Awodey. (2010) \textit{Category theory.} Second edition. Oxford Logic Guides, 52. Oxford University Press, Oxford.

\bibitem [Bar]{Bar} Bralow, H. (1961) ``Possible principles underlying the transformation of sensory messages.'' \textit{Sensory communication}, pp. 217 -- 234.

\bibitem [BD]{BD} Baez, J.C.; Dolan, J. (1995) ``Higher-dimensional algebra and topological quantum field theory.'' \textit{Journal of mathematical physics} vol 36, 6073.

\bibitem [BFL]{BFL} Baez, J.C.; Fritz, T.; Leinster, T. (2011) ``A characterization of entropy in terms of information loss''. \textit{Entropy} 13, no. 11.

\bibitem[BS]{BS} Baez, J.C.; Stay, M. (2011) ``Physics, topology, logic and computation: a Rosetta Stone.'' \textit{New structures for physics}, 95--172. Lecture Notes in Phys., 813, Springer, Heidelberg.

\bibitem [BP1]{BP1} Brown, R.; Porter, T. (2006) ``Category Theory: an abstract setting for
analogy and comparison'', In: \textit{What is Category Theory?} Advanced
Studies in Mathematics and Logic, Polimetrica Publisher, Italy, pp. 257--274.

\bibitem [BP2]{BP2} Brown, R.; Porter, T. (2003) ``Category theory and higher dimensional
algebra: potential descriptive tools in neuroscience'', \textit{Proceedings
of the International Conference on Theoretical Neurobiology, Delhi}, edited by Nandini Singh, National Brain Research
Centre, Conference Proceedings 1 80--92. 

\bibitem [BW]{BW} M. Barr, C. Wells. (1990) \textit{Category theory for computing science.} Prentice Hall International Series in Computer Science. Prentice Hall International, New York.

\bibitem [Big]{Big} Biggs, N.M. (2004) \textit{Discrete mathematics}. Oxford University Press, NY. 

\bibitem [Dia]{Dia} Diaconescu, R. (2008) \textit{Institution-independent model theory} Springer.

\bibitem[DI]{DI} D\"{o}ring, A.; Isham, C. J. ``A topos foundation for theories of physics. I. Formal languages for physics.''
\textit{Journal of mathematical physics} 49 (2008), no. 5, 053515.

\bibitem[EV]{EV} Ehresmann, A.C.; Vanbremeersch, J.P. (2007) \textit{Memory evolutive systems; hierarchy, emergence, cognition}. Elsevier.

\bibitem[Eve]{Eve} Everett III, H. (1973). ``The theory of the universal wave function.'' In \textit{The many-worlds interpretation of quantum mechanics} (Vol. 1, p. 3).

\bibitem [Gog]{Gog} Goguen, J. (1992) ``Sheaf semantics for concurrent interacting objects'' \textit{Mathematical structures in Computer Science} Vol 2, pp. 159 -- 191.

\bibitem [Gro]{Gro} Grothendieck, A. (1971). \textit{S\'eminaire de G\'eom\'etrie Alg\'ebrique du Bois Marie - 1960-61 - Revêtements \'etales et groupe fondamental - (SGA 1)} (Lecture notes in mathematics 224) (in French). Berlin; New York: Springer-Verlag.

\bibitem [Kro]{Kro} Kr\"{o}mer, R. (2007). \textit{Tool and Object: A History and Philosophy of Category Theory}, Birkhauser.

\bibitem [Lam]{Lam} Lambek, J. (1980) ``From $\lambda$-calculus to Cartesian closed categories.'' In \textit{Formalism}, Academic Press, London, pp. 375 -- 402.

\bibitem [Law]{Law} Lawvere, F.W. (2005) ``An elementary theory of the category of sets (long version) with
   commentary." (Reprinted and expanded from Proc. Nat. Acad. Sci. U.S.A. 52
   (1964)) \textit{Repr. Theory Appl. Categ.} \textbf{11}, pp. 1 -- 35.
   
\bibitem [Kho]{Kho} Khovanov, M. (2000) ``A categorificiation of the Jones polynomial'' \textit{Duke Math J.}.

\bibitem [Le1]{Le1} Leinster, T. (2004) \textit{Higher Operads, Higher Categories}. London Mathematical Society Lecture Note Series \textbf{298}, Cambridge University Press.

\bibitem [Le2]{Le2} Leinster, T. (2012) ``Rethinking set theory.'' ePrint available \url{http://arxiv.org/abs/1212.6543}.

\bibitem [Lin]{Lin} Linsker, R. (1988) ``Self-organization in a perceptual network.'' \textit{Computer} \textbf{21}, no. 3, pp. 105 -- 117.

\bibitem [LM]{LM} Landry, E.; Marquis, J-P., 2005, "Categories in Contexts: historical, foundational, and philosophical.'' \textit{Philosophia Mathematica}, (3), vol. 13, no. 1, 1 -- 43.

\bibitem [LS]{LS} F.W. Lawvere, S.H. Schanuel. (2009) \textit{Conceptual mathematics. 
A first introduction to categories.} Second edition. Cambridge University Press, Cambridge.

\bibitem [MacK]{MacK} MacKay, D.J. (2003). \textit{Information theory, inference and learning algorithms.} Cambridge university press.

\bibitem [Mac]{Mac} Mac Lane, S. (1998) \textit{Categories for the working mathematician.} Second edition. Graduate Texts in Mathematics, 5. Springer-Verlag, New York.

\bibitem[Mar1]{Mar1} Marquis, J-P. (2009) \textit{From a Geometrical Point of View: a study in the history and philosophy of category theory}, Springer.

\bibitem [Mar2]{Mar2} Marquis, J-P, ``Category Theory'', \textit{The Stanford Encyclopedia of Philosophy} (Spring 2011 Edition), Edward N. Zalta (ed.), \url{http://plato.stanford.edu/archives/spr2011/entries/category-theory}

\bibitem[Min]{Min} Minsky, M. \textit{The Society of Mind.}  Simon and Schuster, NY 1985.

\bibitem[Mog]{Mog} Moggi, E. (1989) ``A category-theoretic account of program modules.'' \textit{Category theory and computer science (Manchester, 1989),} 101--117, Lecture Notes in Comput. Sci., 389, Springer, Berlin. 

\bibitem [nLa]{nLa} nLab authors.  \url{http://ncatlab.org/nlab/show/HomePage}

\bibitem [Pen]{Pen} Penrose, R. (2006) \textit{The road to reality}. Random house.

\bibitem [RS]{RS} Radul, A.; Sussman, G.J. (2009). ``The art of the propagator.'' \textit{MIT Computer science and artificial intelligence laboratory technical report.}

\bibitem [Sp1]{Sp1} Spivak, D.I. (2012) ``Functorial data migration.'' \textit{Information and communication} 

\bibitem [Sp2]{Sp2} Spivak, D.I. (2012) ``Queries and constraints via lifting problems.'' Submitted to \textit{Mathematical structures in computer science}. ePrint available: \url{http://arxiv.org/abs/1202.2591}

\bibitem [Sp3]{Sp3} Spivak, D.I. (2012) ``Kleisli database instances''. ePrint available: \url{http://arxiv.org/abs/1209.1011}

\bibitem [Sp4]{Sp4} Spivak, D.I. (2013) ``The operad of wiring diagrams: Formalizing a graphical language for databases, recursion, and plug-and-play circuits.'' Available online: \url{http://arxiv.org/abs/1305.0297}

\bibitem[SGWB]{SGWB} Spivak D.I., Giesa T., Wood E., Buehler M.J. (2011) ``Category Theoretic Analysis of Hierarchical Protein Materials and Social Networks.'' PLoS ONE 6(9): e23911. doi:10.1371/journal.pone.0023911

\bibitem[SK]{SK} Spivak, D.I., Kent, R.E. (2012) ``Ologs: A Categorical Framework for Knowledge Representation.'' \textit{PLoS ONE} 7(1): e24274. doi:10.1371/journal.pone.0024274.

\bibitem[WeS]{WeS} Weinberger, S. (2011) ``What is... Persistent Homology?'' AMS.

\bibitem[WeA]{WeA} Weinstein, A. (1996) ``Groupoids: unifying internal and external symmetry. \textit{Notices of the AMS} Vol 43, no. 7, pp. 744 -- 752.

%\bibitem[Wik]{Wik} \href{http://www.wikipedia.org}{\text Wikipedia} (multiple authors). Various articles, all linked with a hyperreference are scattered throughout this text. All accessed December 6, 2012 -- \today.
\bibitem[Wik]{Wik} \href{http://www.wikipedia.org}{\text Wikipedia} (multiple authors). Various articles, all linked with a hyperreference are scattered throughout this text. All accessed 2012-12-06 -- \the \year-{\ifnum \month < 10 0\the\month \else \the\month \fi}-{\ifnum \day < 10 0\the\day \else \the\day \fi}.

\end{thebibliography}

