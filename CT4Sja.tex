\documentclass[a4paper]{ltjsbook}

\usepackage{etex}
\usepackage{savesym}
\usepackage{amssymb, amsmath,amsthm,amscd}
%\savesymbol{lrcorner}
\usepackage{txfonts}
\savesymbol{lrcorner}\savesymbol{llcorner}\savesymbol{urcorner}\savesymbol{ulcorner}
%\usepackage{marvosym}
\usepackage{wasysym}
\savesymbol{Sun}\savesymbol{Mercury}\savesymbol{Venus}\savesymbol{Earth}\savesymbol{Mars}\savesymbol{Jupiter}\savesymbol{Saturn}\savesymbol{Uranus}\savesymbol{Neptune}\savesymbol{Pluto}\savesymbol{leftmoon}\savesymbol{rightmoon}\savesymbol{fullmoon}\savesymbol{newmoon}\savesymbol{Aries}\savesymbol{Taurus}\savesymbol{Gemini}\savesymbol{Leo}\savesymbol{Libra}\savesymbol{Scorpio}\savesymbol{diameter}
\usepackage{mathabx}
%\usepackage{stmaryrd}
\usepackage{setspace}
\usepackage{chngcntr}
\usepackage[tt]{titlepic}
\usepackage{enumerate,makecell}
\usepackage{makeidx,tabularx,dashbox}
\usepackage[usenames,dvipsnames]{xcolor}
\usepackage[bookmarks=true,colorlinks=true, linkcolor=MidnightBlue, citecolor=cyan]{hyperref}
\usepackage{lmodern}
\usepackage{graphicx,float}
\usepackage{multirow}
\usepackage{geometry}
\newgeometry{left=1.6in,right=1.6in,top=1.4in,bottom=1.4in}
\restoresymbol{txfonts}{lrcorner}\restoresymbol{txfonts}{llcorner}\restoresymbol{txfonts}{urcorner}\restoresymbol{txfonts}{ulcorner}

\usepackage{color}
\usepackage[all,poly,color,matrix,arrow]{xy}
\makeindex

%\usepackage{showkeys}

\newcommand{\comment}[1]{}

\newcommand{\longnote}[2][4.9in]{\fcolorbox{black}{yellow}{\parbox{#1}{\color{black} #2}}}
\newcommand{\shortnote}[1]{\fcolorbox{black}{yellow}{\color{black} #1}}
\newcommand{\start}[1]{\shortnote{Start here: #1.}}
\newcommand{\q}[1]{\begin{question}#1\end{question}}
\newcommand{\g}[1]{\begin{guess}#1\end{guess}}
\newcommand{\cfbox}[2]{
    \colorlet{currentcolor}{.}
    {\color{#1}
    \fbox{\color{currentcolor}#2}}
}

\def\tn{\textnormal}
\def\mf{\mathfrak}
\def\mc{\mathcal}
\newcommand{\qt}[1]{\tn{``}#1\tn{"}}

\def\ZZ{{\mathbb Z}}
\def\QQ{{\mathbb Q}}
\def\RR{{\mathbb R}}
\def\CC{{\mathbb C}}
\def\AA{{\mathbb A}}
\def\PP{{\mathbb P}}
\def\NN{{\mathbb N}}


\def\Hom{\tn{Hom}}
\def\Path{\tn{Path}}
\def\Paths{\tn{Paths}}
\def\List{\tn{List}}
\def\Aut{\tn{Aut}}
\def\im{\tn{im}}
\def\Fun{\tn{Fun}}
\def\Ob{\tn{Ob}}
\def\Skel{\tn{Skel}}
\def\Op{\tn{Open}}
\def\PK{\tn{PK}}
\def\FK{\tn{FK}}
\def\SEL*{\tn{SEL*}}
\def\Res{\tn{Res}}
\def\hsp{\hspace{.3in}}
\newcommand{\hsps}[1]{{\hspace{2mm} #1\hspace{2mm}}}
\newcommand{\tin}[1]{\text{\tiny #1}}

\def\singleton{\{\smiley\}}
\newcommand{\boxtitle}[1]{\begin{center}#1\end{center}\vspace{-.1in}}
\newcommand{\singlefun}[1]{\star^{#1}}
\newcommand{\pullb}[1]{\Delta_{#1}}
\newcommand{\lpush}[1]{\Sigma_{#1}}
\newcommand{\rpush}[1]{\Pi_{#1}}
\def\lcone{^\triangleleft}
\def\rcone{^\triangleright}
\def\to{\rightarrow}
\def\from{\leftarrow}
\def\down{\downarrrow}
\def\Down{\Downarrow}
\def\taking{\colon}
\def\inj{\hookrightarrow}
\def\surj{\twoheadrightarrow}
\def\too{\longrightarrow}
\newcommand{\xyright}[1]{\xymatrix{~\ar[r]#1&}}
\newcommand{\xydown}[1]{\xymatrix{~\ar[d]#1\\~}}
\newcommand{\xydoown}[1]{\xymatrix{~\ar[ddd]#1\\\parbox{0in}{~}\\\parbox{0in}{~}\\~}}
\def\fromm{\longleftarrow}
\def\tooo{\longlongrightarrow}
\def\tto{\rightrightarrows}
\def\ttto{\equiv\!\!>}
\def\ss{\subseteq}
\def\superset{\supseteq}
\def\iso{\cong}
\def\down{\downarrow}
\def\|{{\;|\;}}
\def\m1{{-1}}
\def\op{^\tn{op}}
\def\loc{\tn{loc}}
\def\la{\langle}
\def\ra{\rangle}
\def\wt{\widetilde}
\def\wh{\widehat}
\def\we{\simeq}
\def\ol{\overline}
\def\ul{\underline}
\def\plpl{+\!\!+\hspace{1pt}}
\def\acts{\lefttorightarrow}
\def\actsRight{\righttoleftarrow}
\def\vect{\overrightarrow}
\def\qeq{\mathop{=}^?}
\def\del{\partial\,}

\def\rr{\raggedright}

%\newcommand{\LMO}[1]{\bullet^{#1}}
%\newcommand{\LTO}[1]{\bullet^{\tn{#1}}}
\newcommand{\LMO}[1]{\stackrel{#1}{\bullet}}
\newcommand{\LTO}[1]{\stackrel{\tt{#1}}{\bullet}}
\newcommand{\LA}[2]{\ar[#1]^-{\tn {#2}}}
\newcommand{\LAL}[2]{\ar[#1]_-{\tn {#2}}}
\newcommand{\obox}[3]{\stackrel{#1}{\fbox{\parbox{#2}{#3}}}}
\newcommand{\labox}[2]{\obox{#1}{1.6in}{#2}}
\newcommand{\mebox}[2]{\obox{#1}{1in}{#2}}
\newcommand{\smbox}[2]{\stackrel{#1}{\fbox{#2}}}
\newcommand{\fakebox}[1]{\tn{$\ulcorner$#1$\urcorner$}}
\newcommand{\sq}[4]{\xymatrix{#1\ar[r]\ar[d]&#2\ar[d]\\#3\ar[r]&#4}}
\newcommand{\namecat}[1]{\begin{center}$#1:=$\end{center}}

\def\monOb{\blacktriangle}

\def\ullimit{\ar@{}[rd]|(.3)*+{\lrcorner}}
\def\urlimit{\ar@{}[ld]|(.3)*+{\llcorner}}
\def\lllimit{\ar@{}[ru]|(.3)*+{\urcorner}}
\def\lrlimit{\ar@{}[lu]|(.3)*+{\ulcorner}}
\def\ulhlimit{\ar@{}[rd]|(.3)*+{\diamond}}
\def\urhlimit{\ar@{}[ld]|(.3)*+{\diamond}}
\def\llhlimit{\ar@{}[ru]|(.3)*+{\diamond}}
\def\lrhlimit{\ar@{}[lu]|(.3)*+{\diamond}}
\newcommand{\clabel}[1]{\ar@{}[rd]|(.5)*+{#1}}
\newcommand{\TriRight}[7]{\xymatrix{#1\ar[dr]_{#2}\ar[rr]^{#3}&&#4\ar[dl]^{#5}\\&#6\ar@{}[u] |{\Longrightarrow}\ar@{}[u]|>>>>{#7}}}
\newcommand{\TriLeft}[7]{\xymatrix{#1\ar[dr]_{#2}\ar[rr]^{#3}&&#4\ar[dl]^{#5}\\&#6\ar@{}[u] |{\Longleftarrow}\ar@{}[u]|>>>>{#7}}}
\newcommand{\TriIso}[7]{\xymatrix{#1\ar[dr]_{#2}\ar[rr]^{#3}&&#4\ar[dl]^{#5}\\&#6\ar@{}[u] |{\Longleftrightarrow}\ar@{}[u]|>>>>{#7}}}


\newcommand{\arr}[1]{\ar@<.5ex>[#1]\ar@<-.5ex>[#1]}
\newcommand{\arrr}[1]{\ar@<.7ex>[#1]\ar@<0ex>[#1]\ar@<-.7ex>[#1]}
\newcommand{\arrrr}[1]{\ar@<.9ex>[#1]\ar@<.3ex>[#1]\ar@<-.3ex>[#1]\ar@<-.9ex>[#1]}
\newcommand{\arrrrr}[1]{\ar@<1ex>[#1]\ar@<.5ex>[#1]\ar[#1]\ar@<-.5ex>[#1]\ar@<-1ex>[#1]}

\newcommand{\To}[1]{\xrightarrow{#1}}
\newcommand{\Too}[1]{\xrightarrow{\ \ #1\ \ }}
\newcommand{\From}[1]{\xleftarrow{#1}}
\newcommand{\Fromm}[1]{\xleftarrow{\ \ #1\ \ }}

\newcommand{\Adjoint}[4]{\xymatrix@1{#2 \ar@<.5ex>[r]^-{#1} & #3 \ar@<.5ex>[l]^-{#4}}}
\newcommand{\adjoint}[4]{\xymatrix{#1\taking #2\ar@<.5ex>[r]& #3\hspace{1pt}:\hspace{-2pt} #4\ar@<.5ex>[l]}}

\newcommand{\prodmap}[2]{\la#1,#2\ra}
\newcommand{\pb}[3]{\prodmap{#1}{#1}_{#3}}
\newcommand{\coprodmap}[2]{{\left\{\parbox{.1in}{#1\\#2}\right.}}
\newcommand{\po}[3]{\coprodmap{#1}{#2}}

\def\id{\tn{id}}
\def\ids{\tn{ids}}
\def\Top{{\bf Top}}
\def\Cat{{\bf Cat}}
\def\Oprd{{\bf Oprd}}
\def\Str{{\bf Str}}
\def\Mon{{\bf Mon}}
\def\Grp{{\bf Grp}}
\def\Grph{{\bf Grph}}
\def\Type{{\bf Type}}
\def\Supp{{\bf Supp}}
\def\Dist{{\bf Dist}}
\def\Vect{{\bf Vect}}
\def\Kls{{\bf Kls}}
\def\Prop{{\bf Prop}}
\def\FLin{{\bf FLin}}
\def\Set{{\bf Set}}
\def\Sets{{\bf Sets}}
\def\PrO{{\bf PrO}}
\def\Star{{\bf Star}}
\def\Cob{{\bf Cob}}
\def\Qry{{\bf Qry}}
\def\set{{\text \textendash}{\bf Set}}
\def\sSet{{\bf sSet}}
\def\sSets{{\bf sSets}}
\def\Grpd{{\bf Grpd}}
\def\Pre{{\bf Pre}}
\def\Shv{{\bf Shv}}
\def\Rings{{\bf Rings}}
\def\bD{{\bf \Delta}}
\def\dispInt{\parbox{.1in}{$\int$}}
\def\bhline{\Xhline{2\arrayrulewidth}}
\def\bbhline{\Xhline{2.5\arrayrulewidth}}
\def\bbbhline{\Xhline{3\arrayrulewidth}}


\def\colim{\mathop{\tn{colim}}}
\def\hocolim{\mathop{\tn{hocolim}}}

\def\mcA{\mc{A}}
\def\mcB{\mc{B}}
\def\mcC{\mc{C}}
\def\mcD{\mc{D}}
\def\mcE{\mc{E}}
\def\mcF{\mc{F}}
\def\mcG{\mc{G}}
\def\mcH{\mc{H}}
\def\mcI{\mc{I}}
\def\mcJ{\mc{J}}
\def\mcK{\mc{K}}
\def\mcL{\mc{L}}
\def\mcM{\mc{M}}
\def\mcN{\mc{N}}
\def\mcO{\mc{O}}
\def\mcP{\mc{P}}
\def\mcQ{\mc{Q}}
\def\mcR{\mc{R}}
\def\mcS{\mc{S}}
\def\mcT{\mc{T}}
\def\mcU{\mc{U}}
\def\mcV{\mc{V}}
\def\mcW{\mc{W}}
\def\mcX{\mc{X}}
\def\mcY{\mc{Y}}
\def\mcZ{\mc{Z}}

\def\undsc{\rule{2mm}{0.4pt}}
\def\Loop{{\mcL oop}}
\def\LoopSchema{{\parbox{.5in}{\fbox{\xymatrix{\LMO{s}\ar@(l,u)[]^f}}}}}

\newtheorem{theorem}[subsubsection]{Theorem}
\newtheorem{lemma}[subsubsection]{Lemma}
\newtheorem{proposition}[subsubsection]{Proposition}
\newtheorem{corollary}[subsubsection]{Corollary}
\newtheorem{fact}[subsubsection]{Fact}

\theoremstyle{remark}
\newtheorem{remark}[subsubsection]{Remark}
\newtheorem{example}[subsubsection]{Example}
\newtheorem{warning}[subsubsection]{Warning}
\newtheorem{question}[subsubsection]{Question}
\newtheorem{guess}[subsubsection]{Guess}
\newtheorem{answer}[subsubsection]{Answer}
\newtheorem{construction}[subsubsection]{Construction}
\newtheorem{rules}[subsubsection]{Rules of good practice}
\newtheorem{exc}[subsubsection]{Exercise}
\newenvironment{exercise}{\begin{exc}}{\hspace*{\fill}$\lozenge$\end{exc}}
\newtheorem{app}[subsubsection]{Application}
\newenvironment{application}{\begin{app}}{\hspace*{\fill}$\lozenge\lozenge$\end{app}}

%\newenvironment{exercise}{\addtocounter{theorem}{1}\vspace{.1in}\begin{sloppypar}\noindent{\em Exercise}\;\arabic{chapter}.\arabic{section}.\arabic{subsection}.\arabic{theorem}.}{\end{sloppypar}\vspace{.1in}}

\newenvironment{slogan}{\addtocounter{subsubsection}{1}\vspace{.1in}\begin{sloppypar}\noindent{\em Slogan}\;\arabic{chapter}.\arabic{section}.\arabic{subsection}.\arabic{subsubsection}. \begin{quote}``\slshape}{"\end{quote}\end{sloppypar}\vspace{.1in}}

%\newenvironment{application}{\addtocounter{subsubsection}{1}\vspace{.1in}\begin{sloppypar}\noindent{\em Application}\;\arabic{chapter}.\arabic{section}.\arabic{subsection}.\arabic{subsubsection}. \begin{quote}}{\end{quote}\end{sloppypar}\vspace{.1in}}
\makeatletter\let\c@figure\c@equation\makeatother%Aligns figure numbering and equation numbering.

\theoremstyle{definition}
\newtheorem{definition}[subsubsection]{Definition}
\newtheorem{notation}[subsubsection]{Notation}
\newtheorem{conjecture}[subsubsection]{Conjecture}
\newtheorem{postulate}[subsubsection]{Postulate}


%\newtheorem{theorem}{Theorem}[subsection]
%\newtheorem{lemma}[theorem]{Lemma}
%\newtheorem{proposition}[theorem]{Proposition}
%\newtheorem{corollary}[theorem]{Corollary}
%\newtheorem{fact}[theorem]{Fact}
%
%\theoremstyle{remark}
%\newtheorem{remark}[theorem]{Remark}
%\newtheorem{example}[theorem]{Example}
%\newtheorem{warning}[theorem]{Warning}
%\newtheorem{question}[theorem]{Question}
%\newtheorem{guess}[theorem]{Guess}
%\newtheorem{answer}[theorem]{Answer}
%\newtheorem{construction}[theorem]{Construction}
%\newtheorem{rules}[theorem]{Rules of good practice}
%\newtheorem{exc}[theorem]{Exercise}
%%\newenvironment{exercise}{\addtocounter{theorem}{1}\vspace{.1in}\begin{sloppypar}\noindent{\em Exercise}\;\arabic{chapter}.\arabic{section}.\arabic{subsection}.\arabic{theorem}.}{\end{sloppypar}\vspace{.1in}}
%\newenvironment{exercise}{\begin{exc}}{\hspace*{\fill}$\lozenge$\end{exc}}
%
%\theoremstyle{definition}
%\newtheorem{definition}[theorem]{Definition}
%\newtheorem{notation}[theorem]{Notation}
%\newtheorem{conjecture}[theorem]{Conjecture}
%\newtheorem{postulate}[theorem]{Postulate}

\def\Finm{{\bf Fin_{m}}}
\def\Prb{{\bf Prb}}
\def\Prbs{{\wt{\bf Prb}}}
\def\El{{\bf El}}
\def\Gr{{\bf Gr}}
\def\DT{{\bf DT}}
\def\DB{{\bf DB}}
\def\Tables{{\bf Tables}}
\def\Sch{{\bf Sch}}
\def\Fin{{\bf Fin}}
\def\P{{\bf P}}
\def\SC{{\bf SC}}
\def\ND{{\bf ND}}
\def\Poset{{\bf Poset}}


\newcommand{\MainCatLarge}[1]{ 
	\stackrel{#1}{
		\parbox{4.5in}{\fbox{\parbox{4.4in}{\begin{center}\underline{{\tt Employee} manager worksIn $\simeq$ {\tt Employee} worksIn}\hsp  \underline{{\tt Department} secretary worksIn $\simeq$ {\tt Department}}\end{center}~\\\\\\
			\xymatrix@=8pt{&\LTO{Employee}\ar@<.5ex>[rrrrr]^{\tn{worksIn}}\ar@(l,u)[]+<5pt,10pt>^{\tn{manager}}\ar[dddl]_{\tn{first}}\ar[dddr]^{\tn{last}}&&&&&\LTO{Department}\ar@<.5ex>[lllll]^{\tn{secretary}}\ar[ddd]^{\tn{name}}\\\\\\\LTO{FirstNameString}&&\LTO{LastNameString}&~&~&~&\LTO{DepartmentNameString}
			}
		}}}
	}
}
\def\GrIn{{\bf GrIn}}
\def\GrInSchema{\xymatrix{\LMO{Ar}\ar@<.5ex>[r]^{src}\ar@<-.5ex>[r]_{tgt}&\LMO{V\!e}}}
%\CompileMatrices
\pdfoutput=1

\setcounter{secnumdepth}{3}
\setcounter{tocdepth}{1}

\def\sub{\begin{itemize}\item}
\def\sexc{\begin{enumerate}[a.)]\setlength{\itemsep}{.1cm}\setlength{\parskip}{.1cm}\item}
\def\next{\item}
\def\endsub{\end{itemize}}
\def\endsexc{\end{enumerate}}

%%%%%

\begin{document}


\title{~\\~\\Category Theory for Scientists\\(Old Version)}
\author{David I. Spivak}
\titlepic{\vspace{1.3in}\dashbox{\includegraphics[width=.8\textwidth]{ScientificMethod}}\\\vspace{.3in}\Large How can mathematics make this diagram meaningful?%What does this diagram intend to communicate?
\normalsize}
\maketitle


\chapter*{Preface}

An early version of this book was put on line in February 2013 to serve as the textbook for my course \href{http://math.mit.edu/~dspivak/teaching/sp13/}{\text Category Theory for Scientists} taught in the spring semester of 2013 at MIT. During that semester, students provided me with hundreds of comments and questions, which led to a substantial improvement (and the addition of 50 pages) to the original document. 

In the summer of 2013 I signed a contract with the MIT Press to publish a new version of this work under the title {\em Category Theory for the Sciences}. Because I am committed to the open source development model I insisted that a version of this book, namely the one you are reading, remain freely available online. The MIT Press version will of course not be free.

Other than the title, there are two main differences between the present version and the MIT Press version. The first difference is that I will do a full edit with the help of professional editors from the Press. The second difference is that I will write up solutions to the book's (approximately 280) exercises; some of these will be included in the published version, whereas the rest will be available by way of a password-protected page, accessible only to professors who teach the subject.

\tableofcontents

%%%%%%%% Chapter %%%%%%%%

\chapter{緒論}

%The title page of this book contains a graphic that we reproduce here. 
%\begin{align}\label{dia:scientific method}
%\dashbox{\includegraphics[width=.7\textwidth]{ScientificMethod}}
%\end{align}
%It is intended to evoke thoughts of the scientific method. 
%\begin{quote}
%A hypothesis analyzed by a person produces a prediction, which motivates the specification of an experiment, which when executed results in an observation, which analyzed by a person yields a hypothesis.
%\end{quote}
%This sounds valid, and a good graphic can be exceptionally useful for leading a reader through the story that the author wishes to tell. 

この本の題扉には, 次に再掲する図が含まれている.
\begin{align}\label{dia:scientific method}
\dashbox{\includegraphics[width=.7\textwidth]{ScientificMethod}}
\end{align}
この図は科学的手法の思想を呼び起こすことを意図している.
\begin{quote}
仮説の分析によって予言が可能となり, それによって個々の実験の動機が生じ, 実験によって観測結果が得られ, それを分析することにより仮説が生じる.
\end{quote}
これは妥当であろう. そしてこのよき図は, この後に著者が語ろうとすることを理解しようとする際に並外れて便利でありうる.

%Interestingly, a graphic has the power to evoke feelings of understanding, without really meaning much. The same is true for text: it is possible to use a language such as English to express ideas that are never made rigorous or clear. When someone says ``I believe in free will," what does she believe in? We may all have some concept of what she's saying---something we can conceptually work with and discuss or argue about. But to what extent are we all discussing the same thing, the thing she intended to convey?
興味深いことに, 図というものは, 実際にはそれほど意味を持ってなくとも, 理解したという気持ちを呼び起こす力を持っている. これと同じことは文章にも言える. 厳密あるいは明快でない考えを, 例えば英語という言語を使って表現することが可能である. ある人が「私は自由意志を信じている」と言った時, その人はいったい何を信じているのだろうか? おそらく我々みなが, その人が言ったことについてなんらかの概念を持っていることだろう. そしてその何かに対して議論や主義主張などの概念的な行為を行なうことができる. しかし, その人が伝達しようと意図したそのことに関して, 我々がいかほど議論できているだろうか?

%Science is about agreement. When we supply a convincing argument, the result of this convincing is agreement. When, in an experiment, the observation matches the hypothesis---success!---that is agreement. When my methods make sense to you, that is agreement. When practice does not agree with theory, that is disagreement. Agreement is the good stuff in science; it's the high fives.
科学とは一致(agreement)である. 我々が納得した議論を提供するとき, その納得の結果は一致する. 実験において, 観測が仮説と整合した時---成功したぞ!---それは一致である.  実際が理論と合わないとき, それは不一致(disagreement)である. 一致は科学のよき素質である. それはまさにハイタッチだ.

%But it is easy to think we're in agreement, when really we're not. Modeling our thoughts on heuristics and pictures may be convenient for quick travel down the road, but we're liable to miss our turnoff at the first mile. The danger is in mistaking our convenient conceptualizations for what's actually there. It is imperative that we have the ability at any time to ground out in reality. What does that mean?
しかし, 実際にはそうでないのに我々がin agreementである場合を考えることもまた容易である. 道筋を駆け足で簡単にたどるにはheuristicsや図は便利であるかもしれないけれども, 一里目にある分かれ道で間違えることも受けいれなければならなくなる.
危険なのは, 実際にそこにあるものに対する簡易な概念化における間違いである. 現実においては我々は常に内野ゴロでアウトにされてしまう. これは何を意味しているのか?

%Data. Hard evidence. The physical world. It is here that science touches down and heuristics evaporate. So let's look again at the diagram on the cover. It is intended to evoke an idea of how science is performed. Is there hard evidence and data to back this theory up? Can we set up an experiment to find out whether science is actually performed according to such a protocol? To do so we have to shake off the stupor evoked by the diagram and ask the question: ``what does this diagram intend to communicate?"
データ. 確実な証拠. 物理世界. そここそが科学の領域でありそこではheuristicsは霧散する. それでは表紙の図をもう一度見てみよう. 
この図は科学がどのように動いているのかについての発想を呼び起こすことを意図している. この理論を支える確実な証拠やデータはどこにあるのだろうか? 科学が実際にこのようなprotocolに従って動いているかどうかを調べるための実験を, 我々は設計できるだろうか? それを行なうために, 我々は図によって呼び起こされる無感覚な状態を振り払って問いを投げかけなければならない. 「この図は何を伝えることを意図しているのだろうか?」
%In this course I will use a mathematical tool called {\em ologs}, or ontology logs, to give some structure to the kinds of ideas that are often communicated in pictures like the one on the cover. Each olog inherently offers a framework in which to record data about the subject. More precisely it encompasses a {\em database schema}, which means a system of interconnected tables that are initially empty but into which data can be entered. For example consider the olog below
%$$\xymatrix{
%\obox{}{.5in}{a mass}&&\obox{}{1.1in}{an object of mass $m$ held at height $h$ above the ground}\LAL{ll}{\footnotesize has as mass}\LA{rrdd}{\hspace{.4in}\parbox{1in}{\singlespace \footnotesize when dropped has as number of seconds till hitting the ground}}\LAL{dd}{\parbox{.7in}{\singlespace\footnotesize has as height in meters}}&&\\\\
%&&\obox{}{1in}{a real number $h$}\ar@{}[uurr]|(.35){?}\ar[rr]_-{\sqrt{2h\div 9.8}}&\hspace{.3in}&\obox{}{.9in}{a real number}
%}
%$$
%This olog represents a framework in which to record data about objects held above the ground, their mass, their height, and a comparison (the ?-mark in the middle) between the number of seconds till they hit the ground and a certain real-valued function of their height. We will discuss ologs in detail throughout this course. 
この教程では, 著者は\emph{olog}あるいはontology logと呼ばれる数学的な道具を使うことになる. ologを用いることによって, 表紙にあるような絵によってしばしば伝達されるアイデアの類に構造を与えることができる. それぞれのologは本質的には題材についてのrecord dataにおける枠組みを提供する. もっと正確には, \emph{database schema}を含む. これは最初は空のそれぞれの結ばれたテーブルにデータを挿入することができることを意味している. 例として以下のologを考える.
\[
\xymatrix{
\obox{}{.5in}{a mass}&&\obox{}{1.1in}{an object of mass $m$ held at height $h$ above the ground}\LAL{ll}{\footnotesize has as mass}\LA{rrdd}{\hspace{.4in}\parbox{1in}{\singlespace \footnotesize when dropped has as number of seconds till hitting the ground}}\LAL{dd}{\parbox{.7in}{\singlespace\footnotesize has as height in meters}}&&\\\\
&&\obox{}{1in}{a real number $h$}\ar@{}[uurr]|(.35){?}\ar[rr]_-{\sqrt{2h\div 9.8}}&\hspace{.3in}&\obox{}{.9in}{a real number}
}
\]
このologは地上から持ち上げた物体に関してのデータにおける枠組みを表現している. すなわち質量, 高さ, そして地上に落ちるまでの秒数の比較(中心の?マーク)となんらかの高さに関する実函数である.
我々はこの教程を通してologに関する議論を行なう.

%The picture in (\ref{dia:scientific method}) looks like an olog, but it does not conform to the rules that we lay out for ologs in Section \ref{sec:ologs}. In an olog, every arrow is intended to represent a mathematical function. It is difficult to imagine a function that takes in predictions and outputs experiments, but such a function is necessary in order for the arrow
%$$\fbox{a prediction}\To{\tn{motivates the specification of}}\fbox{an experiment}
%$$
%in (\ref{dia:scientific method}) to make sense. To produce an experiment design from a prediction probably requires an expert, and even then the expert may be motivated to specify a different experiment on Tuesday than he is on Monday. But perhaps our criticism has led to a way forward: if we say that every arrow represents a function {\em when in the context of a specific expert who is actually doing the science at a specific time}, then Figure (\ref{dia:scientific method}) begins to make sense. In fact, we will return to the figure in Section \ref{sec:monads} (specifically Example \ref{ex:scientific method}), where background methodological context is discussed in earnest.

\eqref{dia:scientific method}の図はologのように見えるが, \ref{sec:ologs}でologのために容易した設計には整合していない. ologでは, それぞれの矢印は数学での函数を表現することが意図されている. 予言を受け取って実験を出力する函数を想像することは困難であるが, \eqref{dia:scientific method}で矢印
\[
\fbox{a prediction}\To{\tn{motivates the specification of}}\fbox{an experiment}
\]
が意味を持つためにはその函数が必須である. 予言から実験を設計するのはおそらく専門家が必要とされるし, 専門家でさえ火曜日には月曜日と異なった実験をしたくなることがあるだろう. しかし, おそらく我々の批判はさらに先へ進むことになる. \emph{ある文脈の下で特定の専門家が特定の時に実際に科学を行なう}とし時に, もし我々が全ての矢印が函数であると主張するのであるならば, その時には図\eqref{dia:scientific method}は意味を持ち始める. 実際に, 我々は第\ref{sec:monads}章で(具体的には例\ref{ex:scientific method})この図に戻り, そこでまじめにbackground methdological contextを議論することになる.

%This course is an attempt to extol the virtues of a new branch of mathematics, called {\em category theory}, which was invented for powerful communication of ideas between different fields and subfields within mathematics. By powerful communication of ideas I actually mean something precise. Different branches of mathematics can be formalized into categories. These categories can then be connected together by functors. And the sense in which these functors provide powerful communication of ideas is that facts and theorems proven in one category can be transferred through a connecting functor to yield proofs of analogous theorems in another category. A functor is like a conductor of mathematical truth.

この教程は\emph{圏論(category theory)}と呼ばれる数学の新しい分野の価値を褒め称えることを目的としている. 圏論は異なった分野と数学の中の分野でのアイデアの間の強力な情報伝達手段として開発された. 様々な数学の分野が圏(category)によって定式化できる. これらの圏は函手(functor)によって結びついている. そしてこれら函手が強力な思考伝達の手段となる理由は, ある圏の中で事実や証明された定理は, 結びついた函手を通して他の圏でのよく似た定理の証明の導出となる. 函手は数学的事実の導管のようなものだ.

I believe that the language and toolset of category theory can be useful throughout science. We build scientific understanding by developing models, and category theory is the study of basic conceptual building blocks and how they cleanly fit together to make such models. Certain structures and conceptual frameworks show up again and again in our understanding of reality. No one would dispute that vector spaces are ubiquitous. But so are hierarchies, symmetries, actions of agents on objects, data models, global behavior emerging as the aggregate of local behavior, self-similarity, and the effect of methodological context. 

Some ideas are so common that our use of them goes virtually undetected, such as set-theoretic intersections. For example, when we speak of a material that is both lightweight and ductile, we are intersecting two sets. But what is the use of even mentioning this set-theoretic fact? The answer is that when we formalize our ideas, our understanding is almost always clarified. Our ability to communicate with others is enhanced, and the possibility for developing new insights expands. And if we are ever to get to the point that we can input our ideas into computers, we will need to be able to formalize these ideas first.

It is my hope that this course will offer scientists a new vocabulary in which to think and communicate, and a new pipeline to the vast array of theorems that exist and are considered immensely powerful within mathematics. These theorems have not made their way out into the world of science, but they are directly applicable there. Hierarchies are partial orders, symmetries are group elements, data models are categories, agent actions are monoid actions, local-to-global\index{local-to-global} principles are sheaves, self-similarity is modeled by operads, context can be modeled by monads.




\end{document}

