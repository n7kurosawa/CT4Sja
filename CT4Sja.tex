\documentclass[a4paper]{ltjsbook}
\usepackage{luatex85}

\usepackage{etex}
\usepackage{savesym}
\usepackage{amssymb, amsmath,amsthm,amscd}
%\savesymbol{lrcorner}
\usepackage{txfonts}
\savesymbol{lrcorner}\savesymbol{llcorner}\savesymbol{urcorner}\savesymbol{ulcorner}
%\usepackage{marvosym}
\usepackage{wasysym}
\savesymbol{Sun}\savesymbol{Mercury}\savesymbol{Venus}\savesymbol{Earth}\savesymbol{Mars}\savesymbol{Jupiter}\savesymbol{Saturn}\savesymbol{Uranus}\savesymbol{Neptune}\savesymbol{Pluto}\savesymbol{leftmoon}\savesymbol{rightmoon}\savesymbol{fullmoon}\savesymbol{newmoon}\savesymbol{Aries}\savesymbol{Taurus}\savesymbol{Gemini}\savesymbol{Leo}\savesymbol{Libra}\savesymbol{Scorpio}\savesymbol{diameter}
\usepackage{mathabx}
%\usepackage{stmaryrd}
\usepackage{setspace}
\usepackage{chngcntr}
\usepackage[tt]{titlepic}
\usepackage{enumerate,makecell}
\usepackage{makeidx,tabularx,dashbox}
\usepackage[usenames,dvipsnames]{xcolor}
\usepackage[unicode,bookmarks=true,colorlinks=true, linkcolor=MidnightBlue, citecolor=cyan]{hyperref}
\usepackage{lmodern}
\usepackage{graphicx,float}
\usepackage{multirow}
\usepackage{geometry}
\geometry{bottom=35mm}
%\newgeometry{left=1.6in,right=1.6in,top=1.4in,bottom=1.4in}
\restoresymbol{txfonts}{lrcorner}\restoresymbol{txfonts}{llcorner}\restoresymbol{txfonts}{urcorner}\restoresymbol{txfonts}{ulcorner}

\usepackage{color}
\usepackage[all,poly,color,matrix,arrow]{xy}
\makeindex

%\usepackage{showkeys}

\renewcommand{\emph}[1]{\textit{\textgt{#1}}}
\renewcommand{\em}{\itshape\gtfamily}

\newcommand{\comment}[1]{}

\newcommand{\longnote}[2][4.9in]{\fcolorbox{black}{yellow}{\parbox{#1}{\color{black} #2}}}
\newcommand{\shortnote}[1]{\fcolorbox{black}{yellow}{\color{black} #1}}
\newcommand{\start}[1]{\shortnote{Start here: #1.}}
\newcommand{\q}[1]{\begin{question}#1\end{question}}
\newcommand{\g}[1]{\begin{guess}#1\end{guess}}
\newcommand{\cfbox}[2]{
    \colorlet{currentcolor}{.}
    {\color{#1}
    \fbox{\color{currentcolor}#2}}
}

%\def\tn{\textnormal}
\DeclareRobustCommand{\tn}[1]{\ifmmode{\text{\normalfont{#1}}}\else{\normalfont{#1}}\fi}
\newcommand{\mf}[1]{\mathfrak{#1}}
\renewcommand{\mc}[1]{\mathcal{#1}} % overwrite \mc (use 明朝)
\newcommand{\qt}[1]{\tn{``}#1\tn{"}}

\newcommand\ZZ{{\mathbb Z}}
\newcommand\QQ{{\mathbb Q}}
\newcommand\RR{{\mathbb R}}
\newcommand\CC{{\mathbb C}}
\renewcommand\AA{{\mathbb A}} % overwrite Angstroem
\newcommand\PP{{\mathbb P}}
\newcommand\NN{{\mathbb N}}


\newcommand\Hom{\tn{Hom}}
\newcommand\Path{\tn{Path}}
\newcommand\Paths{\tn{Paths}}
\newcommand\List{\tn{List}}
\newcommand\Aut{\tn{Aut}}
\newcommand\im{\tn{im}}
\newcommand\Fun{\tn{Fun}}
\newcommand\Ob{\tn{Ob}}
\newcommand\Skel{\tn{Skel}}
\newcommand\Op{\tn{Open}}
\newcommand\PK{\tn{PK}}
\newcommand\FK{\tn{FK}}
\def\SEL*{\tn{SEL*}}
\newcommand\Res{\tn{Res}}
\newcommand\hsp{\hspace{.3in}}
\newcommand{\hsps}[1]{{\hspace{2mm} #1\hspace{2mm}}}
\newcommand{\tin}[1]{\text{\tiny #1}}

\newcommand\singleton{\{\smiley\}}
\newcommand{\boxtitle}[1]{\begin{center}#1\end{center}\vspace{-.1in}}
\newcommand{\singlefun}[1]{\star^{#1}}
\newcommand{\pullb}[1]{\Delta_{#1}}
\newcommand{\lpush}[1]{\Sigma_{#1}}
\newcommand{\rpush}[1]{\Pi_{#1}}
\def\lcone{^\triangleleft}
\def\rcone{^\triangleright}
\renewcommand\to{\rightarrow}
\newcommand\from{\leftarrow}
\newcommand\down{\downarrrow}
\newcommand\Down{\Downarrow}
\newcommand\taking{\colon}
\newcommand\inj{\hookrightarrow}
\newcommand\surj{\twoheadrightarrow}
\newcommand\too{\longrightarrow}
\newcommand{\xyright}[1]{\xymatrix{~\ar[r]#1&}}
\newcommand{\xydown}[1]{\xymatrix{~\ar[d]#1\\~}}
\newcommand{\xydoown}[1]{\xymatrix{~\ar[ddd]#1\\\parbox{0in}{~}\\\parbox{0in}{~}\\~}}
\newcommand\fromm{\longleftarrow}
\newcommand\tooo{\longlongrightarrow}
\newcommand\tto{\rightrightarrows}
\newcommand\ttto{\equiv\!\!>}
\renewcommand\ss{\subseteq}
\newcommand\superset{\supseteq}
\newcommand\iso{\cong}
%\def\down{\downarrow}
\def\|{{\;|\;}}
\def\m1{{-1}}
\def\op{^\tn{op}}
\newcommand\loc{\tn{loc}}
\newcommand\la{\langle}
\newcommand\ra{\rangle}
\newcommand\wt{\widetilde}
\newcommand\wh{\widehat}
\newcommand\we{\simeq}
\def\ol{\overline}
\def\ul{\underline}
\newcommand\plpl{+\!\!+\hspace{1pt}}
\newcommand\acts{\lefttorightarrow}
\newcommand\actsRight{\righttoleftarrow}
\newcommand\vect{\overrightarrow}
\newcommand\qeq{\mathop{=}^?}
\newcommand\del{\partial\,}

\newcommand\rr{\raggedright}

%\newcommand{\LMO}[1]{\bullet^{#1}}
%\newcommand{\LTO}[1]{\bullet^{\tn{#1}}}
\newcommand{\LMO}[1]{\stackrel{#1}{\bullet}}
\newcommand{\LTO}[1]{\stackrel{\tt{#1}}{\bullet}}
\newcommand{\LA}[2]{\ar[#1]^-{\tn {#2}}}
\newcommand{\LAL}[2]{\ar[#1]_-{\tn {#2}}}
\newcommand{\obox}[3]{\stackrel{#1}{\fbox{\parbox{#2}{#3}}}}
\newcommand{\labox}[2]{\obox{#1}{1.6in}{#2}}
\newcommand{\mebox}[2]{\obox{#1}{1in}{#2}}
\newcommand{\smbox}[2]{\stackrel{#1}{\fbox{#2}}}
\newcommand{\fakebox}[1]{\tn{$\ulcorner$#1$\urcorner$}}
\newcommand{\sq}[4]{\xymatrix{#1\ar[r]\ar[d]&#2\ar[d]\\#3\ar[r]&#4}}
\newcommand{\namecat}[1]{\begin{center}$#1:=$\end{center}}

\newcommand\monOb{\blacktriangle}

\def\ullimit{\ar@{}[rd]|(.3)*+{\lrcorner}}
\def\urlimit{\ar@{}[ld]|(.3)*+{\llcorner}}
\def\lllimit{\ar@{}[ru]|(.3)*+{\urcorner}}
\def\lrlimit{\ar@{}[lu]|(.3)*+{\ulcorner}}
\def\ulhlimit{\ar@{}[rd]|(.3)*+{\diamond}}
\def\urhlimit{\ar@{}[ld]|(.3)*+{\diamond}}
\def\llhlimit{\ar@{}[ru]|(.3)*+{\diamond}}
\def\lrhlimit{\ar@{}[lu]|(.3)*+{\diamond}}
\newcommand{\clabel}[1]{\ar@{}[rd]|(.5)*+{#1}}
\newcommand{\TriRight}[7]{\xymatrix{#1\ar[dr]_{#2}\ar[rr]^{#3}&&#4\ar[dl]^{#5}\\&#6\ar@{}[u] |{\Longrightarrow}\ar@{}[u]|>>>>{#7}}}
\newcommand{\TriLeft}[7]{\xymatrix{#1\ar[dr]_{#2}\ar[rr]^{#3}&&#4\ar[dl]^{#5}\\&#6\ar@{}[u] |{\Longleftarrow}\ar@{}[u]|>>>>{#7}}}
\newcommand{\TriIso}[7]{\xymatrix{#1\ar[dr]_{#2}\ar[rr]^{#3}&&#4\ar[dl]^{#5}\\&#6\ar@{}[u] |{\Longleftrightarrow}\ar@{}[u]|>>>>{#7}}}


\newcommand{\arr}[1]{\ar@<.5ex>[#1]\ar@<-.5ex>[#1]}
\newcommand{\arrr}[1]{\ar@<.7ex>[#1]\ar@<0ex>[#1]\ar@<-.7ex>[#1]}
\newcommand{\arrrr}[1]{\ar@<.9ex>[#1]\ar@<.3ex>[#1]\ar@<-.3ex>[#1]\ar@<-.9ex>[#1]}
\newcommand{\arrrrr}[1]{\ar@<1ex>[#1]\ar@<.5ex>[#1]\ar[#1]\ar@<-.5ex>[#1]\ar@<-1ex>[#1]}

\newcommand{\To}[1]{\xrightarrow{#1}}
\newcommand{\Too}[1]{\xrightarrow{\ \ #1\ \ }}
\newcommand{\From}[1]{\xleftarrow{#1}}
\newcommand{\Fromm}[1]{\xleftarrow{\ \ #1\ \ }}

\newcommand{\Adjoint}[4]{\xymatrix@1{#2 \ar@<.5ex>[r]^-{#1} & #3 \ar@<.5ex>[l]^-{#4}}}
\newcommand{\adjoint}[4]{\xymatrix{#1\taking #2\ar@<.5ex>[r]& #3\hspace{1pt}:\hspace{-2pt} #4\ar@<.5ex>[l]}}

\newcommand{\prodmap}[2]{\la#1,#2\ra}
\newcommand{\pb}[3]{\prodmap{#1}{#2}_{#3}}
\newcommand{\coprodmap}[2]{{\left\{\parbox{.1in}{#1\\#2}\right.}}
\newcommand{\po}[3]{\coprodmap{#1}{#2}}

\newcommand\id{\tn{id}}
\newcommand\ids{\tn{ids}}
\renewcommand\Top{\mathbf{Top}}
\newcommand\Cat{\mathbf{Cat}}
\newcommand\Oprd{\mathbf{Oprd}}
\newcommand\Str{\mathbf{Str}}
\newcommand\Mon{\mathbf{Mon}}
\newcommand\Grp{\mathbf{Grp}}
\newcommand\Grph{\mathbf{Grph}}
\newcommand\Type{\mathbf{Type}}
\newcommand\Supp{\mathbf{Supp}}
\newcommand\Dist{\mathbf{Dist}}
\newcommand\Vect{\mathbf{Vect}}
\newcommand\Kls{\mathbf{Kls}}
\newcommand\Prop{\mathbf{Prop}}
\newcommand\FLin{\mathbf{FLin}}
\newcommand\Set{\mathbf{Set}}
\newcommand\Sets{\mathbf{Sets}}
\newcommand\PrO{\mathbf{PrO}}
\newcommand\Star{\mathbf{Star}}
\newcommand\Cob{\mathbf{Cob}}
\newcommand\Qry{\mathbf{Qry}}
\newcommand\set{{\text \textendash}\mathbf{Set}}
\newcommand\sSet{\mathbf{sSet}}
\newcommand\sSets{\mathbf{sSets}}
\newcommand\Grpd{\mathbf{Grpd}}
\newcommand\Pre{\mathbf{Pre}}
\newcommand\Shv{\mathbf{Shv}}
\newcommand\Rings{\mathbf{Rings}}
\newcommand\bD{\mathbf{\Delta}}
\newcommand\dispInt{\parbox{.1in}{$\int$}}
\newcommand\bhline{\Xhline{2\arrayrulewidth}}
\newcommand\bbhline{\Xhline{2.5\arrayrulewidth}}
\newcommand\bbbhline{\Xhline{3\arrayrulewidth}}


\newcommand\colim{\mathop{\tn{colim}}}
\newcommand\hocolim{\mathop{\tn{hocolim}}}

\newcommand\mcA{\mc{A}}
\newcommand\mcB{\mc{B}}
\newcommand\mcC{\mc{C}}
\newcommand\mcD{\mc{D}}
\newcommand\mcE{\mc{E}}
\newcommand\mcF{\mc{F}}
\newcommand\mcG{\mc{G}}
\newcommand\mcH{\mc{H}}
\newcommand\mcI{\mc{I}}
\newcommand\mcJ{\mc{J}}
\newcommand\mcK{\mc{K}}
\newcommand\mcL{\mc{L}}
\newcommand\mcM{\mc{M}}
\newcommand\mcN{\mc{N}}
\newcommand\mcO{\mc{O}}
\newcommand\mcP{\mc{P}}
\newcommand\mcQ{\mc{Q}}
\newcommand\mcR{\mc{R}}
\newcommand\mcS{\mc{S}}
\newcommand\mcT{\mc{T}}
\newcommand\mcU{\mc{U}}
\newcommand\mcV{\mc{V}}
\newcommand\mcW{\mc{W}}
\newcommand\mcX{\mc{X}}
\newcommand\mcY{\mc{Y}}
\newcommand\mcZ{\mc{Z}}

\def\undsc{\rule{2mm}{0.4pt}}
\newcommand\Loop{{\mcL oop}}
\newcommand\LoopSchema{{\parbox{.5in}{\fbox{\xymatrix{\LMO{s}\ar@(l,u)[]^f}}}}}

\newtheorem{theorem}[subsubsection]{Theorem}
\newtheorem{lemma}[subsubsection]{Lemma}
\newtheorem{proposition}[subsubsection]{Proposition}
\newtheorem{corollary}[subsubsection]{Corollary}
\newtheorem{fact}[subsubsection]{Fact}

\theoremstyle{remark}
\newtheorem{remark}[subsubsection]{Remark}
\newtheorem{example}[subsubsection]{Example}
\newtheorem{warning}[subsubsection]{Warning}
\newtheorem{question}[subsubsection]{Question}
\newtheorem{guess}[subsubsection]{Guess}
\newtheorem{answer}[subsubsection]{Answer}
\newtheorem{construction}[subsubsection]{Construction}
\newtheorem{rules}[subsubsection]{Rules of good practice}
\newtheorem{exc}[subsubsection]{Exercise}
\newenvironment{exercise}{\begin{exc}}{\hspace*{\fill}$\lozenge$\end{exc}}
\newtheorem{app}[subsubsection]{Application}
\newenvironment{application}{\begin{app}}{\hspace*{\fill}$\lozenge\lozenge$\end{app}}

%\newenvironment{exercise}{\addtocounter{theorem}{1}\vspace{.1in}\begin{sloppypar}\noindent{\em Exercise}\;\arabic{chapter}.\arabic{section}.\arabic{subsection}.\arabic{theorem}.}{\end{sloppypar}\vspace{.1in}}

\newenvironment{slogan}{\addtocounter{subsubsection}{1}\vspace{.1in}\begin{sloppypar}\noindent{\em Slogan}\;\arabic{chapter}.\arabic{section}.\arabic{subsection}.\arabic{subsubsection}. \begin{quote}``\slshape}{"\end{quote}\end{sloppypar}\vspace{.1in}}

%\newenvironment{application}{\addtocounter{subsubsection}{1}\vspace{.1in}\begin{sloppypar}\noindent{\em Application}\;\arabic{chapter}.\arabic{section}.\arabic{subsection}.\arabic{subsubsection}. \begin{quote}}{\end{quote}\end{sloppypar}\vspace{.1in}}
\makeatletter\let\c@figure\c@equation\makeatother%Aligns figure numbering and equation numbering.

\theoremstyle{definition}
\newtheorem{definition}[subsubsection]{Definition}
\newtheorem{notation}[subsubsection]{Notation}
\newtheorem{conjecture}[subsubsection]{Conjecture}
\newtheorem{postulate}[subsubsection]{Postulate}


%\newtheorem{theorem}{Theorem}[subsection]
%\newtheorem{lemma}[theorem]{Lemma}
%\newtheorem{proposition}[theorem]{Proposition}
%\newtheorem{corollary}[theorem]{Corollary}
%\newtheorem{fact}[theorem]{Fact}
%
%\theoremstyle{remark}
%\newtheorem{remark}[theorem]{Remark}
%\newtheorem{example}[theorem]{Example}
%\newtheorem{warning}[theorem]{Warning}
%\newtheorem{question}[theorem]{Question}
%\newtheorem{guess}[theorem]{Guess}
%\newtheorem{answer}[theorem]{Answer}
%\newtheorem{construction}[theorem]{Construction}
%\newtheorem{rules}[theorem]{Rules of good practice}
%\newtheorem{exc}[theorem]{Exercise}
%%\newenvironment{exercise}{\addtocounter{theorem}{1}\vspace{.1in}\begin{sloppypar}\noindent{\em Exercise}\;\arabic{chapter}.\arabic{section}.\arabic{subsection}.\arabic{theorem}.}{\end{sloppypar}\vspace{.1in}}
%\newenvironment{exercise}{\begin{exc}}{\hspace*{\fill}$\lozenge$\end{exc}}
%
%\theoremstyle{definition}
%\newtheorem{definition}[theorem]{Definition}
%\newtheorem{notation}[theorem]{Notation}
%\newtheorem{conjecture}[theorem]{Conjecture}
%\newtheorem{postulate}[theorem]{Postulate}

\newcommand\Finm{\mathbf{Fin_{m}}}
\newcommand\Prb{\mathbf{Prb}}
\newcommand\Prbs{{\wt{\mathbf{Prb}}}}
\newcommand\El{\mathbf{El}}
\newcommand\Gr{\mathbf{Gr}}
\newcommand\DT{\mathbf{DT}}
\newcommand\DB{\mathbf{DB}}
\newcommand\Tables{\mathbf{Tables}}
\newcommand\Sch{\mathbf{Sch}}
\newcommand\Fin{\mathbf{Fin}}
\renewcommand\P{\mathbf{P}}
\newcommand\SC{\mathbf{SC}}
\newcommand\ND{\mathbf{ND}}
\newcommand\Poset{\mathbf{Poset}}


\newcommand{\MainCatLarge}[1]{ 
	\stackrel{#1}{
		\parbox{4.5in}{\fbox{\parbox{4.4in}{\begin{center}\underline{{\tt Employee} manager worksIn $\simeq$ {\tt Employee} worksIn}\hsp  \underline{{\tt Department} secretary worksIn $\simeq$ {\tt Department}}\end{center}~\\\\\\
			\xymatrix@=8pt{&\LTO{Employee}\ar@<.5ex>[rrrrr]^{\tn{worksIn}}\ar@(l,u)[]+<5pt,10pt>^{\tn{manager}}\ar[dddl]_{\tn{first}}\ar[dddr]^{\tn{last}}&&&&&\LTO{Department}\ar@<.5ex>[lllll]^{\tn{secretary}}\ar[ddd]^{\tn{name}}\\\\\\\LTO{FirstNameString}&&\LTO{LastNameString}&~&~&~&\LTO{DepartmentNameString}
			}
		}}}
	}
}
\newcommand\GrIn{\mathbf{GrIn}}
\newcommand\GrInSchema{\xymatrix{\LMO{Ar}\ar@<.5ex>[r]^{src}\ar@<-.5ex>[r]_{tgt}&\LMO{V\!e}}}
%\CompileMatrices
\pdfoutput=1

\setcounter{secnumdepth}{3}
\setcounter{tocdepth}{1}

\def\sub{\begin{itemize}\item}
\def\sexc{\begin{enumerate}[a.)]\setlength{\itemsep}{.1cm}\setlength{\parskip}{.1cm}\item}
\def\next{\item}
\def\endsub{\end{itemize}}
\def\endsexc{\end{enumerate}}

%%%%%

\begin{document}


\title{~\\~\\Category Theory for Scientists\\(Old Version)}
\author{David I. Spivak}
\titlepic{\vspace{1.3in}\dashbox{\includegraphics[width=.8\textwidth]{ScientificMethod}}\\\vspace{.3in}\Large How can mathematics make this diagram meaningful?%What does this diagram intend to communicate?
\normalsize}
\maketitle


\chapter*{Preface}

An early version of this book was put on line in February 2013 to serve as the textbook for my course \href{http://math.mit.edu/~dspivak/teaching/sp13/}{\text Category Theory for Scientists} taught in the spring semester of 2013 at MIT. During that semester, students provided me with hundreds of comments and questions, which led to a substantial improvement (and the addition of 50 pages) to the original document. 

In the summer of 2013 I signed a contract with the MIT Press to publish a new version of this work under the title {\em Category Theory for the Sciences}. Because I am committed to the open source development model I insisted that a version of this book, namely the one you are reading, remain freely available online. The MIT Press version will of course not be free.

Other than the title, there are two main differences between the present version and the MIT Press version. The first difference is that I will do a full edit with the help of professional editors from the Press. The second difference is that I will write up solutions to the book's (approximately 280) exercises; some of these will be included in the published version, whereas the rest will be available by way of a password-protected page, accessible only to professors who teach the subject.

\tableofcontents



%%%%%%%% Chapter %%%%%%%%

\chapter{緒論}

%The title page of this book contains a graphic that we reproduce here. 
%\begin{align}\label{dia:scientific method}
%\dashbox{\includegraphics[width=.7\textwidth]{ScientificMethod}}
%\end{align}
%It is intended to evoke thoughts of the scientific method. 
%\begin{quote}
%A hypothesis analyzed by a person produces a prediction, which motivates the specification of an experiment, which when executed results in an observation, which analyzed by a person yields a hypothesis.
%\end{quote}
%This sounds valid, and a good graphic can be exceptionally useful for leading a reader through the story that the author wishes to tell. 

この本の題扉には, 次に再掲する図が含まれている.
\begin{align}\label{dia:scientific method}
\dashbox{\includegraphics[width=.7\textwidth]{ScientificMethod}}
\end{align}
この図は科学的手法の思想を呼び起こすことを意図している.
\begin{quote}
仮説の分析によって予言が可能となり, それによって個々の実験の動機が生じ, 実験によって観測結果が得られ, それを分析することにより仮説が生じる.
\end{quote}
これは妥当であろう. そしてこのよき図は, この後に著者が語ろうとすることを理解しようとする際に並外れて便利でありうる.

%Interestingly, a graphic has the power to evoke feelings of understanding, without really meaning much. The same is true for text: it is possible to use a language such as English to express ideas that are never made rigorous or clear. When someone says ``I believe in free will," what does she believe in? We may all have some concept of what she's saying---something we can conceptually work with and discuss or argue about. But to what extent are we all discussing the same thing, the thing she intended to convey?
興味深いことに, 図というものは, 実際にはそれほど意味を持ってなくとも, 理解したという気持ちを呼び起こす力を持っている. これと同じことは文章にも言える. 厳密あるいは明快でない考えを, 例えば英語という言語を使って表現することが可能である. ある人が「私は自由意志を信じている」と言った時, その人はいったい何を信じているのだろうか? おそらく我々みなが, その人が言ったことについてなんらかの概念を持っていることだろう. そしてその何かに対して議論や主義主張などの概念的な行為を行なうことができる. しかし, その人が伝達しようと意図したそのことに関して, 我々がいかほど議論できているだろうか?

%Science is about agreement. When we supply a convincing argument, the result of this convincing is agreement. When, in an experiment, the observation matches the hypothesis---success!---that is agreement. When my methods make sense to you, that is agreement. When practice does not agree with theory, that is disagreement. Agreement is the good stuff in science; it's the high fives.
科学とは一致(agreement)である. 我々が納得した議論を提供するとき, その納得の結果は一致する. 実験において, 観測が仮説と整合した時---成功したぞ!---それは一致である.  実際が理論と合わないとき, それは不一致(disagreement)である. 一致は科学のよき素質である. それはまさにハイタッチだ.

%But it is easy to think we're in agreement, when really we're not. Modeling our thoughts on heuristics and pictures may be convenient for quick travel down the road, but we're liable to miss our turnoff at the first mile. The danger is in mistaking our convenient conceptualizations for what's actually there. It is imperative that we have the ability at any time to ground out in reality. What does that mean?
しかし, 実際にはそうでないのに我々がin agreementである場合を考えることもまた容易である. 道筋を駆け足で簡単にたどるにはheuristicsや図は便利であるかもしれないけれども, 一里目にある分かれ道で間違えることも受けいれなければならなくなる.
危険なのは, 実際にそこにあるものに対する簡易な概念化における間違いである. 現実においては我々は常に内野ゴロでアウトにされてしまう. これは何を意味しているのか?

%Data. Hard evidence. The physical world. It is here that science touches down and heuristics evaporate. So let's look again at the diagram on the cover. It is intended to evoke an idea of how science is performed. Is there hard evidence and data to back this theory up? Can we set up an experiment to find out whether science is actually performed according to such a protocol? To do so we have to shake off the stupor evoked by the diagram and ask the question: ``what does this diagram intend to communicate?"
データ. 確実な証拠. 物理世界. そここそが科学の領域でありそこではheuristicsは霧散する. それでは表紙の図をもう一度見てみよう. 
この図は科学がどのように動いているのかについての発想を呼び起こすことを意図している. この理論を支える確実な証拠やデータはどこにあるのだろうか? 科学が実際にこのようなprotocolに従って動いているかどうかを調べるための実験を, 我々は設計できるだろうか? それを行なうために, 我々は図によって呼び起こされる無感覚な状態を振り払って問いを投げかけなければならない. 「この図は何を伝えることを意図しているのだろうか?」
%In this course I will use a mathematical tool called {\em ologs}, or ontology logs, to give some structure to the kinds of ideas that are often communicated in pictures like the one on the cover. Each olog inherently offers a framework in which to record data about the subject. More precisely it encompasses a {\em database schema}, which means a system of interconnected tables that are initially empty but into which data can be entered. For example consider the olog below
%$$\xymatrix{
%\obox{}{.5in}{a mass}&&\obox{}{1.1in}{an object of mass $m$ held at height $h$ above the ground}\LAL{ll}{\footnotesize has as mass}\LA{rrdd}{\hspace{.4in}\parbox{1in}{\singlespace \footnotesize when dropped has as number of seconds till hitting the ground}}\LAL{dd}{\parbox{.7in}{\singlespace\footnotesize has as height in meters}}&&\\\\
%&&\obox{}{1in}{a real number $h$}\ar@{}[uurr]|(.35){?}\ar[rr]_-{\sqrt{2h\div 9.8}}&\hspace{.3in}&\obox{}{.9in}{a real number}
%}
%$$
%This olog represents a framework in which to record data about objects held above the ground, their mass, their height, and a comparison (the ?-mark in the middle) between the number of seconds till they hit the ground and a certain real-valued function of their height. We will discuss ologs in detail throughout this course. 
この教程では, 著者は\emph{olog}あるいはontology logと呼ばれる数学的な道具を使うことになる. ologを用いることによって, 表紙にあるような絵によってしばしば伝達されるアイデアの類に構造を与えることができる. それぞれのologは本質的には題材についてのrecord dataにおける枠組みを提供する. もっと正確には, \emph{database schema}を含む. これは最初は空のそれぞれの結ばれたテーブルにデータを挿入することができることを意味している. 例として以下のologを考える.
\[
\xymatrix{
\obox{}{.5in}{a mass}&&\obox{}{1.1in}{an object of mass $m$ held at height $h$ above the ground}\LAL{ll}{\footnotesize has as mass}\LA{rrdd}{\hspace{.4in}\parbox{1in}{\singlespace \footnotesize when dropped has as number of seconds till hitting the ground}}\LAL{dd}{\parbox{.7in}{\singlespace\footnotesize has as height in meters}}&&\\\\
&&\obox{}{1in}{a real number $h$}\ar@{}[uurr]|(.35){?}\ar[rr]_-{\sqrt{2h\div 9.8}}&\hspace{.3in}&\obox{}{.9in}{a real number}
}
\]
このologは地上から持ち上げた物体に関してのデータにおける枠組みを表現している. すなわち質量, 高さ, そして地上に落ちるまでの秒数の比較(中心の?マーク)となんらかの高さに関する実函数である.
我々はこの教程を通してologに関する議論を行なう.

%The picture in (\ref{dia:scientific method}) looks like an olog, but it does not conform to the rules that we lay out for ologs in Section \ref{sec:ologs}. In an olog, every arrow is intended to represent a mathematical function. It is difficult to imagine a function that takes in predictions and outputs experiments, but such a function is necessary in order for the arrow
%$$\fbox{a prediction}\To{\tn{motivates the specification of}}\fbox{an experiment}
%$$
%in (\ref{dia:scientific method}) to make sense. To produce an experiment design from a prediction probably requires an expert, and even then the expert may be motivated to specify a different experiment on Tuesday than he is on Monday. But perhaps our criticism has led to a way forward: if we say that every arrow represents a function {\em when in the context of a specific expert who is actually doing the science at a specific time}, then Figure (\ref{dia:scientific method}) begins to make sense. In fact, we will return to the figure in Section \ref{sec:monads} (specifically Example \ref{ex:scientific method}), where background methodological context is discussed in earnest.

\eqref{dia:scientific method}の図はologのように見えるが, \ref{sec:ologs}でologのために容易した設計には整合していない. ologでは, それぞれの矢印は数学での函数を表現することが意図されている. 予言を受け取って実験を出力する函数を想像することは困難であるが, \eqref{dia:scientific method}で矢印
\[
\fbox{a prediction}\To{\tn{motivates the specification of}}\fbox{an experiment}
\]
が意味を持つためにはその函数が必須である. 予言から実験を設計するのはおそらく専門家が必要とされるし, 専門家でさえ火曜日には月曜日と異なった実験をしたくなることがあるだろう. しかし, おそらく我々の批判はさらに先へ進むことになる. \emph{ある文脈の下で特定の専門家が特定の時に実際に科学を行なう}とし時に, もし我々が全ての矢印が函数であると主張するのであるならば, その時には図\eqref{dia:scientific method}は意味を持ち始める. 実際に, 我々は第\ref{sec:monads}章で(具体的には例\ref{ex:scientific method})この図に戻り, そこでまじめにbackground methdological contextを議論することになる.

%This course is an attempt to extol the virtues of a new branch of mathematics, called {\em category theory}, which was invented for powerful communication of ideas between different fields and subfields within mathematics. By powerful communication of ideas I actually mean something precise. Different branches of mathematics can be formalized into categories. These categories can then be connected together by functors. And the sense in which these functors provide powerful communication of ideas is that facts and theorems proven in one category can be transferred through a connecting functor to yield proofs of analogous theorems in another category. A functor is like a conductor of mathematical truth.

この教程は\emph{圏論(category theory)}と呼ばれる数学の新しい分野の価値を褒め称えることを目的としている. 圏論は異なった分野と数学の中の分野でのアイデアの間の強力な情報伝達手段として開発された. 様々な数学の分野が圏(category)によって定式化できる. これらの圏は函手(functor)によって結びついている. そしてこれら函手が強力な思考伝達の手段となる理由は, ある圏の中で事実や証明された定理は, 結びついた函手を通して他の圏でのよく似た定理の証明の導出となる. 函手は数学的事実の導管のようなものだ.

%I believe that the language and toolset of category theory can be useful throughout science. We build scientific understanding by developing models, and category theory is the study of basic conceptual building blocks and how they cleanly fit together to make such models. Certain structures and conceptual frameworks show up again and again in our understanding of reality. No one would dispute that vector spaces are ubiquitous. But so are hierarchies, symmetries, actions of agents on objects, data models, global behavior emerging as the aggregate of local behavior, self-similarity, and the effect of methodological context. 

著者は圏論の言葉と道具一式は科学の全域において有用だと信じている. 我々は模型を開発することにより科学的な理解を構築する. そして圏論とは基本的な概念の構成要素と, それらがどのように模型にきれいに適用できるかの研究である. ある構造と概念の枠組みは我々の現実の理解において繰り返し繰り返し出現する. ベクトル空間が普遍的であるかどうかを議論する人間はいないであろう. しかしながら, 階層性, 対称性, 物質に対するなんらかの作用, データ模型, 局所的なふるまいの集合によって創発する大域的なふるまい, 自己相似性, そして方法論の文脈による効果.

%Some ideas are so common that our use of them goes virtually undetected, such as set-theoretic intersections. For example, when we speak of a material that is both lightweight and ductile, we are intersecting two sets. But what is the use of even mentioning this set-theoretic fact? The answer is that when we formalize our ideas, our understanding is almost always clarified. Our ability to communicate with others is enhanced, and the possibility for developing new insights expands. And if we are ever to get to the point that we can input our ideas into computers, we will need to be able to formalize these ideas first.

いくつかのアイデア, 例えは集合の共通部分などは, あまりにもありふれているので, それらの使用は実質的には検知できない. 例えば, 我々がある物質について軽くて延性があると言うとき我々は二つの集合の共通部分を取っている. しかしこの淡々とした集合論的事実の記述は何を意味しているのか? その答は, 我々がアイデアを形式化したとき, 我々の理解はほとんど常に明確である. 我々の他者との情報伝達能力は増強され, 新しい洞察を生みだす可能性がひろがっていく. そしてもし我々がアイデアを計算機に入力できるようになった時点では常に, 我々は最初にアイデアを形式化できるようになっている必要がある.

%It is my hope that this course will offer scientists a new vocabulary in which to think and communicate, and a new pipeline to the vast array of theorems that exist and are considered immensely powerful within mathematics. These theorems have not made their way out into the world of science, but they are directly applicable there. Hierarchies are partial orders, symmetries are group elements, data models are categories, agent actions are monoid actions, local-to-global\index{local-to-global} principles are sheaves, self-similarity is modeled by operads, context can be modeled by monads.

著者は, この教程が, 科学者に思考や情報伝達に用いる新しい語彙と, 数学の中に存在していて大いに強力だと考えられている大量の整備された定理への経路をもたらすことを期待している. これらの定理は科学の世界ではこれまでのところは真価を発揮しているとは言えないが, そこに直接適用することができる. 階層性はpartial orderであり, 対称性は群の要素であり, データ模型は圏であり, agent actionはモノイドの作用であり, local-to-global\index{local-to-global}原理は層(sheve)であり, 自己相似性はoperandによってモデル化され, 文脈はモナドによってモデル化される.

%%%%%% Section %%%%%%

%\section{A brief history of category theory}

\section{圏論の簡潔な歴史}

%The paradigm shift brought on by Einstein's theory of relativity brought on the realization that there is no single perspective from which to view the world. There is no background framework that we need to find; there are infinitely many different frameworks and perspectives, and the real power lies in being able to translate between them. It is in this historical context that category theory got its start.
%\footnote{The following history of category theory is far too brief, and perhaps reflects more of the author's aesthetic than any kind of objective truth, whatever that may mean. Here are some much better references: \cite{Kro}, \cite{Mar1}, \cite{LM}.}

Einsteinの相対性理論によってもたらされたパラダイムシフトは, 世界を観るにあたって唯一の観点というものは存在しないという認識をもたらした. 我々が探し求めなければならない背景の枠組みというものは存在しない. 無限に多い異なった枠組みと観点が存在しており, 真なる力はそれらの間を翻訳するというところに存在している. 歴史的には, 圏論はこの文脈から開始した.
\footnote{以下の圏論の歴史は簡潔に過ぎ, それが何を意味するかは別にして, おそらく客観的な真実の類よりも著者の美的感覚を反映していることだろう. はるかによい参考文献を次に挙げる: \cite{Kro}, \cite{Mar1}, \cite{LM}.}. 

%Category theory was invented in the early 1940s by Samuel Eilenberg\index{Eilenberg, Samuel} and Saunders Mac Lane.\index{Mac Lane, Saunders} It was specifically designed to bridge what may appear to be two quite different fields: topology and algebra. Topology is the study of abstract shapes such as 7-dimensional spheres; algebra is the study of abstract equations such as $y^2z=x^3-xz^2$. People had already created important and useful links (e.g. cohomology theory) between these fields, but Eilenberg and Mac Lane needed to precisely compare different links with one another. To do so they first needed to boil down and extract the fundamental nature of these two fields. But the ideas they worked out amounted to a framework that fit not only topology and algebra, but many other mathematical disciplines as well.

圏論は1940年代始めにSamuel Eilenberg\index{Eilenberg, Samuel}とSaunders Mac Lane\index{Mac Lane, Saunders}によって発明された. 圏論は, 具体的には, トポロジーと代数という非常に異なったように見える二つの分野の間に橋を架けるために設計された. トポロジーは, 例えば7次元球のような抽象的な形の研究であり, 代数は$y^2z=x^3-xz^2$のような抽象的な方程式の研究である. これらの分野の間には, 重要で便利な結び付き(e.g. コホモロジー理論)が既に発見され存在していた. しかしEilenbergとMac Laneはそれぞれの結び付きを正確に比較する必要があった. これを遂行するために, 彼等は最初にこれら二つの分野から基本的な性質を煮詰めて抽出する必要があった. ところが, 彼等が考え出した枠組みはトポロジーと代数だけでなく, 他の多くの数学分野にも適用できるものになっていた.

%At first category theory was little more than a deeply clarifying language for existing difficult mathematical ideas. However, in 1957 Alexander Grothendieck\index{Grothendieck!in history} used category theory to build new mathematical machinery (new cohomology theories) that granted unprecedented insight into the behavior of algebraic equations. Since that time, categories have been built specifically to zoom in on particular features of mathematical subjects and study them with a level of acuity that is simply unavailable elsewhere.

圏論は, 最初は既に存在する数学的アイデアを非常に明確するする言語という以上のものではなかった. しかし, 1957年に, Alexander Grothendieck\index{Grothendieck!in history}は圏論を用いて, 代数方程式のふるまいに対してこれまで存在しなかった洞察をもたらす新しい数学的手続き(新しいコホモロジー理論)を構築した. その瞬間から, 圏はとりわけ数学的対象の詳しい特徴を拡大するために, また他では簡単には得られないような鋭敏さのレベルでもって研究を行うために, 構築されてきた.

%Bill Lawvere\index{Lawvere, William} saw category theory as a new foundation for all mathematical thought. Mathematicians had been searching for foundations in the 19th century and were reasonably satisfied with set theory as {\em the foundation}. But Lawvere showed that the category of sets is simply a category with certain nice properties, not necessarily the center of the mathematical universe. He explained how whole algebraic theories can be viewed as examples of a single system. He and others went on to show that higher order logic was beautifully captured in the setting of category theory (more specifically toposes). It is here also that Grothendieck and his school worked out major results in algebraic geometry. 

Bill Lawvere\index{Lawvere, William}は圏論を全ての数学的思考対象の新しい基礎とみなした. 19世紀に数学者は数学の基礎を探し求め, そして\emph{基礎(foundation)}として納得がいく形で集合論に満足した. しかしLawvereは, 集合の圏は単にいくつかのよい性質を持った圏の一つであって, 必ずしも数学の中心にいなければならないものではないということを示した. 彼は, どのようにして代数的理論全体を単一の系の例の一つとしてみることができるかを説明した. 彼と仲間たちは, 高階論理を圏論の設定(より具体的にはトポス)に美しく当てはめることに邁進した. Grothendieckとその学派の代数幾何における重要な結果もまたこの文脈にある.

%In 1980 Joachim Lambek\index{Lambek, Joachim} showed that the types and programs used in computer science form a specific kind of category. This provided a new semantics for talking about programs, allowing people to investigate how programs combine and compose to create other programs, without caring about the specifics of implementation. Eugenio Moggi\index{Moggi, Eugenio} brought the category theoretic notion of monads into computer science to encapsulate ideas that up to that point were considered outside the realm of such theory..

1980年, Joachim Lambek\index{Lambek, Joachim}は型とプログラムが特定の種の圏をなしていることを示した. これによってプログラムについて語るための新しい意味論が得られ, 実装の詳細について考慮することなく, プログラムを結合・合成して他のプログラムを生成することがどのようなことであるかを調べることができるようになった. Eugenio Moggi\index{Moggi, Eugenio}は, 彼が提案するまではこのような理論の適用範囲外だと考えられていた観念をカプセル化するため, 計算機科学にモナドという圏論の概念を導入した.

%It is difficult to explain the clarity and beauty brought to category theory by people like Daniel Kan\index{Kan, Daniel} and Andr\'{e} Joyal\index{Joyal, Andr\'{e}}. They have each repeatedly extracted the essence of a whole mathematical subject to reveal and formalize a stunningly simple yet extremely powerful pattern of thinking, revolutionizing how mathematics is done. 

Daniel Kan\index{Kan, Daniel}やAndr\'{e} Joyal\index{Joyal, Andr\'{e}}といった人々が圏論にもたらした明確さと美しさを説明するのは難しい. 彼らは, 華麗なほどに単純にもかかわらずとてつもなく強力な, 数学的が機能するところの思考と革新の様式を形式化して明らかにするために, それぞれ繰り返し数学的対象の全体からその本質を抽出した.

%All this time, however, category theory was consistently seen by much of the mathematical community as ridiculously abstract. But in the 21st century it has finally come to find healthy respect within the larger community of pure mathematics. It is the language of choice for graduate-level algebra and topology courses, and in my opinion will continue to establish itself as the basic framework in which mathematics is done.

ここまでの全ての時代において, 数学のほとんどのコミュニティにおいて圏論は笑ってしまうくらい抽象的だと一貫してみなされてきた. しかし21世紀になって遂に圏論は純粋数学者の多くのコミュニティにおいて健全であると見いだされることとなった. 圏論は大学院水準の代数とトポロジーの教程のために選択する言語であり, そして著者の意見では, 圏論はその中で数学が機能するところの基本的な枠組みとして, それ自身が確立され続けている. 

%As mentioned above category theory has branched out into certain areas of science as well. Baez\index{Baez, John} and Dolan\index{Dolan, James} have shown its value in making sense of quantum physics, it is well established in computer science, and it has found proponents in several other fields as well. But to my mind, we are the very beginning of its venture into scientific methodology. Category theory was invented as a bridge and it will continue to serve in that role. 

上で述べたように, 圏論は科学の特定の分野にも広がっている. Baez\index{Baez, John}とDolan\index{Dolan, James}は量子物理学の意味づけにおいてその価値を示した. 圏論は計算機科学でよく確立しており, いくつかの他の分野でも支持者が見いだされる. しかし著者の考えでは, まだ人類は科学的方法論への冒険の出発地点にいる. 圏論は異分野の架け橋として発明され, 今後もその役割を果たし続けるだろう.


%%%%%% Section %%%%%%

%\section{Intention of this book}
\section{この本の目的}

%The world of {\em applied mathematics} is much smaller than the world of {\em applicable mathematics}. As alluded to above, this course is intended to create a bridge between the vast array of mathematical concepts that are used daily by mathematicians to describe all manner of phenomena that arise in our studies, and the models and frameworks of scientific disciplines such as physics, computation, and neuroscience. 

\emph{応用数学(applied mathematics)}の世界は\emph{応用可能な数学(applicable mathematics)}の世界よりもはるかに小さい. これまでにほめのかしているように, この教程は研究の中で生じる現象全てを記述するために数学者たちが日々使っている莫大な数学的概念と, 例えば物理学, 計算機科学, 神経科学といった科学的分野の模型や枠組みの間に橋をわたすことを目的としている.

%To the pure mathematician I'll try to prove that concepts such as categories, functors, natural transformations, limits, colimits, functor categories, sheaves, monads, and operads---concepts that are often considered too abstract for even math majors---can be communicated to scientists with no math background beyond linear algebra. If this material is as teachable as I think, it means that category theory is not esoteric but somehow well-aligned with ideas that already make sense to the scientific mind. Note, however, that this book is example-based rather than proof-based, so it may not be suitable as a reference for students of pure mathematics.

純粋数学者に対しては, 圏, 函手, 自然変換, 極限, 余極限, 函手圏, 層, モナド, オペラドのような概念---しばしば数学専攻でさえも抽象的すぎるとみなされるような概念---を, 線形代数以上の数学的背景を持たない科学者に届けられることを, 著者は証明していきたいと思っている. もしこのような概念が著者が考えているように教示可能であるならば, それはすなわち, 圏論は難解なものではなくて, 科学者の脳内において既に意味をなしているアイデアになんらかの形でよくあてはまっているということを意味している. しかしながら, この本は証明中心ではなく例を中心としているため, 純粋数学の学生に対しては参考書として向いていないだろうことには, 注意して欲しい.

%To the scientist I'll try to prove the claim that category theory includes a formal treatment of conceptual structures that the scientist sees often, perhaps without realizing that there is well-oiled mathematical machinery to be employed. We will work on the structure of information; how data is made meaningful by its connections, both internal and outreaching, to other data. Note, however, that this book should most certainly not be taken as a reference on scientific matters themselves. One should assume that any account of physics, materials science, chemistry, etc. has been oversimplified.\index{a warning!oversimplified science} The intention is to give a flavor of how category theory may help us model scientific ideas, not to explain these ideas in a serious way. 

科学者に対しては, 科学者がよく見るところの概念構造, おそらくよく油がさされた数学的機構が供えつけられていることに気付いていないであろうところのその構造の, 形式的な取り扱いを圏論は包含しているという主張を, 著者は証明していきたいと思っている. 我々は情報の構造について取りくむことになる. すなわち, データはその内部および外部の他のデータとの間の結びつきによって, どのように意味を持つことになるのか, ということについてである. しかしながら, おそらくほとんどの場合において, 科学的な対象それ自身に関しての参考文献としてはこの本を使うべきでないことに注意して欲しい. 物理学, 物質科学, 化学, etc. の記述はどれも過度に単純化しすぎていると認識するべきである.\index{a warning!oversimplified science} 本書の目的は, どのようにして圏論の持ち味が我々の科学的思考模型の力になるかを示すことであって, これらのアイデアに真剣に取りくむための説明をするつもりはない.

%Data gathering is ubiquitous in science. Giant databases are currently being mined for unknown patterns, but in fact there are many (many) known patterns that simply have not been catalogued. Consider the well-known case of medical records. A patient's medical history is often known by various individual doctor-offices but quite inadequately shared between them. Sharing medical records often means faxing a hand-written note or a filled-in house-created form between offices. 

データ収集は科学において普遍的である. 現在, 未知のパターンを求めて巨大なデータベースの解析が行なわれているが, 実際のところそこには既知だが未だ名付けられていないとても(とても)多くのパターンが存在している. よく知られている例として医療記録を考えてみよう. 患者の治療履歴はしばしば様々な個々の医院において知られているが, しかしそれらの間では非常に不適切な形でしか共有されていない. 医療記録の共有がは, しばしば手書きのノートあるいは自家製の書式で書き込んだものをファックスでやりとりすることを意味している. 

%Similarly, in science there exists substantial expertise making brilliant connections between concepts, but it is being conveyed in silos of English prose known as journal articles. Every scientific journal article has a methods section, but it is almost impossible to read a methods section and subsequently repeat the experiment---the English language is inadequate to precisely and concisely convey what is being done.

同様に, 科学においては概念の間に光輝く橋をわたす本物の専門知識が存在している. しかしそれらは学術雑誌の記事として知られるそびえたつ散文英語の山によって伝達されている. 科学雑誌の各々の記事には手法(method)と呼ばれる章がある. しかし手法の章を読むのはほとんど不可能だしそれゆえに実験を繰り返すのもほとんど不可能である---なにが行なわれたかを正確かつ簡潔に伝達するにあたって, 英語は適切ではない.\footnote{訳注:もちろん英語以外の自然言語も適切ではない.}

%The first thing to understand in this course is that reusable methodologies can be formalized, and that doing so is inherently valuable. Consider the following analogy. Suppose you want to add up the area of a region in space (or the area under a curve). You break the region down into small squares, each of which you know has area $A$; then you count the number of squares, say $n$, and the result is that the region has an area of about $nA$. If you want a more precise and accurate result you repeat the process with half-size squares. This methodology can be used for any area-finding problem (of which there are more than a first-year calculus student generally realizes) and thus it deserves to be formalized. But once we have formalized this methodology, it can be taken to its limit and out comes integration by Riemann sums. 

この教程で最初に理解するべきことは, 再利用可能な方法論は形式化することができ, そして形式化することは本質的に価値を持つということである. 類推として以下を見てみよう. 空間上の領域の面積(あるいは曲線下面積)を足し上げたいとする. 領域を小さな正方形に分割すれば, そのそれぞれは面積$A$を持つことが分かっている. このとき, 正方形の数$n$を数えれば, 領域は約$nA$の面積を持つというのが, その結果になる. より正確かつ精緻な結果を求めたいのであれば, この半分の大きさの正方形でこの過程を繰り返すことになる. この方法論は(大学一年の解析学の講義の学生が一般に理解できる以上のものまで含む)任意の求積問題に使うことができ, よって形式化する価値がある. この方法を一度形式化してしまえば, その極限をとることができ, そこからRiemann和が出現する.

%I intend to show that category theory is incredibly efficient as a language for experimental design patterns, introducing formality while remaining flexible. It forms a rich and tightly woven conceptual fabric that will allow the scientist to maneuver between different perspectives whenever the need arises. Once one builds that fabric for oneself, he or she has an ability to think about models in a way that simply would not occur without it.  Moreover, putting ideas into the language of category theory forces a person to clarify their assumptions. This is highly valuable both for the researcher and for his or her audience.

この本で著者は, 実験を設計するパターンを記述する言語として圏論が信じられないくらい強力であることを示そうと思っている. 圏論によって形式化を導入できる一方で柔軟性を残すことができる. 圏論は豊富かつ堅固に編み込まれた概念の構造体を形成する. この構造体を使えば, 必要が生じたときはいつでも科学者は異なった観点の間を飛び移ることができるようになる. 一度この構造体を自分のものとして構築できれば, その人はそれなしでは生じえなかったであろうくらいに簡潔な方法で模型について考える能力を得たことになる. 加えて, 圏論の言葉の中にアイデアを投入することによってその前提条件が明瞭化される. これは研究者とその聴衆の両者にとって非常に価値あることだ. 

%What must be recognized in order to find value in this course is that conceptual chaos is a major problem. Creativity demands clarity of thinking, and to think clearly about a subject requires an organized understanding of how its pieces fit together. Organization and clarity also lead to better communication with others. Academics often say they are paid to think and understand, but that is not true. They are paid to think, understand, and {\em communicate their findings}. Universal languages for science---languages such as calculus and differential equations, matrices, or simply graphs and pie-charts---already exist, and they grant us a cohesiveness that makes scientific research worthwhile. In this book I will attempt to show that category theory can be similarly useful in describing complex scientific understandings.

この教程に価値を見いだすために認識しなければならないことは, 概念の混沌は巨大な困難だということである. 創造性には思考の明瞭さが必要であって, 主題を明瞭に思考するためには, その断片々々が互いにどのように組み合っているかを組織化して理解することが必要である. また組織化と明瞭性は他者とのよい情報伝達に繋がる. アカデミックでは, しばしば組織化と明瞭化は思考と理解の対価と語られているが, しかしこれは本当ではない. 組織化と明瞭化は, 思考と理解と\emph{発見したことを伝達する}のための対価である. 科学の普遍的な言語---解析学と微分方程式, 行列, あるいは単純なグラフに円グラフ---は既に存在している. そしてこれらによって我々は科学研究を価値あるものにする結合力を得ることができる. この本では, 複雑な科学的理解を記述する際に, 圏論は上記の言語と同じくらい有用でありうるということを示したいと思う.


%%%%%% Section %%%%%%

%\section{What is requested from the student}
\section{読書にあたって必要なこと}

%I will do my best to make clear the value of category theory in science, but I am not a scientist. To that end I am asking for your help in exploring how category theory may be useful in your specific field. 

著者は科学における圏論の価値を可能な限り明らかにしようと思っているが, しかし著者は科学者ではない. 結局のところ, 読者が従事する科学分野において圏論がどのように有用になるかを探し求める際に, 著者は読者に手助けを求めることになる. 

%I also want you to recognize that the value of mathematics is not generally obvious at first. A good student learning a good subject with a good teacher will see something compelling almost immediately, but may not see how it will be useful in real life. This will come later. I hope you will work hard to understand even without yet knowing what its actual value in your life and research will be. Like a student of soccer is encouraged to spend hours juggling the ball when he or she could be practicing penalty shots, it is important to gain facility with the materials you will be using. Doing exercises is imperative for learning mathematics.

また, 一般に数学の価値は初見では明かではないということを認識して欲しい. よき教師の下でよき主題を学ぶよき学生は, 説得力があることならばたちどころに理解することだろう. しかしそれが実生活でどのように有用になるかに関してはそうではない. これは後日に訪れる. 著者は, 読者の生活や研究におけるその真の価値を知ることなくしても, なお読者は内容の理解に努めて欲しいと願っている. サッカーを学ぶ人達が, ペナルティショットを練習できるときにはボールのリフティングに何時間も費すことを推奨されているように, 使用するものの腕前を磨くことはとても重要である. 数学を学ぶにあたっては, 練習問題を解くことは必要不可欠である.

%%%%%% Section %%%%%%

%\section{Category theory references}
\section{圏論の参考書}

%I wrote this book because the available books on category theory are almost all written for mathematicians (the rest are written for computer scientists). There is one book by Lawvere and Schanuel, called {\em Conceptual Mathematics} \cite{LS}, that offers category theory to a wider audience, but its style is not appropriate for this course. Still, it is very well written and clear. 

著者がこの本を書いた理由は, 現在入手可能な圏論の教科書はそのほぼ全てが数学者に向けて書かれているからである(その残りは計算機科学者に向けて書かれている). 広い分野の読者を対象としている圏論の教科書としては唯一LawvereとSchanuelによる\textit{Conceptual Mathematics} \cite{LS}が存在しているが, その記述スタイルはこの教程にはふさわしくない. それでも, この教科書は非常によく書かれており明快である. 

%The ``bible" of category theory is {\em Categories for the working mathematician} by Mac Lane \cite{Mac}. But as the title suggests, it was written for working mathematicians and will be quite opaque to my target audience. However, once a person has read my book, Mac Lane's book may become a valuable reference. 

圏論の``聖書''はMac Laneによる\textit{Categories for the working mathematician} \cite{Mac}である. しかしその書名から分かるように, この教科書は現役の数学者に向けて書かれており, 本書の対象読者にとっては極めて不明瞭であろう. しかしながら, 一度本書を読みきった読者には, Mac Laneの教科書は価値ある参考書になるであろう.

%Other good books include Steve Awodey's book {\em Category theory} \cite{Awo} and Barr and Wells book {\em Category theory for computing science}, \cite{BW}.  A paper by Brown and Porter called  \href{http://pages.bangor.ac.uk/\%7Emas010/pdffiles/Analogy-and-Comparison.pdf}{\text Category Theory: an abstract setting for analogy and comparison} \cite{BP1} is more in line with the style of this book, only much shorter. Online, I find \href{http://www.wikipedia.org}{\text wikipedia} and a site called \href{http://ncatlab.org/nlab/show/HomePage}{\em the $n$lab} to be quite useful. 

他の素晴しい教科書としてはSteve Awodeyの教科書\textit{Category theory} \cite{Awo}や, BarrとWellsによる教科書\textit{Category theory for computing science} \cite{BW}がある. BrownとPorterによる論文\href{http://pages.bangor.ac.uk/\%7Emas010/pdffiles/Analogy-and-Comparison.pdf}{\text{Category Theory: an abstract setting for analogy and comparison}} \cite{BP1}は, より本書の記述スタイルに沿っていて, ただずっと短いだけだ. インターネット上では, \href{http://www.wikipedia.org}{\text wikipedia}およびWebサイト\href{http://ncatlab.org/nlab/show/HomePage}{\textit{the $n$lab}}は非常に有用であろう. 

%This book attempts to explain category theory by examples and exercises rather than by theorems and proofs. I hope this approach will be valuable to the working scientist.

本書は定理と証明よりもむしろ例と演習によって圏論を説明しようと試みている. 現役の科学者にとって, この方針が価値あるものになることを願っている.

\section{Acknowledgments}

I would like to express my deep appreciation for the many scientists who I have worked with over the past five years. It all started with Paea LePendu who first taught me about databases when I was naively knocking on doors in the University of Oregon computer science department. This book would never have been written if Tristan Nguyen and Dave Balaban had not noticed my work and encouraged me to continue. Dave Balaban and Peter Gates have been my scientific partners since the beginning, working hard to understand what I'm offering and working just as hard to help me understand all that I'm missing. Peter Gates has deepened my understanding of data in profound ways. 

I have also been tremendously lucky to know Haynes Miller, who made it possible for me set down at MIT, with the help of Clark Barwick and Jacob Lurie. I knew that MIT would be the best place in the world for me to pursue this type of research, and it has really come through. Researchers like Markus Buehler and his graduate students Tristan Giesa and Dieter Brommer have been a pleasure to work with, and the many materials science examples scattered throughout this book is a testament to how much our work together has influenced my thinking. 

I'd also like to thank my collaborators and conversation partners with whom I have discussed subjects written about in this book. Other than people mentioned above, these include Steve Awodey, Allen Brown, Adam Chlipala, Carlo Curino, Dan Dugger, Henrik Forssell, David Gepner, Jason Gross, Bob Harper, Ralph Hutchison, Robert Kent, Jack Morava, Scott Morrison, David Platt, Joey Perricone, Dylan Rupel, Guarav Singh, Sam Shames, Nat Stapleton, Patrick Schultz, Ka Yu Tam, Ryan Wisnesky, Jesse Wolfson, and Elizabeth Wood.

I would like to thank Peter Kleinhenz and Peter Gates for reading this book and providing invaluable feedback before I began teaching the 18-S996 class at MIT in Spring 2013. In particular the cover image is a mild alteration of something Gates sent me to help motivate the book to scientists. I would also like to greatly thank the 18-S996 course grader Darij Grinberg, who was not only the best grader I've had in my 14 years of teaching, but gave me more comments than anyone else on the book itself. I'd also like to thank the students from the 18-S996 class at MIT who helped me find typos, pointed me to unclear explanations, and generally helped me improve the book in many ways. Other than the people listed above, these include Aaron Brookner, Leon Dimas, Dylan Erb, Deokhwan Kim, Taesoo Kim, Owen Lewis, Yair Shenfeld, and Adam Strandberg.

I would like to thank my teacher, Peter Ralston, who taught me to repeatedly question the obvious. My ability to commit to a project like this one and to see it to fruition has certainly been enhanced since studying with him.

Finally, I acknowledge my appreciation for support from the Office of Naval Research
\footnote{Grant numbers: N000140910466, N000141010841, N000141310260} 
without which this book would not have been remotely possible. I believe that their funding of basic research is an excellent way of ensuring that the US remains a global leader in the years to come.





%%%%%%%% Chapter %%%%%%%%

%\chapter{The category of sets}\label{chap:sets}

\chapter{集合の圏}\label{chap:sets}

%The theory of sets was invented as a foundation for all of mathematics. The notion of sets and functions serves as a basis on which to build our intuition about categories in general. In this chapter we will give examples of sets and functions and then move on to discuss commutative diagrams. At this point we can introduce ologs which will allow us to use the language of category theory to speak about real world concepts. Then we will introduce limits and colimits, and their universal properties. All of this material is basic set theory, but it can also be taken as an investigation of our first category, the {\em category of sets}, which we call $\Set$. We will end this chapter with some other interesting constructions in $\Set$ that do not fit into the previous sections.

集合論は全数学の基礎として発明された. 集合と函数についての記述は一般の圏について我々の直感を構築するための基礎となる. この章では我々は集合と函数の例を出した後に, 可換図式(commutative diagram)の議論に移行する. この時点で我々はologを導入することができるようになる. ologにより我々は現実世界の概念について語る際に圏論の言葉を使うことができるようになる. 続いて我々は極限(limit)と余極限(colimit), およびその普遍的な性質(universal properties)を導入する. これらの全ては基本的な集合論であるが, 一方で我々の最初の圏---集合の圏$\Set$---の調査ともとらえることができる.
%
それまでの各章にはうまく当てはまらない, また別の興味深い$\Set$の構成をもって, 我々の本章を終えることにする.

%%%%%% Section %%%%%%

%\section{Sets and functions}\index{set} 

\section{集合と函数}\index{set} 

%%%% Subsection %%%%

%\subsection{Sets}

\subsection{集合}

%In this course I'll assume you know what a set is. We can think of a set $X$ as a collection of things $x\in X$, each of which is recognizable as being in $X$ and such that for each pair of named elements $x,x'\in X$ we can tell if $x=x'$ or not.
%\footnote{Note that the symbol $x'$, read ``x-prime", has nothing to do with calculus or derivatives. It is simply notation that we use to name a symbol that is suggested as being somehow like $x$. This suggestion of kinship between $x$ and $x'$ is meant only as an aid for human cognition, and not as part of the mathematics.}
%The set of pendulums is the collection of things we agree to call pendulums, each of which is recognizable as being a pendulum, and for any two people pointing at pendulums we can tell if they're pointing at the same pendulum or not. 

この教程では, 著者は集合が何であるかを読者がよく知っていることを想定している. 我々は集合$X$を要素$x \in X$の集まりだと考えることができる. 要素それぞれは$X$の中にあると認識でき, また名前付きの要素の組$x,x'\in X$に対して我々は$x=x'$であるかそうでないかを答えることができる.
\footnote{記号$x'$---``xプライム''と読む---はcalculusや微分とは全く関係がない. これは単に$x$となにかしら似ていることを示唆する記号として我々が用いる記号の名前である. この$x$と$x'$との間の類似性の示唆が意味するものは人間の認知のみを目的にしており, 数学の一部ではない.}
振り子の集合は我々が振り子と呼ぶことを認めたものの集合であり, 要素のそれぞれは振り子であると認識される. また任意の二人の人間が指差した振り子に対して, 我々は彼らが同じ振り子を指差しているかどうかを答えることができる.

\begin{figure}
\begin{center}
\includegraphics[height=2in]{aSet}
\end{center}
%\caption{A set $X$ with $9$ elements and a set $Y$ with no elements, $Y=\emptyset$.}
\caption{要素が$9$つある集合$X$と要素がない集合$Y=\emptyset$.}
\end{figure}

\begin{notation}\label{not:basic math notation}

%The symbol $\emptyset$\index{a symbol!$\emptyset$} denotes the set with no elements. The symbol $\NN$\index{a symbol!$\NN$} denotes the set of natural numbers, which we can write as 
%$$\NN:=\{0,1,2,3,4,\ldots,877,\ldots\}.$$
%The symbol $\ZZ$\index{a symbol!$\ZZ$} denotes the set of integers, which contains both the natural numbers and their negatives, 
%$$\ZZ:=\{\ldots,-551,\ldots,-2,-1,0,1,2,\ldots\}.$$ 

記号$\emptyset$\index{a symbol!$\emptyset$}は要素がない集合を示す. 記号$\NN$\index{a symbol!$\NN$}は自然数の集合を示す. これは
$$\NN:=\{0,1,2,3,4,\ldots,877,\ldots\}.$$
と書くことができる.
記号$\ZZ$\index{a symbol!$\ZZ$}は整数の集合を示す. これは自然数とその負の要素を両方含む.
$$\ZZ:=\{\ldots,-551,\ldots,-2,-1,0,1,2,\ldots\}.$$ 

%If $A$ and $B$ are sets, we say that $A$ is a {\em subset}\index{subset} of $B$, and write $A\ss B$, if every element of $A$ is an element of $B$. So we have $\NN\ss\ZZ$. Checking the definition, one sees that for any set $A$, we have (perhaps uninteresting) subsets $\emptyset\ss A$ and $A\ss A$. We can use {\em set-builder notation}\index{set!set builder notation} to denote subsets. For example the set of even integers can be written $\{n\in\ZZ\|n\tn{ is even}\}$. The set of integers greater than $2$ can be written in many ways, such as $$\{n\in\ZZ\|n>2\} \hsp\tn{or}\hsp\{n\in\NN\|n>2\}\hsp\tn{or}\hsp\{n\in\NN\|n\geq 3\}.$$

$A$と$B$が集合であり, $A$の全ての要素が$B$の要素であるとき, $A$は$B$の\emph{部分集合}(subset)\index{subset}であると言い, $A\ss B$と書く. よって$\NN\ss\ZZ$が得られる. 定義を確認すれば, 任意の集合$A$について, (おそらく興味をひくことはない)部分集合$\emptyset\ss A$と$A\ss A$が得られることが分かる. 部分集合を示すのに\emph{内含表記(set-builder notation)}\index{set!set builder notation}を使うこともできる. 例えば偶数の集合は$\{n\in\ZZ\|n\tn{ は偶数}\}$と書くことができる. $2$より大きな整数の集合の書き方は数多くある. 例えば$$\{n\in\ZZ\|n>2\} \hsp\tn{or}\hsp\{n\in\NN\|n>2\}\hsp\tn{or}\hsp\{n\in\NN\|n\geq 3\}.$$

%The symbol $\exists$ means ``there exists".\index{a symbol!$\exists$} So we could write the set of even integers as $$\{n\in\ZZ\|n\tn{ is even}\}\hsp=\hsp\{n\in\ZZ\|\exists m\in\ZZ\tn{ such that } 2m=n\}.$$ The symbol $\exists!$\index{a symbol!$\exists$"!} means ``there exists a unique". So the statement ``$\exists! x\in\RR\tn{ such that } x^2=0$" means that there is one and only one number whose square is 0. Finally, the symbol $\forall$ means ``for all".\index{a symbol!$\forall$} So the statement ``$\forall m\in\NN\;\exists n\in\NN\tn{ such that } m<n$" means that for every number there is a bigger one.

記号$\exists$は``少なくとも一つ存在する''ことを意味する.\index{a symbol!$\exists$} よって偶数は$$\{n\in\ZZ\|n\tn{は偶数}\}\hsp=\hsp\{n\in\ZZ\|\exists m\in\ZZ\tn{ such that } 2m=n\}.$$と書くこともできた. 記号$\exists!$\index{a symbol!$\exists$"!}は``一意的に(唯一)存在する''ことを意味する. よって``$\exists! x\in\RR\tn{ such that } x^2=0$''という言明は, 二乗が0である数が存在し, それがただ一つだけであることを意味している. 最後に, 記号$\forall$は``全ての''を意味している.\index{a symbol!$\forall$} よって``$\forall m\in\NN\;\exists n\in\NN\tn{ such that } m<n$''という言明は全ての数に対してそれより大きい数が存在することを意味している.

%As you may have noticed, we use the colon-equals notation `` $A:=XYZ$ " to mean something like ``define $A$ to be $XYZ$".\index{a symbol!:=} That is, a colon-equals declaration is not denoting a fact of nature (like $2+2=4$), but a choice of the speaker. It just so happens that the notation above, such as $\NN:=\{0,1,2,\ldots\}$, is a widely-held choice.

既に気付いている読者もいるであろうが, 我々はコロン-等号記法を用い, `` $A:=XYZ$''が``$A$を$XYZ$と定義する''ことを意味するとしている.\index{a symbol!:=} これは, コロン-等号による宣言は自然界における事実(例えば$2+2=4$というようなもの)を示しているのではなく, 話者の選択を示しているということである. 上記の, 例えば$\NN:=\{0,1,2,\ldots\}$といった記法は, たまたま広く取られる選択である.

\end{notation}

\begin{exercise}
%Let $A=\{1,2,3\}$. What are all the subsets of $A$? Hint: there are 8.
$A=\{1,2,3\}$とする. $A$の部分集合を全て求めよ. ヒント: 8つある.
\end{exercise}

%%%% Subsection %%%%

%\subsection{Functions}\label{sec:functions}
\subsection{函数}\label{sec:functions}

%If $X$ and $Y$ are sets, then a {\em function $f$ from $X$ to $Y$},\index{function} denoted $f\taking X\to Y$, is a mapping that sends each element $x\in X$ to an element of $Y$, denoted $f(x)\in Y$. We call $X$ the {\em domain}\index{function!domain} of the function $f$ and we call $Y$ the {\em codomain}\index{function!codomain} of $f$. 

$X$と$Y$が集合であるならば, \emph{$X$から$Y$への函数$f$}\index{function}は$f\taking X\to Y$と表記され, これはそれぞれの要素$x\in X$を$f(x)\in Y$と表記される$Y$の要素へ送る写像である. $X$は函数$f$の\emph{始域(domain)}\index{function!domain}と呼ばれ, $Y$は$f$の\emph{終域(codomain)}\index{function!codomain}と呼ばれる.

\begin{align}\label{dia:setmap}
\parbox{2.3in}{\includegraphics[height=2in]{SetMap}}
\end{align}

%Note that for every element $x\in X$, there is exactly one arrow emanating from $x$, but for an element $y\in Y$, there can be several arrows pointing to $y$, or there can be no arrows pointing to $y$. 

全ての要素$x\in X$に対して, $x$から出る矢印が正確に一本存在するが, しかしある要素$y\in Y$に対しては, $y$を指す矢印が何本も存在するかもしれないし, また$y$を指す矢印が存在しないかもしれないことに注意しよう.

\begin{application}\label{app:force-extension}\index{materials!force-extension curves}

%In studying the mechanics of materials, one wishes to know how a material responds to tension. For example a rubber band responds to tension differently than a spring does. To each material we can associate a \href{http://en.wikipedia.org/wiki/Stress–strain_curve}{\text force-extension curve}, recording how much force the material carries when extended to various lengths. Once we fix a methodology for performing experiments, finding a material's force-extension curve would ideally constitute a function from the set of materials to the set of curves.
%\footnote{In reality, different samples of the same material, say samples of different sizes or at different temperatures, may have different force-extension curves. If we want to see this as a true function whose codomain is curves it should have as domain something like the set of material samples.}

物質の力学を研究する時に, 張力に対する応答を知りたくなることがある. 例えばゴムバンドは張力に対してバネとは違う応答を示す. いろいろな長さに伸ばした時に, 物質がどれだけ力を生みだしたかを記録することによって, 我々はそれぞれの物質に対して\href{http://en.wikipedia.org/wiki/Stress–strain_curve}{応力-ひずみ曲線(force-extension curve)}を関連付けることができる. 一度実験を行なう方法を固定したのならば, 物質の応力-ひずみ曲線の探索は理想的には物質の集合から曲線の集合への函数の構築となるであろう.
\footnote{実際には, 同じ物質の異なったサンプル, 例えば異なった大きさのサンプルやあるいは異なった温度でのサンプルは異なった応力-ひずみ曲線を持つだろう. もし我々がこれをその終域が曲線である本物の函数であると見なしたいのであれば, それは物質のサンプルの集合といったものを始域としているべきである.}

\end{application}

\begin{exercise}

%Here is a simplified account of how the \href{http://en.wikipedia.org/wiki/Retina}{\text brain receives light}. The eye contains about 100 million photoreceptor (PR) cells. Each connects to a retinal ganglion (RG) cell. No PR cell connects to two different RG cells, but usually many PR cells can attach to a single RG cell. 

以下の記述は, どのようにして\href{http://en.wikipedia.org/wiki/Retina}{脳が光を感じとるか}を単純化したものである. 眼には約1億個の光受容細胞(photoreceptor cell)が含まれている. それぞれの細胞は網膜神経節細胞(retinal ganglion cell)に繋がっている. 二つの異なった網膜神経節細胞に繋がっている光受容細胞は存在しないが, 通常は複数の光受容細胞が一つの網膜神経節細胞に接続することができる.

%Let $PR$ denote the set of photoreceptor cells and let $RG$ denote the set of retinal ganglion cells. 
%\sexc According to the above account, does the connection pattern constitute a function $RG\to PR$, a function $PR\to RG$ or neither one? 
%\next Would you guess that the connection pattern that exists between other areas of the brain are ``function-like"?
%\endsexc

$\mathit{PR}$は光受容細胞の集合, $\textit{RG}$は網膜神経節細胞の集合を示すものとする.
\sexc 上記の記述に従えば, 結合パターンは函数$\textit{RG}\to \textit{PR}$, あるいは函数$\textit{PR}\to \textit{RG}$を構成するだろうか? もしくはどちらも函数ではないのだろうか?
\next 脳の他の領域の間に存在する結合パターンが``函数のように''ふるまうかを推測してみよう.
\endsexc
\end{exercise}

\begin{example}\label{ex:subset as function}\index{subset!as function}

%Suppose that $X$ is a set and $X'\ss X$ is a subset. Then we can consider the function $X'\to X$ given by sending every element of $X'$ to ``itself" as an element of $X$. For example if $X=\{a,b,c,d,e,f\}$ and $X'=\{b,d,e\}$ then $X'\ss X$ and we turn that into the function $X'\to X$ given by $b\mapsto b, d\mapsto d, e\mapsto e$.
%\footnote{This kind of arrow,\;\;$\mapsto$\;\;, is read aloud as ``maps to". A function $f\taking X\to Y$ means a rule for assigning to each element $x\in X$ an element $f(x)\in Y$. We say that ``$x$ maps to $f(x)$" and write $x\mapsto f(x)$.}\index{a symbol!$\mapsto$}

$X$が集合で$X'\ss X$がその部分集合であると仮定する. そのときには$X'$の全ての要素を$X$における``要素それ自身''に対応付ける函数$X'\to X$を考えることができる. 例えば, $X=\{a,b,c,d,e,f\}$として$X'=\{b,d,e\}$としたならば$X'\ss X$
であり, そこから函数$X'\to X$を$b\mapsto b, d\mapsto d, e\mapsto e$とすることによって得ることができる.
\footnote{この\;\;$\mapsto$\;\;という種類の矢印は, ``対応付ける(maps to)''と音読される. 函数$f\taking X\to Y$はそれぞれの要素$x\in X$に要素$f(x)\in Y$を割りあてる法則を意味している. 我々はこれを``$x$を$f(x)$に対応付ける''と読み, $x\mapsto f(x)$と書く.}\index{a symbol!$\mapsto$}

%As a matter of notation, we may sometimes say something like the following: Let $X$ be a set and let $i\taking X'\ss X$ be a subset. Here we are making clear that $X'$ is a subset of $X$, but that $i$ is the name of the associated function.

記法として, 我々は時々``$X$を集合, $i\taking X'\ss X$を部分集合とする''というようなことを言うだろう. ここで$X'$は$X$の部分集合であり, $i$はそれに関連付けられた函数であることを, ここで明確に述べておく.

\end{example}

\begin{exercise}
%Let $f\taking\NN\to\NN$ be the function that sends every natural number to its square, e.g. $f(6)=36$. First fill in the blanks below, then answer a question.
%\sexc $2\mapsto\ul{\hspace{.5in}}$
%\next $0\mapsto\ul{\hspace{.5in}}$
%\next $-2\mapsto\ul{\hspace{.5in}}$
%\next $5\mapsto\ul{\hspace{.5in}}$
%\next Consider the symbol $\to$ and the symbol $\mapsto$. What is the difference between how these two symbols are used in this book?
%\endsexc
$f\taking\NN\to\NN$を, 全ての自然数をその二乗に対応付ける函数とする, e.g. $f(6)=36$. 最初に以下の空白を埋め, それから問題に答えよ.
\sexc $2\mapsto\ul{\hspace{.5in}}$
\next $0\mapsto\ul{\hspace{.5in}}$
\next $-2\mapsto\ul{\hspace{.5in}}$
\next $5\mapsto\ul{\hspace{.5in}}$
\next 記号$\to$と記号$\mapsto$について考えよ. この本の中において, 二つの記号の使用され方の間にある違いはなんだろうか?
\endsexc
\end{exercise}

%Given a function $f\taking X\to Y$, the elements of $Y$ that have at least one arrow pointing to them are said to be {\em in the image} of $f$; that is we have \index{image}
%\begin{align}\label{dia:image}
%\im(f):=\{y\in Y\| \exists x\in X \tn{ such that } f(x)=y\}.
%\end{align} 

函数$f\taking X\to Y$が与えられたとき, $Y$の要素でそれを指す矢印が一つ以上あるものの集合を$f$の\emph{像(image)}と呼ぶ. これは\index{image}
\begin{align}\label{dia:image}
\im(f):=\{y\in Y\| \exists x\in X \tn{ such that } f(x)=y\}
\end{align}
と書ける.

\begin{exercise}
%If $f\taking X\to Y$ is depicted by (\ref{dia:setmap}) above, write its image, $\im(f)$ as a set.
$f\taking X\to Y$が\eqref{dia:setmap}で描写されるとき, 像$\im(f)$を集合として書け.
\end{exercise}

%Given a function $f\taking X\to Y$ and a function $g\taking Y\to Z$, where the codomain of $f$ is the same set as the domain of $g$ (namely $Y$), we say that $f$ and $g$ are composable 
%$$X\Too{f}Y\Too{g}Z.$$ The {\em composition of $f$ and $g$}\label{function composition}\index{function!composition}\index{composition!of functions}\index{a symbol!$\circ$} is denoted by $g\circ f\taking X\to Z$. 

函数$f\taking X\to Y$と函数$g\taking Y\to Z$が与えられ, $f$の終域が$g$の始域と同じ集合($Y$)であるとき, 我々は$f$と$g$は合成可能(composable)であるという.
$$X\Too{f}Y\Too{g}Z.$$ \emph{$f$と$g$の合成(composition of $f$ and $g$)}\label{function composition}\index{function!composition}\index{composition!of functions}\index{a symbol!$\circ$}は$g\circ f\taking X\to Z$と表記される. 

\begin{figure}[h]
\begin{center}
\includegraphics[height=2in]{composition}
\end{center}
\caption{Functions $f\taking X\to Y$ and $g\taking Y\to Z$ compose to a function $g\circ f\taking X\to Z$; just follow the arrows.}
\end{figure}

%Let $X$ and $Y$ be sets. We write $\Hom_\Set(X,Y)$\index{a symbol!$\Hom_\Set$} to denote the set of functions $X\to Y$.
%\footnote{The strange notation $\Hom_\Set(-,-)$ will make more sense later, when it is seen as part of a bigger story.} 
%Note that two functions $f,g\taking X\to Y$ are equal\index{function!equality of} if and only if for every element $x\in X$ we have $f(x)=g(x)$. 

$X$と$Y$を集合とする. 我々は函数$X\to Y$の集合を$\Hom_\Set(X,Y)$\index{a symbol!$\Hom_\Set$}と書く.
\footnote{後程, より大きな筋書きの一部として見たときに, $\Hom_\Set(-,-)$という奇妙な記法がさらなる意味を持つことになる.}
二つの函数$f,g\taking X\to Y$が等しい\index{function!equality of}とは, 全ての要素$x\in X$に対して$f(x)=g(x)$が得られる時かつその時に限ることに注意しよう.

\begin{exercise}
%Let $A=\{1,2,3,4,5\}$ and $B=\{x,y\}.$ 
%\sexc How many elements does $\Hom_\Set(A,B)$ have? 
%\next How many elements does $\Hom_\Set(B,A)$ have?
%\endsexc
$A=\{1,2,3,4,5\}$で$B=\{x,y\}$であるとする.
\sexc $\Hom_\Set(A,B)$の要素はいくつあるか?
\next $\Hom_\Set(B,A)$の要素はいくつあるか?
\endsexc
\end{exercise}

\begin{exercise}~
%\sexc Find a set $A$ such that for all sets $X$ there is exactly one element in $\Hom_\Set(X,A)$. Hint: draw a picture of proposed $A$'s and $X$'s.
%\next Find a set $B$ such that for all sets $X$ there is exactly one element in $\Hom_\Set(B,X)$.
%\endsexc 
\sexc 全ての集合$X$に対して$\Hom_\Set(X,A)$がただ一つだけ要素を持つ, そのような集合$A$を求めよ. ヒント: 考えた $A$と$X$の要素に関しての絵を書け.
\next 全ての集合$X$に対して$\Hom_\Set(B, X)$がただ一つだけ要素を持つ, そのような集合$B$を求めよ.
\endsexc
\end{exercise}

%For any set $X$, we define the {\em identity function on $X$}\index{function!identity}, denoted $\id_X\taking X\to X$, to be the function such that for all $x\in X$ we have $\id_X(x)=x$.\index{a symbol!$\id_X$}

任意の集合$X$に対して, \emph{$X$上の恒等写像(identity function on $X$)}\index{function!identity}を, 全ての$x\in X$に対して$\id_X(x)=x$.\index{a symbol!$\id_X$}である函数として定義し, $\id_X\taking X\to X$で示す.

%\begin{definition}[Isomorphism]\label{def:iso in set}
\begin{definition}[同型写像(Isomorphism)]\label{def:iso in set}

%Let $X$ and $Y$ be sets. A function $f\taking X\to Y$ is called an {\em isomorphism}\index{function!isomorphism}\index{isomorphism!of sets}, denoted $f\taking X\To{\iso}Y$, if there exists a function $g\taking Y\to X$ such that $g\circ f=\id_X$ and $f\circ g=\id_Y$. We also say that $f$ is {\em invertible} and we say that $g$ is {\em the inverse}\index{function!inverse} of $f$. If there exists an isomorphism $X\To\iso Y$ we say that $X$ and $Y$ are {\em isomorphic} sets and may write $X\iso Y$. \index{a symbol!$\iso$}

$X$と$Y$を集合とする. 函数$f\taking X\to Y$は, $g\circ f=\id_X$かつ$f\circ g=\id_Y$である函数$g\taking Y\to X$が存在するとき, \emph{同型写像(isomorphism)}\index{function!isomorphism}\index{isomorphism!of sets}であるといい, $f\taking X\To{\iso}Y$と書いて示す. またこのとき$f$は\emph{可逆(invertible)}であるといい, $g$は$f$の\emph{逆函数(the inverse)}\index{function!inverse}であるという. 同型$X\To\iso Y$が存在するとき, $X$と$Y$は\emph{同型(isomorphic)}な集合であるといい, $X\iso Y$と書くこともある. \index{a symbol!$\iso$}

\end{definition}

\begin{example}

%If $X$ and $Y$ are sets and $f\taking X\to Y$ is an isomorphism then the analogue of Diagram \ref{dia:setmap} will look like a perfect matching, more often called a {\em one-to-one correspondence}\index{one-to-one correspondence}\index{correspondence!one-to-one}. That means that no two arrows will hit the same element of $Y$, and every element of $Y$ will be in the image. For example, the following depicts an isomorphism $X\To{\iso}Y$.

$X$と$Y$が集合で$f\taking X\to Y$が同型写像であるならば, 図式\ref{dia:setmap}の類似物は完全対応となり, よりしばしば\emph{一対一対応(one-to-one correspondence)}\index{one-to-one correspondence}\index{correspondence!one-to-one}と呼ばれる. これが意味するところは, どの二つの矢印も$Y$の同じ要素に当たることがなく, $Y$の全ての要素がその像の中に含まれるということである. 例えば, 次の図は同型写像$X\To{\iso}Y$を描いている.

\begin{align}\label{dia:setmapiso}
\parbox{2.3in}{\includegraphics[height=2in]{SetMapIso}}
\end{align}

\end{example}

\begin{application}\label{app:DNA RNA}\index{RNA transcription}

%There is an isomorphism between the set $\tn{Nuc}_\tn{DNA}$ of \href{http://en.wikipedia.org/wiki/Nucleotides}{\text nucleotides} found in DNA and the set $\tn{Nuc}_\tn{RNA}$ of nucleotides found in RNA. Indeed both sets have four elements, so there are 24 different isomorphisms. But only one is useful. Before we say which one it is, let us say there is also an isomorphism $\tn{Nuc}_\tn{DNA}\iso\{A,C,G,T\}$ and an isomorphism $\tn{Nuc}_\tn{RNA}\iso\{A,C,G,U\}$, and we will use the letters as abbreviations for the nucleotides. 

DNAの中にある\href{http://en.wikipedia.org/wiki/Nucleotides}{ヌクレオチド(nucleotides)}の集合$\tn{Nuc}_\tn{DNA}$と, RNAの中にあるヌクレオチドの集合$\tn{Nuc}_\tn{RNA}$の間には, 同型写像が存在している. 実際にはどちらの集合も4つの要素を持つから, 24個の異なった同型写像が存在しうる. しかしその中で一つだけが有用である. どれがその一つかを述べる前に, また別の同型写像$\tn{Nuc}_\tn{DNA}\iso\{A,C,G,T\}$および同型写像$\tn{Nuc}_\tn{RNA}\iso\{A,C,G,U\}$が存在するということを述べておこう. 今後はこれらの文字をヌクレオチドの略記として用いることにする.

%The convenient isomorphism $\tn{Nuc}_\tn{DNA}\To{\iso}\tn{Nuc}_\tn{RNA}$ is that given by RNA transcription; it sends 
%$$A\mapsto U, C\mapsto G, G\mapsto C, T\mapsto A.$$ 
%(See also Application \ref{app:polymerase}.) There is also an isomorphism $\tn{Nuc}_\tn{DNA}\To{\iso}\tn{Nuc}_\tn{DNA}$ (the matching in the double-helix) given by 
%$$A\mapsto T, C\mapsto G, G\mapsto C, T\mapsto A.$$

有用な同型写像$\tn{Nuc}_\tn{DNA}\To{\iso}\tn{Nuc}_\tn{RNA}$はRNA転写によって与えられる. これは
$$A\mapsto U, C\mapsto G, G\mapsto C, T\mapsto A.$$ 
と移す. (Application \ref{app:polymerase}も見よ.) また同型写像$\tn{Nuc}_\tn{DNA}\To{\iso}\tn{Nuc}_\tn{DNA}$ (二重螺旋の対合)が存在し, 
$$A\mapsto T, C\mapsto G, G\mapsto C, T\mapsto A.$$
で与えられる.

%Protein production can be modeled as a function from the set of 3-nucleotide sequences to the set of eukaryotic amino acids. However, it cannot be an isomorphism because there are $4^3=64$ triplets of RNA nucleotides, but only 21 eukaryotic amino acids. 

たんぱく質の合成は, 3要素のヌクレオチド列から真核生物の持つアミノ酸の集合への函数として模型化される. しかしこれは同型写像ではない. なぜならばRNAの三つ組は$4^3=64$種類存在しているが, 真核生物の持つアミノ酸は21種類しかないからだ.

\end{application}

\begin{exercise}
%Let $n\in\NN$ be a natural number and let $X$ be a set with exactly $n$ elements. 
%\sexc How many isomorphisms are there from $X$ to itself? 
%\next Does your formula from part a.) hold when $n=0$?
%\endsexc

$n\in\NN$を自然数とし, $X$を正確に$n$個の要素を持つ集合とする.
\sexc $X$から$X$自身への同型写像はいくつあるか? 
\next a.) での式は$n=0$の時にも成立するか?
\endsexc
\end{exercise}

\begin{lemma}\label{lemma:isomorphic ER in Set}

%The following facts hold about isomorphism.
%\begin{enumerate}
%\item Any set $A$ is isomorphic to itself; i.e. there exists an isomorphism $A\To{\iso} A$.
%\item For any sets $A$ and $B$, if $A$ is isomorphic to $B$ then $B$ is isomorphic to $A$.
%\item For any sets $A, B,$ and $C$, if $A$ is isomorphic to $B$ and $B$ is isomorphic to $C$ then $A$ is isomorphic to $C$.
%\end{enumerate}

同型写像に関して以下が成立する.
\begin{enumerate}
\item 任意の集合$A$はそれ自身に対して同型である. i.e. 同型写像$A\To{\iso} A$が存在する.
\item 任意の集合$A$と$B$に対して, $A$が$B$に同型ならば$B$は$A$に同型である.
\item 任意の集合$A, B,$および$C$に対して, もし$A$が$B$に同型で$B$が$C$に同型ならば, $A$は$C$に同型である.
\end{enumerate}

\end{lemma}

\begin{proof}

%\begin{enumerate}
%\item The identity function $\id_A\taking A\to A$ is invertible; its inverse is $\id_A$ because $\id_A\circ\id_A=\id_A$.
%\item If $f\taking A\to B$ is invertible with inverse $g\taking B\to A$ then $g$ is an isomorphism with inverse $f$.
%\item If $f\taking A\to B$ and $f'\taking B\to C$ are each invertible with inverses $g\taking B\to A$ and $g'\taking C\to B$ then the following calculations show that $f'\circ f$ is invertible with inverse $g\circ g'$: 
%\begin{align*}
%(f'\circ f)\circ(g\circ g')=f'\circ(f\circ g)\circ g'=f'\circ\id_B\circ g'=f'\circ g'=\id_C\\
%(g\circ g')\circ(f'\circ f)=g\circ(g'\circ f')\circ f=g\circ\id_B\circ f=g\circ f=\id_A
%\end{align*}
%\end{enumerate}

\begin{enumerate}
\item 恒等写像$\id_A\taking A\to A$は可逆である. $\id_A\circ\id_A=\id_A$であるから逆函数は$\id_A$である.
\item もし$f\taking A\to B$が可逆でその逆函数が$g\taking B\to A$であるならば, $g$は同型写像でその逆函数は$f$である.
\item もし$f\taking A\to B$および$f'\taking B\to C$が可逆でそれぞれの逆函数が$g\taking B\to A$および$g'\taking C\to B$であるならば, 以下の計算によって$f'\circ f$が可逆でありその逆函数が$g\circ g'$であることが示される.
\begin{align*}
(f'\circ f)\circ(g\circ g')=f'\circ(f\circ g)\circ g'=f'\circ\id_B\circ g'=f'\circ g'=\id_C\\
(g\circ g')\circ(f'\circ f)=g\circ(g'\circ f')\circ f=g\circ\id_B\circ f=g\circ f=\id_A
\end{align*}
\end{enumerate}

\end{proof}

\begin{exercise}\label{exc:functions are not iso invariant}
%Let $A$ and $B$ be the sets drawn below:
%$$
%\parbox{1.1in}{\boxtitle{A:=}\fbox{\xymatrix@=1pt{\\&\LMO{\;a\;}&&&\LMO{\;\;\;7\;\;}&\\\\\\&&&\LMO{Q}\\&}}}
%\hspace{.8in}
%\parbox{1.2in}{\boxtitle{B:=}\fbox{\xymatrix@=1pt{&&&\LMO{r8}&&\\\\\\\\&\LMO{``Bob"}\\&&\LMO{\clubsuit}}}}
%$$
%Note that the sets $A$ and $B$ are isomorphic. Supposing that $f\taking B\to\{1,2,3,4,5\}$ sends ``Bob" to $1$, sends $\clubsuit$ to $3$, and sends $r8$ to $4$, is there a canonical function $A\to\{1,2,3,4,5\}$ corresponding to $f$?
%\footnote{Canonical means something like ``best choice", a choice that stands out as the only reasonable one.}\index{canonical}
$A$と$B$が集合であり以下のように描写されるとする.
$$
\parbox{1.1in}{\boxtitle{A:=}\fbox{\xymatrix@=1pt{\\&\LMO{\;a\;}&&&\LMO{\;\;\;7\;\;}&\\\\\\&&&\LMO{Q}\\&}}}
\hspace{.8in}
\parbox{1.2in}{\boxtitle{B:=}\fbox{\xymatrix@=1pt{&&&\LMO{r8}&&\\\\\\\\&\LMO{``Bob"}\\&&\LMO{\clubsuit}}}}
$$
$A$と$B$が同型であることに注意しよう. $f\taking B\to\{1,2,3,4,5\}$が, ``Bob''を$1$に移し, $\clubsuit$を$3$に移し, $r8$を$4$に移すと仮定したとき, $f$に対応する標準函数(canonical function)$A\to\{1,2,3,4,5\}$は存在するだろうか?
\footnote{標準(canonical)とは, 唯一の妥当なものとして際だっているような選びかたである, すなわち``もっともよい選択''であるというような意味である.}\index{canonical}
\end{exercise}

\begin{exercise}\label{exc:generator for set}
%Find a set $A$ such that for any set $X$ there is a isomorphism of sets $$X\iso\Hom_\Set(A,X).$$ Hint: draw a picture of proposed $A$'s and $X$'s.
任意の集合$X$に対して同型写像$$X\iso\Hom_\Set(A,X).$$が存在する集合$A$を求めよ. ヒント: 考えた$A$の要素と$X$の要素についての絵を描け.
\end{exercise}

%For any natural number $n\in\NN$, define a set 
%\begin{align}\label{dia:underline n}\index{a symbol!$\ul{n}$}
%\ul{n}:=\{1,2,3,\ldots,n\}.
%\end{align}
%So, in particular, $\ul{2}=\{1,2\}, \ul{1}=\{1\}$, and $\ul{0}=\emptyset$. 

任意の自然数$n\in\NN$に対して, 集合 
\begin{align}\label{dia:underline n}\index{a symbol!$\ul{n}$}
\ul{n}:=\{1,2,3,\ldots,n\}.
\end{align}
を定義する. よって, 特に$\ul{2}=\{1,2\}, \ul{1}=\{1\}$および$\ul{0}=\emptyset$である.

%Let $A$ be any set. A function $f\taking\ul{n}\to A$ can be written as a sequence $$f=(f(1),f(2),\ldots,f(n)).$$

$A$を任意の集合とする. 函数$f\taking\ul{n}\to A$は列$$f=(f(1),f(2),\ldots,f(n)).$$として書くことができる.

\begin{exercise}\label{exc:sequence}~
%\sexc Let $A=\{a,b,c,d\}$. If $f\taking\ul{10}\to A$ is given by $(a,b,c,c,b,a,d,d,a,b)$, what is $f(4)$?
%\next Let $s\taking\ul{7}\to\NN$ be given by $s(i)=i^2$. Write $s$ out as a sequence.
%\endsexc
\sexc $A=\{a,b,c,d\}$とする. もし$f\taking\ul{10}\to A$が$(a,b,c,c,b,a,d,d,a,b)$で与えられたならば, $f(4)$はどうなるか? 
\next $s\taking\ul{7}\to\NN$が$s(i)=i^2$で与えられるとする. $s$を列として書き下せ.
\endsexc
\end{exercise}

%\begin{definition}[Cardinality of finite sets]\label{def:cardinality}
\begin{definition}[有限集合の濃度(Cardinality)]\label{def:cardinality}

%Let $A$ be a set and $n\in\NN$ a natural number. We say that $A$ is {\em has cardinality $n$}\index{cardinality}, denoted $$|A|=n,$$ if there exists an isomorphism of sets $A\iso\ul{n}$. If there exists some $n\in\NN$ such that $A$ has cardinality $n$ then we say that $A$ is {\em finite}. Otherwise, we say that $A$ is {\em infinite} and write $|A|\geq\infty$.

$A$を集合とし$n\in\NN$を自然数とする. 集合間の同型写像$A\iso\ul{n}$が存在するならば, $A$の\emph{濃度(cardinality)は$n$}\index{cardinality}であると言い, $$|A|=n,$$と書く. もし$A$の濃度が$n$である$n\in\NN$が存在するならば$A$は\emph{有限である}と言う. そうでなければ, $A$は\emph{無限である}と言い, $|A|\geq\infty$と書く.

\end{definition}

\begin{exercise}~
%\sexc Let $A=\{5,6,7\}$. What is $|A|$? 
%\next What is $|\NN|$? 
%\next What is $|\{n\in\NN\|n\leq 5\}|$?
%\endsexc
\sexc $A=\{5,6,7\}$とする. $|A|$はどうなるか?
\next $|\NN|$はどうなるか? 
\next $|\{n\in\NN\|n\leq 5\}|$はどうなるか?
\endsexc
\end{exercise}

\begin{lemma}

%Let $A$ and $B$ be finite sets. If there is an isomorphism of sets $f\taking A\to B$ then the two sets have the same cardinality, $|A|=|B|$.

$A$と$B$を有限集合とする. もし集合間の同型写像$f\taking A\to B$が存在したならば, 二つの集合は同じ濃度を持つ, すなわち$|A|=|B|$である.

\end{lemma}

\begin{proof}

%Suppose $f\taking A\to B$ is an isomorphism. If there exists natural numbers $m,n\in\NN$ and isomorphisms $a\taking\ul{m}\To\iso A$ and $b\taking\ul{n}\To\iso B$ then $\ul{m}\To{a^\m1}A\To{f}B\To{b}\ul{n}$ is an isomorphism. One can prove by induction that the sets $\ul{m}$ and $\ul{n}$ are isomorphic if and only if $m=n$. 

$f\taking A\to B$が同型写像であると仮定する. もし自然数$m,n\in\NN$が存在して同型写像$a\taking\ul{m}\To\iso A$および$b\taking\ul{n}\To\iso B$が存在するならば, $\ul{m}\To{a^\m1}A\To{f}B\To{b}\ul{n}$も同型写像である. 帰納法により, 集合$\ul{m}$と$\ul{n}$が同型であるのは$m=n$である時かつその時に限られることが証明できる.

\end{proof}


%%%%%% Section %%%%%%

%\section{Commutative diagrams}\label{sec:comm diag}
\section{可換図式}\label{sec:comm diag}
%\addtocounter{subsection}{1}\setcounter{subsubsection}{0}

%At this point it is difficult to precisely define diagrams or commutative diagrams in general, but we can give the heuristic idea.
%\footnote{We will define commutative diagrams precisely in Section \ref{sec:diagrams in a category}.}
%Consider the following picture: 
%\begin{align}\label{dia:triangle}
%\xymatrix{A\ar[r]^f\ar[rd]_h&B\ar[d]^g\\&C}
%\end{align}
%We say this is a {\em diagram of sets}\index{diagram!in $\Set$} if each of $A,B,C$ is a set and each of $f,g,h$ is a function. We say this diagram {\em commutes}\index{commuting diagram}\index{diagam!commutes} if $g\circ f = h$. In this case we refer to it as a commutative triangle of sets.

この段階では, 図式(diagram)あるいは可換図式(commutative diagram)を正確に定義するのは一般には難しいが, 
heuristicな考えを与えることならばできる.
\footnote{我々は第\ref{sec:diagrams in a category}章で可換図式を正確に定義することになる.}
次の絵を考える.
\begin{align}\label{dia:triangle}
\xymatrix{A\ar[r]^f\ar[rd]_h&B\ar[d]^g\\&C}
\end{align}
$A,B,C$が集合で$f,g,h$が函数であるとき, 我々はこれを\emph{集合の図式(diagram of sets)}\index{diagram!in $\Set$}と言う. もし$g\circ f = h$であるならば, この図式は\emph{可換である(commutes)}\index{commuting diagram}\index{diagam!commutes}と言う. 可換な場合の上図を, 我々は集合の可換三角形として参照する.

\begin{application}

%\href{http://en.wikipedia.org/wiki/Central_dogma_of_molecular_biology}{\text The central dogma of molecular biology} is that ``DNA codes for RNA codes for protein". That is, there is a function from DNA triplets to RNA triplets and a function from RNA triplets to amino acids. But sometimes we just want to discuss the translation from DNA to amino acids, and this is the composite of the other two. The commutative diagram is a picture of this fact.

\href{http://en.wikipedia.org/wiki/Central_dogma_of_molecular_biology}{分子生物学のセントラルドグマ}は``DNAはRNAを符号化しRNAはたんぱく質を符号化する''. これは, DNAの三つ組からRNAの三つ組への函数と, RNAの三つ組からアミノ酸への函数が存在しているということである. しかし, DNAからアミノ酸への変換だけを議論したいこともしばしばある. そしてこれは二つの函数の合成である. 可換図式はこの事実に関する絵である.

\end{application}

%Consider the following picture:
%$$\xymatrix{A\ar[r]^f\ar[d]_h&B\ar[d]^g\\C\ar[r]_i&D}$$
%We say this is a {\em diagram of sets} if each of $A,B,C,D$ is a set and each of $f,g,h,i$ is a function. We say this diagram {\em commutes} if $g\circ f=i\circ h$. In this case we refer to it as a commutative square of sets.

次の絵を考える.
$$\xymatrix{A\ar[r]^f\ar[d]_h&B\ar[d]^g\\C\ar[r]_i&D}$$
$A,B,C,D$が集合であり$f,g,h,i$が函数であるとき, 我々はこれを\emph{集合の図式}と言う. もし$g\circ f=i\circ h$であるならば, この図式は\emph{可換である}と言う. 可換な場合の上図を, 我々は集合の可換四角形として参照する.

\begin{application}

%Given a physical system $S$, there may be two mathematical approaches $f\taking S\to A$ and $g\taking S\to B$ that can be applied to it. Either of those results in a prediction of the same sort, $f'\taking A\to P$ and $g'\taking B\to P$. For example, in \href{http://en.wikipedia.org/wiki/Hamiltonian_mechanics#As_a_reformulation_of_Lagrangian_mechanics}{\text mechanics} we can use either Lagrangian approach or the Hamiltonian approach to predict future states. To say that the diagram 
%$$
%\xymatrix{S\ar[r]\ar[d]&A\ar[d]\\B\ar[r]&P}
%$$
%commutes would say that these approaches give the same result.

物理系$S$が与えられたとき, 適用できる数学的な取り組み方に$f\taking S\to A$と$g\taking S\to B$という二種類があるかもしれない. そのどちらからも, 同じ種類の予言が得られる. すなわち$f'\taking A\to P$と$g'\taking B\to P$. 例えば
\href{http://en.wikipedia.org/wiki/Hamiltonian_mechanics#As_a_reformulation_of_Lagrangian_mechanics}{力学}において, 未来の状態を予言するために我々はLagrangian形式とHamiltonian形式のどちらかを使うことができる. 図式
$$
\xymatrix{S\ar[r]\ar[d]&A\ar[d]\\B\ar[r]&P}
$$
は可換であると言えるならば, どちらの方法をとっても同じ結果が得られることが言えるだろう.

\end{application}

%And so on. Note that diagram (\ref{dia:triangle}) is considered to be the same diagram as each of the following:
%$$
%\xymatrix{A\ar[r]^f\ar[d]_h&B\ar[dl]^g\\C}\hspace{.8in}
%\xymatrix{A\ar[r]^f\ar@/_1pc/[rr]_h&B\ar[r]^g&C}\hspace{.8in}
%\xymatrix{B\ar[rd]^g\\&C\\A\ar[ru]_h\ar[uu]^f}$$

以下も同様である. 図式\eqref{dia:triangle}は次のそれぞれと同じ図式であることに注意しよう.
$$
\xymatrix{A\ar[r]^f\ar[d]_h&B\ar[dl]^g\\C}\hspace{.8in}
\xymatrix{A\ar[r]^f\ar@/_1pc/[rr]_h&B\ar[r]^g&C}\hspace{.8in}
\xymatrix{B\ar[rd]^g\\&C\\A\ar[ru]_h\ar[uu]^f}$$


\section{Olog}\label{sec:ologs}\index{olog}

%In this course we will ground the mathematical ideas in applications whenever possible. To that end we introduce ologs, which will serve as a bridge between mathematics and various conceptual landscapes. The following material is taken from \cite{SK}, an introduction to ologs.
%\begin{align}\label{dia:arginine}\fbox{\xymatrixnocompile{\obox{D}{1in}{\rr an amino acid found in dairy}\LAL{dr}{is}&\obox{A}{.5in}{arginine}\ar@{}[dl]|(.3){\checkmark}\ar@{}[dr]|(.3){\checkmark}\LA{r}{has}\LAL{l}{is}\LA{d}{is}&\obox{E}{.9in}{\rr an electrically-charged side chain}\LA{d}{is}\\&\obox{X}{.9in}{an amino acid}\LAL{dl}{has}\LA{dr}{has}\LA{r}{has}&\smbox{R}{a side chain}\\\mebox{N}{an amine group}&&\mebox{C}{a carboxylic acid}}}\end{align}  

この教程では, 我々は数学的なアイデアを可能な限り実際の応用に基づかせようとしている. その目的に沿って, 我々はologを導入する. ologは様々な数学と概念の風景との間への架け橋となる. 以下の題材はologの入門書である\cite{SK}から取っている.
\footnote{訳注: ologは英語で記述されることを前提としているようである. 訳においては, 日本語でologを記述することは試みておらず, また日本語用に文章を置き換えたりもしていない.}
\begin{align}\label{dia:arginine}\fbox{\xymatrixnocompile{\obox{D}{1in}{\rr an amino acid found in dairy}\LAL{dr}{is}&\obox{A}{.5in}{arginine}\ar@{}[dl]|(.3){\checkmark}\ar@{}[dr]|(.3){\checkmark}\LA{r}{has}\LAL{l}{is}\LA{d}{is}&\obox{E}{.9in}{\rr an electrically-charged side chain}\LA{d}{is}\\&\obox{X}{.9in}{an amino acid}\LAL{dl}{has}\LA{dr}{has}\LA{r}{has}&\smbox{R}{a side chain}\\\mebox{N}{an amine group}&&\mebox{C}{a carboxylic acid}}}\end{align}  


%\cite{SGWB}. 
%\newpage
%\newgeometry{left=.7in,right=.4in,top=1.4in,bottom=1.4in}
%\begin{figure}
%\includegraphics[height=7in]{olog--ProteinSocial}
%\caption{This olog, taken from \cite{SGWB}, describes both a particular kind of social network and a particular kind of protein material in terms of the same arrangement of 23 sets and 44 functions. The elements of these sets should be known to the olog's author, but may not be interpretable by an arbitrary reader of the olog; for example, {\bf V}=\fakebox{a real number} should be interpretable by most readers, but it is unlikely that {\bf U}=\fakebox{a building block} is interpretable without reading that paper.}
%\label{fig:protein social olog}
%\end{figure}
%\newgeometry{left=1.6in,right=1.6in,top=1.4in,bottom=1.4in}

%%%% Subsection %%%%

%\subsection{Types}\index{olog!types}
\subsection{型}\index{olog!types}

%A type is an abstract concept, a distinction the author has made.  We represent each type as a box containing a {\em singular indefinite noun phrase.}   Each of the following four boxes is a type: \begin{align}\label{dia:types}\xymatrixnocompile{\fbox{a man}&\fbox{an automobile}\\\obox{}{1.5in}{a pair $(a,w)$, where $w$ is a woman and $a$ is an automobile}&\obox{}{1.5in}{a pair $(a,w)$ where $w$ is a woman and $a$ is a blue automobile owned by $w$}}\end{align}

型(type)は抽象的な概念であり, ologの作者が行なった区分を示す. 我々はそれぞれの型を\emph{不定な(定冠詞をともなわない)単数の名詞句}を含む箱で表現する. 次の四つの箱は型である. \begin{align}\label{dia:types}\xymatrixnocompile{\fbox{a man}&\fbox{an automobile}\\\obox{}{1.5in}{a pair $(a,w)$, where $w$ is a woman and $a$ is an automobile}&\obox{}{1.5in}{a pair $(a,w)$ where $w$ is a woman and $a$ is a blue automobile owned by $w$}}\end{align}

%Each of the four boxes in (\ref{dia:types}) represents a type of thing, a whole class of things, and the label on that box is what one should call {\em each example} of that class.  Thus \fakebox{a man} does not represent a single man, but the set of men, each example of which is called ``a man".  Similarly, the bottom right box represents an abstract type of thing, which probably has more than a million examples, but the label on the box indicates the common name for each such example.  

\eqref{dia:types}の四つの箱はそれぞれ物の型, すなわち物の種類の全てを表しており, 箱の上の見出しはその種類の\emph{例の一つ}と呼ばれるべきものである. それゆえに, \fakebox{a man}は一人の男性(man)を示しているわければなく, その例の一つが``a man''と呼ばれるであろう男性の集合を表わしている. 同様に右下の箱は抽象的の物---百万以上の例がありうるだろう---の型を示しており, 箱の見出しはそれぞれの個々の例をまとめた一般的な名前を示している.

%Typographical problems emerge when writing a text-box in a line of text, e.g. the text-box \fbox{a man} seems out of place here, and the more in-line text-boxes there are, the worse it gets.  To remedy this, I will denote types which occur in a line of text with corner-symbols; e.g. I will write \fakebox{a man} instead of \fbox{a man}.

一行の中に文章が入った箱を書くときには組版上の問題が生じてくる. 例えば\fbox{a man}という文章が入った箱はここであるべき場所からはみだしているように見えるし, 文書が入った箱がもっとたくさん本文中にある場合には状況はますます悪くなる. これを直すために, 本文中では上隅の鉤の記号をともなった文書で型を示すことにする. 例えば, \fbox{a man}の代わりに\fakebox{a man}と書く.

%% Subsubsection %%

%\subsubsection{Types with compound structures}
\subsubsection{複合的な構造を持つ型}

%Many types have compound structures; i.e. they are composed of smaller units.  Examples include \begin{align}\label{dia:compound}\xymatrixnocompile{\obox{}{.7in}{\rr a man and a woman}&\obox{}{1.3in}{\rr a food portion $f$ and a child $c$ such that $c$ ate all of $f$}&\labox{}{a triple $(p,a,j)$ where $p$ is a paper, $a$ is an author of $p$, and $j$ is a journal in which $p$ was published}}\end{align}  It is good practice to declare the variables in a ``compound type", as I did in the last two cases of (\ref{dia:compound}).  In other words, it is preferable to replace the first box above with something like $$\obox{}{.8in}{a man $m$ and a woman $w$}\hsp\tn{or}\hsp\obox{}{1.1in}{\rr a pair $(m,w)$ where $m$ is a man and $w$ is a woman}$$ so that the variables $(m,w)$ are clear.

多くの型は複合的な構造を持つ. i.e. それらはより小さな単位から構成されている. 例としては, 以下のようなものがある. \begin{align}\label{dia:compound}\xymatrixnocompile{\obox{}{.7in}{\rr a man and a woman}&\obox{}{1.3in}{\rr a food portion $f$ and a child $c$ such that $c$ ate all of $f$}&\labox{}{a triple $(p,a,j)$ where $p$ is a paper, $a$ is an author of $p$, and $j$ is a journal in which $p$ was published}}\end{align} \eqref{dia:compound}の後半二つの例で行なっているように, ``複合的な型''では変数(variable)を宣言するのがよい習慣である. 言いかえれば, 上記の例での最初の箱は$$\obox{}{.8in}{a man $m$ and a woman $w$}\hsp\tn{or}\hsp\obox{}{1.1in}{\rr a pair $(m,w)$ where $m$ is a man and $w$ is a woman}$$のように置き換えるのがよい. これは変数$(m,w)$を明確にするためである.

\begin{rules}\label{rules:types}\index{olog!rules}

%A type is presented as a text box.  The text in that box should 
%\begin{enumerate}[(i)]
%\item begin with the word ``a" or ``an";
%\item refer to a distinction made and recognizable by the olog's author;
%\item refer to a distinction for which instances can be documented;
%\item declare all variables in a compound structure. 
%\end{enumerate}

型は文章が入った箱で表現される. 箱の中の文章は以下の規則を守ることが望ましい.
\begin{enumerate}[(i)]
\item 不定冠詞``a''か``an''で始まるべし.
\item ologの作者がなしている区分と認識を反映しているべし.
\item 区分を目的として事例を説明文章化できる, そのような区分の仕方を反映すべし.
\item 複合的な構造を持っている場合は全ての変数を宣言するべし.
\end{enumerate}

\end{rules}

%The first, second, and third rules ensure that the class of things represented by each box appears to the author as a well-defined set.  The fourth rule encourages good ``readability" of arrows, as will be discussed next in Section \ref{sec:aspects}.  

一番目, 二番目, そして三番目の規則は, それぞれの箱が表現している物の種類が, well-definedな集合として作者には見えていることを保証している. 四番目の規則は矢印の''読み易さ(readability)''が高くあるべきということを推奨している. それについては次の\ref{sec:aspects}章で議論しよう.

%I will not always follow the rules of good practice throughout this document.  I think of these rules being followed ``in the background" but that I have ``nicknamed" various boxes.  So \fakebox{Steve} may stand as a nickname for \fakebox{a thing classified as Steve} and \fakebox{arginine} as a nickname for \fakebox{a molecule of arginine}. However, when pressed, one should always be able to rename each type according to the rules of good practice.

我々はこの文書全体を通して常にこのよき実践のための規則に従うわけではない. これらの規則は``背後では''守られているべきではあると思うが, しかしながら``別名を持つ''様々な箱も使うことにする. だから\fakebox{Steve}は\fakebox{a thing classified as Steve}の別名として, \fakebox{arginine}は\fakebox{a molecule of arginine}の別名として理解されるだろう. しかしながら, 求められた際には, 常によき実践のための規則に従って, それぞれの型を改名できてしかるべきだ.

%%%% Subsection %%%%

%\subsection{Aspects}\label{sec:aspects}\index{olog!aspects}
\subsection{様相}\label{sec:aspects}\index{olog!aspects}

%An aspect of a thing $x$ is a way of viewing it, a particular way in which $x$ can be regarded or measured.  For example, a woman can be regarded as a person; hence ``being a person" is an aspect of a woman.  A molecule has a molecular mass (say in daltons), so ``having a molecular mass" is an aspect of a molecule.  In other words, by {\em aspect} we simply mean a function. The domain $A$ of the function $f\taking A\to B$ is the thing we are measuring, and the codomain is the set of possible ``answers" or results of the measurement. 
%\begin{align}\label{dia:aspect 1}\xymatrixnocompile{\fbox{a woman}\LA{r}{is}&\fbox{a person}}\end{align}\begin{align}\label{dia:aspect 2}\xymatrixnocompile{\fbox{a molecule}\LA{rr}{has as molecular mass (Da)}&\hspace{.7in}&\fbox{a positive real number}}\end{align}

物$x$の様相(aspect)はそれを見方であり, $x$をそうであると見なしたりそれを通じて評価したりする具体的な方法である. 例えば女性(a woman)は人間(a person)であるとみなすことができる. それゆえに``人間である''は女性の様相である. 分子(a molecule)は分子質量(a molecular mass, 単位はダルトン)を持ち, よって``分子質量を持つ''は分子の様相である. 言いかえれば, 我々は\emph{様相}を単純に函数の意味で使う. 函数$f\taking A\to B$の始域$A$は我々が測定する物であり, 終域はありうる``答え''あるいは測定結果の集合である.
\begin{align}\label{dia:aspect 1}\xymatrixnocompile{\fbox{a woman}\LA{r}{is}&\fbox{a person}}\end{align}\begin{align}\label{dia:aspect 2}\xymatrixnocompile{\fbox{a molecule}\LA{rr}{has as molecular mass (Da)}&\hspace{.7in}&\fbox{a positive real number}}\end{align}

%So for the arrow in (\ref{dia:aspect 1}), the domain is the set of women (a set with perhaps 3 billion elements); the codomain is the set of persons (a set with perhaps 6 billion elements).   We can imagine drawing an arrow from each dot in the ``woman" set to a unique dot in the ``person" set, just as in (\ref{dia:setmap}).  No woman points to two different people, nor to zero people --- each woman is exactly one person --- so the rules for a function are satisfied.  Let us now concentrate briefly on the arrow in (\ref{dia:aspect 2}).  The domain is the set of molecules, the codomain is the set $\RR_{>0}$ of positive real numbers.  We can imagine drawing an arrow from each dot in the ``molecule" set to a single dot in the ``positive real number" set.  No molecule points to two different masses, nor can a molecule have no mass: each molecule has exactly one mass.  Note however that two different molecules can point to the same mass.

よって\eqref{dia:aspect 1}での矢印にとって, 始域は女性の集合(おそらく30億の要素がある集合)であり, 終域は人間の集合(おそらく60億の要素がある集合)である. 我々は, \eqref{dia:setmap}でやったのと同じように, ``女性(woman)''の集合の中のそれぞれの点から``人間(person)''の集合の中の一意な点に矢印を描くことを想像できる. いかなる女性からでる矢印も二人の異なった人間を指すことはないし, またどの人間も指さないということもない---それぞれの女性は正確に一人の人間である---だから函数であるための規則は満たされている. 次は, \eqref{dia:aspect 2}での矢印にしばし着目してみよう. 始域は分子の集合であり, 終域は正の実数からなる集合$\RR_{>0}$である. 我々は``分子(molecule)''の集合の中のそれぞれの点から``正の実数(positive real number)''の集合の中の一点に矢印を描くことを想像することができる. どの分子も異なった二つの質量を指し示すことはないし, また分子が質量を持たないということもありえない. それぞれの分子は正確に一つだけ質量を持つ. しかしながら二つの異なった分子が同じ質量を指し示すことがありうることに注意しよう.

%% Subsubsection %%

%\subsubsection{Invalid aspects}\label{sec:invalid aspect}\index{olog!invalid aspects}
\subsubsection{無効な様相}\label{sec:invalid aspect}\index{olog!invalid aspects}

%I tried above to clarify what it is that makes an aspect ``valid", namely that it must be a ``functional relationship."  In this subsection I will show two arrows which on their face may appear to be aspects, but which on closer inspection are not functional (and hence are not valid as aspects).  
 
これまでのところ, 著者はある様相が``有効(valid)''であるのは何によってかということを明確にしようとしてきた. それはすなわち``函数としての関係''でなければならないという点である. この小節では, その表向きは一見したところ様相に見えるものの, より細やかな精査の上では函数ではない(それゆえに有効な様相ではない)二本の矢印を示すことにしよう.
 
%Consider the following two arrows:
%\begin{align}\tag{\arabic{subsection}.\arabic{equation}*}\addtocounter{equation}{1}\label{dia:invalid 1}
%\xymatrixnocompile{\fbox{a person}\LA{r}{has}&\fbox{a child}}
%\end{align}
%\vspace{-.13in}
%\begin{align}\tag{\arabic{subsection}.\arabic{equation}*}\addtocounter{equation}{1}\label{dia:invalid 2}
%\xymatrixnocompile{\fbox{a mechanical pencil}\LA{r}{uses}&\fbox{a piece of lead}}
%\end{align}  
%A person may have no children or may have more than one child, so the first arrow is invalid: it is not a function.  Similarly, if we drew an arrow from each mechanical pencil to each piece of lead it uses, it would not be a function.

以下の二本の矢印を考えよう.
\begin{align}\tag{\arabic{subsection}.\arabic{equation}*}\addtocounter{equation}{1}\label{dia:invalid 1}
\xymatrixnocompile{\fbox{a person}\LA{r}{has}&\fbox{a child}}
\end{align}
\vspace{-.13in}
\begin{align}\tag{\arabic{subsection}.\arabic{equation}*}\addtocounter{equation}{1}\label{dia:invalid 2}
\xymatrixnocompile{\fbox{a mechanical pencil}\LA{r}{uses}&\fbox{a piece of lead}}
\end{align}  
ある人間は子供を持たないかもしれないし, あるいは一人よりも多くの子供を持つかもしれない. よって最初の矢印は無効である. それは函数ではない. 同様に, もし我々がそれぞれのシャープペンシル(mechanical pencil)からそれが使用している芯に矢印を書くことができたのであるのならば, それが函数ではないことが分かるだろう.

\begin{warning}\label{warn:worldview}\index{a warning!different worldviews}

%The author of an olog has a world-view, some fragment of which is captured in the olog.  When person A examines the olog of person B, person A may or may not ``agree with it."  For example, person B may have the following olog $$\fbox{\xymatrix{&\fbox{a marriage}\LA{dr}{ includes}\LAL{dl}{includes }\\\fbox{a man}&&\fbox{a woman}}}$$ which associates to each marriage a man and a woman.  Person A may take the position that some marriages involve two men or two women, and thus see B's olog as ``wrong."  Such disputes are not ``problems" with either A's olog or B's olog, they are discrepancies between world-views.  Hence, throughout this paper, a reader R may see a displayed olog and notice a discrepancy between R's world-view and my own, but R should not worry that this is a problem.  This is not to say that ologs need not follow rules, but instead that the rules are enforced to ensure that an olog is structurally sound, rather than that it ``correctly reflects reality," whatever that may mean.

ologの作者はなんらかの世界観(world-view)を持っており, olongの中にはその一片が捉えられている. ある人物Aが人物Bによるologを検査するとき, 人物Aはそれを``認める''ことがあったりなかったりするだろう. 例えば, 人物Bは以下のolog$$\fbox{\xymatrix{&\fbox{a marriage}\LA{dr}{ includes}\LAL{dl}{includes }\\\fbox{a man}&&\fbox{a woman}}}$$を持っているかもしれない. これはそれぞれの結婚をある男性とある女性に関連付けている. 人物Aは結婚は二人の男性あるいは二人の女性によることもあるという立場を取っており, それゆえにBのologは``悪い''ものと受けとめるかもしれない. このような論争はAのologあるいはBのologにおける``問題点''ではなく, 世界観の間の矛盾である. それゆえに, この文章を通して, 読者Rは表示されたologを見てRの世界観と著者のそれとの間の矛盾に気付くかもしれないが, しかしRはそのことを問題ではないかと気に病むべきではない. これはologが規則に従わなくてもよいと言っているのではない. 規則が課されているのは, それが何を意味しているのであれ``現実を正確に反映する''ためではなく, むしろologが構造的に健全であることを保証するためである.

%Consider the aspect $\fakebox{an object}\Too{\tn{has}}\fakebox{a weight}$. At some point in history, this would have been considered a valid function. Now we know that the same object would have a different weight on the moon than it has on earth. Thus as world-views change, we often need to add more information to our olog. Even the validity of $\fakebox{an object on earth}\Too{\tn{has}}\fakebox{a weight}$ is questionable. However to build a model we need to choose a level of granularity and try to stay within it, or the whole model evaporates into the nothingness of truth!

様相$\fakebox{an object}\Too{\tn{has}}\fakebox{a weight}$を考える. 歴史上のある時点では, これは有効な函数であったであろう. 今や我々は同じ物体でも月の上と地球の上では異なる重さを持つであろうことを知っている. それゆえに世界観の変化に従って我々はしばしばologにさらなる情報を加える必要が出てくる. $\fakebox{an object on earth}\Too{\tn{has}}\fakebox{a weight}$の有効性でさえも疑問である. しかし, 模型を構築するにあたって我々は粒度の水準を選びその中に留まろうとする必要がある. そうでなければ, 全ての模型はなんの真実もない虚無へと霧散してしまう.

\end{warning}

\begin{remark}

%In keeping with Warning \ref{warn:worldview}, the arrows (\ref{dia:invalid 1}) and (\ref{dia:invalid 2}) may not be wrong but simply reflect that the author has a strange world-view or a strange vocabulary.  Maybe the author believes that every mechanical pencil uses exactly one piece of lead.  If this is so, then $\fakebox{a mechanical pencil}\To{\tn{uses}}\fakebox{a piece of lead}$ is indeed a valid aspect!   Similarly, suppose the author meant to say that each person {\em was once} a child, or that a person has an inner child.  Since every person has one and only one inner child (according to the author), the map $\fakebox{a person}\To{\tn{has as inner child}}\fakebox{a child}$ is a valid aspect.  We cannot fault the olog if the author has a view, but note that we have changed the name of the label to make his or her intention more explicit.

Warning \ref{warn:worldview}を念頭に置くと, \eqref{dia:invalid 1}および\eqref{dia:invalid 2}の矢印は悪いものではなく, 作者が奇妙な世界観あるいは奇妙な語彙を持っているということの単純な反映なのであろう. おそらく作者は全てのシャープペンシルは正確に一本だけの芯を使うものだと信じているのだろう. もしそうであるならば, $\fakebox{a mechanical pencil}\To{\tn{uses}}\fakebox{a piece of lead}$は実際に有効な様相となる! 同様に, 全ての人間は\emph{一度は}子供である, あるいはある人間は内なる子(inner child)を持っているということを, 作者が言おうと意図していると仮定してみよう. (このologの作者によると)全ての人間は内なる子をただ一人だけ持つから, 写像$\fakebox{a person}\To{\tn{has as inner child}}\fakebox{a child}$は有効な様相である. ologの作者が世界観を持っているならば, 我々はologを非難することはできない. しかし, 作者の意図するところをより明示的にするために, 我々は見出しの名前は変えている.

\end{remark}

%% Subsubsection %%

%\subsubsection{Reading aspects and paths as English phrases}
\subsubsection{英語の句として様相と経路を読む}

%Each arrow (aspect) $X\To{f} Y$ can be read by first reading the label on its source box (domain of definition) $X$, then the label on the arrow $f$, and finally the label on its target box (set of values) $Y$.  For example, the arrow \begin{align}\label{dia:first author}\fbox{\xymatrixnocompile{\smbox{}{a book}\LA{rrr}{has as first author}&&&\smbox{}{a person}}}\end{align} is read ``a book has as first author a person".  

それぞれの矢印(様相)$X\To{f} Y$は, 最初に出発元の箱の見出し(定義の始域)$X$を読み, 次に矢印$f$の見出しを読み, 最後にその到着先の箱の見出し(値の集合)$Y$を読むことによって, (英語の)文として読むことができる. 例えば, 矢印\begin{align}\label{dia:first author}\fbox{\xymatrixnocompile{\smbox{}{a book}\LA{rrr}{has as first author}&&&\smbox{}{a person}}}\end{align}は``a book has as first author a person''と読むことができる.

\begin{remark}

%Note that the map in (\ref{dia:first author}) is a valid aspect, but that a similarly benign-looking map $\fakebox{a book}\To{\tn{has as author}}\fakebox{a person}$ would not be valid, because it is not functional.  The authors of an olog must be vigilant about this type of mistake because it is easy to miss and it can corrupt the olog.

\eqref{dia:first author}での写像は有効な様相であるが, よく似ていて一見害はなさそうに見える写像$\fakebox{a book}\To{\tn{has as author}}\fakebox{a person}$は函数的ではなく, それゆえに有効ではないことに注意しよう. この種の間違いは非常に犯しやすくかつologを破壊するものであるから, ologの作者は必ず警戒しなければならない.

\end{remark}

%Sometimes the label on an arrow can be shortened or dropped altogether if it is obvious from context.  We will discuss this more in Section \ref{sec:facts} but here is a common example from the way I write ologs. \begin{align}\label{dia:pair of integers}\fbox{\xymatrixnocompile{&\obox{A}{1.2in}{\rr a pair $(x,y)$ where $x$ and $y$ are integers}\ar[dl]_x\ar[dr]^y\\\smbox{B}{an integer}&&\smbox{B}{an integer}}}\end{align}  Neither arrow is readable by the protocol given above (e.g. ``a pair $(x,y)$ where $x$ and $y$ are integers $x$ an integer" is not an English sentence), and yet it is obvious what each map means.  For example, given $(8,11)$ in $A$, arrow $x$ would yield $8$ and arrow $y$ would yield $11$.  The label $x$ can be thought of as a nickname for the full name ``yields, via the value of $x$," and similarly for $y$.  I do not generally use the full name for fear that the olog would become cluttered with text.

文脈が明らかな場合には, 矢印の上の見出しはしばしば縮められたりあるいは完全に落とされてしまいうる. 我々は\ref{sec:facts}章でさらなる議論を行なうが, ここでは著者がologを書く方法から一般的な例をとりあげよう. \begin{align}\label{dia:pair of integers}\fbox{\xymatrixnocompile{&\obox{A}{1.2in}{\rr a pair $(x,y)$ where $x$ and $y$ are integers}\ar[dl]_x\ar[dr]^y\\\smbox{B}{an integer}&&\smbox{B}{an integer}}}\end{align} どの矢印も上記で与えられた規約によって読むことはできない(e.g. ``a pair $(x,y)$ where $x$ and $y$ are integers $x$ an integer''は英語の文ではない). しかしそれでもそれぞれの写像が何を意味しているかは明らかである. 例えば, $A$の要素として$(8,11)$が与えられたとき, 矢印$x$は$8$をもたらし矢印$y$は$11$をもたらすだろう. 見出し$x$はフルネーム``yields, via the value of $x$''のあだ名と考えることができ, また$y$に関しても同様である. ologが文書まみれで取り散らかる虞れがあるため, 著者は一般的にフルネームを使うことはない.

%One can also read paths through an olog by inserting the word ``which" after each intermediate box.
%\footnote{If the intended elements of an intermediate box are humans, it is polite to use ``who" rather than ``which", and other such conventions may be upheld if one so desires.}
%For example the following olog has two paths of length 3 (counting arrows in a chain): \small\begin{align}\label{olog:paths}\fbox{\xymatrixnocompile{\fbox{a child}\LA{r}{is}&\fbox{a person}\LA{rr}{has as parents}\LAL{dr}{has, as birthday}&&\obox{}{.8in}{\rr a pair $(w,m)$ where $w$ is a woman and $m$ is a man}\LA{r}{$w$}&\fbox{a woman}\\&&\fbox{a date}\LA{r}{includes}&\fbox{a year}}}\end{align}  \normalsize The top path is read ``a child is a person, who has as parents a pair $(w,m)$ where $w$ is a woman and $m$ is a man, which yields, via the value of $w$, a woman."  The reader should read and understand the content of the bottom path, which associates to every child a year.  

ologの中に走る経路は, それぞれの間にある箱の後ろ側に単語``which''を挿入するめによっても英文として読むことができる.
\footnote{間に入る箱の要素として意図されるものが人間であるならば, ``which''ではなくむしろ``who''を使うのが礼儀である. 他のこのような慣習についても, もし望まれるのであるならば是認されよう.}
例えば次のologは長さ3 (連鎖における矢印を数える)の二本の経路を含んでいる. {\small\begin{align}\label{olog:paths}\fbox{\xymatrixnocompile{\fbox{a child}\LA{r}{is}&\fbox{a person}\LA{rr}{has as parents}\LAL{dr}{has, as birthday}&&\obox{}{.8in}{\rr a pair $(w,m)$ where $w$ is a woman and $m$ is a man}\LA{r}{$w$}&\fbox{a woman}\\&&\fbox{a date}\LA{r}{includes}&\fbox{a year}}}\end{align} } 上側の経路は``a child is a person, who has as parents a pair $(w,m)$ where $w$ is a woman and $m$ is a man, which yields, via the value of $w$, a woman.''と読むことができる. きっと読者は下側の経路を読んで内容が理解できるはずである. この経路は全ての子をある年に関連付けている.


%% Subsubsection %%

%\subsubsection{Converting non-functional relationships to aspects}\label{sec:relations}
\subsubsection{函数的でない関係を様相へ変換する}\label{sec:relations}

%There are many relationships that are not functional, and these cannot be considered aspects.  Often the word ``has" indicates a relationship --- sometimes it is functional as in $\fakebox{a person}\To{\tn{ has }}\fakebox{a stomach}$, and sometimes it is not, as in $\fakebox{a father}\To{\tn{has}}\fakebox{a child}$. Obviously, a father may have more than one child. This one is easily fixed by realizing that the arrow should go the other way: there is a function $\fakebox{a child}\To{\tn{has}}\fakebox{a father}$. 

多くの函数的ばない関係が存在しており, それらは様相とみなすことはできない. しばしば単語``has''は関係を示唆している---ある時は$\fakebox{a person}\To{\tn{ has }}\fakebox{a stomach}$でみられるように函数的であるし, またある時は$\fakebox{a father}\To{\tn{has}}\fakebox{a child}$でみられるように函数的ではない. 明らかに, 父親は二人以上の子を持つかもしれない. この図は, 矢印を別の方向に走らせればよいということに気がつくことによって簡単に修正される. 函数$\fakebox{a child}\To{\tn{has}}\fakebox{a father}$があるからだ.

%What about $\fakebox{a person}\To{\tn{owns}}\fakebox{a car}$. Again, a person may own no cars or more than one car, but this time a car can be owned by more than one person too. A quick fix would be to replace it by $\fakebox{a person}\To{\tn{owns}}\fakebox{a set of cars}$.   This is ok, but the relationship between \fakebox{a car} and \fakebox{a set of cars} then becomes an issue to deal with later.  There is another way to indicate such ``non-functional" relationships. In this case it would look like this:
%$$
%\fbox{\xymatrix{&\obox{}{1.15in}{a pair $(p,c)$ where $p$ is a person, $c$ is a car, and $p$ owns $c$.}\ar[ddl]_p\ar[ddr]^c\\\\
%\obox{}{.5in}{a person}&&\obox{}{.3in}{a car}}}
%$$
%This setup will ensure that everything is properly organized. In general, relationships can involve more than two types, and the general situation looks like this $$\fbox{\xymatrixnocompile{&&\fbox{$R$}\ar[ddll]\ar[ddl]\ar[ddr]\\\\\fbox{$A_1$}&\fbox{$A_2$}&\cdots&\fbox{$A_n$}}}$$  For example, $$\fbox{\xymatrixnocompile{&\labox{R}{a sequence $(p,a,j)$ where $p$ is a paper, $a$ is an author of $p$, and $j$ is a journal in which $p$ was published}\ar[ddl]_p\ar[dd]_a\ar[ddr]^j\\\\\smbox{A_1}{a paper}&\smbox{A_2}{an author}&\smbox{A_3}{a journal}}}$$ 

$\fakebox{a person}\To{\tn{owns}}\fakebox{a car}$の場合はどうだろうか. ある人物は一台も車を持っていないかもしれないし二台以上持っているかもしれない. しかしこの場合には, ある車は二人以上の人物に所有されているかもしれない. 手短に修正するならば$\fakebox{a person}\To{\tn{owns}}\fakebox{a set of cars}$で置き換えればよいかもしれない. これはこれでよいが, しかし今度は\fakebox{a car}と\fakebox{a set of cars}との間の関係性が, 後々処理すべき問題となる. このような``函数的ではない''関係性を指示するには他の方法もある. この場合は以下のようになるだろう.
$$
\fbox{\xymatrix{&\obox{}{1.15in}{a pair $(p,c)$ where $p$ is a person, $c$ is a car, and $p$ owns $c$.}\ar[ddl]_p\ar[ddr]^c\\\\
\obox{}{.5in}{a person}&&\obox{}{.3in}{a car}}}
$$
この設定は全てが適切に組織されていることを保証するであろう. 一般に, 関係性には二つ以上の型が関わることがあり, これら一般の状況は以下のように見ることができる
$$\fbox{\xymatrixnocompile{&&\fbox{$R$}\ar[ddll]\ar[ddl]\ar[ddr]\\\\\fbox{$A_1$}&\fbox{$A_2$}&\cdots&\fbox{$A_n$}}}$$ 例えば, $$\fbox{\xymatrixnocompile{&\labox{R}{a sequence $(p,a,j)$ where $p$ is a paper, $a$ is an author of $p$, and $j$ is a journal in which $p$ was published}\ar[ddl]_p\ar[dd]_a\ar[ddr]^j\\\\\smbox{A_1}{a paper}&\smbox{A_2}{an author}&\smbox{A_3}{a journal}}}$$


\begin{exercise}
%On page \pageref{dia:invalid 1} we indicate a so-called invalid aspect, namely 
%\begin{align}\tag{\ref{dia:invalid 1}}\xymatrixnocompile{\fbox{a person}\LA{r}{has}&\fbox{a child}}
%\end{align}
%Create a (valid) olog that captures the parent-child relationship; your olog should still have boxes %\fakebox{a person} and \fakebox{a child} but may have an additional box.
\pageref{dia:invalid 1}で我々はいわゆる無効な様相, すなわち\begin{align}\tag{\ref{dia:invalid 1}}\xymatrixnocompile{\fbox{a person}\LA{r}{has}&\fbox{a child}}\end{align}を指摘した.
両親-子供の関係を捉えた(有効な)ologを作成せよ. 作成されたologは依然\fakebox{a person}および\fakebox{a child}を持っているべきだが, 追加の箱も持っているかもしれない.
\end{exercise}

\begin{rules}\label{rules:aspects}\index{olog!rules}

%An aspect is presented as a labeled arrow, pointing from a source box to a target box.  The arrow text should

様相は, 出発元から到着先を指し示す見出し付けられた矢印として表示される. 矢印の文章は以下を守ることが望ましい.

%\begin{enumerate}[(i)]
%\item begin with a verb;
%\item yield an English sentence, when the source-box text followed by the arrow text followed by the target-box text is read; and
%\item refer to a functional relationship: each instance of the source type should give rise to a specific instance of the target type.
%\end{enumerate}

\begin{enumerate}[(i)]
\item 動詞から始まるべし.
\item 出発元の箱の文書, 矢印の文書, 到着先の箱の文書の順で読んだとき, 英語の文書がもたらされるべし. 
\item 函数的な関係が適用されるべし. すなわち出発元の型の具体例は, 到着先の型のある特定の具体例をもたらすべし.
\end{enumerate}

\end{rules}

%%%% Subsection %%%%

%\subsection{Facts}\label{sec:facts}\index{olog!facts}
\subsection{事実}\label{sec:facts}\index{olog!facts}

%In this section I will discuss facts, which are simply ``path equivalences" in an olog. It is the notion of path equivalences that make category theory so powerful. 

この章では``事実(fact)''について議論したい. ologにおいて事実とは単純に``経路等価性(path equivalences)''である. 経路等価性の概念こそが圏論をかくも強力なものとしている.

%A {\em path}\index{olog!path in} in an olog is a head-to-tail sequence of arrows. That is, any path starts at some box $B_0$, then follows an arrow emanating from $B_0$ (moving in the appropriate direction), at which point it lands at another box $B_1$, then follows any arrow emanating from $B_1$, etc, eventually landing at a box $B_n$ and stopping there. The number of arrows is the {\em length} of the path. So a path of length 1 is just an arrow, and a path of length 0 is just a box. We call $B_0$ the {\em source} and $B_n$ the {\em target} of the path.

ologにおいて\emph{経路(path)}\index{olog!path in}とは矢印の先頭から末尾までの列である. これはすなわち, ある箱から$B_0$から出発する任意の経路は$B_0$を出発し, $B_0$から(適切な方向に向かって)発せられる矢印が直後に続き, その矢印が指すところの他の箱$B_1$に到着し, さらに$B_1$から発せられる任意の矢印が後に続き, これらが繰り返された後, ある箱$B_n$に到着しそこで終了する. 矢印の数は経路の\emph{長さ(length)}である. よって長さ1の経路は単なる矢印であり, 長さ0の経路は単なる箱である. 我々は$B_0$を経路の\emph{源泉(source)}, $B_n$を経路の\emph{標的(target)}と呼ぶことにする.

%Given an olog, the author may want to declare that two paths are equivalent.  For example consider the two paths from $A$ to $C$ in the olog 
%\begin{align}\label{olog:commute}\fbox{\xymatrixnocompile{\smbox{A}{a person}\LA{rr}{has as parents}\LAL{drr}{\parbox{.8in}{has as mother}}&&\obox{B}{.8in}{\rr a pair $(w,m)$ where $w$ is a woman and $m$ is a man}\ar@{}[dll]|(.4){\checkmark}\LA{d}{yields as $w$}\\&&\smbox{C}{a woman}}}\end{align}  We know as English speakers that a woman parent is called a mother, so these two paths $A\to C$ should be equvalent.  A more mathematical way to say this is that the triangle in Olog (\ref{olog:commute}) {\em commutes}. That is, path equivalences are simply commutative diagrams as in Section \ref{sec:comm diag}. In the example above we concisely say ``a woman parent is equivalent to a mother."  We declare this by defining the diagonal map in (\ref{olog:commute}) to be {\em the composition} of the horizontal map and the vertical map. 

ologが与えられた時, 作者は二つの経路が等価であると明確に述べたくなるかもしれない. 例えば, 以下のologでの$A$から$C$への二つの経路を考えてみる.
\begin{align}\label{olog:commute}\fbox{\xymatrixnocompile{\smbox{A}{a person}\LA{rr}{has as parents}\LAL{drr}{\parbox{.8in}{has as mother}}&&\obox{B}{.8in}{\rr a pair $(w,m)$ where $w$ is a woman and $m$ is a man}\ar@{}[dll]|(.4){\checkmark}\LA{d}{yields as $w$}\\&&\smbox{C}{a woman}}}\end{align} 我々は英語話者として女性の親は母と呼ばれることを知っている. よって$A$から$C$への経路は等価であるべきである. より数学的な言い方では, Olog \eqref{olog:commute}は{可換である(commute)}. すなわち, 経路等価性とは
単純に\ref{sec:comm diag}章での可換図式である. 上記の例では我々はconcisely``a woman parent is equivalent to a mother''と言える. これを明確に述べるには, 我々は\eqref{olog:commute}での対角線の写像が水平な写像と垂直な写像の\emph{合成(composition)}として定義するとする.

%I generally prefer to indicate a commutative diagram by drawing a check-mark, $\checkmark$, in the region bounded by the two paths, as in Olog (\ref{olog:commute}).  Sometimes, however, one cannot do this unambiguously on the 2-dimensional page.  In such a case I will indicate the commutative diagrams (fact) by writing an equation.  For example to say that the diagram $$\xymatrix{A\ar[r]^f\ar[d]_h&B\ar[d]^g\\C\ar[r]_i&D}$$ commutes, we could either draw a checkmark inside the square or write the equation $A\;f\;g\simeq A\;h\;i$ above it\index{a symbol!$\simeq$}.
%\footnote{We defined function composition on page \ref{function composition}, but here we're using a different notation.\index{a warning!notation for composition} There we would have said $g\circ f = i\circ h$, which is in the backwards-seeming {\em classical order}.\index{composition!classical order} Category theorists and others often prefer the {\em diagrammatic order}\index{composition!diagrammatic order} for writing compositions, which is $f;g = h;i$. For ologs, we follow the latter because it makes for better English sentences, and for the same reason we add the source object to the equation, writing $A f g \simeq A h i$.}
% Either way, it means that ``$f$ then $g$" is equivalent to ``$h$ then $i$".  

著者は, Olog \eqref{olog:commute}でのように, 二つの経路に囲まれた領域の中にチェックマーク$\checkmark$を描くことによって可換図式をしめす記法を一般的に好んでいる. しかしながら, 二次元的な紙の上では, しばしばこれを曖昧なく行うことが不可能になる. このような場合, 可換図式(事実)を等式の形で示すことにしよう. 例えば, 図式$$\xymatrix{A\ar[r]^f\ar[d]_h&B\ar[d]^g\\C\ar[r]_i&D}$$が可換であると言うために, 我々は四角の中にチェックマークを描くか, あるいは等式$A\;f\;g\simeq A\;h\;i$を図の上に書く\index{a symbol!$\simeq$}.
\footnote{我々は\ref{function composition}ページで函数合成を定義したが, ここでは異なった記法を使っている.\index{a warning!notation for composition} \eqref{function composition}ページの記法では$g\circ f = i\circ h$となったであろう. これは後ろからみていく(backwards-seeming) \emph{古典的な順番(classcal order)}である.\index{composition!classical order} 圏論の理論家を代表とする人々は, 合成を書くときにしばしば\emph{図式的な順序(diagrammatic order)}\index{composition!diagrammatic order}を好む. これは$f;g=h;i$である. ologに対しては, 我々は後者を使う. なぜならば後者はよりよく英文になるからである. 同じ理由で等式に源泉の箱を追加して, $A f g \simeq A h i$と書くことにする.
}
どちらの方法をとっても, 意味するところは``$f$ then $g$''と``$h$ then $i$''は等しいということである..  

%Here is another, more scientific example:
%\begin{align*}
%\fbox{\xymatrix{
%\obox{}{1in}{a DNA sequence}\LA{rr}{is transcribed to}\LAL{drr}{codes for}&\hspace{.1in}&\obox{}{1.1in}{an RNA sequence}\ar@{}[dll]|(.35){\checkmark}\LA{d}{is translated to}\\
%&&\obox{}{.6in}{a protein}}}
%\end{align*}
%Note how this diagram gives us the established terminology for the various ways in which DNA, RNA, and protein are related in this context.

また別の, より科学的な例を挙げる.
\begin{align*}
\fbox{\xymatrix{
\obox{}{1in}{a DNA sequence}\LA{rr}{is transcribed to}\LAL{drr}{codes for}&\hspace{.1in}&\obox{}{1.1in}{an RNA sequence}\ar@{}[dll]|(.35){\checkmark}\LA{d}{is translated to}\\
&&\obox{}{.6in}{a protein}}}
\end{align*}
この図によって, DNA, RNA, タンパク質がこの文脈において関係しているところの様々な面で確立している用語が与えられる.

\begin{exercise}\label{exc:family olog}

%Create an olog for human nuclear biological families that includes the concept of person, man, woman, parent, father, mother, and child. Make sure to label all the arrows, and make sure each arrow indicates a valid aspect in the sense of Section \ref{sec:invalid aspect}. Indicate with check-marks ($\checkmark$) the diagrams that are intended to commute. If the 2-dimensionality of the page prevents a check-mark from being unambiguous, indicate the intended commutativity with an equation.

男性(man), 女性(woman), 親(parent), 父(father), 母(mother), 子(child)という概念を含んだ, 人間の生物的核家族のologを作成せよ. 全ての矢印には見出しがつけられていることを確認し, それぞれの矢印は\ref{sec:invalid aspect}章での意味において有効な様相を表していることを確認せよ. 可換図式を意図した図をチェックマーク($\checkmark$)で示せ. もし曖昧さなくチェックマークをつけることが紙の二次元性のせいでできない場合は意図した可換性を等式で示せ.
\end{exercise}

\begin{example}[Non-commuting diagram]

%In my conception of the world, the following diagram does not commute:
%\begin{align}\label{dia:non-commuting}
%\xymatrixnocompile@=50pt{\obox{}{.5in}{a person}\LA{r}{has as father}\LAL{dr}{lives in}&\obox{}{.4in}{a man}\LA{d}{lives in}\\&\obox{}{.4in}{a city}}
%\end{align}
%The non-commutativity of Diagram (\ref{dia:non-commuting}) does not imply that, in my conception, no person lives in the same city as his or her father. Rather it implies that, in my conception, it is not the case that {\em every} person lives in the same city as his or her father.

著者の世界に関する概念においては, 次の図式は可換ではない.
\begin{align}\label{dia:non-commuting}
\xymatrixnocompile@=50pt{\obox{}{.5in}{a person}\LA{r}{has as father}\LAL{dr}{lives in}&\obox{}{.4in}{a man}\LA{d}{lives in}\\&\obox{}{.4in}{a city}}
\end{align}
図式\eqref{dia:non-commuting}での非可換性は, 著者の概念において, その人の父と同じ都市に住む人物がいないということを含意しているわけではない. そうではなくて, 著者の概念において, \emph{全ての}人物がその人の父と同じ都市に住むとは限らないということを含意している.

\end{example}

\begin{exercise}
%Create an olog about a scientific subject, preferably one you think about often. The olog should have at least five boxes, five arrows, and one commutative diagram. 

科学的な話題, 望ましくは読者がよく考える対象についてのologを作成せよ. 作成するologには最低でも5つの箱, 5本の矢印, そして1つの可換図式を含めること.
\end{exercise}

%% Subsubsection %%

%\subsubsection{A formula for writing facts as English}\index{olog!facts in English}
\subsubsection{英語として事実を書くための公式}\index{olog!facts in English}

%Every fact consists of two paths, say $P$ and $Q$, that are to be declared equivalent. The paths $P$ and $Q$ will necessarily have the same source, say $s$, and target, say $t$, but their lengths may be different, say $m$ and $n$ respectively.
%\footnote{If the source equals the target, $s=t$, then it is possible  to have $m=0$ or $n=0$, and the ideas below still make sense.} 
%We draw these paths as 
%\begin{align}\label{dia:two paths for equivalence}
%P:&\hsp\xymatrix@=22pt{\LMO{a_0=s}\ar[r]^{f_1}&\LMO{a_1}\ar[r]^{f_2}&\LMO{a_2}\ar[r]^{f_3}&\cdots\ar[r]^{f_{m-1}}&\LMO{a_{m-1}}\ar[r]^{f_m}&\LMO{a_m=t}}\\\nonumber
%Q:&\hsp\xymatrix@=23pt{\LMO{b_0=s}\ar[r]^{g_1}&\LMO{b_1}\ar[r]^{g_2}&\LMO{b_2}\ar[r]^{g_3}&\cdots\ar[r]^{g_{n-1}}&\LMO{b_{n-1}}\ar[r]^{g_n}&\LMO{b_n=t}}
%\end{align}
%Every part $\ell$ of an olog (i.e. every box and every arrow) has an associated English phrase, which we write as $\qt{\ell}$. Using a dummy variable $x$ we can convert a fact into English too. The following general formula is a bit difficult to understand, see Example \ref{ex:English fact}, but here goes. The fact $P\simeq Q$ from (\ref{dia:two paths for equivalence}) can be Englishified as follows:

全ての事実は二つの経路(ここでは$P$と$Q$としよう)からなり, それらが等しいと宣言することにより成立する. 経路$P$と$Q$は同じ源泉($s$としよう)と同じ標的($t$としよう)を持つ必要があるだろう. しかしその長さ(それぞれ$m$と$n$とする)は異なっているかもしれない.
\footnote{源泉と標的が同じ, すなわち$s=t$である場合, $m=0$あるいは$n=0$であることが可能であり, その場合を以下のアイデアは依然として意味を持つ.} 
我々はこれらの経路を
\begin{align}\label{dia:two paths for equivalence}
P:&\hsp\xymatrix@=22pt{\LMO{a_0=s}\ar[r]^{f_1}&\LMO{a_1}\ar[r]^{f_2}&\LMO{a_2}\ar[r]^{f_3}&\cdots\ar[r]^{f_{m-1}}&\LMO{a_{m-1}}\ar[r]^{f_m}&\LMO{a_m=t}}\\\nonumber
Q:&\hsp\xymatrix@=23pt{\LMO{b_0=s}\ar[r]^{g_1}&\LMO{b_1}\ar[r]^{g_2}&\LMO{b_2}\ar[r]^{g_3}&\cdots\ar[r]^{g_{n-1}}&\LMO{b_{n-1}}\ar[r]^{g_n}&\LMO{b_n=t}}
\end{align}
のように描く.
ologの全ての部分$\ell$(i.e. 全ての箱および全ての矢印)は英語の節に関連しており, 我々はその節を$\qt{\ell}$と書く. ダミー変数$x$を使って, 我々は事実を英語に変換することもできる. 以下の一般的な公式を理解するのは少々難しい. しかし\ref{ex:English fact}ではそれが実践されているのを見ることができる. \eqref{dia:two paths for equivalence}での事実$P\simeq Q$は次のようにして英語化することができる.

\begin{align}\label{dia:Englishification}\index{Englishification}
&\tn{Given }x,\qt{s},\tn{ consider the following. We know that }x\tn{ is }\qt{s}, \\
\nonumber&\tn {which } \qt{f_1}\;\qt{a_1}, \tn{ which } \qt{f_2}\;\qt{a_2}, \tn { which }\ldots \; \qt{f_{m-1}}\;\qt{a_{m-1}}, \tn { which } \qt{f_m}\;\qt{t}\\
\nonumber&\tn{that we'll call } P(x).\\
\nonumber&\tn{We also know that }x\tn{ is } \qt{s},\\
\nonumber&\tn {which } \qt{g_1}\;\qt{b_1}, \tn{ which }\qt{g_2}\;\qt{b_2}, \tn { which }\ldots\;\qt{g_{n-1}}\;\qt{b_{n-1}}, \tn { which } \qt{g_n}\;\qt{t}\\
\nonumber&\tn{that we'll call } Q(x).\\
\nonumber&\tn{Fact: whenever }x\tn{ is }``s",\tn{ we will have }P(x)=Q(x).
\end{align}

\begin{example}\label{ex:English fact}

%Consider the olog
%\begin{align}\label{olog:commute2}\fbox{\xymatrixnocompile{\smbox{A}{a person}\LA{rr}{has}\LAL{drr}{\parbox{.8in}{lives in}}&&\obox{B}{.7in}{\rr an address}\ar@{}[dll]|(.4){\checkmark}\LA{d}{is in}\\&&\smbox{C}{a city}}}
%\end{align}
%To put the fact that Diagram \ref{olog:commute2} commutes into English, we first Englishify the two paths: $F$=``a person has an address which is in a city" and $G$=``a person lives in a city". The source of both is $s$=``a person" and the target of both is $t$=``a city".
%write:
%\begin{align*}
%&\tn{Given }x,\tn{a person, consider the following. We know that } x\tn{ is a person,}\\
%&\tn{which has an address, which is in a city}\\
%&\tn{that we'll call } P(x).\\
%&\tn{We also know that }x\tn{ is a person,}\\
%&\tn{which lives in a city}\\
%&\tn{that we'll call } Q(x).\\
%&\tn{Fact: whenever }x\tn{ is a person, we will have }P(x)=Q(x).
%\end{align*}

次のologを考える. 
\begin{align}\label{olog:commute2}\fbox{\xymatrixnocompile{\smbox{A}{a person}\LA{rr}{has}\LAL{drr}{\parbox{.8in}{lives in}}&&\obox{B}{.7in}{\rr an address}\ar@{}[dll]|(.4){\checkmark}\LA{d}{is in}\\&&\smbox{C}{a city}}}
\end{align}
図式\ref{olog:commute2}が交換する事実を英語に落とすために, 我々は最初に二つの経路$F$=``a person has an address which is in a city''および$G$=``a person lives in a city''を英語化する. 両者の源泉は$s$=``a person''であり, 標的は$t$=``a city''である. 書き下すと以下となる.
\begin{align*}
&\tn{Given }x,\tn{a person, consider the following. We know that } x\tn{ is a person,}\\
&\tn{which has an address, which is in a city}\\
&\tn{that we'll call } P(x).\\
&\tn{We also know that }x\tn{ is a person,}\\
&\tn{which lives in a city}\\
&\tn{that we'll call } Q(x).\\
&\tn{Fact: whenever }x\tn{ is a person, we will have }P(x)=Q(x).
\end{align*}

\end{example}

\begin{exercise}
%This olog was taken from \cite{Sp1}.
%\begin{align}\label{dia:phone paths}\xymatrix{&\obox{N}{1in}{a phone number}\LA{rr}{has}&&\obox{C}{.8in}{an area code}\ar@{}[dll]|{\checkmark}\LA{d}{corresponds to}\\\obox{OLP}{1.2in}{an operational landline phone}\LA{ru}{is assigned}\LAL{r}{is}&\obox{P}{1in}{a physical phone}\LAL{rr}{\parbox{.55in}{\scriptsize is currently located in}}&&\obox{R}{.5in}{a region}}
%\end{align} 
%It says that a landline phone is physically located in the region that its phone number is assigned. Translate this fact into English using the formula from \ref{dia:Englishification}.

このologは\cite{Sp1}から取った.
\begin{align}\label{dia:phone paths}\xymatrix{&\obox{N}{1in}{a phone number}\LA{rr}{has}&&\obox{C}{.8in}{an area code}\ar@{}[dll]|{\checkmark}\LA{d}{corresponds to}\\\obox{OLP}{1.2in}{an operational landline phone}\LA{ru}{is assigned}\LAL{r}{is}&\obox{P}{1in}{a physical phone}\LAL{rr}{\parbox{.55in}{\scriptsize is currently located in}}&&\obox{R}{.5in}{a region}}
\end{align} 
これは, 固定電話は電話番号が割り当てられた地域に物理的に設置されていることを言っている. \ref{dia:Englishification}の公式を使って, この事実を英語に翻訳せよ.
\end{exercise}

\begin{exercise}
%In the above olog (\ref{dia:phone paths}), suppose that the box \fakebox{an operational landline phone} is replaced with the box \fakebox{an operational mobile phone}. Would the diagram still commute?
上記のolog\eqref{dia:phone paths}において, \fakebox{an operational landline phone}を\fakebox{an operational mobile phone}で置換したと仮定する. 図式はそれでも可換だろうか?
\end{exercise}

%% Subsubsection %%

\subsubsection{Images}\label{sec:images}\index{olog!images}\index{image!in olog}

%In this section we discuss a specific kind of fact, generated by any aspect. Recall that every function has an image, meaning the subset of elements in the codomain that are ``hit" by the function. For example the function $f(x)=2*x\taking \ZZ\to\ZZ$ has as image the set of all even numbers.

この章では我々は任意の様相から生成される特殊な種類の事実について議論する. 全ての函数は像, すなわち函数が``命中''する, 終域の要素の部分集合を持つことを思いだそう. 例えば函数$f(x)=2*x\taking \ZZ\to\ZZ$は全偶数の集合を像として持つ.

%Similarly the set of mothers arises as is the image of the ``has as mother" function, as shown below 
%$$
%\xymatrix{\obox{P}{.5in}{a person}\LAL{rd}{has}\LA{rr}{$\stackrel{f\taking P\to P}{\tn{has as mother}}$}&&\obox{P}{.5in}{a person}\\
%&\obox{M=\im(f)}{.6in}{a mother}\LAL{ur}{is}\ar@{}[u]|(.6){\checkmark}
%}$$

同様に母(mother)からなる集合は, 函数``has as mother''の像として, 以下に示されるようにして生起する.
$$
\xymatrix{\obox{P}{.5in}{a person}\LAL{rd}{has}\LA{rr}{$\stackrel{f\taking P\to P}{\tn{has as mother}}$}&&\obox{P}{.5in}{a person}\\
&\obox{M=\im(f)}{.6in}{a mother}\LAL{ur}{is}\ar@{}[u]|(.6){\checkmark}
}$$

\begin{exercise}
%For each of the following types, write down a function for which it is the image, or say ``not clearly an image type" 
%\sexc \fakebox{a book}
%\next \fakebox{a material that has been fabricated by a process of type $T$}
%\next \fakebox{a bicycle owner}
%\next \fakebox{a child}
%\next \fakebox{a used book}
%\next \fakebox{an inhabited residence}
%\endsexc
以下の型それぞれについて, それぞれが像となる函数を書き下すか, あるいは``はっきりとは像としての型にならない''ことを述べよ.
\sexc \fakebox{a book}
\next \fakebox{a material that has been fabricated by a process of type $T$}
\next \fakebox{a bicycle owner}
\next \fakebox{a child}
\next \fakebox{a used book}
\next \fakebox{an inhabited residence}
\endsexc
\end{exercise}


%%%%%% Section %%%%%%

%\section{Products and coproducts}\label{sec:prods and coprods in set}
\section{積と余積}\label{sec:prods and coprods in set}

%In this section we introduce two concepts that are likely to be familiar, although perhaps not by their category-theoretic names, product and coproduct. Each is an example of a large class of ideas that exist far beyond the realm of sets.

この章で我々は, 圏論的な名前はそうでもないかもしれないが, おそらくは馴染み深いであろう二つの概念, 積と余積を導入する. それぞれは集合の領域を超えて存在しているアイデアの大きな種類の一つの例となっている.

%%%% Subsection %%%%

%\subsection{Products}\label{sec:products}\index{products!of sets}
\subsection{積}\label{sec:products}\index{products!of sets}

\begin{definition}

%Let $X$ and $Y$ be sets. The {\em product of $X$ and $Y$}, denoted $X\times Y$,\index{a symbol!$\times$} is defined as the set of ordered pairs $(x,y)$ where $x\in X$ and $y\in Y$. Symbolically, $$X\times Y=\{(x,y)\|x\in X,\;\; y\in Y\}.$$ There are two natural {\em projection functions} $\pi_1\taking X\times Y\to X$ and $\pi_2\taking X\times Y\to Y$.\index{projection functions}\index{product!projection functions}
$$\xymatrix@=15pt{&X\times Y\ar[ddr]^{\pi_2}\ar[ddl]_{\pi_1}\\\\X&&Y}$$

$X$と$Y$を集合とする. \emph{$X$と$Y$の積あるいは直積(product of $X$ and $Y$)}を, 順序付きの対の集合$(x,y)$と定義し, $X\times Y$\index{a symbol!$\times$}と書く. ここで$x\in X$および$y\in Y$である. 記号的に書くと$$X\times Y=\{(x,y)\|x\in X,\;\; y\in Y\}$$である. 二つの自然な\emph{射影函数(projection functions)} $\pi_1\taking X\times Y\to X$と$\pi_2\taking X\times Y\to Y$が存在する. \index{projection functions}\index{product!projection functions}
$$\xymatrix@=15pt{&X\times Y\ar[ddr]^{\pi_2}\ar[ddl]_{\pi_1}\\\\X&&Y}$$

\end{definition}

\begin{example}\label{ex:grid1}[Grid of dots]\index{product!as grid}

%Let $X=\{1,2,3,4,5,6\}$ and $Y=\{\clubsuit,\diamondsuit,\heartsuit,\spadesuit\}$. Then we can draw $X\times Y$ as a 6-by-4 grid of dots, and the projections as projections
$X=\{1,2,3,4,5,6\}$および$Y=\{\clubsuit,\diamondsuit,\heartsuit,\spadesuit\}$とする. $X\times Y$は$6\times4$の点の配列として書くことができ, 射影函数は以下の射影となる.
\begin{align}
\parbox{2.9in}{\begin{center}\small $X\times Y$\vspace{-.1in}\end{center}\fbox{
\xymatrix@=10pt{
\LMO{(1,\clubsuit)}&\LMO{(2,\clubsuit)}&\LMO{(3,\clubsuit)}&\LMO{(4,\clubsuit)}&\LMO{(5,\clubsuit)}&\LMO{(6,\clubsuit)}\\
\LMO{(1,\diamondsuit)}&\LMO{(2,\diamondsuit)}&\LMO{(3,\diamondsuit)}&\LMO{(4,\diamondsuit)}&\LMO{(5,\diamondsuit)}&\LMO{(6,\diamondsuit)}\\
\LMO{(1,\heartsuit)}&\LMO{(2,\heartsuit)}&\LMO{(3,\heartsuit)}&\LMO{(4,\heartsuit)}&\LMO{(5,\heartsuit)}&\LMO{(6,\heartsuit)}\\
\LMO{(1,\spadesuit)}&\LMO{(2,\spadesuit)}&\LMO{(3,\spadesuit)}&\LMO{(4,\spadesuit)}&\LMO{(5,\spadesuit)}&\LMO{(6,\spadesuit)}\\
}}}
\parbox{.9in}{
\xymatrix{~\ar[rr]^{\pi_2}&&~}
}
\parbox{.3in}{\begin{center}\small $Y$\vspace{-.1in}\end{center}\fbox{
\xymatrix@=10pt{
\LMO{\clubsuit}\\\LMO{\diamondsuit}\\\LMO{\heartsuit}\\\LMO{\spadesuit}
}}}
\\\nonumber
\parbox{1in}{\hspace{-1.95in}\xymatrix{~\ar[dd]_{\pi_1}\\\\~}}
\\\nonumber
\parbox{2.9in}{\hspace{-1.2in}\fbox{
\xymatrix@=24pt{
\LMO{1}&\LMO{2}&\LMO{3}&\LMO{4}&\LMO{5}&\LMO{6}
}}\begin{center}\hspace{-2.6in}\small$X$\end{center}}
\end{align}

\end{example}

\begin{application}
%A traditional (Mendelian) way to predict the genotype of offspring based on the genotype of its parents is by the use of \href{http://en.wikipedia.org/wiki/Punnett_square}{Punnett squares}. If $F$ is the set of possible genotypes for the female parent and $M$ is the set of possible genotypes of the male parent, then $F\times M$ is drawn as a square, called a Punnett square, in which every combination is drawn. 
親の遺伝子型からその子の遺伝子型を予言するための伝統的な(Mendel的な)方法は, \href{http://en.wikipedia.org/wiki/Punnett_square}{パンネットの方形(Punnett squares)}によるものである. もし$F$が母方の遺伝子型が取りうる集合であり$M$が父方の遺伝子型が取りうる集合であるならば, $F\times M$はパンネットの方形(Punnett squares)と呼ばれる四角形として書くことができ, これによって全ての組み合わせを書くことができる.
\end{application}

\begin{exercise}
%How many elements does the set $\{a,b,c,d\}\times\{1,2,3\}$ have?
集合$\{a,b,c,d\}\times\{1,2,3\}$の要素の数は幾つだろうか?
\end{exercise}

\begin{application}

%Suppose we are conducting experiments about the mechanical properties of materials, as in Application \ref{app:force-extension}. For each material sample we will produce multiple data points in the set $\fakebox{extension}\times\fakebox{force}\iso\RR\times\RR$.

Application \ref{app:force-extension}のような, 物質の力学特性を測定する実験を行なっていると仮定しよう. それぞれの物質サンプルに対して, 集合$\fakebox{extension}\times\fakebox{force}\iso\RR\times\RR$の中のデータ点が複数得られる.

\end{application}

\begin{remark}

%It is possible to take the product of more than two sets as well. For example, if $A,B,$ and $C$ are sets then $A\times B\times C$ is the set of triples, 
%$$A\times B\times C:=\{(a,b,c)\|a\in A, b\in B, c\in C\}.$$

二つ以上の集合の積をとることも可能である. 例えば, $A$, $B$, および$C$を集合としたとき, 
$A\times B\times C$ は三つ組の集合 
$$A\times B\times C:=\{(a,b,c)\|a\in A, b\in B, c\in C\}$$
である.

%This kind of generality is useful in understanding multiple dimensions, e.g. what physicists mean by 10-dimensional space. It comes under the heading of {\em limits}, which we will see in Section \ref{sec:lims and colims in a cat}.

この種の一般性は多次元性, e.g. 物理学者が10次元空間という言葉で意味するところのもの, を理解するのに便利である. これは\ref{sec:lims and colims in a cat}章で見る\emph{極限(limits)}の項目で表れることになる.

\end{remark}

\begin{example}\label{ex:R2}

%Let $\RR$\index{a symbol!$\RR$} be the set of real numbers. By $\RR^2$ we mean $\RR\times\RR$ (though see Exercise \ref{exc:two R2s}). Similarly, for any $n\in\NN$, we define $\RR^n$ to be the product of $n$ copies of $\RR$. 

$\RR$\index{a symbol!$\RR$}を実数の集合とする. $\RR^2$によって, $\RR\times\RR$を意味するとする(Exercise \ref{exc:two R2s}も見よ). 同様に, 任意の$n\in\NN$に対して$\RR^n$を$n$個容易した$\RR$の積であると定義する.

%According to \cite{Pen}, Aristotle seems to have conceived of space as something like $S:=\RR^3$ and of time as something like $T:=\RR$. Spacetime, had he conceived of it, would probably have been $S\times T\iso\RR^4$. He of course did not have access to this kind of abstraction, which was probably due to Descartes. 

\cite{Pen}によると, アリストテレスは空間を$S:=\RR^3$のようなもの, また時間を$T:=\RR$のようなものだと理解していたようである. アリストテレスが理解していたところの時空は, おそらく$S\times T\iso\RR^4$であったであろう. もちろんアリストテレスはこの種の抽象化を利用する手段を持たなかった. この抽象化はおそらくデカルトに帰されるものである.

\end{example}

\begin{exercise}
%Let $\ZZ$ denote the set of integers, and let $+\taking\ZZ\times\ZZ\to\ZZ$ denote the addition function and $\cdot\taking\ZZ\times\ZZ\to\ZZ$ denote the multiplication function. Which of the following diagrams commute?

$\ZZ$が整数の集合を表すとし, $+\taking\ZZ\times\ZZ\to\ZZ$は足し算の函数を, $\cdot\taking\ZZ\times\ZZ\to\ZZ$をかけ算の函数を表すとする. 以下の図式で可換なものはどれか?
\sexc $$\xymatrix{
\ZZ\times\ZZ\times\ZZ\ar[rr]^-{(a,b,c)\mapsto(a\cdot b,a\cdot c)}\ar[d]_{(a,b,c)\mapsto(a+b,c)}&\hsp&\ZZ\times\ZZ\ar[d]^{(x,y)\mapsto x+y}\\
\ZZ\times\ZZ\ar[rr]_{(x,y)\mapsto xy}&&\ZZ}
$$
\next $$
\xymatrix{
\ZZ\ar[rr]^{x\mapsto (x,0)}\ar[drr]_{\id_\ZZ}&&\ZZ\times\ZZ\ar[d]^{(a,b)\mapsto a\cdot b}\\&&\ZZ}
$$
\next$$
\xymatrix{
\ZZ\ar[rr]^{x\mapsto (x,1)}\ar[drr]_{\id_\ZZ}&&\ZZ\times\ZZ\ar[d]^{(a,b)\mapsto a\cdot b}\\&&\ZZ}
$$
\endsexc
\end{exercise}

%% Subsubsection %%

%\subsubsection{Universal property for products}\index{products!universal property of}\index{universal property!products}

\subsubsection{積における普遍的性質}\index{products!universal property of}\index{universal property!products}

%\begin{lemma}[Universal property for product]\label{lemma:up for prod}

\begin{lemma}[積における普遍的性質%
\footnote{訳注: 圏論の文脈では``\emph{universality property}''は多くの``普遍性''と訳される. 一方で自然科学の他の分野では普遍性は``universality''の訳として用いられることも多いため, ここでは普遍的性質と訳すことにした.}%
]\label{lemma:up for prod}

%Let $X$ and $Y$ be sets. For any set $A$ and functions $f\taking A\to X$ and $g\taking A\to Y$, there exists a unique function $A\to X\times Y$ such that the following diagram commutes \footnote{The symbol $\forall$ is read ``for all"; the symbol $\exists$ is read ``there exists", and the symbol $\exists!$ is read ``there exists a unique". So this diagram is intended to express the idea that for any functions $f\taking A\to X$ and $g\taking A\to Y$, there exists a unique function $A\to X\times Y$ for which the two triangles commute.}
%\begin{align}\label{dia:univ prop for products}
%\xymatrix@=15pt{&X\times Y\ar[ldd]_{\pi_1}\ar[rdd]^{\pi_2}\\\\X\ar@{}[r]|{\checkmark}&&Y\ar@{}[l]|{\checkmark}\\\\&A\ar[luu]^{\forall f}\ar[ruu]_{\forall g}\ar@{-->}[uuuu]^{\exists !}}
%\end{align}
%We might write the unique function as $$\prodmap{f}{g}\taking A\to X\times Y.$$

$X$と$Y$を集合とする. 任意の集合$A$と函数$f\taking A\to X$および$g\taking A\to Y$に対して, 以下の図式が可換となる函数$A\to X\times Y$が一意に存在する. \footnote{記号$\forall$は``for all'' (``全ての$\dots$に対して''), 記号$\exists$は``there exists'' (``存在する''), 記号$\exists!$は''there exists a unique'' (``一意に存在する'', あるいは``唯一つだけ存在する'')と読む. よってこの図式は任意の函数$f\taking A\to X$と$g\taking A\to Y$に対して, 二つの三角形が可換となる函数$A\to X\times Y$が一意に存在するというアイデアを表現している.}
\begin{align}\label{dia:univ prop for products}
\xymatrix@=15pt{&X\times Y\ar[ldd]_{\pi_1}\ar[rdd]^{\pi_2}\\\\X\ar@{}[r]|{\checkmark}&&Y\ar@{}[l]|{\checkmark}\\\\&A\ar[luu]^{\forall f}\ar[ruu]_{\forall g}\ar@{-->}[uuuu]^{\exists !}}
\end{align}
我々はこの一意に存在する函数を$$\prodmap{f}{g}\taking A\to X\times Y.$$と書くことがある.

\end{lemma}

\begin{proof}

%Suppose given $f,g$ as above. To provide a function $\ell\taking A\to X\times Y$ is equivalent to providing an element $\ell(a)\in X\times Y$ for each $a\in A$. We need such a function for which $\pi_1\circ \ell=f$ and $\pi_2\circ \ell=g$. An element of $X\times Y$ is an ordered pair $(x,y)$, and we can use $\ell(a)=(x,y)$ if and only if $x=\pi_1(x,y)=f(a)$ and $y=\pi_2(x,y)=g(a)$. So it is necessary and sufficient to define $$\prodmap{f}{g}(a):=(f(a),g(a))$$ for all $a\in A$.

$f$, $g$が上のように与えられたと仮定する. 函数$\ell\taking A\to X\times Y$を与えるということは, それぞれの$a\in A$に対して値$\ell(a)\in X\times Y$を与えることに等しい. 我々は$\pi_1\circ \ell=f$および$\pi_2\circ \ell=g$となる函数を必要としている. $X\times Y$の要素は順序付けられた対$(x,y)$であり, よって我々は, $x=\pi_1(x,y)=f(a)$および$y=\pi_2(x,y)=g(a)$の時かつその時に限り$\ell(a)=(x,y)$を使うことができる. よって全ての$a\in A$に対して$$\prodmap{f}{g}(a):=(f(a),g(a))$$を定義すれば必要かつ十分である.

\end{proof}

\begin{example}[Grid of dots, continued]\label{ex:grid2}

%We need to see the universal property of products as completely intuitive. Recall that if $X$ and $Y$ are sets, say of cardinalities $|X|=m$ and $|Y|=n$ respectively, then $X\times Y$ is an $m\times n$ grid of dots, and it comes with two canonical projections $X\From{\pi_1}X\times Y\To{\pi_2}Y$. These allow us to extract from every grid element $z\in X\times Y$ its column $\pi_1(z)\in X$ and its row $\pi_2(z)\in Y$.

我々は, 積の普遍的性質を完全に直感的に理解できるようになる必要がある. $X$と$Y$が集合であることを思い出し, 濃度をそれぞれ$|X|=m$および$|Y|=n$とすれば, $X\times Y$は$m\times n$の点の格子となり, そして標準射影$X\From{\pi_1}X\times Y\To{\pi_2}Y$が付随する. これらによって, 全ての要素$z\in X\times Y$からその列$\pi_1(z)\in X$および行$\pi_2(z)\in Y$を取り出すことができる.

%Suppose that each person in a classroom picks an element of $X$ and an element of $Y$. Thus we have functions $f\taking C\to X$ and $g\taking C\to Y$. But isn't picking a column and a row the same thing as picking an element in the grid? The two functions $f$ and $g$ induce a unique function $C\to X\times Y$. And how does this function $C\to X\times Y$ compare with the original functions $f$ and $g$? The commutative diagram (\ref{dia:univ prop for products}) sums up the obvious connection. 

教室にいる人間それぞれが$X$の要素と$Y$の要素を選び出すと仮定する. ゆえに我々は函数$f\taking C\to X$と$g\taking C\to Y$を得た. しかし列および行を選び出すのは格子の要素を選ぶのと同じことではないのではなかろうか? 二つの函数$f$および$g$は一意な函数$C\to X\times Y$を誘導する. この函数$C\to X\times Y$は元の函数$f$および$g$と比較するとどうなるだろうか? 可換図式\eqref{dia:univ prop for products}は, その明白なる関係性を要約している.

\end{example}

\begin{example}

%Let $\RR$ be the set of real numbers. The origin in $\RR$ is an element of $\RR$. As you showed in Exercise \ref{exc:generator for set}, we can view this (or any) element of $\RR$ as a function $z\taking\singleton\to\RR$, where $\singleton$ is any set with one element. Our function $z$ ``picks out the origin". Thus we can draw functions 
%$$\xymatrix@=15pt{&\singleton\ar[ddr]^z\ar[ddl]_z\\\\\RR&&\RR}
%$$
%The universal property for products guarantees a function $\singleton\to\RR\times\RR$, which will be the origin in $\RR^2.$

$\RR$を実数の集合とする. $\RR$の起源(origin)は$\RR$の要素である. Exercise \ref{exc:generator for set}で示したように, $\RR$の中のこの(あるいは任意の)要素は$z\taking\singleton\to\RR$と見ることができる. ここで$\singleton$は任意の要素が一つの集合である. 函数$z$は``起源を選び出す''. よって
$$\xymatrix@=15pt{&\singleton\ar[ddr]^z\ar[ddl]_z\\\\\RR&&\RR}
$$
などの函数を書ける. 積の普遍的性質は函数$\singleton\to\RR\times\RR$を保証する. これは$\RR^2$の起源となるだろう.

\end{example}

\begin{remark}

%Given sets $X, Y,$ and $A$, and functions $f\taking A\to X$ and $g\taking A\to Y$, there is a unique function $A\to X\times Y$ that commutes with $f$ and $g$. We call it {\em the induced function $A\to X\times Y$},\index{induced function} meaning the one that arises in light of $f$ and $g$.

集合$X$, $Y$, $A$, 函数$f\taking A\to X$, $g\taking A\to Y$が与えられたとき, $f$と$g$に可換な函数$A\to X\times Y$が一意に存在する. 我々はこの函数を\emph{誘導函数(the induced function) $A\to X\times Y$},\index{induced function}と呼ぶことにする. これが指すものは$f$と$g$を考慮したときに浮かび上がるそれである.

\end{remark}

\begin{exercise}
%For every set $A$ there is some nice relationship between the following three sets: $$\Hom_{\Set}(A,X), \hsp \Hom_\Set(A,Y), \hsp \text{and} \hsp\Hom_\Set(A,X\times Y).$$ What is it?

全ての集合$A$に対して次の三つの集合の間には素敵な関係がある. それは何か?
$$\Hom_{\Set}(A,X), \hsp \Hom_\Set(A,Y), \hsp \text{and} \hsp\Hom_\Set(A,X\times Y).$$

%Hint: Do not be alarmed: this problem is a bit ``recursive" in that you'll use products in your formula.
ヒント: 慌てないこと. この問題は, 式の中で積を使うということを少しだけ繰り返している.

\end{exercise}

\begin{exercise}~
%\sexc Let $X$ and $Y$ be sets. Construct the ``swap map" $s\taking X\times Y\to Y\times X$ using only the universal property for products. If $\pi_1\taking X\times Y\to X$ and $\pi_2\taking X\times Y\to Y$ are the projection functions, write $s$ in terms of the symbols $``\pi_1",``\pi_2", ``(\ ,\ )",$ and $``\circ"$. 
%\next Can you prove that $s$ is a isomorphism using only the universal property for product?
%\endsexc
\sexc $X$と$Y$を集合とする. ``swap map'' $s\taking X\times Y\to Y\times X$を, 積の普遍的性質のみを用いて構成せよ. $\pi_1\taking X\times Y\to X$および$\pi_2\taking X\times Y\to Y$が射影函数であるとして, $s$を記号``$\pi_1$'', ``$\pi_2$'', ``$(\ ,\ )$'', および``$\circ$''を用いて書け. 
\next $s$が同型写像であることを, 積の普遍的性質のみを用いて証明できるだろうか?
\endsexc
\end{exercise}

\begin{example}\label{ex:product to product}
%Suppose given sets $X,X', Y, Y'$ and functions $m\taking X\to X'$ and $n\taking Y\to Y'$. We can use the universal property of products to construct a function $s\taking X\times Y\to X'\times Y'$.  Here's how.

集合$X$, $X'$, $Y$, $Y'$, 函数$m\taking X\to X'$, $n\taking Y\to Y'$が与えられたと仮定する. 我々は積の普遍的性質を用いて函数$s\taking X\times Y\to X'\times Y'$を構成することができる. 以下にその方法を示す.

%The universal property (Lemma \ref{lemma:up for prod}) says that to get a function from any set $A$ to $X'\times Y'$, we need two functions, namely some $f\taking A\to X'$ and some $g\taking A\to Y'$. Here $A=X\times Y$. 

普遍的性質(Lemma \ref{lemma:up for prod})により, 任意の集合$A$から$X'\times Y'$への函数を得るためには. 我々は二種の函数, すなわちなんらかの$f\taking A\to X'$となんらかの$g\taking A\to Y'$を必要とする. ここで$A=X\times Y$である.

%What we have readily available are the two projections $\pi_1\taking X\times Y\to X$ and $\pi_2\taking X\times Y\to Y$. But we also have $m\taking X\to X'$ and $n\taking Y\to Y'$. Composing, we set $f:=m\circ \pi_1$ and $g:=n\circ\pi_2$.
%$$\xymatrix{
%&X'\times Y'\ar[dl]_{\pi_1'}\ar[dr]^{\pi_2'}\\
%X'&&Y'\\
%X\ar[u]^m&&Y\ar[u]_n\\
%&X\times Y\ar[ul]^{\pi_1}\ar[ur]_{\pi_2}\ar@{-->}[uuu]
%}
%$$
%The dotted arrow is often called the {\em product} of $m\taking X\to X'$ and $n\taking Y\to Y'$ and is denoted simply by 
%$$m\times n\taking X\times Y\to X'\times Y'.$$

我々が使用可能なのは二つの射影函数$\pi_1\taking X\times Y\to X$と$\pi_2\taking X\times Y\to Y$である. しかし我々は$m\taking X\to X'$と$n\taking Y\to Y'$も利用できる. 合成して, $f:=m\circ \pi_1$および$g:=n\circ\pi_2$と設定する.
$$\xymatrix{
&X'\times Y'\ar[dl]_{\pi_1'}\ar[dr]^{\pi_2'}\\
X'&&Y'\\
X\ar[u]^m&&Y\ar[u]_n\\
&X\times Y\ar[ul]^{\pi_1}\ar[ur]_{\pi_2}\ar@{-->}[uuu]
}
$$
点線の矢印はしばしば$m\taking X\to X'$と$n\taking Y\to Y'$の\emph{積(product)}と呼ばれ, 簡潔に
$$m\times n\taking X\times Y\to X'\times Y'.$$
と書かれる.

\end{example}

%% Subsubsection %%

%\subsubsection{Ologging products}\label{sec:ologging products}
\subsubsection{Ologと積}\label{sec:ologging products}

%Given two objects $c,d$ in an olog, there is a canonical label $\qt{c\times d}$ for their product $c\times d$, written in terms of the labels $\qt{c}$ and $\qt{d}$. Namely, $$\qt{c\times d}:=\tn{a pair }(x,y)\tn{ where }x\tn{ is }\qt{c}\tn{ and }y\tn{ is }\qt{d}.$$ The projections $c\from c\times d\to d$ can be labeled ``yields, as $x$," and ``yields, as $y$," respectively.

olog内の対象$c$, $d$が与えられたとき, その積$c\times d$に対して, 見出し$\qt{c}$および$\qt{d}$によって書かれる標準見出し(canonical label)$\qt{c\times d}$が存在する. すなわち$$\qt{c\times d}:=\tn{a pair }(x,y)\tn{ where }x\tn{ is }\qt{c}\tn{ and }y\tn{ is }\qt{d}.$$ 射影$c\from c\times d\to d$はそれぞれ``yields, as $x$''および``yields, as $y$''と見出し付けられる.

%Suppose that $e$ is another object and $p\taking e\to c$ and $q\taking e\to d$ are two arrows. By the universal property of products (Lemma \ref{lemma:up for prod}), $p$ and $q$ induce a unique arrow $e\to c\times d$ making the evident diagrams commute. This arrow can be labeled
%\begin{center}
%yields, insofar as it $\qt{p}\;\qt{c}$ and $\qt{q}\;\qt{d}$, 
%\end{center}

$e$が別の対象であり$p\taking e\to c$と$q\taking e\to d$が二本の矢印であると仮定する. 積の普遍的性質(Lemma \ref{lemma:up for prod})により, $p$と$q$は自明に得られる図式を可換にする矢印$e\to c\times d$を一意に誘導する. この矢印は以下のように見出し付けられる.
\begin{center}
yields, insofar as it $\qt{p}\;\qt{c}$ and $\qt{q}\;\qt{d}$, 
\end{center}

\begin{example}

%Every car owner owns at least one car, but there is no obvious function $\fakebox{a car owner}\to\fakebox{a car}$ because he or she may own more than one. One good choice would be the car that the person drives most often, which we'll call his or her primary car. Also, given a person and a car, an economist could ask how much utility the person would get out of the car. From all this we can put together the following olog involving products:
%$$
%\xymatrixnocompile{\obox{O}{.7in}{a car owner}\LAL{dd}{is}\ar[ddrr]_(.35){\parbox{.45in}{\scriptsize owns, as primary,}}\LA{rr}{\parbox{.8in}{\rr\scriptsize yields, insofar as it is a person and owns, as primary, a car,}}&\ar@{}[d]^(.4){\checkmark}&
%\obox{P\times C}{1in}{a pair $(x,y)$ where $x$ is a person and $y$ is a car}\ar@/^1pc/[ddll]^(.7){\tn{yields, as }x,}\LA{dd}{\tn{yields, as }y,}\LA{rr}{\parbox{.7in}{\scriptsize has as associated utility}}&&\obox{V}{.8in}{a dollar value}\\&&\\\obox{P}{.5in}{a person}&&\obox{C}{.4in}{a car}
%}
%$$

全ての自動車保有者は最低一台の自動車を保有しているが, 自明な函数$\fakebox{a car owner}\to\fakebox{a car}$は存在しない. なぜならばその人は二台以上保有しているかもしれないからだ. 一つのよい自動車の選択は, その人が最もよく運転している自動車, 我々がその人の主要な自動車と呼ぶところの自動車であろう. また, 人物と自動車が与えられたとき, その人がその車からどれだけの利便性が得られたかと, 経済学者は問うかもしれない. これら全てから, われわれは積にかかわる以下のologをまとめあげることができる.
$$
\xymatrixnocompile{\obox{O}{.7in}{a car owner}\LAL{dd}{is}\ar[ddrr]_(.35){\parbox{.45in}{\scriptsize owns, as primary,}}\LA{rr}{\parbox{.8in}{\rr\scriptsize yields, insofar as it is a person and owns, as primary, a car,}}&\ar@{}[d]^(.4){\checkmark}&
\obox{P\times C}{1in}{a pair $(x,y)$ where $x$ is a person and $y$ is a car}\ar@/^1pc/[ddll]^(.7){\tn{yields, as }x,}\LA{dd}{\tn{yields, as }y,}\LA{rr}{\parbox{.7in}{\scriptsize has as associated utility}}&&\obox{V}{.8in}{a dollar value}\\&&\\\obox{P}{.5in}{a person}&&\obox{C}{.4in}{a car}
}
$$

\end{example}

%%%% Subsection %%%%

%\subsection{Coproducts}\label{sec:coproducts}\index{coproducts!of sets}
\subsection{余積}\label{sec:coproducts}\index{coproducts!of sets}

\begin{definition}\label{def:coproduct}

%Let $X$ and $Y$ be sets. The {\em coproduct of $X$ and $Y$}, denoted $X\sqcup Y$,\index{a symbol!$\sqcup$} is defined as the ``disjoint union" of $X$ and $Y$, i.e. the set for which an element is either an element of $X$ or an element of $Y$. If something is an element of both $X$ and $Y$ then we include both copies, and distinguish between them, in $X\sqcup Y$. See Example \ref{ex:coproduct}

$X$と$Y$を集合とする. \emph{$X$と$Y$の余積(coproduct of $X$ and $Y$)}を, $X$と$Y$の``直和(disjoint union)''あるいは非交和, i.e. その要素が$X$あるいは$Y$のどちらかの要素である集合と定義し, $X\sqcup Y$,\index{a symbol!$\sqcup$}と書く. もしある要素が$X$と$Y$の両方の要素であるならば, 我々はその両方ともを, お互いを区別できるようにして, $X\sqcup Y$に含めることにする. Example \ref{ex:coproduct}を見よ.

%There are two natural inclusion functions $i_1\taking X\to X\sqcup Y$ and $i_2\taking Y\to X\sqcup Y$.\index{inclusion functions}\index{coproduct!inclusion functions}
%$$\xymatrix@=15pt{X\ar[ddr]_{i_1}&&Y\ar[ddl]^{i_2}\\\\&X\sqcup Y}$$

二つの包含函数(inclusion function) $i_1\taking X\to X\sqcup Y$および$i_2\taking Y\to X\sqcup Y$.\index{inclusion functions}\index{coproduct!inclusion functions}が存在する.
$$\xymatrix@=15pt{X\ar[ddr]_{i_1}&&Y\ar[ddl]^{i_2}\\\\&X\sqcup Y}$$

\end{definition}

\begin{example}\label{ex:coproduct}

%The coproduct of $X:=\{a,b,c,d\}$ and $Y:=\{1,2,3\}$ is $$X\sqcup Y\iso\{a,b,c,d,1,2,3\}.$$ The coproduct of $X$ and itself is $$X\sqcup X\iso\{i_1a,i_1b,i_1c,i_1d,i_2a,i_2b,i_2c,i_2d\}$$ 
%The names of the elements in $X\sqcup Y$ are not so important. What's important are the inclusion maps $i_1,i_2$, which ensure that we know where each element of $X\sqcup Y$ came from.

$X:=\{a,b,c,d\}$と$Y:=\{1,2,3\}$の余積は$$X\sqcup Y\iso\{a,b,c,d,1,2,3\}.$$である. $X$とそれ自身の余積は$$X\sqcup X\iso\{i_1a,i_1b,i_1c,i_1d,i_2a,i_2b,i_2c,i_2d\}$$である.
$X\sqcup Y$の要素の名前はそれほど重要ではない. 重要なことは, $X\sqcup Y$のそれぞれの要素がどこから来たかを知ることができることを保証する, 包含写像$i_1$, $i_2$の存在である.

\end{example}

\begin{example}[Airplane seats\footnote{訳注: ビジネスクラス等の中間的な上級客席は無視するとする.}]\label{ex:airplanes}

\begin{align}\label{dia:airplane}
\xymatrix@=15pt{
\obox{X}{.8in}{an economy-class seat in an airplane}\LAL{ddr}{is}&&\obox{Y}{.7in}{a first-class seat in an airplane}\LA{ddl}{is}\\\\
&\obox{X\sqcup Y}{.7in}{a seat in an airplane}
}
\end{align}

\end{example}

\begin{exercise}
%Would you say that \fakebox{a phone} is the coproduct of \fakebox{a cellphone} and \fakebox{a landline phone}? 
\fakebox{a phone} (電話)は\fakebox{a cellphone} (携帯電話)と\fakebox{a landline phone} (固定電話)の余積と言えるだろうか?
\end{exercise}

\begin{example}[Disjoint union of dots]\label{ex:coprod of dots}

\begin{align}
\parbox{2.4in}{\begin{center}\small $X\sqcup Y$\vspace{-.1in}\end{center}\fbox{
\xymatrix@=15pt{
\LMO{\clubsuit}&\LMO{1}&\LMO{2}&\LMO{3}&\LMO{4}&\LMO{5}&\LMO{6}\\\LMO{\diamondsuit}\\\LMO{\heartsuit}\\\LMO{\spadesuit}
}}}
\parbox{.9in}{
\xymatrix{~&&\ar[ll]_{i_2}~}
}
\parbox{.3in}{\begin{center}\small $Y$\vspace{-.1in}\end{center}\fbox{
\xymatrix@=15pt{
\LMO{\clubsuit}\\\LMO{\diamondsuit}\\\LMO{\heartsuit}\\\LMO{\spadesuit}
}}}
\\\nonumber
\parbox{1in}{\hspace{-1.4in}\xymatrix{~\\\\\ar[uu]_{i_1}}}
\\\nonumber
\parbox{2.1in}{\hspace{-1.3in}\fbox{
\xymatrix@=15pt{
\LMO{1}&\LMO{2}&\LMO{3}&\LMO{4}&\LMO{5}&\LMO{6}
}}\begin{center}\hspace{-2.6in}\small$X$\end{center}}
\end{align}

\end{example}

%% Subsubsection %%

%\subsubsection{Universal property for coproducts}\index{coproducts!universal property of}
\subsubsection{余積の普遍的性質}\index{coproducts!universal property of}

\begin{lemma}[Universal property for coproduct]\label{lemma:up for coprod}

%Let $X$ and $Y$ be sets. For any set $A$ and functions $f\taking X\to A$ and $g\taking Y\to A$, there exists a unique function $X\sqcup Y\to A$ such that the following diagram commutes
%$$
%\xymatrix@=15pt{&A\\\\X\ar[uur]^{\forall f}\ar[ddr]_{i_1}&&Y\ar[uul]_{\forall g}\ar[ddl]^{i_2}\\\\&X\sqcup Y\ar@{-->}[uuuu]^{\exists!}}
%$$
%We might write the unique function as 
%\footnote{We are about to use a two-line symbol, which is a bit unusual. In what follows a certain function $X\sqcup Y\to A$ is being denoted by the symbol $\coprodmap{f}{g}$.}
%$$\coprodmap{f}{g}\taking X\sqcup Y\to A.$$

$X$と$Y$を集合とする. 任意の集合$A$と函数$f\taking X\to A$, $g\taking Y\to A$に対して, 以下の図式を可換にするところの函数$X\sqcup Y\to A$が一意に存在する.
$$
\xymatrix@=15pt{&A\\\\X\ar[uur]^{\forall f}\ar[ddr]_{i_1}&&Y\ar[uul]_{\forall g}\ar[ddl]^{i_2}\\\\&X\sqcup Y\ar@{-->}[uuuu]^{\exists!}}
$$
我々はこの一意に存在する函数を
$$\coprodmap{f}{g}\taking X\sqcup Y\to A$$
と書くことがある\footnote{我々は二行にわたる記法を使用しようとしている. これは少しばかり通常ではない. これ以降ではある函数$X\sqcup Y\to A$は記法$\coprodmap{f}{g}$によって示されるとする.}.
\end{lemma}

\begin{proof}

%Suppose given $f,g$ as above. To provide a function $\ell\taking X\sqcup Y\to A$ is equivalent to providing an element $f(m)\in A$ is for each $m\in X\sqcup Y$. We need such a function such that $\ell\circ i_1=f$ and $\ell\circ i_2=g$. But each element $m\in X\sqcup Y$ is either of the form $i_1x$ or $i_2y$, and cannot be of both forms. So we assign 
%$$\coprodmap{f}{g}(m)=\begin{cases}f(x)&\tn{if } m=i_1x,\\ g(y) &\tn{if }m=i_2y.\end{cases}$$
%This assignment is necessary and sufficient to make all relevant diagrams commute.

$f$, $g$が上記のように与えられたとする. 函数$\ell\taking X\sqcup Y\to A$を与えるということは, それぞれの$m\in X\sqcup Y$に対して要素$f(m)\in A$を与えることに等しい. 我々は$\ell\circ i_1=f$および$\ell\circ i_2=g$となる函数を必要としている. しかしそろぞれの要素$m\in X\sqcup Y$は$i_1x$かあるいは$i_2y$のどちらかの形を取り, 両方の形を取ることはできない. よって我々は
$$\coprodmap{f}{g}(m)=\begin{cases}f(x)&\tn{if } m=i_1x,\\ g(y) &\tn{if }m=i_2y.\end{cases}$$
と割り当てる. この割り当ては全ての関連する図式を可換にするのに必要かつ十分である.

\end{proof}

\begin{example}[Airplane seats, continued]

%The universal property of coproducts says the following. Any time we have a function $X\to A$ and a function $Y\to A$, we get a unique function $X\sqcup Y\to A$. For example, every economy class seat in an airplane and every first class seat in an airplane is actually {\em in a particular airplane}. Every economy class seat has a price, as does every first class seat.
%\begin{align}
%\xymatrix{
%&\obox{A}{.9in}{a dollar figure}&\\
%\obox{X}{.8in}{an economy-class seat in an airplane}\LA{ru}{has as price}\LA{r}{is}\LAL{dr}{is in}&\obox{X\sqcup Y}{.7in}{a seat in an airplane}\ar@{-->}[d]_{\exists!}\ar@{-->}[u]^{\exists!}\ar@{}[ur]|(.35){\checkmark}\ar@{}[dl]|(.35){\checkmark}\ar@{}[dr]|(.35){\checkmark}\ar@{}[ul]|(.35){\checkmark}&\obox{Y}{.7in}{a first-class seat in an airplane}\LAL{l}{is}\LAL{lu}{has as price}\LA{dl}{is in}\\
%&\obox{B}{.7in}{an airplane}&
%}
%\end{align}
%The universal property of coproducts formalizes the following intuitively obvious fact:
%\begin{quote}
%If we know how economy class seats are priced and we know how first class seats are priced, and if we know that every seat is either economy class or first class, then we automatically know how all seats are priced.
%\end{quote}
%To say it another way (and using the other induced map):
%\begin{quote}
%If we keep track of which airplane every economy class seat is in and we keep track of which airplane every first class seat is in, and if we know that every seat is either economy class or first class, then we require no additional tracking for any airplane seat whatsoever.
%\end{quote}

余積の普遍的性質により, 以下が言える. 函数$X\to A$および$Y\to A$があるときは常に, 函数$X\sqcup Y\to A$が一意に得られる. 例えば, ある航空機の全てのエコノミークラスの座席と, ある航空機の全てのファーストクラスの座席が, 実際には\emph{同じ航空機の中にあるとする}. 全てのエコノミークラスの座席は料金が設定されているし, ファーストクラスもそうである.
\begin{align}
\xymatrix{
&\obox{A}{.9in}{a dollar figure}&\\
\obox{X}{.8in}{an economy-class seat in an airplane}\LA{ru}{has as price}\LA{r}{is}\LAL{dr}{is in}&\obox{X\sqcup Y}{.7in}{a seat in an airplane}\ar@{-->}[d]_{\exists!}\ar@{-->}[u]^{\exists!}\ar@{}[ur]|(.35){\checkmark}\ar@{}[dl]|(.35){\checkmark}\ar@{}[dr]|(.35){\checkmark}\ar@{}[ul]|(.35){\checkmark}&\obox{Y}{.7in}{a first-class seat in an airplane}\LAL{l}{is}\LAL{lu}{has as price}\LA{dl}{is in}\\
&\obox{B}{.7in}{an airplane}&
}
\end{align}
余積の普遍的性質は, 以下の直感的に自明な事実を形式化する.
\begin{quote}
もし我々が, エコノミークラスの料金がどのように設定されるかを知っており, またファーストクラスの料金がどのように設定されるかを知っており, そして全ての座席がエコノミークラスかファーストクラスかのどちらかに分類されることを知っているならば, 我々は全ての座席について料金がどのように設定されるかを自動的に知ることになる.
\end{quote}
違った言い方で(他の誘導写像を使って)表現してみよう.
\begin{quote}
もし我々が全てのエコノミークラスの座席についてその座席が存在しているところの航空機を把握しており, 全てのファーストクラスの座席についてその座席が存在しているところの航空機を把握しており, さらに全ての座席がエコノミークラスかファーストクラスのどちらかに分類されることを知っているならば, それ以上に把握する必要がある航空機の座席は一席もない.
\end{quote}

\end{example}

\begin{application}[Piecewise defined curves]

%In science, curves are often defined or considered piecewise. For example in testing the mechanical properties of a material, we might be interested in various regions of \href{http://en.wikipedia.org/wiki/Deformation_(engineering)}{deformation}, such as elastic, plastic, or post-fracture. These are three intervals on which the material displays different kinds of properties. 

科学において, 曲線はしばしばその部分部分の塊として考察されたり定義されたりする. 例えば, 物質の力学的性質を試験する際, 我々は物質が示す\href{http://en.wikipedia.org/wiki/Deformation_(engineering)}{変形(deformation)}の相, 例えば弾性的(elastic)であるか, 塑性的(plastic)であるか, あるいは破断後か(post-fracture)といったことに興味があるであろう.  これらは物質の性質が異なったふるまいを見せる3つの領域である.

%For real numbers $a<b\in\RR$, let $[a,b]:=\{x\in\RR\|a\leq x\leq b\}$ denote the closed interval. Given a function $[a,b]\to\RR$ and a function $[c,d]\to\RR$, the universal property of coproducts implies that they extend uniquely to a function $[a,b]\sqcup[c,d]\to\RR$, which will appear as a piecewise defined curve.

実数$a<b\in\RR$に対して$[a,b]:=\{x\in\RR\|a\leq x\leq b\}$は閉区間を示す. 函数$[a,b]\to\RR$と$[c,d]\to\RR$が与えられたとき, 余積の普遍的性質によって, これらの函数が函数$[a,b]\sqcup[c,d]\to\RR$に一意に拡張できることが示唆される. これらは曲線を定義する部分と見ることができるだろう.

%Often we are given a curve on $[a,b]$ and another on $[b,c]$, where the two curves agree at the point $b$. This situation is described by pushouts, which are mild generalizations of coproducts; see Section \ref{sec:pushouts}.

しばしば$[a,b]$上の曲線と別の$[b,c]$上の曲線が得られ, そしてその二つの曲線が点$b$で一致することがある. この状況は押し出し(pushout)によって表現される. これは余積のゆるやかな一般化である. \ref{sec:pushouts}章を見よ.

\end{application}

\begin{exercise}\label{exc:coprod}

%Write the universal property for coproduct in terms of a relationship between the following three sets: $$\Hom_{\Set}(X,A), \hsp \Hom_\Set(Y,A), \hsp \text{and} \hsp\Hom_\Set(X\sqcup Y,A).$$ 

以下の3つの集合の間の関係の観点で余積の普遍的性質を書き下せ. $$\Hom_{\Set}(X,A), \hsp \Hom_\Set(Y,A), \hsp \text{and} \hsp\Hom_\Set(X\sqcup Y,A).$$ 
\end{exercise}

\begin{example}\label{ex:coproduct1}

%In the following olog the types $A$ and $B$ are disjoint, so the coproduct $C=A\sqcup B$ is just the union. $$\fbox{\xymatrix{\smbox{A}{a person}\LA{r}{is}&\smbox{C=A\sqcup B}{a person or a cat}&\smbox{B}{a cat}\LAL{l}{is}}}$$

以下のologで型$A$と$B$は互いに素であり(共通部分集合を持たず), よって余積$C=A\sqcup B$は単なる合併集合である. $$\fbox{\xymatrix{\smbox{A}{a person}\LA{r}{is}&\smbox{C=A\sqcup B}{a person or a cat}&\smbox{B}{a cat}\LAL{l}{is}}}$$
\end{example}

\begin{example}\label{ex:coproduct2}

%In the following olog, $A$ and $B$ are not disjoint, so care must be taken to differentiate common elements. $$\fbox{\xymatrixnocompile{\obox{A}{.7in}{\rr an animal that can fly}\LA{rr}{labeled ``A" is}&&\obox{C=A\sqcup B}{1.3in}{an animal that can fly (labeled ``A") or an animal that can swim (labeled ``B")}&&\obox{B}{.9in}{\rr an animal that can swim}\LAL{ll}{labeled ``B" is}}}$$  Since ducks can both swim and fly, each duck is found twice in $C$, once labeled as a flyer and once labeled as a swimmer.  The types $A$ and $B$ are kept disjoint in $C$, which justifies the name ``disjoint union."

以下のologにおいて, $A$と$B$は互いに素ではなく, よって共通要素を区別して取らなければならないことに注意する. $$\fbox{\xymatrixnocompile{\obox{A}{.7in}{\rr an animal that can fly}\LA{rr}{labeled ``A" is}&&\obox{C=A\sqcup B}{1.3in}{an animal that can fly (labeled ``A") or an animal that can swim (labeled ``B")}&&\obox{B}{.9in}{\rr an animal that can swim}\LAL{ll}{labeled ``B" is}}}$$ なぜならアヒルは泳げかつ飛べるから, それぞれのアヒルは$C$において, かつて飛べるものと見出し付けられた分と, かつて泳ぐものと見出し付けられていた分とで, 計二回見出されるからである. 型$A$と型$B$は$C$でも互いに素なままであり, これは``直和(disjoint union)''という用語で正当化される.

\end{example}

\begin{exercise}

%Understand Example \ref{ex:coproduct2} and see if a similar idea would make sense for particles and waves. Make an olog, and choose your wording in accordance with Rules \ref{rules:types}. How do photons, which exhibit properties of both waves and particles, fit into the coproduct in your olog?

Example \ref{ex:coproduct2}を理解し, 同じようなアイデアが粒子と波動に対して意味を持つかどうかを見よ. ologを作り, Rules \ref{rules:types}に調和する単語を選べ. 光子は波動と粒子の双方の性質を示すが\footnote{訳注: 波動と粒子の双方の性質を示す物体は光子に限らない.}, 作成したologの余積の中に当てはめられるだろうか?

\end{exercise}

\begin{exercise}
%Following the section above, ``Ologging products" page \pageref{sec:ologging products}, come up with a naming system for coproducts, the inclusions, and the universal maps. Try it out by making an olog (involving coproducts) discussing the idea that both a .wav file and a .mp3 file can be played on a modern computer. Be careful that your arrows are valid in the sense of Section \ref{sec:invalid aspect}.

\pageref{sec:ologging products}ページの``Ologと積''に従って, 余積, 包含, および普遍写像に対する名前付けシステムを構築せよ. (余積を含む)ologとして, 現代的な計算機上で.wavファイルと.mp3ファイルの双方が再生可能であるというアイデアを議論するologを作り, そこで作成したシステムの実践を試みよ. \ref{sec:invalid aspect}章における意味で矢印が有効であるかどうかに十分注意せよ.
\end{exercise}

%%%%%% Section %%%%%%

%\section{Finite limits in $\Set$}\label{sec:finite limits}
\section{$\Set$の有限極限}\label{sec:finite limits}

%In this section we discuss what are called {\em limits} of variously-shaped diagrams of sets. We will make all this much more precise when we discuss limits in arbitrary categories in Section \ref{sec:lims and colims in a cat}.

この章で我々は, 集合の様々な形の図式の\emph{極限(limit)}と呼ばれるものについて議論する. ここでの議論は全て, \ref{sec:lims and colims in a cat}章で任意の圏における極限を議論した時に, はるかに精緻なものになるであろう.

%%%% Subsection %%%%

%\subsection{Pullbacks}
\subsection{引き戻し}

%\begin{definition}[Pullback]\label{def:pullback}\index{pullback!of sets}
\begin{definition}[引き戻し]\label{def:pullback}\index{pullback!of sets}

%Suppose given the diagram of sets and functions below.
%\begin{align}\label{dia:fp sets}
%\xymatrix{&Y\ar[d]^g\\
%X\ar[r]_f&Z}
%\end{align}
%Its {\em fiber product}\index{fiber product} is the set 
%$$X\times_ZY:=\{(x,w,y)\|f(x)=w=g(y)\}.$$ There are obvious projections $\pi_1\taking X\times_ZY\to X$ and $\pi_2\taking X\times_ZY\to Y$ (e.g. $\pi_2(x,w,y)=y$). Note that if $W=X\times_ZY$ then the diagram 
%\begin{align}\label{dia:pullback sets}
%\xymatrix{W\ullimit\ar[r]^-{\pi_2}\ar[d]_{\pi_1}&Y\ar[d]^g\\
%X\ar[r]_f&Z}
%\end{align}
%commutes. Given the setup of Diagram \ref{dia:fp sets} we define the {\em pullback of $X$ and $Y$ over $Z$} to be any set $W$ for which we have an isomorphism $W\To{\iso}X\times_ZY$. The corner symbol $\lrcorner$ in Diagram \ref{dia:pullback sets} indicates that $W$ is the pullback.\index{a symbol!$\lrcorner$}

以下のような集合の函数の図式が与えられたとする.
\begin{align}\label{dia:fp sets}
\xymatrix{&Y\ar[d]^g\\
X\ar[r]_f&Z}
\end{align}
この\emph{ファイバー積(fiber product)}\index{fiber product}は集合
$$X\times_ZY:=\{(x,w,y)\|f(x)=w=g(y)\}$$である. 自明な写像$\pi_1\taking X\times_ZY\to X$と$\pi_2\taking X\times_ZY\to Y$ (e.g. $\pi_2(x,w,y)=y$)が存在する. もし$W=X\times_ZY$ならば図式
\begin{align}\label{dia:pullback sets}
\xymatrix{W\ullimit\ar[r]^-{\pi_2}\ar[d]_{\pi_1}&Y\ar[d]^g\\
X\ar[r]_f&Z}
\end{align}
は可換であることに注意せよ. Diagram \ref{dia:fp sets}の設定が与えられたとき, \emph{$Z$上の$X$と$Y$の引き戻し(pullback of $X$ and $Y$ over $Z$)}を, 同型写像$W\To{\iso}X\times_ZY$を得ることができるところの任意の写像$W$として定義する. Diagram \ref{dia:pullback sets}での鉤記号$\lrcorner$は$W$が引き戻しであることを示す.\index{a symbol!$\lrcorner$}

\end{definition}

\begin{exercise}
%Let $X,Y,Z$ be as drawn and $f\taking X\to Z$ and $g\taking Y\to Z$ the indicated functions. 
%\begin{center}
%\includegraphics[height=2in]{setPullback}
%\end{center}
%What is the pullback of the diagram $X\Too{f}Z\Fromm{g}Y$?
$X$, $Y$, $Z$を描かれているような集合とし, $f\taking X\to Z$と$g\taking Y\to Z$を示さている函数とする. 
\begin{center}
\includegraphics[height=2in]{setPullback}
\end{center}
図式$X\Too{f}Z\Fromm{g}Y$の引き戻しは何だろうか?
\end{exercise}

\begin{exercise}~
%\sexc Draw a set $X$ with five elements and a set $Y$ with three elements. Color each element of $X$ and each element of $Y$ either red, blue, or yellow,
%\footnote{You can use shadings rather than coloring, if coloring would be annoying.}
%and do so in a ``random-looking" way. Considering your coloring of $X$ as a function $X\to C$, where $C=\{\tn{red, blue, yellow}\}$, and similarly obtaining a function $Y\to C$, draw the fiber product $X\times_CY$. Make sure it is colored appropriately.
%\next The universal property for products guarantees a function $X\times_CY\to X\times Y$, which I can tell you will be an injection. This means that the drawing you made of the fiber product can be imbedded into the $5\times 3$ grid; please draw the grid and indicate this subset.
%\endsexc
\sexc 5要素の集合$X$と3要素の集合$Y$を描け. $X$のそれぞれの要素と$Y$のそれぞれの要素を赤, 青, 黄のいづれかの色を塗り, 
\footnote{色を塗るのが面倒であるならば影を付けるだけでもよい.}
``無秩序に見える''ようなやり方をとれ. $C=\{\tn{red, blue, yellow}\}$として$X$の色の塗り方を$X\to C$とみなし, 同様に函数$Y\to C$を得, ファイバー積$X\times_CY$を描け. 適切に色が塗られていることを確認せよ.
\next 積の普遍的性質は函数$X\times_CY\to X\times Y$の存在を保証し, これは単射であると示すことができる. これが 意味するのは, $5\times 3$の格子の中に, 先程描いたファイバー積の図式を埋め込むことができるということである. よければ格子を描いてこの部分集合を示してみよう.
\endsexc
\end{exercise}

\begin{remark}

%Some may prefer to denote this fiber product by $f\times_Zg$ rather than $X\times_ZY$. The former is  mathematically better notation, but human-readability is often enhanced by the latter, which is also more common in the literature. We use whichever is more convenient.

ファイバー積を, $X\times_ZY$よりもむしろ$f\times_Zg$と書くことを好む人もいる. 後者は数学的にはよりよい記法であるが, 人間の可読性はしばしば前者の方が高められており, また前者は文献でより一般的な記法でもある. 我々はどちらかに限らずより便利な方を用いる.

\end{remark}

\begin{exercise}~
%\sexc Suppose that $Y=\emptyset$; what can you say about $X\times_ZY$? 
%\next Suppose now that $Y$ is any set but that $Z$ has exactly one element; what can you say about $X\times_ZY$?
%\endsexc
\sexc $Y=\emptyset$とする. $X\times_ZY$について何が言えるだろうか?
\next 今度は$Y$を任意の集合とするが, $Z$は要素を丁度一つだけ持つとする. $X\times_ZY$について何が言えるだろうか?
\endsexc
\end{exercise}

\begin{exercise}
%Let $S=\RR^3, T=\RR$, and think of them as (Aristotelian) space and time, with the origin in $S\times T$ given by the center of mass of MIT at the time of its founding. Let $Y=S\times T$ and let $g_1\taking Y\to S$ be one projection and $g_2\taking Y\to T$ the other projection. Let $X=\singleton$ be a set with one element and let $f_1\taking X\to S$ and $f_2\taking X\to T$ be given by the origin in both cases. 
%\sexc What are the fiber products $W_1$ and $W_2$:
%$$
%\xymatrix{W_1\ar[r]\ar[d]\ullimit&Y\ar[d]^{g_1}\\X\ar[r]_{f_1}&S}\hspace{1in}
%\xymatrix{W_2\ar[r]\ar[d]\ullimit&Y\ar[d]^{g_2}\\X\ar[r]_{f_2}&T}
%$$
%\next Interpret these sets in terms of the center of mass of MIT at the time of its founding.
%\endsexc
$S=\RR^3$, $T=\RR$とし, これらを(アリストテレス的な)空間と時間とみなし, $S\times T$の原点を創立時のマサチューセッツ工科大学(MIT)の重心によって与えるとする. $Y=S\times T$とし, $g_1\taking Y\to S$を一つ目の射影, $g_2\taking Y\to T$をもう一つの射影とする. $X=\singleton$を要素が一つだけの集合とし, $f_1\taking X\to S$および$f_2\taking X\to T$が両者とも原点によって与えられるとする. 
\sexc $W_1$と$W_2$のファイバー積は何だろうか?
$$
\xymatrix{W_1\ar[r]\ar[d]\ullimit&Y\ar[d]^{g_1}\\X\ar[r]_{f_1}&S}\hspace{1in}
\xymatrix{W_2\ar[r]\ar[d]\ullimit&Y\ar[d]^{g_2}\\X\ar[r]_{f_2}&T}
$$
\next これらの集合を創立時のMITの重心という観点で解釈せよ.
\endsexc
\end{exercise}

%% Subsubsection %%

%\subsubsection{Using pullbacks to define new ideas from old}
\subsubsection{既存のアイデアから新しいアイデアを定義するための引き戻しの使用}

%In this section we will see that the fiber product of a diagram can serve to define a new concept. For example, in (\ref{dia:bad battery}) we define what it means for a cellphone to have a bad battery, in terms of the length of time for which it remains charged. By being explicit, we reduce the chance of misunderstandings between different groups of people. This can be useful in situations like audits and those in which one is trying to reuse or understand data gathered by others.

この章では, 我々は図式のファイバー積が新しい概念の定義を提供するのに役立つことを見る. 例えば, \eqref{dia:bad battery}では充電器の調子が悪い携帯電話を意味するものを, 十分に充電されている時間の長さという観点から定義する. 概念を明示的にすることで, 我々は異なった集団に属する人々の間で誤解がしょうじる機会を減らす. これは監査や, あるいは他者によって集められたデータを理解や再利用しようと試みるといった状況において有用になりうる.

\begin{example}

%Consider the following two ologs. The one on the right is the pullback of the one on the left. 
%\begin{align}\label{dia:wealthy and loyal}
%\fbox{\xymatrixnocompile{&\obox{C}{.7in}{\rr a loyal customer}\LA{d}{is}\\\obox{B}{.7in}{\rr a wealthy customer}\LA{r}{is}&\smbox{D}{a customer}}}\hsp&\fbox{\xymatrix{\obox{A=B\times_DC}{.9in}{\rr a customer that is wealthy and loyal}\LAL{d}{is}\LA{r}{is}&\obox{C}{.7in}{\rr a loyal customer}\LA{d}{is}\\\obox{B}{.7in}{\rr a wealthy customer}\LA{r}{is}&\smbox{D}{a customer}}}
%\end{align}
%Check from Definition \ref{def:pullback} that the label, ``a customer that is wealthy and loyal", is fair and straightforward as a label for the fiber product $A=B\times_DC$, given the labels on $B,C$, and $D$.

以下の二つのologを考える. 右側のものは左側のものの引き戻しである.
\begin{align}\label{dia:wealthy and loyal}
\fbox{\xymatrixnocompile{&\obox{C}{.7in}{\rr a loyal customer}\LA{d}{is}\\\obox{B}{.7in}{\rr a wealthy customer}\LA{r}{is}&\smbox{D}{a customer}}}\hsp&\fbox{\xymatrix{\obox{A=B\times_DC}{.9in}{\rr a customer that is wealthy and loyal}\LAL{d}{is}\LA{r}{is}&\obox{C}{.7in}{\rr a loyal customer}\LA{d}{is}\\\obox{B}{.7in}{\rr a wealthy customer}\LA{r}{is}&\smbox{D}{a customer}}}
\end{align}
Definition \ref{def:pullback}から, $B$, $C$, $D$の上に与えられた見出しに対して, ファイバー積$A=B\times_DC$についての見出しとして, 見出し``a customer that is wealthy and loyal''が妥当かつ直接的であることを確認せよ.

\end{example}

\begin{remark}\label{rem:defining using pullbacks}

%Note that in Diagram (\ref{dia:wealthy and loyal}) the top-left box could have been (non-canonically named) \fakebox{a good customer}. If it was taken to be the fiber product, then the author would be effectively {\em defining} a good customer to be one that is wealthy and loyal. 

Diagram \eqref{dia:wealthy and loyal}において, 左上の箱は(非標準的な名前として)\fakebox{a good customer}であったであろうことに注意せよ. これがファイバー積であるとされているのならば, ologの作者はよき顧客(a good customer)は裕福(wealthy)で義理堅い(loyal)顧客であると効果的に\emph{定義}していることになるであろう.

\end{remark}

\begin{exercise}
%For each of the following, an author has proposed that the diagram on the right is a pullback. Do you think their labels are appropriate or misleading; that is, is the label on the upper-left box reasonable given the rest of the olog, or is it suspect in some way?
以下のそれぞれのologにおいて, その作者は右側の図式は引き戻しであると提案している. これらの見出しをどう思うだろうか? 適切かあるいは誤解を招かねないだろうか? つまり, 左上の箱の見出しは与えられたologの残り部分に対して妥当だろうか, それともなんらかの点で疑問だろうか?
\sexc\begin{align*}\footnotesize\fbox{\xymatrix{&&\smbox{C}{blue}\LA{d}{is}\\\smbox{B}{a person}\LA{rr}{\parbox{.7in}{\rr has as favorite color}}&&\smbox{D}{a color}}}\hsp&
\footnotesize\fbox{\xymatrixnocompile{\obox{A=B\times_DC}{1.1in}{\rr a person whose favorite color is blue}\LAL{d}{is}\LA{rr}{\parbox{.7in}{\rr has as favorite color}}&&\smbox{C}{blue}\LA{d}{is}\\\smbox{B}{a person}\LA{rr}{\parbox{.7in}{\rr has as favorite color}}&&\smbox{D}{a color}}}
\end{align*}
\next\begin{align*}
\footnotesize\fbox{\xymatrixnocompile{&&\smbox{C}{a woman}\LA{d}{is}\\\smbox{B}{a dog}\LA{rr}{\parbox{.7in}{\rr has as owner}}&&\smbox{D}{a person}}}\hsp&
\footnotesize\fbox{\xymatrixnocompile{\obox{A=B\times_DC}{1in}{\rr a dog whose owner is a woman}\LAL{d}{is}\LA{rr}{\parbox{.7in}{\rr has as owner}}&&\smbox{C}{a woman}\LA{d}{is}\\\smbox{B}{a dog}\LA{rr}{\parbox{.7in}{\rr has as owner}}&&\smbox{D}{a person}}}
\end{align*}
\next\begin{align*}
\footnotesize\fbox{\xymatrixnocompile{&\obox{C}{.5in}{\rr a piece of furniture}\LA{d}{has}\\\obox{B}{.6in}{\rr a space in our house}\LA{r}{has}&\smbox{D}{a width}}}\hsp&
\footnotesize\fbox{\xymatrixnocompile{\obox{A=B\times_DC}{.5in}{\rr a good fit}\LAL{d}{$s$}\LA{r}{$f$}&\obox{C}{.5in}{\rr a piece of furniture}\LA{d}{has}\\\obox{B}{.6in}{\rr a space in our house}\LA{r}{has}&\smbox{D}{a width}}}
\end{align*}
\endsexc
\end{exercise}

\begin{exercise}~
%\sexc Consider your olog from Exercise \ref{exc:family olog}. Are any of the commutative squares there actually pullback squares? 
%\next Now use ologs with products and pullbacks to define what a brother is and what a sister is (again in a human biological nuclear family), in terms of types such as \fakebox{an offspring of mating pair $(a,b)$}, \fakebox{a person}, \fakebox{a male person}, \fakebox{a female person}, and so on.
%\endsexc
\sexc Exercise \ref{exc:family olog}で作ったologを考える. そこにある任意の可換四角形は実際に引き戻しの四角形だろうか?
\next (同様に人間の生物学的核家族において)男兄弟(brother)とは何かと女兄弟(sister)とは何かを定義するために積と引き戻しを含むologを使え. 例えば\fakebox{an offspring of mating pair $(a,b)$}, \fakebox{a person}, \fakebox{a male person}, \fakebox{a female person}, などといった型によって表現せよ.
\endsexc
\end{exercise}

\begin{definition}[Preimage]\label{def:preimage}

%Let $f\taking X\to Y$ be a function and $y\in Y$ an element. The {\em preimage of y under $f$}\index{preimage}, denoted $f^\m1(y)$,\index{a symbol!$f^\m1$} is  the subset $f^\m1(y):=\{x\in X\|f(x)=y\}$. If $Y'\ss Y$ is any subset, the {\em preimage of $Y'$ under $f$}, denoted $f^\m1(Y')$, is the subset $f^\m1(Y')=\{x\in X\|f(x)\in Y'\}$.

$f\taking X\to Y$を函数とし, $y\in Y$を集合の要素とする. \emph{$f$による$y$の原像(preimage of $y$ under $f$)}\index{preimage}は部分集合$f^\m1(y):=\{x\in X\|f(x)=y\}$であり, $f^\m1(y)$,\index{a symbol!$f^\m1$}と書く. $Y'\ss Y$を任意の部分集合とすると, \emph{$f$による$Y'$の原像(preimage of $Y'$ under $f$)}は部分集合$f^\m1(Y')=\{x\in X\|f(x)\in Y'\}$であり, $f^\m1(Y')$と書かれる.

\end{definition}

\begin{exercise}
%Let $f\taking X\to Y$ be a function and $y\in Y$ an element. Draw a pullback diagram in which the fiber product is isomorphic to the preimage $f^\m1(y)$.
$f\taking X\to Y$を函数とし$y\in Y$を集合の要素とする. 含まれるファイバー積が原像$f^\m1(y)$と同型であるような引き戻しの図式を描け.
\end{exercise}

\begin{lemma}[Universal property for pullback]\label{lemma:up for fp}

%Suppose given the diagram of sets and functions as below.
%\begin{align*}
%\xymatrix{&Y\ar[d]^u\\
%X\ar[r]_t&Z}
%\end{align*}
%For any set $A$ and commutative solid arrow diagram as below (i.e. functions $f\taking A\to X$ and $g\taking A\to Y$ such that $t\circ f=u\circ g$), 
%\begin{align}\label{dia:universal property of fp}
%\xymatrix{
%&X\times_ZY\ar@/_1pc/[lddd]_{\pi_1}\ar@/^1pc/[rddd]^{\pi_2}\\\\
%&A\ar@{-->}[uu]^{\exists!}\ar[dl]_{\forall f}\ar[dr]^{\forall g}&\\
%X\ar[rd]_t&&Y\ar[ld]^u\\
%&Z&}
%\end{align}
%there exists a unique arrow $\pb{f}{g}{Z}\taking A\to X\times_ZY$ making everything commute, i.e. 
%$$f=\pi_1\circ \pb{f}{g}{Z}\hsp\text{and}\hsp g=\pi_2\circ\pb{f}{g}{Z}.$$

以下の集合と函数の図式が与えられたとする.
\begin{align*}
\xymatrix{&Y\ar[d]^u\\
X\ar[r]_t&Z}
\end{align*}
任意の集合$A$と, 以下のような可換な実線の矢印の図式(i.e. $t\circ f=u\circ g$となるところの函数$f\taking A\to X$と$g\taking A\to Y$)に対して,
\begin{align}\label{dia:universal property of fp}
\xymatrix{
&X\times_ZY\ar@/_1pc/[lddd]_{\pi_1}\ar@/^1pc/[rddd]^{\pi_2}\\\\
&A\ar@{-->}[uu]^{\exists!}\ar[dl]_{\forall f}\ar[dr]^{\forall g}&\\
X\ar[rd]_t&&Y\ar[ld]^u\\
&Z&}
\end{align}
全てを可換にする一意な矢印
$\pb{f}{g}{Z}\taking A\to X\times_ZY$が存在する\footnote{訳注: 原文の$\pb{f}{f}{Z}$を修正.}. すなわち
$$f=\pi_1\circ \pb{f}{g}{Z}\hsp\text{および}\hsp g=\pi_2\circ\pb{f}{g}{Z}.$$

\end{lemma}

\begin{exercise}
%Create an olog whose underlying shape is a commutative square. Now add the fiber product so that the shape is the same as that of Diagram (\ref{dia:universal property of fp}). Assign English labels to the projections $\pi_1,\pi_2$ and to the dotted map $A\To{\pb{f}{g}{Z}}X\times_ZY$, such that these labels are as canonical as possible.
骨組が可換四角形であるologを作成せよ. 次に\eqref{dia:universal property of fp}と同じ形になるようにファイバー積を追加せよ. 射影$\pi_1$, $\pi_2$および点線の写像$A\To{\pb{f}{g}{Z}}X\times_ZY$に英語の見出しを割り当て, 可能な限りそれらの見出しがcananicalであるようにせよ.
\end{exercise}

%% Subsubsection %%

\subsubsection{Pasting diagrams for pullback}

%Consider the diagram drawn below, which includes a left-hand square, a right-hand square, and a big rectangle.
%$$
%\xymatrix{
%A'\ar[r]^{f'}\ar[d]_i\ullimit&B'\ar[r]^{g'}\ar[d]_j\ullimit&C'\ar[d]^k\\
%A\ar[r]_f&B\ar[r]_g&C}
%$$
%The right-hand square has a corner symbol indicating that $B'\iso B\times_CC'$ is a pullback. But the corner symbol on the left is ambiguous; it might be indicating that the left-hand square is a pullback, or it might be indicating that the big rectangle is a pullback. It turns out that if $B'\iso B\times_CC'$ then it is not ambiguous because the left-hand square is a pullback if and only if the big rectangle is.

以下の図式を考える. ここには左側の四角形, 右側の四角形, そし全体の四角形が含まれる.
$$
\xymatrix{
A'\ar[r]^{f'}\ar[d]_i\ullimit&B'\ar[r]^{g'}\ar[d]_j\ullimit&C'\ar[d]^k\\
A\ar[r]_f&B\ar[r]_g&C}
$$
右側の四角形には$B'\iso B\times_CC'$が引き戻しであることを示す鉤記号がある. しかし左側の鉤記号は曖昧である. これは左側の四角形が引き戻しであることを示しているのかもしれないし, 全体の四角形が引き戻しであることを示しているのかもしれない. しかし, もし$B'\iso B\times_CC'$ならば曖昧でないことが分かる. なぜならば左側の四角形が可換である時かつその時に限り全体の四角形が可換であるからである.

\begin{proposition}\label{prop:pasting}

%Consider the diagram drawn below
%$$
%\xymatrix{
%&B'\ar[r]^{g'}\ar[d]_j\ullimit&C'\ar[d]^k\\
%A\ar[r]_f&B\ar[r]_g&C}
%$$
%where $B'\iso B\times_CC'$ is a pullback. Then there is an isomorphism $A\times_BB'\iso A\times_CC'$. Said another way, $$A\times_B(B\times_CC')\iso A\times_CC'.$$

以下に描かれた図式を考える.
$$
\xymatrix{
&B'\ar[r]^{g'}\ar[d]_j\ullimit&C'\ar[d]^k\\
A\ar[r]_f&B\ar[r]_g&C}
$$
ここで$B'\iso B\times_CC'$は引き戻しである. このとき同型写像$A\times_BB'\iso A\times_CC'$が存在する. 別の言い方をすれば$$A\times_B(B\times_CC')\iso A\times_CC'.$$である.

\end{proposition}

\begin{proof}

%We first provide a map $\phi\taking A\times_B(B\times_CC')\to A\times_CC'$. An element of $A\times_B(B\times_CC')$ is of the form $(a,b,(b,c,c'))$ such that $f(a)=b, g(b)=c$ and $k(c')=c$. But this implies that $g\circ f(a)=c=k(c')$ so we put $\phi(a,b,(b,c,c')):=(a,c,c')\in A\times_CC'$. Now we provide a proposed inverse, $\psi\taking A\times_CC'\to A\times_B(B\times_CC')$. Given $(a,c,c')$ with $g\circ f(a)=c=k(c')$, let $b=f(a)$ and note that $(b,c,c')$ is an element of $B\times_CC'$. So we can define $\psi(a,c,c')=(a,b,(b,c,c'))$. It is easy to see that $\phi$ and $\psi$ are inverse.
 
最初に写像$\phi\taking A\times_B(B\times_CC')\to A\times_CC'$を与える. $A\times_B(B\times_CC')$の要素は, $f(a)=b$, $g(b)=c$, $k(c')=c$として$(a,b,(b,c,c'))$の形を持つ. しかしこれは$g\circ f(a)=c=k(c')$を暗に意味し, よって$\phi(a,b,(b,c,c')):=(a,c,c')\in A\times_CC'$と置ける. さて, 次は逆写像$\psi\taking A\times_CC'\to A\times_B(B\times_CC')$を与えよう. $g\circ f(a)=c=k(c')$なる$(a,c,c')$が与えられたとき, $b=f(a)$として, また$(b,c,c')$が$B\times_CC'$の要素であることに注意しよう. よって我々は$\psi(a,c,c')=(a,b,(b,c,c'))$と定義できる. $\phi$と$\psi$が逆写像の関係であることは簡単に分かる.
\end{proof}

%Proposition \ref{prop:pasting} can be useful in authoring ologs. For example, the type \fakebox{a cellphone that has a bad battery} is vague, but we can lay out precisely what it means using pullbacks:
%\small
%\begin{align}\label{dia:bad battery}
%\fbox{\xymatrixnocompile{\obox{A\iso B\times_DC}{1in}{a cellphone that has a bad battery}\ar[r]\ar[d]&\smbox{C\iso D\times_FE}{a bad battery}\ar[r]\ar[d]&\obox{E\iso F\times_HG}{.5in}{less than 1 hour}\ar[r]\ar[d]&\obox{G}{.5in}{between 0 and 1}\ar[d]\\\smbox{B}{a cellphone}\LA{r}{has}&\smbox{D}{a battery}\LA{r}{\parbox{.4in}{\rr remains charged for}}&\obox{F}{.6in}{a duration of time}\LA{r}{\hspace{.07in}\parbox{.4in}{\rr in hours yields}}&\obox{H}{.6in}{a range of numbers}}}
%\end{align}\normalsize

Proposition \ref{prop:pasting}はologを正当化するのに役に立つ. 例えば\fakebox{a cellphone that has a bad battery}は明確ではない. しかし我々は引き戻しを使ってそれが意味するところを正確に展開できる.
\small
\begin{align}\label{dia:bad battery}
\fbox{\xymatrixnocompile{\obox{A\iso B\times_DC}{1in}{a cellphone that has a bad battery}\ar[r]\ar[d]&\smbox{C\iso D\times_FE}{a bad battery}\ar[r]\ar[d]&\obox{E\iso F\times_HG}{.5in}{less than 1 hour}\ar[r]\ar[d]&\obox{G}{.5in}{between 0 and 1}\ar[d]\\\smbox{B}{a cellphone}\LA{r}{has}&\smbox{D}{a battery}\LA{r}{\parbox{.4in}{\rr remains charged for}}&\obox{F}{.6in}{a duration of time}\LA{r}{\hspace{.07in}\parbox{.4in}{\rr in hours yields}}&\obox{H}{.6in}{a range of numbers}}}
\end{align}\normalsize

%The category-theoretic fact described above says that since $A\iso B\times_DC$ and $C\iso D\times_FE$, it follows that $A\iso B\times_FE$.  That is, we can deduce the definition ``a cellphone that has a bad battery is defined as a cellphone that has a battery which remains charged for less than one hour."  

上記の圏論的な事実の記述は, $A\iso B\times_DC$および$C\iso D\times_FE$であることから$A\iso B\times_FE$が従うということを述べている. これは, 定義``調子が悪い充電池を持つ携帯電話は, 一時間未満しか充電が持たない充電池を持つ携帯電話として定義される''\footnote{英文では``a cellphone that has a bad battery is defined as a cellphone that has a battery which remains charged for less than one hour.''}ことが推論できたということである.


\begin{exercise}~
%\sexc Create an olog that defines two people to be ``of approximately the same height" if and only if their height difference is less than half an inch, using a pullback. Your olog can include the box \fakebox{a real number $x$ such that $-.5<x<.5$}. 
%\next In the same olog, make a box for those people whose height is approximately the same as a person named ``The Virgin Mary". You may need to use images, as in Section \ref{sec:images}.
%\endsexc
\sexc 二人の人間が``ほぼ同じ身長である(of approximately the same height)''時かつその時に限り身長差が3cm以下である, ということをを定義するologを引き戻しを使って定義せよ. 作成したologには箱\fakebox{a real number $x$ such that $-1.5<x<1.5$}が含まれていること.
\next 同じolog内に, ``The Virgin Mary''という名前の人物とほほおなじ身長を持つ人間の箱を作れ. Section \ref{sec:images}のように像を使う必要があるかもしれない.
\endsexc
\end{exercise}

\begin{exercise}\label{exc:pointwise map of fp}
%Consider the diagram on the left below, where both squares commute. 
左下の図式を考える. ここで両方の四角形か可換とする.
$$
\xymatrix@=15pt{
&&&Y'\ar[dd]\\
&&Y\ar[ru]\ar[dd]\\
&X'\ar'[r][rr]&&Z'\\
X\ar[rr]\ar[ru]&&Z\ar[ru]
}
\hspace{1in}
\xymatrix@=15pt{
&W'\ar[rr]\ar'[d][dd]\ullimit&&Y'\ar[dd]\\
W\ar[rr]\ar[dd]\ullimit&&Y\ar[ru]\ar[dd]\\
&X'\ar'[r][rr]&&Z'\\
X\ar[rr]\ar[ru]&&Z\ar[ru]
}
$$
%Let $W=X\times_ZY$ and $W'=X'\times_{Z'}Y'$, and form the diagram to the right. Use the universal property of fiber products to construct a map $W\to W'$ such that all squares commute.
$W=X\times_ZY$, $W'=X'\times_{Z'}Y'$として, これらが右側の図式をなしているとする. ファイバー積の普遍的性質を用いて, 全ての四角形を可換にする写像$W\to W'$を構成せよ.
\end{exercise}

%%%% Subsection %%%%

%\subsection{Spans, experiments, and matrices}
\subsection{スパン, 実験, 行列}

\begin{definition}\label{def:span}\index{span}

%Given sets $A$ and $B$, a {\em span on $A$ and $B$} is a set $R$ together with functions $f\taking R\to A$ and $g\taking R\to B$. 
%$$\xymatrix@=15pt{&R\ar[ddl]_f\ar[ddr]^g\\\\A&&B}$$

集合$A$, $B$が与えられた時, \emph{$A$と$B$のスパン(span on $A$ and $B$)}は, 函数$f\taking R\to A$ and $g\taking R\to B$を伴う集合$R$である.
$$\xymatrix@=15pt{&R\ar[ddl]_f\ar[ddr]^g\\\\A&&B}$$

\end{definition}

\begin{application}\label{app:exp temp press}

%Think of $A$ and $B$ as observables and $R$ as a set of experiments performed on these two variables. For example, let's say $T$ is the set of possible temperatures of a \href{http://en.wikipedia.org/wiki/Ideal_gas_law}{\text gas} in a fixed container and let's say $P$ is the set of possible pressures of the gas. We perform 1000 experiments in which we change and record the temperature and we simultaneously also record the pressure; this is a span $T\From{f}E\To{g}P$. The results might look like this:

$A$と$B$を可観測量とし, $R$をこれら二つの変数を対象として実行される実験とする. 例えば$T$を, 固定サイズの容器のなかで, \href{http://en.wikipedia.org/wiki/Ideal_gas_law}{気体}がとれる温度とし, $P$をその気体がとれる圧力とする. 1000回温度を変えて実験を実行し, 同時に圧力を記録する. これはスパン$T\From{f}E\To{g}P$である. 結果は以下のようになるだろう.
$$
\begin{tabular}{| l || l | l |}
\bhline
\multicolumn{3}{| c |}{Experiment}\\\bhline
{\bf ID}&{\bf Temperature}&{\bf Pressure}\\\bbhline
1&100& 72\\\hline
2&100&73\\\hline
3&100&72\\\hline
4&200&140\\\hline
5&200&138\\\hline
6&200&141\\\hline
\vdots&\vdots&\vdots\\\bhline
\end{tabular}
$$
%TODO: 有効数字の指摘

\end{application}

\begin{definition}\label{def:composite span}

%Let $A,B,$ and $C$ be sets, and let $A\From{f}R\To{g}B$ and $B\From{f'}R'\To{g'}C$ be spans. Their {\em composite span}\index{span!composite} is given by the fiber product $R\times_BR'$ as in the diagram below:
$A$, $B$, $C$を集合とし, $A\From{f}R\To{g}B$, $B\From{f'}R'\To{g'}C$をスパンとする. これらの\emph{合成スパン(composite span)}\index{span!composite}はファイバー積$R\times_BR'$によって与えられ, 下の図式のようになる.
$$
\xymatrix@=10pt{&&R\times_BR'\ar[ldd]\ar[rdd]\\\\&R\ar[ddl]_f\ar[ddr]^g&&R'\ar[ddl]_{f'}\ar[ddr]^{g'}\\\\A&&B&&C
}$$

\end{definition}

\begin{application}\label{app:exp temp press 2}

%Let's look back at our lab's experiment from Application \ref{app:exp temp press}, which resulted in a span $T\From{f}E\To{g}P$. Suppose we notice that something looks a little wrong. The pressure should be linear in the temperature but it doesn't appear to be. We hypothesize that the volume of the container is increasing under pressure. We look up this container online and see that experiments have been done to measure the volume as the interior pressure changes. The data has generously been made available online, which gives us a span $P\From{f'}E'\To{g'}V$. 

Application \ref{app:exp temp press}で行なわれた, 我々の研究室の実験を振り返ってみよう. 実験結果は$T\From{f}E\To{g}P$である. 我々は何かが少々おかしく見えることに気が付いたと仮定する. 圧力は温度に比例すべきはずだが, そのようには見えない. 我々は容器の体積が圧力下で増加するという仮説を立てる. 我々はこの容器をインターネットで調べ, そして内部圧力の変化に伴って体積を測定する実験が行われていたということを知る. 気前よいことにデータはインターネット上で取得可能であり, これによってスパン$P\From{f'}E'\To{g'}V$が与えられる.

%The composite of our lab's span with the online data span yields a span $T\from E''\to V$, where $E'':=E\times_PE'$. What information does this span give us? In explaining it, one might say ``whenever an experiment in our lab yielded the same pressure as one they recorded, let's call that a data point. Every data point has an associated temperature (from our lab) and an associated volume (from their experiment). This is the best we can do." 

我々の研究室ではのスパンとインターネット上のデータのスパンの合成によってスパン$T\from E''\to V$がもたらされる. ここで$E'':=E\times_PE'$である. このスパンから何の情報が得られるだろうか? これを説明するのにあたって, 次のようなことが言えるだろう. ``彼らが記録した圧力と同じ圧力が我々の研究室での実験から得られているならば, それをデータ点と呼ぶことにしよう. 全てのデータ点は(我々の実験によって)温度と, (彼らの実験からによって)体積と結びついている. これが我々ができる精一杯のことである.''

%The information we get this way might be seen by some as unscientific, but it certainly is the kind of information people use in business and in every day life calculation---we get our data from multiple sources and put it together. Moreover, it is scientific in the sense that it is reproducible. The way we obtained our $T$-$V$ data is completely transparent.

この方法で我々が得た情報は非科学的に見えるかもしれない. しかしこれは確かに仕事や日常生活での計算で使う情報の類である---我々は複数の情報源からデータを取得しそれを結合する. 加えて, 再現可能という点ではこれは科学的でもある. 我々が$T$-$V$データを得た方法は完全に透過的である.


\end{application}

%We can relate spans to matrices of natural numbers, and see a natural ``categorification" of matrix addition and matrix multiplication. If our spans come from experiments as in Applications \ref{app:exp temp press} and \ref{app:exp temp press 2} the matrices involved will look like huge but sparse matrices. Let's go through that.

我々はスパンを自然数の行列に関連付けることができ, そして行列の足し算とかけ算を自然に``categorification''できることが分かる. Applications \ref{app:exp temp press}や\ref{app:exp temp press 2}でのような実験からスパンが得られたのならば, 関連する行列は巨大でかつ疎な行列のようになるだろう. それでは検討してみよう.

%Let $A$ and $B$ be sets and let $A\from R\to B$ be a span. By the universal property of products, we have a unique map $R\To{p}A\times B$. 

$A$と$B$を集合とし$A\from R\to B$をスパンとする. 積の普遍的性質により, 一意な写像$R\To{p}A\times B$を得ることができる.

%We make a matrix of natural numbers out of this data as follows. The set of rows is $A$, the set of columns is $B$. For elements $a\in A$ and $b\in B$, the $(a,b)$-entry is the cardinality of its preimage, $|p^\m1(a,b)|$, i.e. the number of elements in $R$ that are sent by $p$ to $(a,b)$. 

我々は以下のようにしてこのデータから自然数の行列を作ることができる. 行の集合は$A$であり, 列の集合は$B$である. 要素$a\in A$および$b\in B$に対して, 行列要素$(a,b)$はその原像$|p^\m1(a,b)|$の濃度, i.e. $p$によって$(a,b)$に送られた$R$の要素の数である.

%Suppose we are given two $(A,B)$-spans, i.e. $A\from R\to B$ and $A\from R'\to B$; we might think of these has having the same {\em dimensions}, i.e. they are both $|A|\times|B|$-matrices. We can take the disjoint union $R\sqcup R'$ and by the universal property of coproducts we have a unique span $A\from R\sqcup R'\to B$ making the requisite diagram commute.
%\footnote{
%$$\xymatrix{
%&R\ar[dl]\ar[dr]\ar[d]\\
%A&R\sqcup R'\ar[l]\ar[r]&B\\
%&R'\ar[ur]\ar[ul]\ar[u]}
%$$
%}
%The matrix corresponding to this new span will be the sum of the matrices corresponding to the two previous spans out of which it was made.

$(A,B)$のスパンが二つ与えられたと仮定する, i.e. $A\from R\to B$ and $A\from R'\to B$. 二つが同じ\emph{次元(dimension)}を持つときを考えたいかもしれない, i.e. 両者は共に$|A|\times|B|$-行列であるとする. 我々は直和$R\sqcup R'$をとることができ, 余積の普遍性によって我々は一意なスパン$A\from R\sqcup R'\to B$を得ることができ, これは必要とされる図式を可換にする.
\footnote{
$$\xymatrix{
&R\ar[dl]\ar[dr]\ar[d]\\
A&R\sqcup R'\ar[l]\ar[r]&B\\
&R'\ar[ur]\ar[ul]\ar[u]}
$$
}
新しいスパンに対応する行列は, そのスパンが生成される元となった二つのスパンに対応する行列の和に相当することになる.

%Given a span $A\from R\to B$ and a span $B\from S\to C$, the composite span can be formed as in Definition \ref{def:composite span}. It will correspond to the usual multiplication of matrices.

スパン$A\from R\to B$およびスパン$B\from S\to C$が与えられたとき, 合成スパンはDefinition \ref{def:composite span}の形で得られる. これは通常の行列の積に対応することになる.

\begin{construction}\label{const:bipartite}\index{graph!bipartite}

%Given a span $A\From{f} R\To{g} B$, one can draw a {\em bipartite graph} with each element of $A$ drawn as a dot on the left, each element of $B$ drawn as a dot on the right, and each element $r\in R$ drawn as an arrow connecting vertex $f(r)$ on the left to vertex $g(r)$ on the right.

スパン$A\From{f} R\To{g} B$が与えられたとき, $A$のそれぞれの要素を左側の点として, $B$のそれぞれの要素を右側の点として, そして要素$r\in R$のそれぞれを左側の頂点$f(r)$と右側の頂点$g(r)$を結ぶ矢印として描くことにより, \emph{二部グラフ(bipartite graph)}を描くことができる.

\end{construction}

\begin{exercise}~
%\sexc Draw the bipartite graph (as in Construction \ref{const:bipartite}) corresponding to the span $T\From{f}E\To{g}P$ in Application \ref{app:exp temp press}.
%\next Now make up your own span $P\From{f'}E'\To{g'}V$ and draw it. Finally, draw the composite span below. 
%\next Can you say how the composite span graph relates to the graphs of its factors?
%\endsexc
\sexc (Construction \ref{const:bipartite}のように) Application \ref{app:exp temp press}のスパン$T\From{f}E\To{g}P$に対応する二部グラフを描け.
\next 今度は自作のスパン$P\From{f'}E'\To{g'}V$を作ってみよう. 最後にその下に合成スパンを描け.
\next 合成スパンのグラフとその部分要素のグラフの関係について何か言えるだろうか?
\endsexc
\end{exercise}

%%%% Subsection %%%%

%\subsection{Equalizers and terminal objects}
\subsection{等価子と終対象}

\begin{definition}\label{def:equalizer}\index{equalizer}

%Suppose given two parallel arrows 
%\begin{align}\label{dia:equalizer}
%\xymatrix{X\ar@<.5ex>[r]^f\ar@<-.5ex>[r]_g&Y.}\hspace{1in}\xymatrix{Eq(f,g)\ar[r]^-p&X\ar@<.5ex>[r]^f\ar@<-.5ex>[r]_g&Y}
%\end{align}
%The {\em equalizer of $f$ and $g$} is the commutative diagram as to the right in (\ref{dia:equalizer}), where we define $$Eq(f,g):=\{x\in X\|f(x)=g(x)\}$$ and where $p$ is the canonical inclusion.

二本の平行な矢印が与えられたと仮定する. 
\begin{align}\label{dia:equalizer}
\xymatrix{X\ar@<.5ex>[r]^f\ar@<-.5ex>[r]_g&Y.}\hspace{1in}\xymatrix{Eq(f,g)\ar[r]^-p&X\ar@<.5ex>[r]^f\ar@<-.5ex>[r]_g&Y}
\end{align}
\emph{$f$と$g$の等価子(equalizer of $f$ and $g$)}は\eqref{dia:equalizer}の右側のような可換図式である. ここで$$Eq(f,g):=\{x\in X\|f(x)=g(x)\}$$と定義し, また$p$はcanonical inclusionである.

\end{definition}

\begin{example}

%Suppose one has designed an experiment to test a theoretical prediction. The question becomes, ``when does the theory match the experiment?" The answer is given by the equalizer of the following diagram:
%$$\xymatrix{
%\obox{}{.5in}{an input}\ar@<1ex>[rr]^{\tn{should, according to theory, yield}}\ar@<-1ex>[rr]_{\tn{according to experiment yields}}&\hspace{1in}&\obox{}{.6in}{an output}
%}$$
%The equalizer is the set of all inputs for which the theory and the experiment yield the same output.

理論的予想を検証するための実験を立案したと仮定しよう. 問題は``どんな時に理論は実験と整合するか?''になる. その答えは以下の図式の等価子によって与えられる.
$$\xymatrix{
\obox{}{.5in}{an input}\ar@<1ex>[rr]^{\tn{should, according to theory, yield}}\ar@<-1ex>[rr]_{\tn{according to experiment yields}}&\hspace{1in}&\obox{}{.6in}{an output}
}$$
等価子は, 理論と実験が同じ出力をもたらすところの入力全てからなる集合である.

\end{example}

\begin{exercise}
%Come up with an olog that uses equalizers in a reasonably interesting way. Alternatively, use an equalizer to specify those published authors who have published exactly one paper. Hint: find a function from authors to papers; then find another.
等価子を筋道だっていて興味深い方法で用いているologを思い付け. あるいはその代わりに, 論文をちょうど一本だけ出版した著者を特徴づけるために等価子を使用せよ. ヒント: 著者から論文への函数を見出せ. 次にもう一つ別の函数を見出せ.
\end{exercise}

\begin{exercise}
%Find a universal property enjoyed by the equalizer of two arrows, and present it in the style of Lemmas \ref{lemma:up for prod}, \ref{lemma:up for coprod}, and \ref{lemma:up for fp}.
二つの矢印の等価子が持っている普遍的性質を見出し, Lemmas \ref{lemma:up for prod}, \ref{lemma:up for coprod}, および\ref{lemma:up for fp}の様式で表示せよ.
\end{exercise}

\begin{exercise}[終対象(terminal object)]\index{terminal object!in $\Set$}~
%\sexc A terminal set is a set $S$ such that for every set $X$, there exists a unique function $X\to S$. Find a terminal set. 
%\next Do you think that the notion {\em terminal set} belongs in this section (Section \ref{sec:finite limits})? How so? If products, pullbacks, and equalizers are all limits, what do limits have in common?
%\endsexc
\sexc 終集合は, 全ての集合$X$に対して函数$X\to S$が一意に存在するところの集合$S$である. 終集合を見出せ.
\next この章(Section \ref{sec:finite limits})における\emph{終集合}の概念についてなにか思っただろうか? どのように思っただろうか? もし積, 引き戻し, 等価子を全て極限(limit)と呼ぶならば, 極限が共通して持つものはなんだろうか?
\endsexc
\end{exercise}


%%%%%% Section %%%%%%

%\section{Finite colimits in $\Set$}\label{sec:finite colimits}
\section{$\Set$の有限余極限}\label{sec:finite colimits}

%This section will parallel Section \ref{sec:finite limits}---I will introduce several types of finite colimits and hope that this gives the reader some intuition about them, without formally defining them yet. Before doing so, I must define equivalence relations and quotients.

この章はSection \ref{sec:finite limits}と並立する章となる---著者はいくつかの種類の有限余極限を紹介し, それによって読者がそれらに対するなんらかの直感を, 形式的な定義を抜きにして得ることを期待する. ただしその前にまず, 我々は同値関係と商を定義しなければならない.

%%%% Subsection %%%%

%\subsection{Background: equivalence relations}\index{equivalence relation}\index{relation!equivalence}
\subsection{背景: 同値関係}\index{equivalence relation}\index{relation!equivalence}

\begin{definition}[Equivalence relations and equivalence classes]

%Let $X$ be a set. An {\em equivalence relation on $X$} is a subset $R\ss X\times X$ satisfying the following properties for all $x,y,z\in X$:
%\begin{description}
%\item[Reflexivity:] $(x,x)\in R$;
%\item[Symmetry:] $(x,y)\in R$ if and only if $(y,x)\in R$; and
%\item[Transitivity:] if $(x,y)\in R$ and $(y,z)\in R$ then $(x,z)\in R$.
%\end{description}
%If $R$ is an equivalence relation, we often write $x\sim_R y$, or simply $x\sim y$, to mean $(x,y)\in R$. For convenience we may refer to the equivalence relation by the symbol $\sim$, saying that $\sim$ is an equivalence relation on $X$.\index{a symbol!$\sim$}

$X$を集合とする. \emph{$X$上の同値関係(equivalence relation on $X$)}とは全ての$x,y,z\in X$に対して以下の性質を満たす部分集合$R\ss X\times X$である.
\begin{description}
\item[反射性(Reflexivity):] $(x,x)\in R$.
\item[対称性(Symmetry):] $(x,y)\in R$のときかつそのときに限り$(y,x)\in R$.
\item[推移性(Transitivity):] もし$(x,y)\in R$でかつ$(y,z)\in R$ならば$(x,z)\in R$.
\end{description}
もし$R$が同値関係ならば, $(x,y)\in R$を示すために我々はしばしば$x\sim_R y$, あるいはもっと簡単に$x\sim y$と書く. 便利のため, 同値関係を記号$\sim$によって参照し, $\sim$は$X$上の同値関係と言うこともあるだろう.\index{a symbol!$\sim$}

%An {\em equivalence class of $\sim$}\index{equivalence relation!equivalence classes} is a subset $A\ss X$ such that
%\begin{itemize}
%\item $A$ is nonempty, $A\neq\emptyset$;
%\item if $x\in A$ and $x'\in A$, then $x\sim x'$; and 
%\item if $x\in A$ and $x\sim y$, then $y\in A$.
%\end{itemize}
%Suppose that $\sim$ is an equivalence relation on $X$. The {\em quotient of $X$ by $\sim$}\index{equivalence relation!quotient by}, denoted $X/\sim$\index{a symbol!$X/\sim$} is the set of equivalence classes of $\sim$.

\emph{$\sim$の同値類(equivalence class of $\sim$)}\index{equivalence relation!equivalence classes}とは以下を満たす部分集合$A\ss X$である.
\begin{itemize}
\item $A$は空集合ではない, $A\neq\emptyset$.
\item もし$x\in A$であり$x'\in A$ならば$x\sim x'$. 
\item もし$x\in A$であり$x\sim y$ならば$y\in A$.
\end{itemize}
$\sim$が$X$上の同値関係だと仮定する. \emph{$X$の$\sim$による商(quotient of $X$ by $\sim$)}\index{equivalence relation!quotient by}とは, $\sim$による同値類の集合であり, $X/\sim$\index{a symbol!$X/\sim$}と書く.

\end{definition}

\begin{example}

%Let $\ZZ$ denote the set of integers. Define a relation $R\ss\ZZ\times\ZZ$ by $$R=\{(x,y)\|\exists n\in\ZZ \tn{ such that } x+7n=y\}.$$ Then $R$ is an equivalence relation because $x+7*0=x$ (reflexivity); $x+7*n=y$ if and only if $y+7*(-n)= x$ (symmetry); and $x+7n=y$ and $y+7m=z$ together imply that $x+7(m+n)=z$ (transitivity).

$\ZZ$が整数の集合を示すとする. $$R=\{(x,y)\|\exists n\in\ZZ \tn{ such that } x+7n=y\}$$によって関係$R\ss\ZZ\times\ZZ$を定義する. この時$R$は同値関係となる. なぜなら$x+7*0=x$ (反射性), $x+7*n=y$のときかつそのときに限り$y+7*(-n)= x$であり(対称性), $x+7n=y$と$y+7m=z$は$x+7(m+n)=z$を伴う(推移性).

\end{example}

\begin{exercise}
%Let $X$ be the set of people on earth; define a binary relation $R\ss X\times X$ on $X$ as follows. For a pair $(x,y)$ of people, say $(x,y)\in R$ if $x$ spends a lot of time thinking about $y$. 
%\sexc Is this relation reflexive? 
%\next Is it symmetric? 
%\next Is it transitive?
%\endsexc
$X$を地球上の人間の集合とする. $X$上の二項関係$R\ss X\times X$を次のように定義する. 人間の対$(x,y)$に対して, もし$x$が$y$について考えることに多くの時間を費やしているならば$(x,y)\in R$とする.
\sexc この関係は反射的か? 
\next 対称的か?
\next 推移的か?
\endsexc
\end{exercise}

\begin{example}[Partitions]\label{ex:partition}

%An equivalence relation on a set $X$ can be thought of as a way of partitioning $X$. A {\em partition of $X$}\index{equivalence relation!as partition} consists of a set $I$, called {\em the set of parts}, and for every element $i\in I$ a subset $X_i\ss X$ such that two properties hold:
%\begin{itemize}
%\item every element $x\in X$ is in some part (i.e. for all $x\in X$ there exists $i\in I$ such that $x\in X_i$); and
%\item no element can be found in two different parts (i.e. if $x\in X_i$ and $x\in X_j$ then $i=j$).
%\end{itemize}

%Given a partition of $X$, we define an equivalence relation $\sim$ on $X$ by saying $x\sim x'$ if $x$ and $x'$ are in the same part (i.e. if there exists $i\in I$ such that $x,x'\in X_i$). The parts become the equivalence classes of this relation. Conversely, given an equivalence relation, one makes a partition on $X$ by taking $I$ to be the set of equivalence classes and for each $i\in I$ letting $X_i$ be the elements in that equivalence class.

集合$X$の上の同値関係は$X$の分配の方法と考えることができる. \emph{partition of $X$}\index{equivalence relation!as partition}は, \emph{the set of parts}と呼ばれる集合$I$と, 全ての要素$i\in I$に対する部分集合$X_i\ss X$からなり, これらは以下の二つの性質を持つ.
\begin{itemize}
\item それぞれの要素$x\in X$はどれかの部分に属している (i.e. 全ての要素$x\in X$に対して$x\in X_i$なるところの$i\in I$が存在している).
\item 二つの異なった部分に属する要素はない (i.e. もし$x\in X_i$かつ$x\in X_j$ならば$i=j$).
\end{itemize}

partition of $X$が与えられたとき, 我々は$X$上の同値関係$\sim$を, もし$x$と$x'$が同じ部分に属しているならば(i.e. もし$x,x'\in X_i$なるところの$i\in I$が存在しているならば) $x\sim x'$ として定義できる. これらの部分はこの関係の同値類になる. 逆に, 同値関係が与えられたとき, $I$を同値類の集合, それぞれの$i\in I$に対して$X_i$を同値類の要素となるようにすることによって, $X$上の部分を作ることができる.

\end{example}

\begin{exercise}
%Let $X$ and $B$ be sets and let $f\taking X\to B$ be a function. Define a subset $R\ss X\times X$ by $$R=\{(x,y)\|f(x)=f(y)\}.$$ 
%\sexc Is $R$ an equivalence relation? 
%\next Are all equivalence relations on $X$ obtainable in this way (as the fibers of some function having domain $X$)?
%\next Does this viewpoint on equivalence classes relate to that of Example \ref{ex:partition}?
%\endsexc
$X$と$B$を集合とし, $f\taking X\to B$を函数とする. 部分集合$R\ss X\times X$を$$R=\{(x,y)\|f(x)=f(y)\}$$によって定義する.
\sexc $R$は同値関係だろうか? 
\next $X$上の全ての同値関係をこの方法で(始域$X$を持つある函数のファイバーとして)得ることができるだろうか?
\next 同値関係に対するこの観点はExample \ref{ex:partition}でのそれと関係しているだろうか?
\endsexc
\end{exercise}

\begin{exercise}
%Take a set $I$ of sets; i.e. suppose that for each element $i\in I$ you are given a set $X_i$. For every two elements $i,j\in I$ say that $i\sim j$ if $X_i$ and $X_j$ are isomorphic. Is this relation an equivalence relation on $I$?
ある集合に対して集合$I$を取る. i.e. それぞれの要素$i\in I$に対して集合$X_i$が与えられると仮定する. 全ての二つの要素$i,j\in I$に対して$X_i$と$X_j$が同型ならば$i\sim j$であるとする. この関係は$I$上の同値関係だろうか?
\end{exercise}

\begin{lemma}[Generating equivalence relations]\label{lemma:generating ERs}

%Let $X$ be a set and $R\ss X\times X$ a subset. There exists a relation $S\ss X\times X$ such that
%\begin{itemize}
%\item $S$ is an equivalence relation,
%\item $R\ss S$, and
%\item for any equivalence relation $S'$ such that $R\ss S'$, we have $S\ss S'$.
%\end{itemize}
%The relation $S'$ will be called {\em the equivalence relation generated by $R$}.\index{equivalence relation!generated}

$X$を集合, $R\ss X\times X$を部分集合とする. 以下を満たすところの関係$S\ss X\times X$が存在する.
\begin{itemize}
\item $S$は同値関係であるる
\item $R\ss S$.
\item $R\ss S'$なる任意の同値関係$S'$に対して$S\ss S'$.
\end{itemize}

\end{lemma}

\begin{proof}

%Let $L_R$ be the set of all equivalence relations on $X$ that contain $R$; in other words, each element $\ell\in L_R$ is an equivalence relation, $\ell\in X\times X$. The set $L_R$ is non-empty because $X\times X\ss X\times X$ is an equivalence relation. Let $S$ denote the set of pairs $(x_1,x_2)\in X\times X$ that appear in every element of $L_R$. Note that $R\ss S$ by definition. We need only show that $S$ is an equivalence relation.

$L_R$を, $R$を含む$X$上の全ての同値関係の集合だと仮定する. 別の言い方をすれば, それぞれの要素$\ell\in L_R$は同値関係であり, $\ell\in X\times X$である. 集合$L_R$は空ではない. なぜならば$X\times X\ss X\times X$は同値関係だからである. $S$は$L_R$の全ての要素の中に現われる対$(x_1,x_2)\in X\times X$の集合を表示するものとする. 定義より$R\ss S$である. 我々は$S$が同値関係であることさえ示せばよい.

%It is clearly reflexive, because $R$ is. If $(x,y)\in S$ then $(x,y)\in\ell$ for all $\ell\in L_R$. But since each $\ell$ is an equivalence relation, $(y,x)\in\ell$ too, so $(y,x)\in S$. This shows that $S$ is symmetric. The proof that it is transitive is similar: if $(x,y)\in S$ and $(y,z)\in S$ then they are both in each $\ell$ which puts $(x,z)$ in each $\ell$, which puts it in $S$.

これは明らかに反射的である. なぜなら$R$がそうだからである. もし$(x,y)\in S$ならば全ての$\ell\in L_R$に対して$(x,y)\in\ell$である. しかし$\ell$は同値関係だから, $(y,x)\in\ell$でもあり, よって$(y,x)\in S$である. これは$S$が対称的であることを示している. 推移的であることの証明も同様である. もし$(x,y)\in S$かつ$(y,z)\in S$ならば, 両者はともに$\ell$に含まれ, これは$(x,z)$がそれぞれの$\ell$にあることを意味し, これは$(x,z)$が$S$に含まれることを意味する.

\end{proof}

\begin{remark}

%Let $X$ be a set and $R\ss X\times X$ a relation. The proof of Lemma \ref{lemma:generating ERs} has the benefit of working even if $|X|\geq\infty$, but it has the cost that it is not very intuitive, nor useful in practice when $X$ is finite. The intuitive way to think about the idea of equivalence relation generated by $R$ is as follows.
%\begin{enumerate}
%\item First add to $R$ what is demanded by reflexivity, $R_1:=R\cup\{(x,x)\|x\in X\}$.
%\item Then add to $R$ what is demanded by symmetry, $R_2:=R_1\cup\{(x,y)\|(y,x)\in R_1\}.$
%\item Finally, add to $R$ what is demanded by transitivity, $$S=R_2\cup\{(x,z)\|(x,y)\in R_2, \tn{ and } (y,z)\in R_2\}.$$
%\end{enumerate}

$X$を集合とし$R\ss X\times X$を関係とする. Lemma \ref{lemma:generating ERs}の証明は$|X|\geq\infty$のときでさえうまくいくという利点を持つが, その代わりにそれほど直感的でなくまた$X$が有限集合のときにはそれほど便利ではないという犠牲を払っている. $R$によって生成される同値関係というアイデアを考える直感的な道は以下である.
\begin{enumerate}
\item 最初に反射性に必要なものを$R$に加える, $R_1:=R\cup\{(x,x)\|x\in X\}$.
\item 次に対称性に必要なものを$R$に加える, $R_2:=R_1\cup\{(x,y)\|(y,x)\in R_1\}.$
\item 最後に推移性に必要なものを$R$に加える, $$S=R_2\cup\{(x,z)\|(x,y)\in R_2, \tn{ and } (y,z)\in R_2\}.$$
\end{enumerate}

\end{remark}

\begin{exercise}
%Consider the set $\RR$ of real numbers. Draw the coordinate plane $\RR\times\RR$, give it coordinates $x$ and $y$. A binary relation on $\RR$ is a subset $S\ss\RR\times\RR$, which can be drawn as a set of points in the plane. 
%\sexc Draw the relation $\{(x,y)\|y=x^2\}$. 
%\next Draw the relation $\{(x,y)\|y\geq x^2\}.$
%\next Let $S_0$ be the equivalence relation on $\RR$ generated (in the sense of Lemma \ref{lemma:generating ERs}) by the empty set. Draw $S$ as a subset of the plane.
%\next Consider the equivalence relation $S_1$ generated by $\{(1,2),(1,3)\}$. Draw $S_1$ in the plane. Highlight the equivalence class containing $(1,2)$.
%\next The reflexivity property and the symmetry property have pleasing visualizations in $\RR\times\RR$; what are they? 
%\next Is there a nice heuristic for visualizing the transitivity property?
%\endsexc
実数の集合$\RR$を考える. 座標平面$\RR\times\RR$を描き, 座標を$x$と$y$とする. $\RR$上の二項関係は$S\ss\RR\times\RR$の部分集合であり, これは平面上の点の集合として書くことができる.
\sexc 関係$\{(x,y)\|y=x^2\}$を描け.
\next 関係$\{(x,y)\|y\geq x^2\}$を描け.
\next $S_0$を, 空集合から(Lemma \ref{lemma:generating ERs}の意味で)生成された$\RR$上の同値関係とする. $S$を平面の分集合として描け. 
\next $\{(1,2),(1,3)\}$によって生成される同値関係$S_1$について考える $S_1$を平面上に描け. $(1,2)$を含む同値類を強調せよ.
\next 反射性と対称性は$\RR\times\RR$上で気持ちよく視覚化できる性質がある. それはなにか?
\next 推移性の視覚化に対してはよい発見的方法はあるだろうか?
\endsexc
\end{exercise}

\begin{exercise}
%Consider the binary relation $R=\{(n,n+1)\|n\in\ZZ\}\ss\ZZ\times\ZZ$. 
%\sexc What is the equivalence relation generated by $R$? 
%\next How many equivalence classes are there?
%\endsexc
二項関係$R=\{(n,n+1)\|n\in\ZZ\}\ss\ZZ\times\ZZ$を考える. 
\sexc $R$によって生成される同値関係は何か? 
\next 同値類の数はどれだけあるか?
\endsexc
\end{exercise}

\begin{exercise}
%Suppose $N$ is a network (or graph). Let $X$ be the nodes of the network, and let $R\ss X\times X$ denote the relation such that $(x,y)\in R$ iff there exists an arrow connecting $x$ to $y$.
%\footnote{The word {\em iff} means ``if and only if". In this case we are saying that the pair $(x,y)$ is in $R$ if and only if there exists an arrow connecting $x$ and $y$.\index{iff}}
%\sexc What is the equivalence relation $\sim$ generated by $R$? 
%\next What is the quotient $X/\sim$?
%\endsexc
$N$をネットワーク(あるいはグラフ)と仮定する. $X$をネットワークのノードとし, $R\ss X\times X$を関係とし, $(x,y)\in R$なのは$x$と$y$を結んでいる矢印が存在しているときかつその時に限るとする. 
\footnote{原注: 単語\emph{iff}は``if and only if''を意味する. この場合は, 対$(x,y)$が$R$に存在するのは$x$と$y$を結ぶ矢が存在するときかつその時に限るということを言っている.\index{iff}}
\sexc $R$によって生成される同値関係$\sim$は何か?
\next 商$X/\sim$は何か?
\endsexc
\end{exercise}

%%%% Subsection %%%%

%\subsection{Pushouts}\label{sec:pushouts}
\subsection{押し出し}\label{sec:pushouts}

%\begin{definition}[Pushout]\label{def:pushout}
\begin{definition}[押し出し]\label{def:pushout}

%Suppose given the diagram of sets and functions below:
%\begin{align}\label{dia:pushout}
%\xymatrix{W\ar[r]^f\ar[d]_g&X\\Y}
%\end{align}
%Its {\em fiber sum},\index{fiber sum} denoted $X\sqcup_WY$, is defined as the quotient of $X\sqcup W\sqcup Y$ by the equivalence relation $\sim$ generated by $w\sim f(w)$ and $w\sim g(w)$ for all $w\in W$.
%$$X\sqcup_WY:=(X\sqcup W\sqcup Y)/\sim \hsp\tn{where } \forall w\in W,\;\;  w\sim f(w)\;\;\tn{ and }\;\; w\sim g(w).$$ 
%There are obvious inclusions $i_1\taking X\to X\sqcup_WY$ and $i_2\taking Y\to X\sqcup_WY$.
%\footnote{Note that our term inclusions is not too good, because it seems to suggest that $i_1$ and $i_2$ are injective (see Definition \ref{def:inj,surj,bij}) and this is not always the case.}
%Note that if $Z=X\sqcup_WY$ then the diagram
%\begin{align}\label{dia:pushout sets}
%\xymatrix{W\ar[r]^g\ar[d]_f&Y\ar[d]^{i_2}\\X\ar[r]_-{i_1}&Z\lrlimit}
%\end{align} 
%commutes. Given the setup of Diagram \ref{dia:pushout} we define the {\em pushout of $X$ and $Y$ over $W$} to be any set $Z$ for which we have an isomorphism $Z\To{\iso}X\sqcup_WY$. The corner symbol $\ulcorner$ in Diagram \ref{dia:pushout sets} indicates that $Z$ is the pushout.\index{a symbol!$\ulcorner$}

以下の集合と函数の図式が与えられたとする.
\begin{align}\label{dia:pushout}
\xymatrix{W\ar[r]^f\ar[d]_g&X\\Y}
\end{align}
\emph{ファイバー和(fiber sum)}\index{fiber sum}は, 全ての$w\in W$に対して$w\sim f(w)$および$w\sim g(w)$によって生成される同値関係$\sim$による$X\sqcup W\sqcup Y$の商によって定義され, $X\sqcup_WY$と表記される.
$$X\sqcup_WY:=(X\sqcup W\sqcup Y)/\sim \hsp\tn{where } \forall w\in W,\;\;  w\sim f(w)\;\;\tn{ and }\;\; w\sim g(w).$$ 
明らかな包含(inclusion) $i_1\taking X\to X\sqcup_WY$および$i_2\taking Y\to X\sqcup_WY$が存在する.
\footnote{我々が用語として使った包含はそれほどよいものではない. なぜならこの用語は$i_1$と$i_2$が単射(Definition \ref{def:inj,surj,bij}を見よ)であることを示唆しているようにみえるが, これは常にはなりたたないからだ.}
もし$Z=X\sqcup_WY$ならば, 図式
\begin{align}\label{dia:pushout sets}
\xymatrix{W\ar[r]^g\ar[d]_f&Y\ar[d]^{i_2}\\X\ar[r]_-{i_1}&Z\lrlimit}
\end{align} 
が可換であることに注意せよ. Diagram \ref{dia:pushout}の設定が与えられたとき, 我々は\emph{$Z$上の$X$と$Y$の押し出し(pushout of $X$ and $Y$ over $W$)}を, 同型写像$Z\To{\iso}X\sqcup_WY$があるところの任意の集合$Z$として定義することができる. Diagram \ref{dia:pushout sets}の鉤記号$\ulcorner$は$Z$が押し出しであることを示す.\index{a symbol!$\ulcorner$}

\end{definition}

\begin{example}

%Let $X=\{x\in\RR\|0\leq x\leq1\}$ be the set of numbers between 0 and 1, inclusive, let $Y=\{y\in\RR\|1\leq y\leq 2\}$ by the set of numbers between 1 and 2, inclusive, and let $W=\{1\}$. Then the pushout $X\From{f} W\To{g} Y$, where $f$ and $g$ are the ``obvious" functions ($1\mapsto 1$) is $X\sqcup_WY\iso\{z\in\RR\|0\leq z\leq 2\}$, as expected. When we eventually get to general colimits, one can check that the whole real line can be made by patching together intervals in this way.

$X=\{x\in\RR\|0\leq x\leq1\}$を0と1の間の数の閉集合とし, $Y=\{y\in\RR\|1\leq y\leq 2\}$を1と2の間の数の閉集合とし, $W=\{1\}$とする. このとき押し出し$X\From{f} W\To{g} Y$は, 予想される通り$X\sqcup_WY\iso\{z\in\RR\|0\leq z\leq 2\}$である. ここで$f$と$g$は``自明な''函数($1\mapsto 1$)である. 最終的に一般的な余極限をとる際には, この方法による区間の切り貼りによって全実軸が作れることを確かめることができる.

\end{example}

\begin{example}[Pushout]\label{ex:pushout}

%In each example below, the diagram to the right is intended to be a pushout of the diagram to the left.  The new object, $D$, is the union of $B$ and $C$, but instances of $A$ are equated to their $B$ and $C$ aspects.  This will be discussed after the two diagrams.

以下のそれぞれの例では, 右側の図式は左側の図式の押し出しであることを意図している. 新しい対象$D$は$B$と$C$の合併であるが, $A$の実例はその$B$と$C$の様相と同等に扱われる. これに関しては以下の二つの図式の後に議論することにする.

\begin{align}
\label{dia:po1}\fbox{\xymatrixnocompile{\obox{A}{.7in}{a cell in the shoulder}\LA{r}{is}\LAL{d}{is}&\obox{C}{.6in}{a cell in the arm}\\\obox{B}{.7in}{a cell in the torso}}}\hsp&\fbox{\xymatrix{\obox{A}{.7in}{a cell in the shoulder}\LA{r}{is}\LAL{d}{is}&\obox{C}{.6in}{a cell in the arm}\LA{d}{}\\\obox{B}{.7in}{a cell in the torso}\LA{r}{}&\obox{D=B\sqcup_AC}{.8in}{a cell in the torso or arm}}}
\end{align}
%In the left-hand olog (\ref{dia:po1}, the two arrows are inclusions: the author considers every cell in the shoulder to be both in the arm and in the torso. The pushout is then just the union, where cells in the shoulder are not double-counted.
左手側のolog \eqref{dia:po1}では, 二つの矢印は単射である. すなわちologの作者は肩の全ての細胞は腕と胴体の両方にあると考えてる. このとき押し出しは単純に合併であり, 肩の細胞は二重に数えられてはいない.

\begin{align}\label{dia:po2}\fbox{\xymatrixnocompile@=18pt{\obox{A}{.8in}{\rr a college mathematics course}\LA{r}{yields}\LAL{d}{is}&\obox{C}{.8in}{an utterance of the phrase ``too hard"}\\\obox{B}{.6in}{\rr a college course}}}\hsp&\fbox{\xymatrixnocompile@=18pt{\obox{A}{.8in}{\rr a college mathematics course}\LA{r}{yields}\LAL{d}{is}&\obox{C}{.8in}{an utterance of the phrase ``too hard"}\LA{d}{}\\\obox{B}{.6in}{\rr a college course}\LA{r}{}&\obox{\parbox{.6in}{\vspace{.1in}\tiny$D=B\!\sqcup_A\!C$}}{1in}{\rr a college course, where every mathematics course is replaced by an utterance of the phrase ``too hard"}}}
\end{align}

%In Olog (\ref{dia:po1}), the shoulder is seen as part of the arm and part of the torso.  When taking the union of these two parts, we do not want to ``double-count" the shoulder (as would be done in the coproduct $B\sqcup C$, see Example \ref{ex:coproduct2}).  Thus we create a new type $A$ for cells in the shoulder, which are considered the same whether viewed as cells in the arm or cells in the torso.  In general, if one wishes to take two things and glue them together, with $A$ as the glue and with $B$ and $C$ as the two things to be glued, the union is the pushout $B\sqcup_AC$. (A nice image of this can be seen in the setting of topological spaces, see Example \ref{ex:pushout in Top}.)

olog \eqref{dia:po1}では, 肩は腕の一部部分でありまた胴体の一部分であるようである. この時はこれら二つの合併をとる際に(余積$B\sqcup C$を取った時のように)肩を二重に数えたくない(Example \ref{ex:coproduct2}も見よ). よって我々は肩の細胞に対して新しい型$A$を作り, それを腕の細胞か胴体の細胞かのどちらかとみなすことにする. 一般的に, もし二つのものを取ってきてそれをのり付けし一緒にしたいならば, $A$はのりとして$B$と$C$はそれによってのり付けられる二つのものとして, その合併は押し出し$B\sqcup_AC$である. (これについのよいイメージが位相空間の設定で見ることができる. Example \ref{ex:pushout in Top}を見よ.)

%In Olog (\ref{dia:po2}), if every mathematics course is simply ``too hard," then when reading off a list of courses, each math course will not be read aloud but simply read as ``too hard."  To form $D$ we begin by taking the union of $B$ and $C$, and then we consider everything in $A$ to be the same whether one looks at it as a course or as the phrase ``too hard."  The math courses are all blurred together as one thing.  Thus we see that the power to equate different things can be exercised with pushouts.

olog \eqref{dia:po2}で, もし全ての数学コースが単純に``難しすぎる''ならば, その時はコースの一覧を読み上げる際, それぞれの数学のコースは声に出して読みあげる必要はなく単に``難しすぎる''と読めばよいだろう. $D$を形成するのにはまず$B$と$C$の合併をとり, 次に$A$の全てがコースか単語``too hard''で表現できるかのどちらかであるものは全て同じものとみなす. 数学コースはすべて一緒になって一つのものにぼやかされる. よって異なるもの同等視する力は押し出しによって行使可能であることが分かる.

\end{example}

\begin{exercise}
%Let $W,X,Y$ be as drawn and $f\taking W\to X$ and $g\taking W\to Y$ the indicated functions. 
$W$, $X$, $Y$が図のようであるとし, $f\taking W\to X$および$g\taking W\to Y$が図示された函数とする. 
\begin{center}
\includegraphics[height=2in]{setPushout}
\end{center}
%The pushout of the diagram $X\Fromm{f}W\Too{g}Y$ is a set $P$. Write down the cardinality of $P\iso\ul{n}$ as a natural number $n\in\NN$.
図式の押し出し$X\Fromm{f}W\Too{g}Y$は集合$P$である. $P\iso\ul{n}$の濃度を自然数$n\in\NN$として書き下せ.
\end{exercise}

\begin{exercise}
%Suppose that $W=\emptyset$; what can you say about $X\sqcup_WZ$? 
$W=\emptyset$と仮定する. $X\sqcup_WZ$について何が言えるだろうか?
\end{exercise}

\begin{exercise}
%Let $W:=\NN=\{0,1,2,\ldots\}$ denote the set of natural numbers, let $X=\ZZ$ denote the set of integers, and let $Y=\singleton$ denote a one-element set. Define $f\taking W\to X$ by $f(w)= -(w+1)$, and define $g\taking W\to Y$ to be the unique map. Describe the set $X\sqcup_WY$.
$W:=\NN=\{0,1,2,\ldots\}$は自然数を示しているとし, $X=\ZZ$は整数を示しているとし, $Y=\singleton$は一要素集合を示しているとする. $f\taking W\to X$を$f(w)= -(w+1)$によって定義し, また$\taking W\to Y$を一意な写像になるように定義せよ. $X\sqcup_WY$を表現せよ.
\end{exercise}

\begin{exercise}
%Let $i\taking R\ss X\times X$ be an equivalence relation (see Example \ref{ex:subset as function} for notation). Composing with the projections $\pi_1,\pi_2\taking X\times X\to X$, we have two maps $\pi_1\circ i,\taking R\to X$ and $\pi_2\circ i\taking R\to X$. 
%\sexc What is the pushout $$X\From{\pi_1\circ i}R\To{\pi_2\circ i}X?$$ 
%\next If $i\taking R\ss X\times X$ is not assumed to be an equivalence relation, we can still define the pushout above. Is there a relationship between the pushout $X\From{\pi_1\circ i}R\To{\pi_2\circ i}X$ and the equivalence relation generated by $R\ss X\times X$?
%\endsexc
$i\taking R\ss X\times X$を同値関係とする(記法に関してはExample \ref{ex:subset as function}を見よ). 射影$\pi_1,\pi_2\taking X\times X\to X$を合成することにより, 二つの写像$\pi_1\circ i,\taking R\to X$と$\pi_2\circ i\taking R\to X$が得られる.
\sexc $X\From{\pi_1\circ i}R\To{\pi_2\circ i}X$の押し出しは何か?
\next $i\taking R\ss X\times X$が同値関係と仮定できない場合も, 上記の押し出しを定義することはできる. 押し出し$X\From{\pi_1\circ i}R\To{\pi_2\circ i}X$と$R\ss X\times X$によって生成される同値関係との間に何か関係があるだろうか?
\endsexc
\end{exercise}


%\begin{lemma}[Universal property for pushout]\label{lemma:up for po}
\begin{lemma}[押し出しの普遍的性質]\label{lemma:up for po}

%Suppose given the diagram of sets and functions as below.
以下のように集合と函数の図式が与えられたとする.
\begin{align*}
\xymatrix{W\ar[r]^u\ar[d]_t&Y\\
X}
\end{align*}
%For any set $A$ and commutative solid arrow diagram as below (i.e. functions $f\taking X\to A$ and $g\taking Y\to A$ such that $f\circ t=g\circ u$), 
任意の集合$A$と, 以下のような可換な実線の矢印の図式(i.e. $f\circ t=g\circ u$であるところの函数$f\taking X\to A$および$g\taking Y\to A$)に対して, 
\begin{align}\label{dia:universal property of po}
\xymatrix{
&W\ar[dr]^u\ar[dl]_t\\
X\ar@/_1pc/[dddr]_{i_1}\ar[rd]_f&&Y\ar@/^1pc/[dddl]^{i_2}\ar[dl]^g\\
&A\\\\
&X\sqcup_WY\ar@{-->}[uu]^{\exists!}
}
\end{align}
%there exists a unique arrow $\po{f}{g}{W}\taking X\sqcup_WY\to A$ making everything commute, $$f=\po{f}{g}{W}\circ i_1\hsp\text{and}\hsp g=\po{f}{g}{W}\circ i_2.$$
全てを可換にする矢印$\po{f}{g}{W}\taking X\sqcup_WY\to A$が一意に存在する.
$$f=\po{f}{g}{W}\circ i_1\hsp\text{and}\hsp g=\po{f}{g}{W}\circ i_2.$$
\end{lemma}

%%%% Subsection %%%%

%\subsection{Other finite colimits}
\subsection{他の有限余極限}

%\begin{definition}\label{def:coequalizer}[Coequalizer]\index{coequalizer}
\begin{definition}[余等価子]\label{def:coequalizer}\index{coequalizer}

%Suppose given two parallel arrows 
二本の平行な矢印が与えられたとする.
\begin{align}\label{dia:coequalizer}
\xymatrix{X\ar@<.5ex>[r]^f\ar@<-.5ex>[r]_g&Y.}\hspace{1in}\xymatrix{X\ar@<.5ex>[r]^f\ar@<-.5ex>[r]_g&Y\ar[r]^-q&Coeq(f,g)}
\end{align}
%The {\em coequalizer of $f$ and $g$} is the commutative diagram as to the right in (\ref{dia:coequalizer}), where we define $$Coeq(f,g):=Y\;/\;f(x)\sim g(x)$$ i.e. the coequalizer of $f$ and $g$ is the quotient of $Y$ by the equivalence relation generated by $\{(f(x),g(x))\|x\in X\}\ss Y\times Y$
\emph{$f$と$g$の余等価子(coequalizer of $f$ and $g$)}は\eqref{dia:coequalizer}の右側のような可換図式である. ここで$$Coeq(f,g):=Y\;/\;f(x)\sim g(x)$$と定義した. i.e. $f$と$g$の余等価子は, $\{(f(x),g(x))\|x\in X\}\ss Y\times Y$によって生成される同値関係による, $Y$の商である.

\end{definition}

\begin{exercise}
%Let $X=\RR$ be the set of real numbers. What is the coequalizer of the two maps $X\to X$ given by $x\mapsto x$ and $x\mapsto (x+1)$ respectively?
$X=\RR$を実数の集合とする. $x\mapsto x$と$x\mapsto (x+1)$によって与えられる二つの写像$X\to X$の余等価子は何だろうか?
\end{exercise}

\begin{exercise}
%Find a universal property enjoyed by the coequalizer of two arrows.
二つの矢印の余等価子が持っている普遍的性質を見出せ.
\end{exercise}

%\begin{exercise}[Initial object]\label{exc:initial set}
\begin{exercise}[始対象(initial object)]\label{exc:initial set}\index{initial object!in $\Set$}
%An initial set is a set $S$ such that for every set $A$, there exists a unique function $S\to A$. 
%\sexc Find an initial set. 
%\next Do you think that the notion {\em initial set} belongs in this section (Section \ref{sec:finite colimits})? How so? If coproducts, pushouts, and coequalizers are all colimits, what do colimits have in common?
%\endsexc
始集合は, 全ての集合$A$に対して函数$S\to A$が一意に存在するところの集合$S$である.
\sexc 始集合を見出せ.
\next この章(Section \ref{sec:finite colimits})における始集合の概念についてなにか思っただろうか? もし余積, 押し出し, 余等価子を全て余極限(colimit)と呼ぶならば, 余極限が共通して持つものはなんだろうか?
\endsexc
\end{exercise}


%%%%%% Section %%%%%%

%\section{Other notions in $\Set$}
\section{$\Set$におけるその他の様々な概念}

%In this section we discuss some left-over notions in the category of Sets.
この章では, 我々はまだ触れられていない集合の圏における様々な概念について議論する.

%%%% Subsection %%%%

\subsection{Retractions}

\begin{definition}

%Suppose we have a function $f\taking X\to Y$ and a function $g\taking Y\to X$ such that $g\circ f=\id_X$. In this case we call $f$ a {\em retract section} and we call $g$ a {\em retract projection}. \index{retraction}

$g\circ f=\id_X$であるところの函数$f\taking X\to Y$と函数$g\taking Y\to X$があるとする. この場合, $f$を\emph{retract section}と呼び, $g$を\emph{retract projection}と呼ぶ. \index{retraction}

\end{definition}

\begin{exercise}
%Create an olog that includes sets $X$ and $Y$, and functions $f\taking X\to Y$ and $g\taking Y\to X$ such that $g\circ f=\id_X$ but such that $f\circ g\neq\id_Y$; that is, such that $f$ is a retract section but not an isomorphism.
集合$X$, $Y$と, 函数$f\taking X\to Y$, $g\taking Y\to X$を含み, $g\circ f=\id_X$であるが$f\circ g\neq\id_Y$であるologを作成せよ. これは, $f$はretract sectionであるが同型写像でないことを意味している.
\end{exercise}

%%%% Subsection %%%%

\subsection{Curry化}\label{sec:currying}\index{currying}\index{materials!force extension curves}

%Currying is the idea that when a function takes many inputs, we can input them one at a time or all at once. For example, consider the function that takes a material $M$ and an extension $E$ and returns the force transmitted through the material when it is pulled to that extension. This is a function $e\taking \fakebox{a material}\times\fakebox{an extension}\to\fakebox{a force}$. This function takes two inputs at once, but it is convenient to ``curry" the second input. Recall that $\Hom_\Set(\fakebox{an extension},\fakebox{a force})$ is the set of theoretical force-extension curves. Currying transforms $e$ into a function $$e'\taking\fakebox{a material}\to\Hom_\Set(\fakebox{an extension},\fakebox{a force}).$$ This is a more convenient way to package the same information. 

Curry化(currying)は, 函数が複数の入力を取るときに, 一回につき一つ入力するか一度に全部入力するかのどちらかが可能である, というアイデアである. 例えば, 物質$M$と伸び$E$を取り, その伸びを得るまで物質をひっぱったときに物質が生む力を返す函数を考える. これは函数$e\taking \fakebox{a material}\times\fakebox{an extension}\to\fakebox{a force}$である. この函数は二つの入力を一度にとるが, 二番目の入力を``curry化''するのもまた便利である. $\Hom_\Set(\fakebox{an extension},\fakebox{a force})$は理論的な応力-ひずみ曲線の集合であることを思い出そう. Curry化は$e$を函数$$e'\taking\fakebox{a material}\to\Hom_\Set(\fakebox{an extension},\fakebox{a force})$$に変換する. これは同じ情報をパッケージ化するのにあたってより便利な方法である.

%In fact, it may be convenient to repackage this information another way. For any extension, we may want the function that takes a material and returns how much force it can transmit at that extension. This is a function $$e''\taking\fakebox{an extension}\to\Hom_\Set(\fakebox{a material},\fakebox{a force}).$$ 

実際のところ, この情報はまた別の方法で再パッケージ化するのが便利かもしれない. 任意の伸びに対して, なんらかの物質を受け取ってその伸びに対応して生み出された力を返す函数が欲しいかもしれない. これは函数$$e''\taking\fakebox{an extension}\to\Hom_\Set(\fakebox{a material},\fakebox{a force})$$である.

\begin{notation}\index{exponentials ! in $\Set$}
%Let $A$ and $B$ be sets. We sometimes denote the set of functions from $A$ to $B$ by 
$A$と$B$を集合とする. 我々はしばしば$A$から$B$への函数の集合を以下のように表記する.
\begin{align}\label{dia:exponential sets}
B^A:=\Hom_\Set(A,B).
\end{align}
\end{notation}

\begin{exercise}
%For a finite set $A$, let $|A|\in\NN$ denote the cardinality of (number of elements in) $A$. If $A$ and $B$ are both finite (including the possibility that one or both are empty), is it always true that $|B^A|=|B|^{|A|}$?
有限集合$A$に対し, $|A|\in\NN$は$A$の濃度(要素数)を表すとする. $A$と$B$がともに有限であるとき(どちらか一方あるいは両方が空集合である場合を含む), $|B^A|=|B|^{|A|}$は常に真だろうか?
\end{exercise}

\begin{proposition}[Currying]\label{prop:curry}

%Let $A$ denote a set. For any sets $X,Y$ there is a bijection 
%\begin{align}\label{dia:curry bijection}
%\phi\taking\Hom_\Set(X\times A,Y)\To{\iso}\Hom_\Set(X,Y^A).
%\end{align}

$A$は集合を表すとする. 任意の集合$X$, $Y$に対し, 全単射
\begin{align}\label{dia:curry bijection}
\phi\taking\Hom_\Set(X\times A,Y)\To{\iso}\Hom_\Set(X,Y^A).
\end{align}
が存在する.

\end{proposition}

\begin{proof}

%Suppose given $f\taking X\times A\to Y$. Define $\phi(f)\taking X\to Y^A$ as follows: for any $x\in X$ let $\phi(f)(x)\taking A\to Y$ be defined as follows: for any $a\in A$, let $\phi(f)(x)(a):=f(x,a)$. 

$f\taking X\times A\to Y$が与えられたとする. $\phi(f)\taking X\to Y^A$を以下のように定義する. 任意の$x\in X$に対して$\phi(f)(x)\taking A\to Y$を以下のように定義する. 任意の$a\in A$に対して$\phi(f)(x)(a):=f(x,a)$.

%We now construct the inverse, $\psi\taking\Hom_\Set(X,Y^A)\to\Hom_\Set(X\times A,Y)$. Suppose given $g\taking X\to Y^A$. Define $\psi(g)\taking X\times A\to Y$ as follows: for any pair $(x,a)\in X\times A$ let $\psi(g)(x,a):=g(x)(a)$. 

今度は逆写像, $\psi\taking\Hom_\Set(X,Y^A)\to\Hom_\Set(X\times A,Y)$を構成しよう. $g\taking X\to Y^A$が与えられたとする. $\psi(g)\taking X\times A\to Y$を以下のように定義する. 任意の対$(x,a)\in X\times A$に対して$\psi(g)(x,a):=g(x)(a)$.

%Then for any $f\in\Hom_\Set(X\times A,Y)$ we have $\psi\circ\phi(f)(x,a)=\phi(f)(x)(a)=f(x,a)$, and for any $g\in\Hom_\Set(X,Y^A)$ we have $\phi\circ\psi(g)(x)(a)=\psi(g)(x,a)=g(x)(a)$, Thus we see that $\phi$ is an isomorphism as desired.

この時, 任意の$f\in\Hom_\Set(X\times A,Y)$に対して$\psi\circ\phi(f)(x,a)=\phi(f)(x)(a)=f(x,a)$が得られ, 任意の$g\in\Hom_\Set(X,Y^A)$に対して$\phi\circ\psi(g)(x)(a)=\psi(g)(x,a)=g(x)(a)$が得られる. これによって$\phi$は期待した通り同型写像であることが分かる.

\end{proof}

\begin{exercise}
%Let $X=\{1,2\}, A=\{a,b\}$, and $Y=\{x,y\}$. 
%\sexc\label{part:three distinct} Write down three distinct elements of $L:=\Hom_\Set(X\times A,Y)$. 
%\next Write down all the elements of $M:=\Hom_\Set(A,Y)$. 
%\next For each of the three elements $\ell\in L$ you chose in part (\ref{part:three distinct}), write down the corresponding function $\phi(\ell)\taking X\to M$ guaranteed by Proposition \ref{prop:curry}.
%\endsexc
$X=\{1,2\}, A=\{a,b\}$, $Y=\{x,y\}$とする.
\sexc\label{part:three distinct} $L:=\Hom_\Set(X\times A,Y)$の異なる要素を三つ書き下せ.
\next $M:=\Hom_\Set(A,Y)$の要素を全て書き下せ.
\next \eqref{part:three distinct}で選んだ三要素$\ell\in L$のそれぞれについて, Proposition \ref{prop:curry}によって存在が保証されているところの対応する函数$\phi(\ell)\taking X\to M$を書き下せ.
\endsexc
\end{exercise}

\begin{exercise}\label{exc:evaluation}\index{exponentials!evaluation of}
%Let $A$ and $B$ be sets. We know that $\Hom_\Set(A,B)=B^A$, so we have a function $\id_{B^A}\taking\Hom_\Set(A,B)\to B^A$. Look at Proposition \ref{prop:curry}, making the substitutions $X=\Hom_\Set(A,B)$, $Y=B$, and  $A=A$. Consider the function $$\phi^\m1\taking\Hom_\Set(\Hom_\Set(A,B),B^A)\to\Hom_\Set(\Hom_\Set(A,B)\times A,B)$$ obtained as the inverse of (\ref{dia:curry bijection}). We have a canonical element $\id_{B^A}$ in the domain of $\phi^\m1$. We can apply the function $\phi^\m1$ and obtain an element $ev=\phi^\m1(\id_{B^A})\in\Hom_\Set(\Hom_\Set(A,B)\times A,B)$, which is itself a function, $$ev\taking\Hom_\Set(A,B)\times A\to B.$$ 
%\sexc Describe the function $ev$ in terms of how it operates on elements in its domain. 
%\next Why might one be tempted to denote this function by $ev$?
%\endsexc
%
$A$と$B$を集合とする. 我々は$\Hom_\Set(A,B)=B^A$であることを知っている. よって函数$\id_{B^A}\taking\Hom_\Set(A,B)\to B^A$が得られる. Proposition \ref{prop:curry}において, $X=\Hom_\Set(A,B)$, $Y=B$, $A=A$の代入を行なってみよう. \eqref{dia:curry bijection}の逆として得られる函数$$\phi^\m1\taking\Hom_\Set(\Hom_\Set(A,B),B^A)\to\Hom_\Set(\Hom_\Set(A,B)\times A,B)$$を考える. $\phi^\m1$の始域にはcanonicalな要素$\id_{B^A}$がある. 函数$\phi^\m1$を適用すれば要素$ev=\phi^\m1(\id_{B^A})\in\Hom_\Set(\Hom_\Set(A,B)\times A,B)$を得ることができる. この要素はそれ自体が函数である. $$ev\taking\Hom_\Set(A,B)\times A\to B.$$ 
\sexc 函数$ev$を, その始域の要素に対してどのように演算が行なわれるかという観点で記述せよ.
\next $ev$によってこの函数を記述するのが魅力的だと考えられるのはなぜだろうか?
\endsexc
\end{exercise}

%If $n\in\NN$ is a natural number, recall from (\ref{dia:underline n}) that there is a nice set $\ul{n}=\{1,2,\ldots,n\}$. If $A$ is a set, we often make the abbreviation 
$n\in\NN$を自然数とする. \eqref{dia:underline n}から, よい性質を持つ集合$\ul{n}=\{1,2,\ldots,n\}$が存在することを思いだそう. もし$A$が集合ならば, 我々はしばしば以下の略記を使用する.
\begin{align}\label{dia:exponential abbrev}
A^n:=A^{\ul{n}}.
\end{align}

\begin{exercise}\label{exc:two R2s}
%
%In Example \ref{ex:R2} we said that $\RR^2$ is an abbreviation for $\RR\times\RR$, but in (\ref{dia:exponential abbrev}) we say that $\RR^2$ is an abbreviation for $\RR^{\ul{2}}$. Use Exercise \ref{exc:generator for set}, Proposition \ref{prop:curry}, Exercise \ref{exc:coprod}, and the fact that 1+1=2, to prove that these are isomorphic, $\RR^{\ul{2}}\iso\RR\times\RR$.
%
Example \ref{ex:R2}では$\RR^2$は$\RR\times\RR$の略記と言ったが, しかし\eqref{dia:exponential abbrev}では$\RR^2$は$\RR^{\ul{2}}$の略記と言った. Exercise \ref{exc:generator for set}, Proposition \ref{prop:curry}, Exercise \ref{exc:coprod}, そして$1+1=2$という事実を用い, 同型写像$\RR^{\ul{2}}\iso\RR\times\RR$が存在することを証明せよ.

%(The answer to Exercise \ref{exc:generator for set} was $A=\singleton$: i.e. $\Hom_\Set(\singleton,X)\iso X$ for all $X$.)
%
(Exercise \ref{exc:generator for set}の答えは$A=\singleton$だった. i.e. 全ての$X$に対し$\Hom_\Set(\singleton,X)\iso X$.)
%
\end{exercise}

%%%% Subsection %%%%

%\subsection{Arithmetic of sets}\label{sec:arithmetic of sets}\index{set!arithmetic of}
\subsection{集合の算術}\label{sec:arithmetic of sets}\index{set!arithmetic of}

%Proposition \ref{prop:arithmetic of sets} summarizes the properties of products, coproducts, and exponentials, and shows them all in a familiar light, namely that of arithmetic. In fact, one can think of the natural numbers as literally being the isomorphism classes of finite sets---that's what they are used for in counting. Consider the standard procedure for counting the elements of a set $S$, say cows in a field: one points to an element in $S$ and simultaneously says ``1", points to another element in $S$ and simultaneously says ``2", and so on until finished. This procedure amounts to nothing more than creating an isomorphism (one-to-one mapping) between $S$ and some set $\ul{n}$. 

Proposition \ref{prop:arithmetic of sets}は積, 余積, 指数の性質をまとめ, そしてそれら全てを身近な観点, すなわち集合の算術という観点の下で提示する. 事実, 自然数は有限集合に対する同型写像の類まさにそのものとして考えることができる---これは数え上げに使っていることそのものだ. 集合$S$, 例えば牧場にいる牛の数え上げに使う標準的な手続きを考える. 数え上げる人は$S$内のある要素を指差して同時に``1''と言い, $S$の別の要素を指差して同時に``2''と言い, そして終了するまで続ける. この手続きは$S$とある集合$\ul{n}$の間の同型写像(一対一写像)がそれ以上作れなくなるところまで続く. 

%Again, the natural numbers are the isomorphism classes of finite sets. Their behavior, i.e. the arithmetic of natural numbers, reflects the behavior of sets. For example the fact that multiplication distributes over addition is a fact about grids of dots as in Example \ref{ex:grid1}. The following proposition lays out such arithmetic properties of sets.

繰り返すと, 自然数は有限集合の同型写像の類である. これらのふるまい, i.e. 自然数の算術は, 集合のふるまいを反映している. 例えば乗算の加算への分配則はExample \ref{ex:grid1}での点の格子に関する事実である. 以下のpropositionは, 集合のこれらの算術的性質を列挙したものである.

%In this proposition, we denote the coproduct of two sets $A$ and $B$ by the notation $A+B$ rather than $A\sqcup B$. It is a reasonable notation in general, and one that is often used. 

このpropositionでは, われわれは二つの集合$A$および$B$の余積を$A\sqcup B$ではなく$A+B$によって表記する. これは一般的に読みやすく, しばしば使われる記法である.

\begin{proposition}\label{prop:arithmetic of sets}

%The following isomorphisms exist for any sets $A,B,$ and $C$ (except for one caveat, see Exercise \ref{exc:0 to the 0}). 
以下の同型写像は任意の集合$A$, $B$, $C$に対して存在する(一点の注意事項を例外として. Exercise \ref{exc:0 to the 0}を見よ.).
\begin{itemize}
\item $A+\ul{0}\iso A$
\item $A + B\iso B + A$
\item $(A + B) + C \iso A + (B + C)$
\item $A\times\ul{0}\iso\ul{0}$
\item $A\times\ul{1}\iso A$
\item $A\times B\iso B\times A$
\item $(A\times B)\times C \iso A\times (B\times C)$
\item $A\times(B+C)\iso (A\times B)+(A\times C)$
\item $A^{\ul{0}}\iso \ul{1}$
\item $A^{\ul{1}}\iso A$
\item $\ul{0}^A\iso\ul{0}$
\item $\ul{1}^A\iso\ul{1}$
\item $A^{B+C}\iso A^B\times A^C$
\item $(A^B)^C\iso A^{B\times C}$
\end{itemize}

\end{proposition}

\begin{exercise}\label{exc:0 to the 0}
%Everything in Proposition \ref{prop:arithmetic of sets} is true except in one case, namely that of $$\ul{0}^{\ul{0}}.$$ In this case, we get conflicting answers, because for any set $A$, including $A=\emptyset=\ul{0}$, we have claimed both that $A^{\ul{0}}\iso\ul{1}$ and that $\ul{0}^A\iso\ul{0}.$ 
Proposition \ref{prop:arithmetic of sets}は一つの例外を除いて全て真である. その例外とはすなわち$$\ul{0}^{\ul{0}}$$である. この場合, 我々は矛盾した答えを得る. なぜなら$A=\emptyset=\ul{0}$も含めた任意の集合$A$に対して, $A^{\ul{0}}\iso\ul{1}$と$\ul{0}^A\iso\ul{0}$の両方を主張できるからである.

%What is the correct answer for $\ul{0}^{\ul{0}}$, based on the definitions of $\ul{0}$ and $\ul{1}$, given in (\ref{dia:underline n}), and of $A^B$, given in (\ref{dia:exponential sets})?
\eqref{dia:underline n}で与えられた$\ul{0}$ and $\ul{1}$の定義と, \eqref{dia:exponential sets}で与えられた$A^B$の定義に基くと, $\ul{0}^{\ul{0}}$の正しい答はどうなるだろうか?
\end{exercise}

\begin{exercise}
It is also true of natural numbers that if $a,b\in\NN$ and $ab=0$ then either $a=0$ or $b=0$. Is the analogous statement true of all sets?
\end{exercise}

Proposition \ref{prop:arithmetic of sets} is in some sense about isomorphisms. It says that understanding isomorphisms of sets reduces to understanding natural numbers. But note that there is much more going on in $\Set$ than isomorphisms; in particular there are functions that are not invertible. 

In grade school you probably never saw anything that looked like this:
$$5^3\times 3\too 5$$
And yet in Exercise \ref{exc:evaluation} we found a function $ev\taking B^A\times A\to B$ that exists for any sets $A,B$. This function $ev$ is not an isomorphism so it somehow does not show up as an equation of natural numbers. But it still has important meaning.
\footnote{Roughly, the existence of $ev\taking\ul{5}^{\ul{3}}\times\ul{3}\too \ul{5}$ says that given a dot in a $5\times 5\times 5$ grid of dots, and given one of the three axes, you can tell me the coordinate of that dot along that axis.} In terms of mere number, it looks like we are being told of an important function $\ul{575}\to\ul{5}$, which is bizarre. The issue here is precisely the one you confronted in Exercise \ref{exc:functions are not iso invariant}.

\begin{exercise}
Explain why there is a canonical function $\ul{5}^{\ul{3}}\times\ul{3}\too \ul{5}$ but not a canonical function $\ul{575}\to\ul{5}$.
\end{exercise}

\begin{slogan}
It is true that a set is isomorphic to any other set with the same number of elements, but don't be fooled into thinking that the study of sets reduces to the study of numbers. Functions that are not isomorphisms cannot be captured within the framework of numbers. 
\end{slogan}

%%%% Subsection %%%%

\subsection{Subobjects and characteristic functions}

\begin{definition}\label{def:power set}

For any set $B$, define the {\em power set of $B$}\index{power set}, denoted $\PP(B)$,\index{a symbol!$\PP$} to be the set of subsets of $B$.

\end{definition}

\begin{exercise}\label{exc:size of power sets}~
\sexc How many elements does $\PP(\emptyset)$ have? 
\next How many elements does $\PP(\singleton)$ have? 
\next How many elements does $\PP(\{1,2,3,4,5,6\})$ have? 
\next Any idea why they may have named it ``power set"?
\endsexc
\end{exercise}

%% Subsubsection %%

\subsubsection{Simplicial complexes}\label{sec:simplicial complex}

\begin{definition}\label{def:simplicial complex}\index{simplicial complex}

Let $V$ be a set and let $\PP(V)$ be its powerset. A subset $X\ss\PP(V)$ is called {\em downward-closed} if, for every $u\in X$ and every $u'\ss u$, we have $u'\in X$. We say that $X$ {\em contains all atoms} if for every $v\in V$ the singleton set $\{v\}$ is an element of $X$. 

A {\em simplicial complex} is a pair $(V,X)$ where $V$ is a set and $X\ss\PP(V)$ is a downward-closed subset that contains all atoms. The elements of $X$ are called {\em simplices} (singular: {\em simplex}).\index{simplex} Any subset $u\ss V$ has a cardinality $|u|$, so we have a function $X\to\NN$ sending each simplex to its cardinality. The set of simplices with cardinality $n+1$ is denoted $X_n$ and each element $x\in X_n$ is called an {\em $n$-simplex}.
\footnote{It is annoying at first that the set of subsets with cardinality 1 is denoted $X_0$, etc. But this is standard convention because as we will see, $X_n$ will be $n$-dimensional.}
Since $X$ contains all atoms (subsets of cardinality 1), we have $X_0\iso V$, and we may also call the 0-simplices {\em vertices}. We sometimes call the 1-simplices {\em edges}.
\footnote{The reason we wrote $X_0\iso V$ rather than $X_0=V$ is that $X_0$ is the set of 1-element subsets of $V$. So if $V=\{a,b,c\}$ then $X_0=\{\{a\},\{b\},\{c\}\}$. This is really just pedantry.}

Since $X_0\iso V$, we may denote a simplicial complex $(V,X)$ simply by $X$.

\end{definition}

\begin{example}

Let $n\in\NN$ be a natural number and let $V=\ul{n+1}$. Define {\em the $n$-simplex}, denoted $\Delta^n$, to be the simplicial complex $\PP(V)\ss\PP(V)$, i.e. the whole power set, which indeed is downward-closed and contains all atoms. 

\end{example}


We can draw a simplicial complex $X$ by first putting all the vertices on the page as dots. Then for every $x\in X_1$, we see that $x=\{v,v'\}$ consists of 2 vertices, so we draw an edge connecting $v$ and $v'$. For every $y\in X_2$ we see that $y=\{w,w',w''\}$ consists of 3 vertices, so we draw a (filled-in) triangle connecting them. All three edges will be drawn too because $X$ is assumed to be downward closed.

Thus, the 0-simplex $\Delta^0$, the 1-simplex $\Delta^1$, the 2-simplex $\Delta^2$, and the 3-simplex $\Delta^3$ are drawn here:
\begin{center}
\includegraphics[height=1.1in]{simplices}
\end{center} 

The $n$-simplices for various $n$'s are in no way all of the simplicial complexes. In general a simplicial complex is a union or ``gluing together" of simplices in a prescribed manner. For example, consider the simplicial complex $X$ with vertices $X_0=\{1,2,3,4\},$ edges $X_1=\{\{1,2\},\{2,3\},\{2,4\}\},$ and no higher simplices $X_2=X_3=\cdots=\emptyset$. We might draw $X$ as follows:
$$\xymatrix{\LMO{1}\ar@{-}[r]&\LMO{2}\ar@{-}[r]\ar@{-}[d]&\LMO{3}\\&\LMO{4}}$$

\begin{exercise}
Let $X$ be the following simplicial complex, so that $X_0=\{A,B,\ldots,M\}$. 
\begin{center}
\includegraphics[height=3in]{OlogNetwork5}
\end{center} 
In this case $X_1$ consists of elements like $\{A,B\}$ and $\{D,K\}$ but not $\{D,J\}$. 

Write out $X_2$ and $X_3$ (hint: the drawing of $X$ indicates that $X_3$ should have one element).
\end{exercise}

\begin{exercise}
The 2-simplex $\Delta^2$ is drawn as a filled-in triangle with vertices $V=\{1,2,3\}$. There is a simplicial complex $X=\partial\Delta^2$ that would be drawn as an empty triangle with the same set of vertices. 
\sexc Draw $\Delta^2$ and $X$ side by side and make clear the difference.
\next Write down the data for $X$ as a simplicial complex. In other words what are the sets $X_0, X_1, X_2, X_3,\ldots$?
\endsexc
\end{exercise}

%% Subsubsection %%

\subsubsection{Subobject classifier}

\begin{definition}\label{def:subobject classifier}\index{subobject classifier!in $\Set$}

Define the {\em subobject classifier} for $\Set$, denoted $\Omega$\index{a symbol!$\Omega$}, to be the set $\Omega:=\{True,False\}$, together with the function $\singleton\to\Omega$ sending the unique element to $True$.

\end{definition}


\begin{proposition}\label{prop:characteristic function}

Let $B$ be a set. There is an isomorphism $$\phi\taking\Hom_\Set(B,\Omega)\To{\iso}\PP(B).$$

\end{proposition}

\begin{proof}

Given a function $f\taking B\to\Omega$, let $\phi(f)=\{b\in B\|f(b)=True\}\ss B$. We now construct a function $\psi\taking\PP(B)\to\Hom_\Set(B,\Omega)$ to serve as the inverse of $\phi$. Given a subset $B'\ss B$, define $\psi(B')\taking B\to\Omega$ as follows: 
$$\psi(i)(b)=\begin{cases}
True&\tn{ if } b\in B',\\
False&\tn{ if } b\not\in B'.
\end{cases}
$$
One checks easily that $\phi$ and $\psi$ are mutually inverse.

\end{proof}

\begin{definition}[Characteristic function]\index{characteristic function}\index{subset!characteristic function of}

Given a subset $B'\ss B$, we call the corresponding function $B\to\Omega$ the {\em characteristic function of $B'$ in $B$.}

\end{definition}

Let $B$ be any set and let $\PP(B)$ be its power set. By Proposition \ref{prop:characteristic function} there is a bijection between $\PP(B)$ and $\Omega^B$. Since $\Omega$ has cardinality 2, the cardinality of $\PP(B)$ is $2^{|B|}$, which explains the correct answer to Exercise \ref{exc:size of power sets}.

\begin{exercise}
Let $f\taking A\to\Omega$ denote the characteristic function of some $A'\ss A$, and define $A''\ss A$ to be its complement, $A'':=A-A'$ (i.e. $a\in A''$ if and only if $a\not\in A'$). 
\sexc What is the characteristic function of $A''\ss A$? 
\next Can you phrase it in terms of some function $\Omega\to\Omega$?
\endsexc
\end{exercise}

%%%% Subsection %%%%

\subsection{Surjections, injections}

The classical definition of injections and surjections involves elements, which we give now. But a more robust notion involves all maps and will be given in Proposition \ref{prop:inj and surj}.

\begin{definition}\label{def:inj,surj,bij}\index{function!injection}\index{function!surjection}\index{function!bijection}

Let $f\taking X\to Y$ be a function. We say that $f$ is {\em surjective} if, for all $y\in Y$ there exists some $x\in X$ such that $f(x)=y$. We say that $f$ is {\em injective} if, for all $x\in X$ and all $x'\in X$ with $f(x)=f(x')$ we have $x=x'$.

A function that is both injective and surjective is called {\em bijective}.

\end{definition}

\begin{remark}

It turns out that a function that is bijective is always an isomorphism and that all isomorphisms are bijective. We will not show that here, but it is not too hard; see for example \cite[Theorem 5.4]{Big}.

\end{remark}

\begin{definition}[Monomorphisms, epimorphisms]\label{def:mono, epi in set}\index{epimorphism!in $\Set$}\index{monomorphism!in $\Set$}

Let $f\taking X\to Y$ be a function. 

We say that $f$ is a {\em monomorphism} if for all sets $A$ and pairs of functions $g,g'\taking A\to X$,
$$
\xymatrix{A\ar@/^1pc/[r]^g\ar@/_1pc/[r]_{g'}&X\ar[r]^f&Y}
$$
if $f\circ g=f\circ g'$ then $g=g'$.

We say that $f$ is an {\em epimorphism} if for all sets $B$ and pairs of functions $h,h'\taking Y\to B$, 
$$
\xymatrix{X\ar[r]^f&Y\ar@/^1pc/[r]^h\ar@/_1pc/[r]_{h'}&B}
$$
if $h\circ f=h'\circ f$ then $h=h'$.

\end{definition}

\begin{proposition}\label{prop:inj and surj}

Let $f\taking X\to Y$ be a function. Then $f$ is injective if and only if it is a monomorphism; $f$ is surjective if and only if it is an epimorphism.

\end{proposition}

\begin{proof}

If $f$ is a monomorphism it is clearly injective by putting $A=\singleton$. Suppose that $f$ is injective and let $g,g'\taking A\to X$ be functions such that $f\circ g=f\circ g'$, but suppose for contradiction that $g\neq g'$. Then there is some element $a\in A$ such $g(a)\neq g'(a)\in X$. But by injectivity $f(g(a))\neq f(g'(a))$, contradicting $f\circ g=f\circ g'$.

Suppose that $f\taking X\to Y$ is an epimorphism and choose some $y_0\in Y$ (noting that if $Y$ is empty then the claim is vacuously true). Let $h\taking Y\to\Omega$ denote the characteristic function of the subset $\{y_0\}\ss Y$ and let $h'\taking Y\to\Omega$ denote the characteristic function of $\emptyset\ss Y$; note that $h(y)=h'(y)$ for all $y\neq y_0$. Then since $f$ is an epimorphism and $h\neq h'$, we must have $h\circ f\neq h'\circ f$, so there exists $x\in X$ with $h(f(x))\neq h'(f(x))$, which implies that $f(x)=y_0$. This proves that $f$ is surjective.

Finally, suppose that $f$ is surjective, and let $h,h'\taking Y\to B$ be functions with $h\circ f=h'\circ f$. For any $y\in Y$, there exists some $x\in X$ with $f(x)=y$, so $h(y)=h(f(x))=h'(f(x))=h'(y)$. This proves that $f$ is an epimorphism.

\end{proof}

\begin{proposition}\label{prop:pb preserve mono}

Let $f\taking X\to Y$ be a monomorphism. Then for any function $g\taking A\to Y$, the top map $f'\taking X\times_YA\to A$ in the diagram
$$
\xymatrix{X\times_YA\ar[r]^-{f'}\ar[d]_{g'}\ullimit&A\ar[d]^g\\X\ar[r]_f&Y}
$$
is a monomorphism.

\end{proposition}

\begin{proof}

To show that $f'$ is a monomorphism, we take an arbitrary set $B$ and two maps $m,n\taking B\to X\times_YA$ such that $f'\circ m=f'\circ n$, denote that function by $p:=f'\circ m\taking B\to A$. Now let $q=g'\circ m$ and $r=g'\circ n$. The diagram looks like this:
$$
\xymatrix{B\ar@<.5ex>[rr]^(.4)m\ar@<-.5ex>[rr]_(.4)n\ar@/^2pc/[rrr]^p\ar@<.5ex>[drr]^(.6)q\ar@<-.5ex>[drr]_(.6)r&&X\times_YA\ar[r]^{f'}\ar[d]_{g'}\ullimit&A\ar[d]^g\\&&X\ar[r]_f&Y}
$$
We have that 
\begin{align*}f\circ q=f\circ g'\circ m=g\circ f'\circ m=g\circ f'\circ n=f\circ g'\circ n=f\circ r\end{align*} 
But we assumed that $f$ is a monomorphism so this implies that $q=r$. By the universal property of pullbacks, Lemma \ref{lemma:up for fp}, we have $m=n$.

\end{proof}

\begin{exercise}
Show, in analogy to Proposition \ref{prop:pb preserve mono}, that pushouts preserve epimorphisms.
\end{exercise}

\begin{example}\label{exc:olog pullbacks}

Suppose an olog has a fiber product square
$$\xymatrix{X\times_ZY\ar[r]^-{g'}\ar[d]_{f'}&Y\ar[d]^f\\X\ar[r]_g&Z}$$ such that $f$ is intended to be an injection and $g$ is any map.
\footnote{Of course, this diagram is symmetrical, so the same ideas hold if $g$ is an injection and $f$ is any map.} 
In this case, there are nice labeling systems for $f', g'$, and $X\times_ZY$. Namely:
\begin{itemize}
\item ``is" is an appropriate label for $f'$, 
\item the label for $g$ is an appropriate label for $g'$,
\item (the label for $X$, then ``which", then the label for $g$, then the label for $Y$) is an appropriate label for $X\times_ZY$.
\end{itemize}

To give an explicit example, 
$$\xymatrix{
\obox{X\times_ZY}{.9in}{a rib which is made by a cow}\LA{rr}{is made by}\LAL{d}{is}&&\obox{Y}{.4in}{a cow}\LA{d}{is}\\
\obox{X}{.3in}{a rib}\LAL{rr}{is made by}&&\obox{Z}{.6in}{an animal}
}
$$

\end{example}

\begin{corollary}\label{cor:monos are pullbacks of true}

Let $i\taking A\to X$ be a monomorphism. Then there is a fiber product square of the form 
\begin{align}\label{dia:monos are pbs of true}
\xymatrix{A\ar[r]^{f'}\ar[d]_i\ullimit&\singleton\ar[d]^{True}\\X\ar[r]_f&\Omega.}
\end{align}

\end{corollary}

\begin{proof}

Let $X'\ss X$ denote the image of $i$ and let $f\taking X\to\Omega$ denote the characteristic function of $X'\ss X$. Then it is easy to check that Diagram \ref{dia:monos are pbs of true} is a pullback.

\end{proof}

\begin{exercise}
Consider the subobject classifier $\Omega$, the singleton $\singleton$ and the map $\singleton\To{True}\Omega$ from Definition \ref{def:subobject classifier}. Look at diagram \ref{dia:monos are pbs of true} and in the spirit of Exercise \ref{exc:olog pullbacks}, come up with a label for $\Omega$, a label for $\singleton$, and a label for $True$. Given a label for $X$ and a label for $f$, come up with a label for $A$, a label for $i$ and a label for $f'$, such that the English smoothly fits the mathematics.
\end{exercise}

%%%% Subsection %%%%

\subsection{Multisets, relative sets, and set-indexed sets}

In this section we prepare ourselves for considering categories other than $\Set$, by looking at some categories related to $\Set$. 

%% Subsubsection %%

\subsubsection{Multisets}\index{multiset}

Consider the set $X$ of words in a given document. If $WC(X)$ is the wordcount of the document, we will not generally have $WC(X)=|X|$. The reason is that a set cannot contain the same element more than once, so words like ``the" might be undercounted in $|X|$. A {\em multiset} is a set in which elements can be assigned a multiplicity, i.e. a number of times they are to be counted. 

But if $X$ and $Y$ are multisets, what is the appropriate type of mapping from $X$ to $Y$? Since every set is a multiset (in which each element has multiplicity 1), let's restrict ourselves to notions of mapping that agree with the usual one on sets. That is, if multisets $X$ and $Y$ happen to be sets then our mappings $X\to Y$ should just be functions.

\begin{exercise}\label{exc:multiset 1}~
\sexc Come up with some notion of mapping for multisets that generalizes functions when the notion is restricted to sets. 
\next Suppose that $X=(1,1,2,3)$ and $Y=(a,b,b,b)$, i.e. $X=\{1,2,3\}$ with $1$ having multiplicity 2, and $Y=\{a,b\}$ with $b$ having multiplicity 3. What are all the maps $X\to Y$ in your notion?
\endsexc
\end{exercise}

In Chapter \ref{chap:categories} we will be getting to the definition of category, and you can test whether your notion of mapping in fact defines a category. Here is my definition of mapping for multisets.

\begin{definition}\label{def:multiset}

A {\em multiset} is a sequence $X:=(E,B,\pi)$ where $E$ and $B$ are sets and $\pi\taking E\to B$ is a surjective function. We refer to $E$ as the set of {\em element instances of $X$}, we refer to $B$ as the set of {\em element names of $X$}, and we refer to $\pi$ as the {\em naming function for $X$}. Given an element name $x\in B$, let $\pi^\m1(x)\ss E$ be the preimage; the number of elements in $\pi^\m1(x)$ is called the {\em multiplicity of $x$}.

Suppose that $X=(E,B,\pi)$ and $X'=(E',B',\pi')$ are multisets. A {\em mapping from $X$ to $Y$}, denoted $f\taking X\to Y$, consists of a pair $(f_1,f_0)$ such that $f_1\taking E\to E'$ and $f_0\taking B\to B'$ are functions and such that the following diagram commutes:
\begin{align}\label{dia:multiset map}
\xymatrix{E\ar[r]^{f_1}\ar[d]_{\pi}&E'\ar[d]^{\pi'}\\B\ar[r]_{f_0}&B'.}
\end{align}

\end{definition}

\begin{exercise}
Suppose that a pseudo-multiset is defined to be almost the same as a multiset, except that $\pi$ is not required to be surjective. 
\sexc Write down a pseudo-multiset that is not a multi-set. 
\next Describe the difference between the two notions in terms of multiplicities. 
\next Complexity of names aside, which do you think is a more useful notion: multiset or pseudo-multisets? 
\endsexc
\end{exercise}

\begin{exercise}
Consider the multisets described in Exercise \ref{exc:multiset 1}. 
\sexc Write each of them in the form $(E,B,\pi)$, as in Definition \ref{def:multiset}. 
\next In terms of the same definition, what are the mappings $X\to Y$? 
\next If we remove the restriction that diagram \ref{dia:multiset map} must commute, how many mappings $X\to Y$ are there?
\endsexc
\end{exercise}

%\begin{example}[Using multisets to approach probability]
%
%Let $S=\{a,b,c,d\}$, and call each element of $S$ a {\em suit}. Let $R=\{1,2,\ldots,13\}$, and call each element of $R$ a {\em rank}. Let $C=S\times R$, and call each element of $C$ a {\em card}. For each $n\in\NN$, let $$A_n=\{f\taking \{1,2,\ldots,n\}\to C\|f \tn{ is injective}\}$$ and call each element of $A_n$ an {\em $n$-card arrangement}. So $A_{52}$ has $52!:=52*51*\cdots*1$ elements and $A_1$ has 52 elements.
%
%The {\em 4-handed poker deal} is the function $A_{52}\to A_5\times A_5\times A_5\times A_5$ given as follows. For each $i\in\{1,2,3,4\}$, let $p_i\taking\{1,2,3,4,5\}\to\{1,2,\ldots,52\}$ be the (injective) function given by the following matrix
%\begin{align}
%\begin{array}{l || l | l | l | l | l}
%i&p_i(1)&p_i(2)&p_i(3)&p_i(4)&p_i(5)\\\hline
%p_1&1&5&9&13&17\\
%p_2&2&6&10&14&18\\
%p_3&3&7&11&15&19\\
%p_4&4&8&12&16&20
%\end{array}
%\end{align}
%Then if $f\taking\{1,2,\ldots,52\}\to C$ is a 52-card arrangement, then $f\circ p_i\taking\{1,2,3,4,5\}\to C$ is a 5-card arrangement, and putting them together we get our 4-handed poker deal.
%
%We could similarly define a $k$-handed poker deal for any $1\leq k\leq 10$. We focus here on $1$-handed poker deals. For any $n\leq 52$, let $\sim_n$ denote the equivalence relation on $A_n$ for which two arrangements are equivalent if one is a permutation (reordering) of the other. Let $H_n=A_n/\sim_n$ be the quotient; we call its elements {\em $n$-card hands}. 
%
%In ``5-card poker", the 5-card hands are classified by patterns of suits and ranks in the cards. More precisely, given a 5-card arrangement $f\taking \{1,2,3,4,5\}\to C$, we consider the compositions with $\pi_1\taking C\to S$ and $\pi_2\taking C\to R$, which we denote $f_S:=\pi_1\circ f$ and $f_R:=\pi_2\circ f$. We are concerned only with the cardinalities of the fibers of $f_S$ and $f_R$. For example if some suit $s\in S$ has $|f_S^\m1(s)|=5$ we say that $f$ (or its equivalence class in $H_5$) is a {\em flush}. If for ranks $r_1,r_2$ we have $|f_R^\m1(r_1)|=2$ and $|f_R^\m1(r_2)|=3$, we call it a {\em full house}. 
%
%\end{example}

%% Subsubsection %%

\subsubsection{Relative sets}\label{sec:relative sets}\index{relative set}

Let's continue with our ideas from multisets, but now suppose that we have a fixed set $B$ of names that we want to keep once and for all. Whenever someone discusses a set, each element must have a name in $B$. And whenever someone discusses a mapping, it must preserve the names. For example, if $B$ is the set of English words, then every document consists of an ordered set mapping to $B$ (e.g. $1\mapsto \tn{Suppose}, 2\mapsto\tn{that}, 3\mapsto\tn{we},$ etc.) A mapping from document $A$ to document $B$ would send each word found somewhere in $A$ to the same word found somewhere in $B$. This notion is defined carefully below.

\begin{definition}[Relative set]\label{def:relative sets}\index{relative set}

Let $B$ be a set. A {\em relative set over $B$}, or simply a {\em set over $B$}, is a pair $(E,\pi)$ such that $E$ is a set and $\pi\taking E\to B$ is a function. A {\em mapping of relative sets over $B$}, denoted $f\taking (E,\pi)\to(E',\pi')$, is a function $f\taking E\to E'$ such that the triangle below commutes, i.e. $\pi=\pi'\circ f$,
$$
\xymatrix@=10pt{E\ar[rr]^f\ar[rdd]_{\pi}&&E'\ar[ldd]^{\pi'}\\\\&B
}
$$

\end{definition}

\begin{exercise}
Given sets $X,Y,Z$ and functions $f\taking X\to Y$ and $g\taking Y\to Z$, we can compose them to get a function $X\to Z$. If $B$ is a set, if $(X,p), (Y,q),$ and $(Z,r)$ are relative sets over $B$, and if $f\taking (X,p)\to (Y,q)$ and $g\taking (Y,q)\to (Z,r)$ are mappings, is there a reasonable notion of composition such that we get a mapping of relative sets $(X,p)\to (Z,r)$? Hint: draw diagrams.
\end{exercise}

\begin{exercise}~
\sexc Let $\singleton$ denote a set with one element. What is the difference between sets over $\singleton$ and simply sets?
\next Describe the sets relative to $\emptyset$. How many are there?
\endsexc
\end{exercise}

%% Subsubsection %%

\subsubsection{Indexed sets}\label{sec:indexed sets}\index{indexed set}

Let $A$ be a set. Suppose we want to assign to each element $a\in A$ a set $S_a$. This is called an $A$-indexed set. In category theory we are always interested in the legal mappings between two different structures of the same sort, so we need a notion of $A$-indexed mappings; we do the ``obvious thing".

\begin{example}\label{ex:classroom seats}

Let $C$ be a set of classrooms. For each $c\in C$ let $P_c$ denote the set of people in classroom $c$, and let $S_c$ denote the set of seats (chairs) in classroom $c$. Then $P$ and $S$ are $C$-indexed sets. The appropriate kind of mapping between them respects the indexes. That is, a mapping of multi-sets $P\to S$ should, for each classroom $c\in C$, be a function $P_c\to S_c$.\footnote{If we wanted to allow people from any classroom to choose a chair from just any classroom, category theory would tell us to reconsider $P$ and $S$ as sets, forgetting their indices. See Section \ref{sec:left push}.}

\end{example}

\begin{definition}\label{def:indexed sets}\index{indexed set}

Let $A$ be a set. An {\em $A$-indexed set} is a collection of sets $S_a$, one for each element $a\in A$; for now we denote this by $(S_a)_{a\in A}$. If $(S'_a)_{a\in A}$ is another $A$-indexed set, a {\em mapping of $A$-indexed sets from $(S_a)_{a\in A}$ to $(S'_a)_{a\in A}$}, denoted $$(f_a)_{a\in A}\taking(S_a)_{a\in A}\to (S'_a)_{a\in A}$$ is a collection of functions $f_a\taking S_a\to S'_a$, one for each element $a\in A$.

\end{definition}

\begin{exercise}
Let $\singleton$ denote a one element set. What are $\singleton$-indexed sets and mappings between them?
\end{exercise}

\begin{exercise}
There is a strong relationship between $A$-indexed sets and relative sets over $A$. What is it? 
\end{exercise}





%%%%%%%% Chapter %%%%%%%%

\chapter{様々な圏と函手---“圏”を使わずに}\label{chap:categories and functors without admitting it}

%In this chapter we begin to use our understanding of sets to build more interesting mathematical devices, each of which organizes our understanding of a certain kind of domain. For example, monoids organize our thoughts about agents acting on objects; groups are monoids except restricted to only allow agents to act reversibly. We will then study graphs, which are systems of nodes and arrows that can capture ideas like information flow through a network or model connections between building blocks in a material. We will discuss orders, which can be used to study taxonomies or hierarchies. Finally we take a mathematical look at databases, which actually subsume everything else in the chapter. Databases are connection patterns for structuring information.

この章で我々は, それぞれがある種の領域における我々の理解を構成しているところの, より興味深い数学的装置を構築するために, 集合に対する理解を用い始めることになる. 例えば, モノイド(monoid)は対象に作用するエージェント(agent)に関する思考を構成する. 群(group)はエージェントが可逆的に作用することのみを許すように制限されている点が異なるモノイドである. そのあと我々はグラフ(graph)を調べる. これはノードと矢の系で, ネットワーク上の情報の流れや, 物質の構成ブロックの間の結合の模型といったアイデアを捉えることができる. 我々は順序(order)を議論する. これは分類や階層の研究に用いることができる. 最後に, 我々はデータベース(database)の数学的な面を学ぶ. これは実際にはこの章の残り全てを包括している. データベースは構造化された情報に対するパターンの結合である.

%We will see in Chapter \ref{chap:categories} that everything we study in the present chapter is an example of a category. So is $\Set$, the category of sets studied in Chapter \ref{chap:sets}. One way to think of a category is as a set of objects and a connection pattern between them; sets are objects (ovals full of dots if you wish) connected by functions. But each set is itself a category: the objects inside it are just disconnected! Just like a set has an interior view and an exterior view, so will all the categories in this chapter. Each monoid {\em is} a category, but there is also a category {\em of} monoids. 

我々はChapter \ref{chap:categories}で, 我々がそれまでの章で学んだ全ては圏の例であることを見る. $\Set$, Chapter \ref{chap:sets}で学んだ集合の圏もまたそうである. 圏を考察する一つのやり方は, 対象の集合とそれらの間の結合パターンとみなすことである. 集合は函数によって結びつけられた対象(もし望むなら点で埋められた楕円形と思ってもよい)である. しかしそれぞれの集合はそれ自体が圏である. その内部の対象は単に結合していないだけだ! このように集合には内側での視点と外側での視点があり, この章での全ての圏もまたそうである. それぞれのモノイド\emph{は}圏であるが, しかしモノイド\emph{の}圏でもある.

%However, we will not really say the word ``category" much if at all in this chapter. It seems preferable to let the ideas rise on their own accord as interesting structures in their own right before explaining that everything in site fits into a single framework. That will be the pleasant reward to come in Chapter \ref{chap:categories}.

しかし, この章で我々は, たとえ使うにしても``圏''という単語を実際には多くは使用しない. ここでの全てが単一の枠組みの中に適合することを説明する前に, それ自身からそれが本質的に持つ興味深い構造としてアイデアが浮び上がってくる方が, より好ましいだろう. これはChapter \ref{chap:categories}で嬉しい報酬となって返ってくる.

%%%%%% Section %%%%%%

\section{モノイド}\label{sec:monoids}\index{monoid}

%A common way to interpret phenomena we see around us is to say that agents are acting on objects. For example, in a computer drawing program, the user {\em acts on} the canvas in certain prescribed ways. Choices of actions from an available list can be performed in sequence to transform one image into another. As another example, one might investigate the notion that time {\em acts on} the position of hands on a clock in a prescribed way. A first rule for actions is this: the performance of a sequence of several actions is itself the performance of an action---a more complex action, but an action nonetheless.

我々が身の回りの現象を解釈する通常の方法の一つは, エージェントが対象に作用するとすることである. たとえば, コンピュータ描画プログラムでは, 使用者はある定められた方法でキャンバスに\emph{作用する}. 可能動作の一覧からの選択は, ある画像から他の画像へ変換するために一連の動作として実行されるだろう. また別の例として, 時間は時計の針の位置に対して定められた方法で\emph{作用する}という概念を, 調べる人がいるかもしれない. 作用に対する最初の法則は以下である. 一連のいくつかの作用の動作は, それ自身がある一つの作用---より複雑だがそれにもかかわらず一つの---作用である.

%Mathematical objects called {\em monoids} and {\em groups} are tasked with encoding the agent's perspective in all this, i.e. what the agent can do, and what happens when different actions are done in succession. A monoid can be construed as a set of actions, together with a formula that encodes how a sequence of actions is itself considered an action. A group is the same as a monoid, except that every action is required to be reversible. In this section we concentrate on monoids; we will get to groups in Section \ref{sec:groups}.

\emph{モノイド(monoid)}および\emph{群(groups)}と呼ばれる数学的対象は, この観点においてエージェントを符号化することが課されている. i.e. エージェントができることはなにか, そして異なった作用が連続してなされたとき何が起こるのか. モノイドは作用の集合に, 一連の作用をそれ自身作用としてみなすことを符号化する式が伴ったものとして構成することができる. 群はモノイドと同じものだが, それぞれの作用が可逆的であることが要求されている点が異なる. この章では我々はモノイドに集中する. 我々はSection \ref{sec:groups}で群にとりかかることにする.

%%%% Subsection %%%%

%\subsection{Definition and examples}
\subsection{定義と例}

%\begin{definition}[Monoid]\label{def:monoid}
\begin{definition}[モノイド]\label{def:monoid}\index{monoid}

%A {\em monoid} is a sequence $(M,e,\star)$, where $M$ is a set, $e\in M$ is an element, and $\star\taking M\times M\to M$ is a function, such that the following conditions hold for all $m,n,p\in M$:
%\begin{itemize}
%\item $m\star e=m$,
%\item $e\star m=m$, and
%\item $(m\star n)\star p=m\star(n\star p)$.
%\end{itemize}
%We refer to $e$ as the {\em identity element}\index{monoid!identity element of} and to $\star$ as the {\em multiplication formula} for the monoid.\index{monoid!multiplication formula}
%\footnote{Although the function $\star\taking M\times M\to M$ is called the multiplication formula, it may have nothing to do with multiplication. It is nothing more than a formula for taking two inputs and returning an output; calling it ``multiplication" is suggestive of its origins, rather than prescriptive of its behavior.} 
%We call the first two rules {\em identity laws} and the third rule the {\em associativity law} for monoids. 

\emph{モノイド(monoid)}は列$(M,e,\star)$である. ここで$M$は集合, $e\in M$は要素, $\star\taking M\times M\to M$は, 全ての$m,n,p\in M$に対して以下の条件を満たすところの函数である.
\begin{itemize}
\item $m\star e=m$,
\item $e\star m=m$,
\item $(m\star n)\star p=m\star(n\star p)$.
\end{itemize}
我々は$e$を\emph{単位元(identity element)}\index{monoid!identity element of}, $\star$をモノイドに対する\emph{積の式(multiplication formula)}\index{monoid!multiplication formula}として参照する.
\footnote{函数$\star\taking M\times M\to M$は積の式と呼ばれるものの, これは複製とは全く関係がないだろう. これは二つの入力を取って一つの出力を返す式それ以上のものではない. ``積''と呼ぶのはそのふるまいの規約よりも, むしろその出自を示唆している.}
我々は最初の二つの法則をモノイドに対する\emph{単位則(identity laws)}, 三番目の法則をモノイドに対する\emph{結合則(associativity law)}と呼ぶ

\end{definition}

\begin{remark}

%To be pedantic, the conditions from Definition \ref{def:monoid} should be stated 
%\begin{itemize}
%\item $\star(m,e)=m$,
%\item $\star(e,m)=m$, and 
%\item $\star(\star(m,n),p)=\star(m,(\star(n,p))$.
%\end{itemize} The way they are written in Definition \ref{def:monoid} is called {\em infix notation},\index{infix notation} and we often use infix notation without mentioning it. That is, given a function $\cdot\taking A\times B\to C$, we may write $a\cdot b$ rather than $\cdot(a,b)$.

教育的であろうとすれば, Definition \ref{def:monoid}の条件は
\begin{itemize}
\item $\star(m,e)=m$,
\item $\star(e,m)=m$,
\item $\star(\star(m,n),p)=\star(m,(\star(n,p))$.
\end{itemize}
と述べられるべきである. Definition \ref{def:monoid}での書き方の流儀は\emph{中置記法(infix notation)},\index{infix notation}と呼ばれ, 我々はしばしば特に言及することなく中置記法を用いる. これはすなわち, 函数$\cdot\taking A\times B\to C$が与えられたとき, $\cdot(a,b)$よりもむしろ$a\cdot b$と書くであろう, ということである.

\end{remark}

%\begin{example}[Additive monoid of natural numbers]\label{ex:monoid 0}\index{monoid!additive natural numbers}
\begin{example}[自然数の加法モノイド]\label{ex:monoid 0}\index{monoid!additive natural numbers}

%Let $M=\NN$ be the set of natural numbers. Let $e=0$ and let $\star\taking M\times M\to M$ denote addition, so that $\star(4,18)=22$. Then the equations $m\star 0=m$ and $0\star m=m$ hold, and $(m\star n)\star p=m\star (n\star p)$. By assigning $e$ and $\star$ in this way, we have ``given $\NN$ the structure of a monoid".

$M=\NN$を自然数の集合とする. $e=0$であり$\star\taking M\times M\to M$が加法を表すとする. よって$\star(4,18)=22$である. このとき等式$m\star 0=m$および$0\star m=m$がなりたち, $(m\star n)\star p=m\star (n\star p)$である. この方法で$e$と$\star$を割り当てることによって, 我々は``$\NN$にモノイドの構造を与えた''.

\end{example}

\begin{remark}

%Sometimes we are working with a monoid $(M,e,\star)$, and the identity $e$ and multiplication $\star$ are somehow clear from context. In this case we might refer to the set $M$ as though it were the whole monoid. For example, if we were discussing the monoid from Example \ref{ex:monoid 0}, we might refer to it as $\NN$. The danger comes because sets may have multiple monoid structures, as we see below in Exercise \ref{exc:monoid 1}. 

時々, モノイド$(M,e,\star)$を扱う際に, 単位元$e$と積$\star$がなんらかの文脈上明らかなことがある. この場合, 我々は集合$M$をそれがあたかもモノイドの全てであるかのように参照するかもしれない. 例えば, もしExample \ref{ex:monoid 0}からモノイドを議論しようとすれば, 我々は$\NN$としてそれを参照するだろう. これは危険をともなう。なぜならば以下のExercise \ref{exc:monoid 1}で見るように, 集合は複数のモノナド構造を持つかもしれないからだ. 

\end{remark}

%\begin{example}[Non-monoid]
\begin{example}[非モノイド]

%If $M$ is a set, we might call a function $f\taking M\times M\to M$ an {\em operation on $M$}. For example, if $M=\NN$ is the set of natural numbers, we can consider the operation $f\taking\NN\to\NN$ called exponentiation. For example $f(2,5)=2*2*2*2*2=32$ and $f(7,2)=49.$ This is indeed an operation, but it is not part of any monoid. For one thing there is no possible unit. Trying the obvious choice of $e=1$, we see that $a^1=a$ (good), but that $1^a=1$ (bad: we need it to be $a$). For another thing, this operation is not associative because in general $a^{b^c}\neq (a^b)^c$. For example, $2^{1^2}=2$ but $(2^1)^2=4$. 

$M$が集合であるとき, 我々は函数$f\taking M\times M\to M$を\emph{operation on $M$}\index{operation!on $\Set$}と呼ぶことがある. 例えば, $M=\NN$が自然数の集合であるとき, 我々は累乗と呼ばれるoperation $f\taking\NN\times\NN\to\NN$を考えることができる. 例えば$f(2,5)=2*2*2*2*2=32$であり$f(7,2)=49$である. これは実際operationであるが, しかしいかなるモノイドの一部でもない. 一例を挙げると単位元の可能性がある要素が存在しない. 明らかな選択として$e=1$を試みると, $a^1=a$は得られる(good)ものの, $1^a=1$も得られる(bad: 我々はこれが$a$であることを必要とする). また別の点を挙げると, このoperationは結合的ではない. なぜなら一般に$a^{b^c}\neq (a^b)^c$である. 例えば$2^{1^2}=2$しかし$(2^1)^2=4$である.

%One might also attempt to consider an operation $f\taking M\times M\to M$ that, upon closer inspection, aren't even operations. For example, if $M=\ZZ$ then exponentiation is not even an operation. Indeed, $f(2,-1)=2^{-1}=\frac{1}{2}$, and this is not an integer. To have a function $f\taking M\times M\to M$, we need that every element of the domain, in this case every pair of integers, has an output under $f$. So there is no such function $f$. 

operation $f\taking M\times M\to M$として, 詳しく見てみるとoperationですらないようなものを, 考えてみたくなるかもしれない. 例えば, $M=\ZZ$とすると, 累乗はoperationですらない. 実際, $f(2,-1)=2^{-1}=\frac{1}{2}$であり, これは整数ではない. 函数$f\taking M\times M\to M$を得るにあたって, 始域の全ての要素, この場合は全ての整数の対が, $f$の下で出力を(終域内に)持つ必要がある. よって(この場合)そのような函数$f$は存在しない. 

\end{example}

\begin{exercise}\label{exc:monoid 1}
%Let $M=\NN$ be the set of natural numbers. Taking $e=1$, come up with a formula for $\star$ that gives $\NN$ the structure of a monoid.
$M=\NN$を自然数の集合とする. $e=1$をとり, $\NN$にモノイド構造を与える$\star$に対する式を考案せよ.
\end{exercise}

\begin{exercise}
%Come up with an operation on the set $M=\{1,2,3,4\}$, i.e. a legitimate function $f\taking M\times M\to M$, such that $f$ cannot be the multiplication formula for a monoid on $M$. That is, either it is not associative, or no element of $M$ can serve as a unit.
集合$M=\{1,2,3,4\}$の上のoperation, i.e. 正当な函数$f\taking M\times M\to M$で$M$上のモノイドに対する積の式を満しえないところの$f$を, 考案せよ. これは, 結合的ではないか, あるいは$M$のどの要素も単位元となりえないかの, どちらかであればよい.
\end{exercise}

\begin{exercise}\label{ex:commutative monoid}
%In both Example \ref{ex:monoid 0} and Exercise \ref{exc:monoid 1}, the monoids $(M,e,\star)$ satisfied an additional rule called {\em commutativity},\index{monoid!commutative} namely $m\star n=n\star m$ for every $m,n\in M$. There is a monoid $(M,e,\star)$ lurking in linear algebra textbooks that is not commutative; if you have background in linear algebra try to answer this: what $M, e$, and $\star$ might I be referring to?
Example \ref{ex:monoid 0}と\ref{exc:monoid 1}の両者ともに, モノイド$(M,e,\star)$は\emph{可換性(commutativity)}\index{monoid!commutative}, すなわち全ての$m,n\in M$に対して$m\star n=n\star m$であるという, 追加の法則を満している. 線形代数の教科書には可換でないモノイド$(M,e,\star)$が潜んでいる. もしも線形代数の背景知識があるならば, 以下の問いに答えてみよう. 著者が参照しているであろう$M$, $e$, そして$\star$は何であろうか?
\end{exercise}

\begin{exercise}
%Recall the notion of commutativity for monoids from Exercise \ref{ex:commutative monoid}. 
%\sexc What is the smallest set $M$ that you can give the structure of a non-commutative monoid? 
%\next What is the smallest set $M$ that you can give the structure of a monoid?
%\endsexc
\ref{ex:commutative monoid}でのモノイドの可換性の概念を思い出そう.
\sexc 非可換モノイドの構造を与えることができる最小の集合$M$は何だろうか?
\next モノイドの構造を与えることができる最小の集合$M$は何だろうか?
\endsexc
\end{exercise}

%\begin{example}[Trivial monoid]\label{ex:trivial monoid}
\begin{example}[自明なモノイド]\label{ex:trivial monoid}

%There is a monoid with only one element, $M=(\{e\},e,\star)$ where $\star\taking\{e\}\times\{e\}\to\{e\}$ is the unique function. We call this monoid {\em the trivial monoid},\index{monoid!trivial} and sometimes denote it $\ul{1}$.

一要素しかないモノイド$M=(\{e\},e,\star)$が存在する. ここで$\star\taking\{e\}\times\{e\}\to\{e\}$は一意な函数である. 我々はこのモノイドを\emph{自明なモノイド(the trivial monoid)}\index{monoid!trivial}と呼び, しばしば$\ul{1}$と表記する.

\end{example}

\begin{example}

%Suppose that $(M,e,\star)$ is a monoid. Given elements $m_1,m_2,m_3,m_4$ there are five different ways to parenthesize the product $m_1\star m_2\star m_3\star m_4$, and the associativity law for monoids will show them all to be the same. We have
$(M,e,\star)$がモノイドであると仮定する. 要素$m_1,m_2,m_3,m_4$が与えられたとき, 積$m_1\star m_2\star m_3\star m_4$に括弧をつけるのに5種類の異なった方法が存在する. モノイドに対する結合則によってそれら全てが同じであることが示される. 以下が得られる.
\begin{align*}
((m_1\star m_2)\star m_3)\star m_4&=(m_1\star m_2)\star (m_3\star m_4)\\
&=(m_1\star(m_2\star m_3))\star m_4\\
&=m_1\star(m_2\star (m_3\star m_4))\\
&=m_1\star((m_2\star m_3)\star m_4)
\end{align*}

%In fact, the product of any list of monoid elements is the same, regardless of parenthesization. Therefore, we can unambiguously write $m_1m_2m_3m_4m_5$ rather than any given parenthesization of it. This is known as the \href{http://en.wikipedia.org/wiki/Coherence_theorem}{\text coherence theorem} and can be found in \cite{Mac}.

実際, 任意のモノイドの要素の列の積は, 括弧づけにかかわらず同じである. それゆえに, 括弧を付けなくとも, あいまいさなく$m_1m_2m_3m_4m_5$と書くことができる. これは\href{http://en.wikipedia.org/wiki/Coherence_theorem}{\text coherence theorem}として知られており, \cite{Mac}で見出される.

\end{example}

%% Subsubsection %%

%\subsubsection{Free monoids and finitely presented monoids}\label{sec:free monoid}
\subsubsection{自由モノイドと有限表現モノイド}\label{sec:free monoid}

\begin{definition}\label{def:list}\index{list}

%Let $X$ be a set. A {\em list in $X$} is a pair $(n,f)$ where $n\in\NN$ is a natural number (called the {\em length of the list}) and $f\taking\ul{n}\to X$ is a function, where $\ul{n}=\{1,2,\ldots,n\}$. We may denote such a list by 
%$$(n,f)=[f(1),f(2),\ldots,f(n)].$$ 
%The {\em empty list} is the unique list in which $n=0$; we may denote it by $[\;]$. Given an element $x\in X$ the {\em singleton list on $x$} is the list $[x]$. Given a list $L=(n,f)$ and a number $i\in\NN$ with $i\leq n$, the {\em $i$th entry of $L$} is the element $f(i)\in X$. \index{entry!in list}

$X$を集合とする. \emph{$X$でのリスト(list in $X$)}とは対$(n,f)$である. ここで$n\in\NN$は(\emph{リストの長さ(length of the list)}と呼ばれる)自然数であり, $f\taking\ul{n}\to X$は函数, また$\ul{n}=\{1,2,\ldots,n\}$である. われわれはこのようなリストを
$$(n,f)=[f(1),f(2),\ldots,f(n)]$$
によって表記する. \emph{空リスト(empty list)}は$n=0$である一意なリストである. 我々はこれを$[\;]$によって表記しよう. 要素$x\in X$が与えられたとき, \emph{$x$の単一要素リスト(singleton list on $x$)}はリスト$[x]$である. リスト$L=(n,f)$と$i\leq n$なる自然数$i\in\NN$が与えられたとき, \emph{$i$th entry of $L$}は要素$f(i)\in X$である.\index{entry!in list}

%Given two lists $L=(n,f)$ and $L'=(n',f')$, define the {\em concatenation of $L$ and $L'$}\index{list!concatenation}\index{concatenation!of lists}, denoted $L\plpl L'$,\index{a symbol!$\plpl$} to be the list $(n+n',f\plpl f')$, where $f\plpl f'\taking \ul{n+n'}\to X$ is given on $i\leq n+n'$ by
%$$(f\plpl f')(i):=
%\begin{cases}
%f(i)&\tn{ if }i\leq n\\
%f'(i-n)&\tn{ if }i\geq n+1
%\end{cases}
%$$

二つのlist $L=(n,f)$と$L'=(n',f')$が与えられたとき, \emph{$L$と$L'$の連結(concatenation of $L$ and $L'$)}\index{list!concatenation}\index{concatenation!of lists}を, $L\plpl L'$,\index{a symbol!$\plpl$}と表記し, $(n+n',f\plpl f')$と定義する. ここで$f\plpl f'\taking \ul{n+n'}\to X$は$i\leq n+n'$について以下で与えられる.
$$(f\plpl f')(i):=
\begin{cases}
f(i)&\tn{ if }i\leq n\\
f'(i-n)&\tn{ if }i\geq n+1
\end{cases}
$$
\end{definition}

\begin{example}

%Let $X=\{a,b,c,\ldots,z\}$. The following are elements of $\List(X)$: $$[a,b,c],\;\; [p],\;\; [p,a,a,a,p],\;\; [\;],\;\;\dots$$ The concatenation of $[a,b,c]$ and $[p,a,a,a,p]$ is $[a,b,c,p,a,a,a,p]$. The concatenation of any list $A$ with $[\;]$ is just $A$.

$X=\{a,b,c,\ldots,z\}$とする. 以下は$\List(X)$の要素である. $$[a,b,c],\;\; [p],\;\; [p,a,a,a,p],\;\; [\;],\;\;\dots$$ $[a,b,c]$と$[p,a,a,a,p]$の連結は$[a,b,c,p,a,a,a,p]$である. 任意のリスト$A$と$[\;]$の連結は$A$である.

\end{example}

\begin{definition}\label{def:free monoid}\index{monoid!free}

%Let $X$ be a set. The {\em free monoid generated by $X$} is the sequence $M:=(\List(X),[\;],\plpl)$, where $\List(X)$ is the set of lists of elements in $X$, where $[\;]\in\List(X)$ is the empty list, and where $\plpl$ is the operation of list concatenation. We refer to $X$ as the set of generators for the monoid $M$.

$X$を集合とする. \emph{$X$によって生成される自由モノイド(free monoid generated by $X$)}とは列$M:=(\List(X),[\;],\plpl)$である. ここで$\List(X)$は$X$内の要素からなるリストの集合であり, $[\;]\in\List(X)$は空リスト, $\plpl$はリスト連結のoperationである. 我々は$X$をモノイド$M$に対する生成子(generator)の集合として参照する.

\end{definition}

\begin{exercise}
%Let $\singleton$ denote a one-element set. 
%\sexc What is the free monoid generated by $\singleton$? 
%\next What is the free monoid generated by $\emptyset$?
%\endsexc
$\singleton$が一要素集合を表すとする.
\sexc $\singleton$によって生成される自由モノイドは何か?
\next $\emptyset$によって生成される自由モノイドは何か?
\endsexc
\end{exercise}

%In the definition below, we will define a monoid $M$  by specifying some generators and some relations. Lists of generators provide us all the possible ways to write elements of $M$. The relations allow us to have two such ways of writing the same element. The following definition is a bit dense, so see Example \ref{ex:presented monoid} for a concrete example.

以下の定義では, 我々はモノイド$M$を, なんらかの生成子となんらかの関係(relation)を指定することによって定義する. 生成子のリストによって、$M$の要素を書き出す全ての方法が得られる. 関係によって, 同じ要素を書き出す二つの異なった方法を得ることができる. 以下の定義は少々密度が濃い. 完全な例としてはExample \ref{ex:presented monoid}を見るのがよい. 

\begin{definition}[Presented monoid]\label{def:presented monoid}\index{monoid!presented}

%Let $G$ be a finite set, let $n\in\NN$ be a natural number,
%\footnote{The number $n\in\NN$ is going to stand for the number of relations we declare.} 
%and for each $1\leq i\leq n$, let $m_i$ and $m_i'$ be elements of $\List(G)$.
%\footnote{Each $m_i$ and $m_i'$ are going to be made equal in the set $M$.} 
%The {\em monoid presented by generators $G$ and relations $\{(m_i,m_i')\|1\leq i\leq n\}$} is the monoid $\mcM=(M,e,\star)$ defined as follows. Let $\sim$ denote the equivalence relation on $\List(G)$ generated by $\{(xm_iy\sim xm_i'y)\|x,y\in\List(G), 1\leq i\leq n\}$, and define $M=\List(G)/\sim$. Let $e=[\;]$ and let $a * b$ be obtained by concatenating representing lists. 

$G$を有限集合, $n\in\NN$を自然数とし\footnote{数$n\in\NN$は我々が宣言する関係の数を意味することになる.}, それぞれの$1\leq i\leq n$に対して$m_i$と$m_i'$は$\List(G)$の要素であるとする. \footnote{$m_i$と$m_i'$のそれぞれは集合$M$の中で等号を作ることになる.} \emph{生成子$G$と関係$\{(m_i,m_i')\|1\leq i\leq n\}$によって表現されるモノイド(monoid presented by generators $G$ and relations $\{(m_i,m_i')\|1\leq i\leq n\}$)}は, モノイド$\mcM=(M,e,\star)$であり, 以下のように定義される. $\sim$が$\{(xm_iy\sim xm_i'y)\|x,y\in\List(G), 1\leq i\leq n\}$によって生成される$\List(G)$上の同値関係を示すとし, $M=\List(G)/\sim$と定義する. $e=[\;]$とし、また$a * b$を表現しているリストの結合によって得られるとする.

\end{definition}

\begin{remark}

%Every free monoid is a presented monoid, because we can just take the set of relations to be empty.

全ての自由モノイドは表現モノイドである. なぜなら関係の集合として単に空集合をとることができるからである.

\end{remark}

\begin{example}\label{ex:presented monoid}

%Let $G=\{a,b,c,d\}$. Think of these as buttons that can be pressed. The free monoid $\List(G)$ is the set of all ways of pressing buttons, e.g. pressing $a$ then $a$ then $c$ then $c$ then $d$ corresponds to the list $[a,a,c,c,d]$. The idea of presented monoids is that you notice that pressing $[a,a,c]$ always gives the same result as pressing $[d,d]$. You also notice that pressing $[c,a,c,a]$ is the same thing as doing nothing. 

$G=\{a,b,c,d\}$とする. これらを押すことのできるボタンであると考える. 自由モノイド$\List(G)$はボタンを押す全てやり方の集合である, e.g. $a$を押して$a$を押して$c$を押して$c$を押して$d$を押すのは, リスト$[a,a,c,c,d]$に対応する. 表現モノイドのアイデアは, $[a,a,c]$と押すのは常に$[d,d]$と押すのと同じ結果を与える, ということである. また, $[c,a,c,a]$と押すことは何も押さないのと同じことであるということに気が付いたとしよう. 

%In this case, we would have $m_1=[a,a,c]$, $m_1'=[d,d]$, and $m_2=[c,a,c,a], m_2'=[\;]$ and relations $\{(m_1,m_1'), (m_2,m_2')\}$. Really this means that we're equating $m_1$ with $m_1'$ and $m_2$ with $m_2'$, which for convenience we'll write out:
%$${\color{blue}{[a,a,c]}}={\color{blue}{[d,d]}}\hsp\tn{and}\hsp{\color{red}{[a,c,a,c]}}={\color{red}{[\;]}}
%$$ 

この場合, $m_1=[a,a,c]$, $m_1'=[d,d]$, $m_2=[c,a,c,a]$, $m_2'=[\;]$, および関係$\{(m_1,m_1'), (m_2,m_2')\}$が得られている. 実際にこれが意味するところは, $m_1$と$m_1'$, $m_2$と$m_2'$を等しいと見なすということである. 我々は便利のため以下のように書き出す.
$${\color{blue}{[a,a,c]}}={\color{blue}{[d,d]}}\hsp\tn{and}\hsp{\color{red}{[a,c,a,c]}}={\color{red}{[\;]}}
$$ 

%To see how this plays out, we give an example of a calculation in $M=\List(G)/\sim$. Namely, 
これがどのように働くかをみるために, $M=\List(G)/\sim$での計算の例を与えよう. すなわち,
\begin{align*}
[b,c,b,{\color{blue}{d,d}},a,c,a,a,c,d] = [b,c,b,a,a,{\color{red}{c,a,c,a}},a,c,d] &= [b,c,b,a,{\color{blue}{a,a,c}},d]\\
&= [b,c,b,a,d,d,d].
\end{align*}

\end{example}

\begin{application}[Buffer]\label{app:buffer}

%Let $G=\{a,b,c,\ldots\,z\}$. Suppose we have a \href{http://en.wikipedia.org/wiki/Data_buffer}{\text buffer} of 32 characters and we want to consider the set of lists of length at most 32 to be a monoid. We simply have to decide what happens when someone types a list of length more than 32. 

$G=\{a,b,c,\ldots\,z\}$とする. 32文字の\href{http://en.wikipedia.org/wiki/Data_buffer}{\text buffer}が手元にあると仮定する. 我々は最大長が32である文字リストの集合をモノイドとみなしたい. 簡単には我々は32以上の長さが入力されたときに何が起きるかを決める必要がある. 

%One option is to say that the last character typed overwrites the 32nd entry, $$[a_1,a_2,\ldots,a_{31},a_{32},b]\sim_1[a_1,a_2,\ldots,a_{31},b].$$ Another option is to say that any character typed after\_32 entries is discarded, $$[a_1,a_2,\ldots,a_{31},a_{32},b]\sim_2[a_1,a_2,\ldots,a_{31},a_{32}].$$ Both of these yield finitely presented monoids, generated by $G$. (In case it's useful, the number of necessary relations in both cases is $26^{33}$.)

一つの選択肢は, 入力された最後の文字は32番目のentryを上書きすることである. $$[a_1,a_2,\ldots,a_{31},a_{32},b]\sim_1[a_1,a_2,\ldots,a_{31},b]$$ 別の選択肢は32番目のentryを越えて入力された全ての文字を廃棄することである. $$[a_1,a_2,\ldots,a_{31},a_{32},b]\sim_2[a_1,a_2,\ldots,a_{31},a_{32}].$$ これら両者のいずれも, $G$によって生成される有限な表現モノイドを与える. (これが使える場合には, 必要な関係の数はどちらの場合も$26^{33}$である.)

\end{application}

\begin{exercise}\label{exc:buffer3}

%Let's consider the buffer concept again (see Application \ref{app:buffer}), but this time only having size 3 rather than size 32. Show using Definition \ref{def:presented monoid} that with relations given by $\sim_1$ we indeed have $[a,b,c,d,e,f]=[a,b,f]$ and that with relations given by $\sim_2$ we indeed have $[a,b,c,d,e,f]=[a,b,c]$.

再びbufferの概念を考えてみよう(Application \ref{app:buffer}を見よ). ただし今回は大きさが32ではなく3であるとする. Definition \ref{def:presented monoid}を用いて, $\sim_1$によって与えられる関係によって実際に$[a,b,c,d,e,f]=[a,b,f]$が得られること, そして$\sim_2$によって実際に$[a,b,c,d,e,f]=[a,b,c]$が得られることを示せ.
\end{exercise}

\begin{exercise}
%Let $K:=\{BS,a,b,c,\ldots,z\}$, a set having 27 elements. Suppose you want to think of $BS\in K$ as the ``backspace key" and the elements $a,b,\ldots z\in K$ as the letter keys on a keyboard. Then the free monoid $\List(K)$ is not quite appropriate as a model because we want $[a,b,d,BS]=[a,b]$. 
%\sexc Choose a set of relations for which the monoid presented by generators $K$ and the chosen relations is appropriate to this application. 
%\next Under your relations, how does $[BS]$ compare with $[\;]$? Is that suitable?
%\endsexc
$K:=\{\mathit{BS},a,b,c,\ldots,z\}$を, 27要素の集合とする. $\mathit{BS}\in K$は``backspace key'', 要素$a,b,\ldots z\in K$は文字キーでキーボード上にあるとみなしたいと仮定する. このとき自由モノイド$\List(K)$は模型としてそれほど適切ではないだろう. なぜなら$[a,b,d,\mathit{BS}]=[a,b]$であって欲しいからだ.
\sexc 生成子$K$で表現されるモノイドと選んだ関係がこの応用に対して適切であるところの関係の集合を選べ. 
\next その関係の下で, $[\mathit{BS}]$と$[\;]$を比較するとどうなるか? これは適切だろうか?
\endsexc
\end{exercise}

%% Subsubsection %%

%\subsubsection{Cyclic monoids}
\subsubsection{巡回モノイド}

\begin{definition}

%A monoid is called {\em cyclic}\index{monoid!cyclic} if it has a presentation involving only one generator. 

モノイドは, その表現がただ一つの生成子によって表されるとき, \emph{巡回モノイド(cyclic monoid)}\index{monoid!cyclic}あるいは単項生成モノイドと呼ばれる.

\end{definition}

\begin{example}\label{ex:cyclic}

%Let $Q$ be a symbol; we look at some cyclic monoids generated by $\{Q\}$. With no relations the monoid would be the free monoid on one generator, and would have underlying set $\{[\;],[Q],[Q,Q],[Q,Q,Q],\ldots\}$, with identity element $[\;]$ and multiplication given by concatenation (e.g. $[Q,Q,Q]\plpl[Q,Q]=[Q,Q,Q,Q,Q]$). This is just $\NN$, the additive monoid of natural numbers.

$Q$を符号とする. $\{Q\}$によって生成されるある巡回モノイドについて見てみよう. 関係がないならばこのモノイドは生成子一つの自由モノイドで, 集合$\{[\;],[Q],[Q,Q],[Q,Q,Q],\ldots\}$を元にしており, 単位元$[\;]$をともない, 積は結合(e.g. $[Q,Q,Q]\plpl[Q,Q]=[Q,Q,Q,Q,Q]$)によって与えられるだろう. これはまさしく$\NN$, 自然数の加法モノイドである. 

%With the really strong relation $[Q]\sim[\;]$ we would get the trivial monoid, a monoid having only one element (see Example \ref{ex:trivial monoid}).

実に強力な関係$[Q]\sim[\;]$があるとき, 我々は自明なモノイド, すなわち一要素しか持たないモノイドを得る(Example \ref{ex:trivial monoid}を見よ).

%Another possibility is given in the first part of Example \ref{ex:clocks}, where the relation $Q^{12}\sim[\;]$ is used, where $Q^{12}$ is shorthand for $[Q,Q,Q,Q,Q,Q,Q,Q,Q,Q,Q,Q]$.

他の可能性が, Example \ref{ex:clocks}の最初の部分で与えられている. ここでは関係$Q^{12}\sim[\;]$が使われている. ここで$Q^{12}$は$[Q,Q,Q,Q,Q,Q,Q,Q,Q,Q,Q,Q]$の短縮形である.

\end{example}

\begin{example}\label{ex:cyclic monoid (7,4)}

%Consider the cyclic monoid with generator $Q$ and relation $Q^7=Q^4$. This monoid has seven elements, $\{e=Q^0,Q=Q^1, Q^2, Q^3, Q^4, Q^5, Q^6\}$, and we know that $Q^6\star Q^5=Q^7*Q^4=Q^4*Q^4=Q^7*Q=Q^5.$ One might depict this monoid as follows
%$$\xymatrix@=15pt{
%\LMO{e}\ar[rr]&&\LMO{Q}\ar[rr]&&\LMO{Q^2}\ar[rr]&&\LMO{Q^3}\ar[rr]&&\LMO{Q^4}\ar[dr]\\
%&&&&&&&\LMO{Q^6}\ar[ur]&&\LMO{Q^5}\ar[ll]
%}
%$$
%To see the mathematical source of this intuitive depiction, see Example \ref{ex:yoneda for cyclic monoid}.

生成子$Q$と関係$Q^7=Q^4$による巡回モノイドを考える. このモノイドは7要素$\{e=Q^0,Q=Q^1, Q^2, Q^3, Q^4, Q^5, Q^6\}$を持ち, また$Q^6\star Q^5=Q^7*Q^4=Q^4*Q^4=Q^7*Q=Q^5$であることが分かっている. このモノイドは以下のように書くことができるだろう.
$$\xymatrix@=15pt{
\LMO{e}\ar[rr]&&\LMO{Q}\ar[rr]&&\LMO{Q^2}\ar[rr]&&\LMO{Q^3}\ar[rr]&&\LMO{Q^4}\ar[dr]\\
&&&&&&&\LMO{Q^6}\ar[ur]&&\LMO{Q^5}\ar[ll]
}
$$
この直感的な描画の数学的起源を知るには, Example \ref{ex:yoneda for cyclic monoid}を見よ.

\end{example}

%\begin{exercise}[Classify the cyclic monoids]\label{exc:classify cyclic}
\begin{exercise}[巡回モノイドの分類]\label{exc:classify cyclic}

%Classify all the cyclic monoids up to isomorphism. That is, come up with a naming system such that every cyclic monoid can be given a name in your system, such that no two non-isomorphic cyclic monoids have the same name, and such that no name exists in the system unless it refers to a cyclic monoid. 

同型写像によって全ての巡回モノイドを分類せよ. これは名前付け体系を考案することである. この体系において, 全ての巡回モノイドは名前を与えられ, 互いに同型でない二つの巡回モノイドは決して同じ名前を持たず, 巡回モノイドを参照しない名前はその内部に存在しない. 

%Hint: one might see a pattern in which the three monoids in Example \ref{ex:cyclic} correspond respectively to $\infty$, $1$, and $12$, and then think ``Cyclic monoids can be classified by (i.e. systematically named by elements of) the set $\NN\sqcup\{\infty\}$." That idea is on the right track, but is not correct.

ヒント: Example \ref{ex:cyclic}の, それぞれ$\infty$, $1$, $12$に対応する3つのモノイドにあるパターンを理解しよう. そして``巡回モノイドは集合$\NN\sqcup\{\infty\}$によって分類できる(i.e. その要素によって体系的に名前付けられる).''かを考えよ. このアイデアは正しい道筋にあるが, しかし正確ではない.
\end{exercise}

%%%% Subsection %%%%

%\subsection{Monoid actions}
\subsection{モノイド作用}

%\begin{definition}[Monoid action]\label{def:monoid action}\index{monoid!action}\index{action!of a monoid}
\begin{definition}[モノイド作用]\label{def:monoid action}\index{monoid!action}\index{action!of a monoid}

%Let $(M,e,\star)$ be a monoid and let $S$ be a set. An {\em action of $(M,e,\star)$ on $S$}, or simply an {\em action of $M$ on $S$} or an {\em $M$-action on $S$}, is a function $$\acts\;\;\taking M\times S\to S$$\index{a symbol!$\acts$} such that the following conditions hold for all $m,n\in M$ and all $s\in S$:

$(M,e,\star)$をモノイドとし, $S$を集合とする. \emph{$S$への$(M,e,\star)$の作用(action of $(M,e,\star)$ on $S$)}あるいは\emph{$S$への$M$-作用($M$-action on $S$)}とは, 全ての$m,n\in M$および$s\in S$に足して以下の条件を全て満たすところの函数$$\acts\;\;\taking M\times S\to S$$\index{a symbol!$\acts$}である.
\begin{itemize}
\item $e\acts s=s$
\item $m\acts(n\acts s)=(m\star n)\acts s$.
%\footnote{
%Definition \ref{def:monoid action} actually defines a {\em left action}\index{action!left} of $(M,e,\star)$ on $S$. A {\em right action}\index{action!right} is like a left action except the order of operations is somehow reversed. We will not really use right-actions in this text, but we briefly define it here for completeness. With notation as above, the only difference is in the second condition. We replace it by the condition that for all $m,n\in M$ and all $s\in S$ we have 
%$$m\acts(n\acts s)=(n\star m)\acts s
%$$}
\footnote{
Definition \ref{def:monoid action}は実際には$S$への\emph{左作用(left action)}\index{action!left}$(M,e,\star)$を定義している. \emph{右作用(right action)}\index{action!right}は左作用と似ているが, 演算の順番がどういうわけか逆であるという点が異なる. 我々はこの文書では右作用を実際に使うことはないが, 完全を期すためにここで手短に定義をしておこう. 上記と同じ記法を用いたとき, 唯一の違いは二番目の条件である. 我々は二番目の条件を以下のように置きかえる. 全ての$m,n\in M$と$s\in S$に対して
$$m\acts(n\acts s)=(n\star m)\acts s
$$}
\end{itemize}

\end{definition}

\begin{remark}\label{rmk:monoid action}

%To be pedantic (and because it's sometimes useful), we may rewrite $\acts$ as $\alpha\taking M\times S\to S$ and restate the conditions from Definition \ref{def:monoid action} as
教育的である(かつ時々役にたつため), 我々は$\acts$を$\alpha\taking M\times S\to S$と書き直し, Definition \ref{def:monoid action}での条件を以下に再掲する.
\begin{itemize}
\item $\alpha(e,s)=s$, and
\item $\alpha(m,\alpha(n,s))=\alpha(m\star n,s)$.
\end{itemize}

\end{remark}

\begin{example}\label{ex:clocks}

%Let $S=\{0,1,2,\ldots,11\}$ and let $N=(\NN,0,+)$ be the additive monoid of natural numbers (see Example \ref{ex:monoid 0}).  We define a function $\acts\taking\NN\times S\to S$ by taking a pair $(n,s)$ to the remainder that appears when $n+s$ is divided by 12. For example $4\acts 2=6$ and $8\acts 9 = 5$. This function has the structure of a monoid action because the two rules from Definition \ref{def:monoid action} hold.

$S=\{0,1,2,\ldots,11\}$とし, $N=(\NN,0,+)$を自然数の加法モノイドとする(see Example \ref{ex:monoid 0}). 函数$\acts\taking\NN\times S\to S$を, 対$(n,s)$を取り, $n+s$を12で割ったときの余りを返す函数として定義する. 例えば, $4\acts 2=6$であり$8\acts 9 = 5$である. この函数はモノイド作用の構造を持つ. なぜならDefinition \ref{def:monoid action}の二つの法則を満たすからだ.

%Similarly, let $T$ denote the set of points on a circle, elements of which are denoted by a real number in the interval $[0,12)$, i.e. $$T=\{x\in\RR\|0\leq x< 12\}$$ and let $R=(\RR,0,+)$ denote the additive monoid of real numbers. Then there is an action $R\times T\to T$, similar to the one above (see Exercise \ref{exc:clock}).

同じように, $T$が円周上の点の集合を表すとし, その要素は区間$[0,12)$の実数を表示しているとする. i.e. $$T=\{x\in\RR\|0\leq x< 12\}$$ そして$R=(\RR,0,+)$が実数の加法モノイドであるとする. このとき上記のそれと似て作用$R\times T\to T$が存在している(Exercise \ref{exc:clock}を見よ).

%One can think of this as an action of the monoid of time on the clock.
これは, 時計の上に表示される時間のモノイドの作用として考えることができる.

\end{example}

\begin{exercise}\label{exc:clock}~
%\sexc Realize the set $T:=[0,12)\ss\RR$ as the coequalizer of a pair of arrows $\RR\tto\RR$. 
%\next For any $x\in\RR$, realize the mapping $x\cdot-\taking T\to T$, implied by Example \ref{ex:clocks}, using the universal property of coequalizers. 
%\next Prove that it is an action.
%\endsexc
\sexc 集合$T:=[0,12)\ss\RR$を矢の対$\RR\tto\RR$の余等価子として実現せよ. 
\next 任意の$x\in\RR$に対して, Example \ref{ex:clocks}によってほめのかされる写像$x\cdot-\taking T\to T$を, 余等価子の普遍性を用いて実現せよ. 
\next これが作用であることを証明せよ.
\endsexc
\end{exercise}

\begin{exercise}
%Let $B$ denote the set of buttons (or positions) of a video game controller (other than, say `start' and `select'), and consider the free monoid $\List(B)$ on $B$. 
%\sexc What would it mean for $\List(B)$ to act on the set of states of some game? Imagine a video game $G'$ that uses the controller, but for which $\List(B)$ would not be said to act on the states of $G'$. Now imagine a simple game $G$ for which $\List(B)$ would be said to act.
%\next Can you think of a state $s$ of $G$, and two distinct elements $\ell,\ell'\in\List(B)$ such that $\ell\acts s=\ell'\acts s$? In video game parlance, what would you call an element $b\in B$ such that, for every state $s\in G$, one has $b\acts s=s$? 
%\next In video game parlance, what would you call a state $s\in S$ such that, for every sequence of buttons $\ell\in\List(B)$, one has $\ell\acts s=s$?
%\endsexc
$B$がテレビゲームのコントローラのボタン(あるいはその位置, あるいはそれに加えてスタートボタンやセレクトボタン)の集合を示しているとし, $B$上の自由モノイド$\List(B)$を考える. 
\sexc $\List(B)$に対して, あるゲームの状態の集合に作用するとは何を意味しているだろうか? コントローラを使うテレビゲーム$G'$で, しかしそれに対して$\List(B)$は$G'$の状態の集合への作用とは言えないようなものを想像せよ. 次に単純なゲーム$G$で, それに対して$\List(B)$が作用であると言えるものを想像せよ.
\next $G$の状態$s$と, $\ell\acts s=\ell'\acts s$であるところの二つの異なった状態$\ell,\ell'\in\List(B)$を考えることができるだろうか. テレビゲームの用語では, 全ての$s\in G$に対して$b\acts s=s$が得られるところの要素$b\in B$をなんと呼ぶだろうか?
\next テレビゲームの用語では, 全てのボタン列$\ell\in\List(B)$に対して$\ell\acts s=s$が得られるところの要素$s\in S$はなんと呼ぶだろうか?
\endsexc
\end{exercise}

\begin{application}

%Let $f\taking\RR\to\RR$ be a differentiable function of which we want to find roots (points $x\in\RR$ such that $f(x)=0$). Let $x_0\in\RR$ be a starting point. For any $n\in\NN$ we can apply \href{http://en.wikipedia.org/wiki/Newton's_method}{\text Newton's method} to $x_n$ to get $$x_{n+1}=x_n-\frac{f(x_n)}{f'(x_n)}.$$ 
%This is a monoid (namely $\NN$, the free monoid on one generator) acting on a set (namely $\RR$). 

$f\taking\RR\to\RR$を, 根($f(x)=0$であるところの$x\in\RR$)を求めたい対象の微分可能な函数としている. $x_0\in\RR$を初期点とする. 任意の$n\in\NN$に対して, $x_n$に\href{http://en.wikipedia.org/wiki/Newton's_method}{Newton法(\text Newton's method)}を適用することができそれによって$$x_{n+1}=x_n-\frac{f(x_n)}{f'(x_n)}$$が得られる.
これは集合(すなわち$\RR$)に作用するモノイド(すなわち$\NN$, 一要素上の自由モノイド)である.

%However, Newton's method can get into trouble. For example at a critical point it causes division by 0, and sometimes it can oscillate or overshoot. In these cases we want to perturb a bit to the left or right. To have these actions available to us, we would add ``perturb" elements to our monoid. Now we have more available actions at any point, but at the cost of using a more complicated monoid.

しかし, Newton法には問題が生じることがある. 例えば停留点(critical point)では0除算を起こしうるし, 時々振動したり目標を飛びこしてしまったりもする. これらの場合は, 我々は右や左に少し摂動を加えたくなる. これらの操作が使用可能になるようにするために, ``摂動''要素をモノイドに追加しよう. いまや我々は任意の点でより適用範囲が広い操作を手に入れた. ただしより複雑なモノイドを使うという対価を払っている. 

%When publishing an experimental finding, there may be some deep methodological questions that are not considered suitably important to mention. For example, one may not publish the kind solution finding method (e.g. Newton's method or Runge-Kutta) that was used, nor the set of available actions, e.g. what kinds of perturbation were used by the researcher. However, these may actually influence the reproducibility of results. By using a language such as that of monoid actions, we can align our data model with our unspoken assumptions about how functions are analyzed.

実験的発見が公表されるときには, 言及するにふさわしいほど重要だと考えられていなかったなんらかの深い方法論への疑問がしばしば出現する. 例えば, 解を求めるために使用された方法(e.g. Newton法あるいは\href{https://en.wikipedia.org/wiki/Runge\%E2\%80\%93Kutta_methods}{Runge-Kutta法}), あるいは可能な操作の集合, e.g. その研究者がどのような種類の摂動を用いたか, という類のことを公表しないかしれない. しかしながら, これらは実際には結果の再現性に影響するかしれない. モノイド作用のような言語を使用することにより, 我々はどのような函数が解析されたかといった言及されなかった仮定とデータ模型とを連携させることができる.

\end{application}

\begin{remark}

%A monoid is useful for understanding how an agent acts on the set of states of an object, but there is only one {\em kind} of action. At any point, all actions are available. In reality it is often the case that contexts can change and different actions are available at different times. For example on a computer, the commands available in one application have no meaning in another. This will get us to categories in the next chapter. 

モノイドは, 対象の状態の集合に対してagentがどのように作用するかを理解するのに便利だが, そこには一つの\emph{類}としての作用しか存在していない. すなわち任意の時点で全ての作用が適用可能である. 現実では, 文脈が変更を受けたり異なった時間に異なった作用が適用可能になることもしばしば起こる. 例えば計算機では, あるアプリケーションで可能な命令は他のアプリケーションでは意味を持たない. 次の章で、我々はこれによって圏へと導かれるだろう.

\end{remark}


%% Subsubsection %%

%\subsubsection{Monoids actions as ologs}
\subsubsection{ologとしてのモノイド作用}

%If monoids are understood in terms of how they act on sets, then it is reasonable to think of them in terms of ologs. In fact, the ologs associated to monoids are precisely those ologs that have exactly one type (and possibly many arrows and commutative diagrams). 

もしモノイドが集合へどのように作用するかという観点で理解できるとすれば, モノイドをologのかんんで考えるのは妥当である. 実際のところ正確に, モノイドに関連したologはちょうど1つの型を持つ(そしてたくさんの矢と可換図式を持ちうる) ologである. 

\begin{example}\label{ex:monoid as olog}\index{monoid!olog of}

%In this example we show how to associate an olog to a monoid action. Consider the monoid $M$ generated by the set $\{u,d,r\}$, standing for ``up, down, right", and subject to the relations $$[u,d]\sim[\;],\hsp[d,u]\sim[\;],\hsp[u,r]=[r,u],\hsp \tn{and}\hsp [d,r]=[r,d].$$
%We might imagine that $M$ acts on the set of positions for a character in an old video game. In that case the olog corresponding to this action should look something like the following:
%\begin{center}
%\includegraphics[width=\textwidth]{monoidOlog}
%\end{center}

この例では我々はどのようにologとモノイド作用を関連つければいいかを示す. ``up'', ``down'', ``right''とし理解される集合$\{u,d,r\}$によって生成され, 関係$$[u,d]\sim[\;],\hsp[d,u]\sim[\;],\hsp[u,r]=[r,u],\hsp \tn{and}\hsp [d,r]=[r,d]$$に従うモノイド$M$を考える.
$M$はある古いテレビゲーム中での文字の位置の集合に作用すると想像してみよう. この場合, この作用に対応するologは以下のようになるべきであろう.
\begin{center}
\includegraphics[width=\textwidth]{monoidOlog}
\end{center}

\end{example}

%% Subsubsection %%

%\subsubsection{Finite state machines}\label{sec:FSMs}\index{finite state machine}
\subsubsection{有限状態機械}\label{sec:FSMs}\index{finite state machine}

%According to Wikipedia, a \href{http://en.wikipedia.org/wiki/Finite_state_machine#Mathematical_model}{\em deterministic finite state machine} is a quintuple $(\Sigma,S,s_0,\delta,F)$, where
%\begin{enumerate}
%\item $\Sigma$ is a finite non-empty set of symbols, called the {\em input alphabet},
%\item $S$ is a finite, non-empty set, called {\em the state set},
%\item $\delta\taking \Sigma\times S\to S$ is a function, called the {\em state-transition function}, and
%\item $s_0\in S$ is an element, called {\em the initial state},
%\item $F\ss S$ is a subset, called the {\em set of final states}.
%\end{enumerate}

Wikipediaによると, \href{http://en.wikipedia.org/wiki/Finite_state_machine#Mathematical_model}{\emph{有限状態機械(deterministic finite state machine)}}とは五つ組$(\Sigma,S,s_0,\delta,F)$である. ここで
\begin{enumerate}
\item $\Sigma$は空ではない有限の記号の集合であり, \emph{入力文字集合(input alphabet)}と呼ばれる.
\item $S$は空ではない有限の集合であり, \emph{状態集合(the state set)}と呼ばれる.
\item $\delta\taking \Sigma\times S\to S$は函数であり, \emph{状態遷移函数(state-transition function)}と呼ばれる.
\item $s_0\in S$は要素であり, \emph{初期状態(the initial state)}と呼ばれる.
\item $F\ss S$は部分集合であり, \emph{終了状態の集合(set of final states)}と呼ばれる.
\end{enumerate}

%In this book we will not worry about the initial state and the set of final states, concerning ourselves more with the interaction via $\delta$ of the alphabet $\Sigma$ on the set $S$ of states.

この本では我々は初期状態と終状態の集合については気をめぐらせないことにして, 状態の集合$S$への文字$\Sigma$の作用$\delta$を通じた相互作用に, 関心を集中することにする.

\begin{figure}[h]
%\begin{center}
\centering
\includegraphics[height=2in]{FSM1}
%\end{center}
%\caption{A finite state machine with alphabet $\Sigma=\{a,b\}$ and state set $S=\{\tn{State 0, State 1, State 2}\}$. If pressed, we will make State 0 the initial state and \{State 2\} the set of final states.}\label{fig:fsa}
\caption{文字$\Sigma=\{a,b\}$と状態$S=\{\tn{State 0, State 1, State 2}\}$を持つ有限状態機械の例. もし要求されたならば, State 0を初期状態とし, $\{\text{State 2}\}$を終状態の集合とする.}
\end{figure}\label{fig:fsa}
%The following proposition expresses the notion of finite state automata in terms of free monoids and their actions on finite sets.
以下のpropsitionは自由モノイドと有限集合へのそれらの作用という観点から有限状態オートマトンの概念を表現している.

\begin{proposition}

%Let $\Sigma, S$ be finite non-empty sets. Giving a function $\delta\taking\Sigma\times S\to S$ is equivalent to giving an action of the free monoid $\List(\Sigma)$ on $S$. 

$\Sigma$と$S$を有限な空でない集合とする. 函数$\delta\taking\Sigma\times S\to S$を与えることは$S$への自由モノイド$\List(\Sigma)$の作用を与えることと等しい.

\end{proposition}

\begin{proof}

%By Definition \ref{def:monoid action}, we know that function $\epsilon\taking\List(\Sigma)\times S\to S$ constitutes an action of the monoid $\List(\Sigma)$ on the set $S$ if and only if, for all $s\in S$ we have $\epsilon([\;],s)=s$, and for any two elements $m,m'\in\List(\Sigma)$ we have $\epsilon(m,\epsilon(m',s))=\epsilon(m\star m',s)$, where $m\star m'$ is the concatenation of lists. Let $$A=\{\epsilon\taking \List(\Sigma)\times S\to S\|\epsilon\tn{ constitutes an action}\}.$$ We need to prove that there is an isomorphism of sets $$\phi\taking A\To{\iso}\Hom_\Set(\Sigma\times S,S).$$

Definition \ref{def:monoid action}により, 函数$\epsilon\taking\List(\Sigma)\times S\to S$が$S$への自由モノイド$\List(\Sigma)$の作用をなすのは, 全ての$s\in S$に対して$\epsilon([\;],s)=s$が得られ, 全ての二要素の組$m,m'\in\List(\Sigma)$に対して$\epsilon(m,\epsilon(m',s))=\epsilon(m\star m',s)$が得られるときであり, かつその時に限る. ここで$m\star m'$はリストの結合である. $$A=\{\epsilon\taking \List(\Sigma)\times S\to S\|\epsilon\tn{ constitutes an action}\}$$とする. 我々は, 集合の同型写像$$\phi\taking A\To{\iso}\Hom_\Set(\Sigma\times S,S)$$が存在することを証明する必要がある.

%Given an element $\epsilon\taking\List(\Sigma)\times S\to S$ in $A$, define $\phi(\epsilon)$ on an element $(\sigma,s)\in\Sigma\times S$ by $\phi(\epsilon)(\sigma,s):=\epsilon([\sigma],s)$, where $[\sigma]$ is the one-element list. We now define $\psi\taking\Hom_\Set(\Sigma\times S,S)\to A$.

$A$の要素$\epsilon\taking\List(\Sigma)\times S\to S$が与えられたとき, 要素$(\sigma,s)\in\Sigma\times S$上に$\phi(\epsilon)(\sigma,s):=\epsilon([\sigma],s)$によって$\phi(\epsilon)$を定義する. ここで$[\sigma]$は一要素リストである. 次に$\psi\taking\Hom_\Set(\Sigma\times S,S)\to A$を定義する.

%Given an element $f\in\Hom_\Set(\Sigma\times S,S)$, define $\psi(f)\taking\List(\Sigma)\times S\to S$ on a pair $(L,s)\in\List(\Sigma)\times S$, where $L=[\epsilon_1,\ldots,\epsilon_n]$ as follows. By induction, if $n=0$, put $\psi(f)(L,s)=s$; if $n\geq 1$, let $L'=[\epsilon_1,\ldots,\epsilon_{n-1}]$ and put $\psi(f)(L,s)=\psi(f)(L',f(\epsilon_n,s))$. One checks easily that $\psi(f)$ satisfies the two rules above, making it an action of $\List(\Sigma)$ on $S$. It is also easy to check that $\phi$ and $\psi$ are mutually inverse, completing the proof.

要素$f\in\Hom_\Set(\Sigma\times S,S)$が与えられたとき, 対$(L,s)\in\List(\Sigma)\times S$上に$\psi(f)\taking\List(\Sigma)\times S\to S$を定義する. ここで$L=[\epsilon_1,\ldots,\epsilon_n]$は以下のように定められる. 帰納的に, $n=0$ならば$\psi(f)(L,s)=s$, $n\geq 1$ならば$L'=[\epsilon_1,\ldots,\epsilon_{n-1}]$, そして$\psi(f)(L,s)=\psi(f)(L',f(\epsilon_n,s))$とする. $\psi(f)$が上記の二つの法則を満たし, $S$への$\List(\Sigma)$の作用を成立させることは簡単に確認できる. 証明を完了するにあたって, $\phi$と$\psi$が互いに逆函数であることも簡単に確認できる.

\end{proof}

%We sum up the idea of this section as follows:
この節のアイデアは以下のようにまとめるあげることができる.
\begin{slogan}
%A finite state machine is an action of a free monoid on a finite set.
有限状態機械は有限集合への自由モノイドの作用である.
\end{slogan}

\begin{exercise}
%Consider the functions $\phi$ and $\psi$ above. 
%\sexc Show that for any $f\taking\Sigma\times S\to S$, the map $\psi(f)\taking\List(\Sigma)\times S\to S$ constitutes an action. 
%\next Show that $\phi$ and $\psi$ are mutually inverse functions (i.e. $\phi\circ\psi=\id_{\Hom(\Sigma\times S,S)}$ and $\psi\circ\phi=\id_{A}.$)
%\endsexc
上記の函数$\phi$と$\psi$について考える.
\sexc 任意の$f\taking\Sigma\times S\to S$に対して, 写像$\psi(f)\taking\List(\Sigma)\times S\to S$が作用をなすことを示せ.
\next $\phi$と$\psi$が互いに逆函数(i.e. $\phi\circ\psi=\id_{\Hom(\Sigma\times S,S)}$および$\psi\circ\phi=\id_{A}$)であることを示せ.
\endsexc
\end{exercise}

%%%% Subsection %%%%

%\subsection{Monoid action tables}\label{sec:monoid action table}\index{action table}
\subsection{モノイド作用表}\label{sec:monoid action table}\index{action table}

%Let $M$ be a monoid generated by the set $G=\{g_1,\ldots,g_m\}$, and with some relations, and suppose that $\alpha\taking M\times S\to S$ is an action of $M$ on a set $S=\{s_1,\ldots,s_n\}$. We can represent the action $\alpha$ using an {\em action table} whose columns are the elements of $G$ and whose rows are the elements of $S$. In each cell $(row,col)$, where $row\in S$ and $col\in G$, we put the element $\alpha(col,row)\in S$. 

$M$を集合$G=\{g_1,\ldots,g_m\}$によって生成されモノイドでいくつかの関係を伴うものとし, $\alpha\taking M\times S\to S$を集合$S=\{s_1,\ldots,s_n\}$への$M$の作用と仮定する. 我々は作用$\alpha$を\emph{作用表(action table)}をも用いて表現できる. この表の列は$G$の要素であり, 行は$S$の要素である. $\mathit{row}\in S$および$\mathit{col}\in G$としたとき, それぞれの升目$(\mathit{row},\mathit{col)}$の中には, 要素$\alpha(\mathit{col},\mathit{row})\in S$を入れる. 

%\begin{example}[Action table]\label{ex:action table}
\begin{example}[作用表]\label{ex:action table}

%If $\Sigma$ and $S$ are the sets from Figure \ref{fig:fsa}, the displayed action of $\List(\Sigma)$ on $S$ would be given by the action table
$\Sigma$と$S$がFigure \ref{fig:fsa}の集合であるならば, そこで示されている$S$への$\List(\Sigma)$の作用は以下の作用表によって与えられるだろう.
\begin{align}\label{dia:action table for FSM}
\begin{tabular}{| l || l | l |}\bhline
%\multicolumn{3}{|c|}{Action from \ref{fig:fsa}}\\\bhline
\multicolumn{3}{|c|}{\ref{fig:fsa}の作用}\\\bhline
%{\bf ID}&{\bf a}&{\bf b}\\\bbhline
\textbf{ID}&\textbf{a}&\textbf{b}\\\bbhline
State 0&State 1&State 2\\\hline
State 1& State 2& State 1\\\hline
State 2&State 0&State 0\\\bhline
\end{tabular}
\end{align}

\end{example}

%\begin{example}[Multiplication action table]\label{ex:multiplication table}
\begin{example}[積の作用表]\label{ex:multiplication table}

%Every monoid acts on itself by its multiplication formula, $M\times M\to M$. If $G$ is a generating set for $M$, we can write the elements of $G$ as the columns and the elements of $M$ as rows, and call this a multiplication table. For example, let $(\NN,1,*)$ denote the multiplicative monoid of natural numbers. The multiplication table is as follows:
全てのモノイドは, 積の式$M\times M\to M$によってそれ自身への作用となる. $G$が$M$を生成する集合であるとすると, 我々は$G$の要素を列として書き, $M$の要素を行として書きことができる. これは積の表と呼ぶことができる. 例えば, $(\NN,1,*)$が自然数の乗法モノイドを示しているとする. 積の表は以下のようになる.
\begin{align}
\begin{tabular}{| l || l | l | l | l | l | l | l |}\bhline
%\multicolumn{8}{|c|}{Multiplication of natural numbers}\\\bhline
\multicolumn{8}{|c|}{自然数の積}\\\bhline
%{\bf $\NN$}&{\bf 0}&{\bf 1}&{\bf 2}&{\bf 3}&{\bf 4}&{\bf 5}&{\bf $\cdots$}\\\bbhline
{$\NN$}&\textbf{0}&\textbf{1}&\textbf{2}&\textbf{3}&\textbf{4}&\textbf{5}&$\cdots$\\\bbhline
0&0&0&0&0&0&0&$\cdots$\\\hline
1&0&1& 2& 3 & 4&5&$\cdots$\\\hline
2&0&2&4&6&8&10&$\cdots$\\\hline
3&0&3&6&9&12&15&$\cdots$\\\bhline
4&0&4&8&12&16&20&$\cdots$\\\bhline
\vdots&\vdots&\vdots&\vdots&\vdots&\vdots&\vdots&$\ddots$\\\hline
21&0&21&42&63&84&105&$\cdots$\\\hline
\vdots&\vdots&\vdots&\vdots&\vdots&\vdots&\vdots&$\ddots$\\\bhline
\end{tabular}
\end{align}
%Try to understand what is meant by this: ``applying column $2$ and then column $2$ returns the same thing as applying column $4$."
``列$2$を作用しそれからまた列$2$を作用させることは列$4$を作用させたものと同じものを返す''が何を意味しているかの理解を試みてみよう.

%In the above table, we were implicitly taking every element of $\NN$ as a generator (since we had a column for every natural number). In fact, there is a smallest generating set for the monoid $(\NN,1,*)$, so that every element of the monoid is a product of some combination of these generators, namely the primes and 0.
上の表では, 暗黙的に$\NN$のそれぞれの要素を生成子とみなした(なぜならそれぞれの自然数に対して列があるからだ). 実際, モノイド$(\NN,1,*)$を生成する最小の集合が存在し, よってモノイドのそれぞれの要素はその生成子のなんらかの組み合わせの積となる. その生成子はすなわち素数と0である.
\begin{align*}
\begin{tabular}{| l || l | l | l | l | l | l | l |}\bhline
%\multicolumn{8}{|c|}{Multiplication of natural numbers}\\\bhline
\multicolumn{8}{|c|}{自然数の積}\\\bhline
%{\bf $\NN$}&{\bf 0}&{\bf 2}&{\bf 3}&{\bf 5}&{\bf 7}&{\bf 11}&{\bf $\cdots$}\\\bbhline
{$\NN$}&\textbf{0}&\textbf{2}&\textbf{3}&\textbf{5}&\textbf{7}&\textbf{11}&{$\cdots$}\\\bbhline
0&0&0&0&0&0&0&$\cdots$\\\hline
1&0&2& 3& 5 & 7&11&$\cdots$\\\hline
2&0&4&6&10&14&22&$\cdots$\\\hline
3&0&6&9&15&21&33&$\cdots$\\\bhline
4&0&8&12&20&28&44&$\cdots$\\\bhline
\vdots&\vdots&\vdots&\vdots&\vdots&\vdots&\vdots&$\ddots$\\\hline
21&0&42&63&105&147&231&$\cdots$\\\hline
\vdots&\vdots&\vdots&\vdots&\vdots&\vdots&\vdots&$\ddots$\\\bhline
\end{tabular}
\end{align*}

\end{example}

\begin{exercise}
%Let $\NN$ be the additive monoid of natural numbers, let $S=\{0,1,2,\ldots,11\}$, and let $\cdot\taking\NN\times S\to S$ be the action given in Example \ref{ex:clocks}. Using a nice small generating set for the monoid, write out the corresponding action table.
$\NN$を自然数の加法モノイドとする. $S=\{0,1,2,\ldots,11\}$とし, $\cdot\taking\NN\times S\to S$をExample \ref{ex:clocks}で与えられた作用とする. モノイドを生成する小さなよい集合をを用いて, 対応する作用表を書き出せ.
\end{exercise}

%%%% Subsection %%%%

%\subsection{Monoid homomorphisms}
\subsection{モノイド準同型}

%A monoid $(M,e,\star)$ involves a set, an identity element, and a multiplication formula. For two monoids to be comparable, their sets, their identity elements, and their multiplication formulas should be appropriately comparable.\index{appropriate comparison} For example the additive monoids $\NN$ and $\ZZ$ should be comparable because $\NN\ss\ZZ$ is a subset, the identity elements in both cases are the same $e=0$, and the multiplication formulas are both integer addition. 

モノイド$(M,e,\star)$は集合, 単位元, 積の式を要件とする. 二つのモノイドが比較可能であるとき, その集合, 単位元, 積の式は適切に比較できるべきである.\index{appropriate comparison} 例えば, 加法モノイド$\NN$と$\ZZ$は比較可能であるべきである. なぜなら$\NN\ss\ZZ$は部分集合であり, 単位元は両方の場合で同じ$e=0$であり, 積の式は両者ともに整数の足し算であるからだ.

\begin{definition}\label{def:monoid hom}\index{monoid!homomorphism}

%Let $\mcM:=(M,e,\star)$ and $\mcM':=(M',e',\star')$ be monoids. A {\em monoid homomorphism $f$ from $\mcM$ to $\mcM'$}, denoted $f\taking\mcM\to\mcM'$, is a function $f\taking M\to M'$ satisfying two conditions:
%\begin{itemize}
%\item $f(e)=e'$, and 
%\item $f(m_1\star m_2)=f(m_1)\star'f(m_2)$, for all $m_1,m_2\in M$.
%\end{itemize}

$\mcM:=(M,e,\star)$と$\mcM':=(M',e',\star')$をモノイドとする. \emph{$\mcM$から$\mcM'$へのモノイド準同型$f$(monoid homomorphism $f$ from $\mcM$ to $\mcM'$)}は, 以下の二つの条件を満たす函数$f\taking M\to M'$であり, $f\taking\mcM\to\mcM'$と表記される.
\begin{itemize}
\item $f(e)=e'$,
\item 全ての$m_1,m_2\in M$に対して$f(m_1\star m_2)=f(m_1)\star'f(m_2)$.
\end{itemize}

%The set of monoid homomorphisms from $\mcM$ to $\mcM'$ is denoted $\Hom_{\Mon}(\mcM,\mcM')$.

$\mcM$から$\mcM'$へのモノイド準同型の集合は$\Hom_{\Mon}(\mcM,\mcM')$と表記される.

\end{definition}

\begin{example}[From $\NN$ to $\ZZ$]\label{ex:nat to int}

%As stated above, the inclusion map $i\taking\NN\to\ZZ$ induces a monoid homomorphism $(\NN,0,+)\to(\ZZ,0,+)$ because $i(0)=0$ and $i(n_1+n_2)=i(n_1)+i(n_2)$. 

上で述べられたように, 包含写像$i\taking\NN\to\ZZ$はモノイド準同型$(\NN,0,+)\to(\ZZ,0,+)$を誘導する. なぜなら$i(0)=0$かつ$i(n_1+n_2)=i(n_1)+i(n_2)$であるからだ.

%Let $i_5\taking\NN\to\ZZ$ denote the function $i_5(n)=5*n$, so $i_5(4)=20$. This is also a monoid homomorphism because $i_5(0)=5*0=0$ and $i_5(n_1+n_2)=5*(n_1+n_2)=5*n_1+5*n_2=i_5(n_1)+i_5(n_2).$

$i_5\taking\NN\to\ZZ$が函数$i_5(n)=5*n$を示すとする. よって$i_5(4)=20$である. これはまたモノイド準同型でもある. なぜなら$i_5(0)=5*0=0$かつ$i_5(n_1+n_2)=5*(n_1+n_2)=5*n_1+5*n_2=i_5(n_1)+i_5(n_2)$であるからだ.

\end{example}

\begin{application}\label{app:RNA reader 1}

%Let $R=\{a,c,g,u\}$ and let $T=R^3$, the set of triplets in $R$. Let $\mcR=\List(R)$ be the free monoid on $R$ and let $\mcT=\List(T)$ denote the free monoid on $T$. There is a monoid homomorphism $F\taking\mcT\to\mcR$ given by sending $t=(r_1,r_2,r_3)$ to the list $[r_1,r_2,r_3]$.
%\footnote{More precisely, the monoid homomorphism $F$ sends a list $[t_1,t_2,\ldots,t_n]$ to the list $[r_{1,1},r_{1,2},r_{1,3},r_{2,1},r_{2,2},r_{2,3},\ldots,r_{n,1},r_{n,2},r_{n,3}]$, where for each $0\leq i\leq n$ we have $t_i=(r_{i,1},r_{i,2},r_{i,3})$.}

$R=\{a,c,g,u\}$とし, $T=R^3$すなわち$R$の三つ組の集合とする. $\mcR=\List(R)$を$R$上の自由モノイドとし, $\mcT=\List(T)$が$T$上の自由モノイドを示しているとする. $t=(r_1,r_2,r_3)$をリスト$[r_1,r_2,r_3]$に送ることによって得られるモノイド準同型$F\taking\mcT\to\mcR$が存在している.
\footnote{より正確には, モノイド準同型$F$はリスト$[t_1, t_2, \ldots, t_n]$をリスト$[r_{1,1}, r_{1,2}, r_{1,3}, r_{2,1}, r_{2,2}, r_{2,3},\allowbreak \ldots,\allowbreak r_{n,1}, r_{n,2}, r_{n,3}]$に送る. ここでそれぞれの$0\leq i\leq n$に対し$t_i=(r_{i,1}, r_{i,2}, r_{i,3})$とした.}

%If $A$ be the set of amino acids and $\mcA=\List(A)$ the free monoid on $A$, the process of \href{http://en.wikipedia.org/wiki/Translation_(biology)}{\text translation} gives a monoid homomorphism $G\taking\mcT\to\mcA$, turning a list of RNA triplets into a polypeptide. But how do we go from a list of RNA nucleotides to a polypeptide? The answer is that there is no good way to do this mathematically. So what is going wrong?

$A$がアミノ酸の集合で$\mcA=\List(A)$が$A$上の自由モノイドであるとき, \href{http://en.wikipedia.org/wiki/Translation_(biology)}{翻訳(\text{translation})}の機構によって, RNAの三つ組をポリペプチド\footnote{訳注: 小さなタンパク質はポリペプチド(polypeptide)とも呼ばれる.}へ変換するモノイド準同型$G\taking\mcT\to\mcA$が与えられる. しかしどのようにしてRNAヌクレオチドのリストからポリペプチドに移行できるだろうか? これを数学的に行なうよい方法はないというのがその答である. では何が悪くなるのであろうか?

%The answer is that there should not be a monoid homomorphism $\mcR\to\mcA$ because not all sequences of nucleotides produce a polypeptide; for example if the sequence has only two elements, it does not code for a polypeptide. There are several possible remedies to this problem. One is to take the image of $F$, which is a submonoid $\mcR'\ss\mcR$. It is not hard to see that there is a monoid homomorphism $F'\taking\mcR'\to\mcT$, and we can compose it with $G$ to get our desired monoid homomorphism $G\circ F'\taking\mcR'\to\mcA$. 
%\footnote{Adding stop-codons to the mix we can handle more of $\mcR$, e.g. sequences that don't have a multiple-of-three many nucleotides.}

モノイド準同型$\mcR\to\mcA$は存在するべきではないというのがその答である. なぜなら全てのヌクレオチドの列がポリペプチドを生成するわけではないからである. 例えば, もし列に二つしか要素が含まれていないならば, それはポリペプチドを符号化していない. この問題を修正するにはいくつかの方法がある. 一つは部分モノイド$\mcR'\ss\mcR$であるところの$F$の像を取ることである. モノイド準同型$F'\taking\mcR'\to\mcT$が存在することを見るのは難しくはない. そして$G$と組みあわせて欲しかったモノイド準同型$G\circ F'\taking\mcR'\to\mcA$を得ることができる.
\footnote{停止コドンを含めることによってより多くの$\mcR$, e.g. ヌクレオチドの数が3の倍数ではない列, を扱えることができる.}

\end{application}

\begin{example}\label{ex:trivial monoid homomorphism}\index{monoid!trivial homomorphism}

%Given any monoids $\mcM$ there is a unique monoid homomorphism from $\mcM$ to the trivial monoid $\ul{1}$ (see Example \ref{ex:trivial monoid}). There is also a unique homomorphism $\ul{1}\to\mcM$. These facts together have an upshot: between any two monoids $\mcM$ and $\mcM'$ we can always construct a homomorphism 
%$$\mcM\Too{!}\ul{1}\Too{!}\mcM'$$
%which we call the {\em trivial homomorphism $\mcM\to\mcM'$}.\index{trivial homomorphism!of monoids} A morphism $\mcM\to\mcM'$ that is not trivial is called a {\em nontrivial homomorphism}.

任意のモノイド$\mcM$が与えられたとき, $\mcM$から自明なモノイド$\ul{1}$(Example \ref{ex:trivial monoid}を見よ)へのモノイド準同型が一意に存在する. また準同型$\ul{1}\to\mcM$も一意に存在する. この事実を併せるとある結論が得られる. すなわち任意の二つのモノイド$\mcM$と$\mcM'$の間には, 常に\emph{自明な準同型(trivial homomorphism) $\mcM\to\mcM'$}\index{trivial homomorphism!of monoids}と呼ばれる準同型$$\mcM\Too{!}\ul{1}\Too{!}\mcM'$$を構築することができる.
自明でない射$\mcM\to\mcM'$は\emph{非自明な準同型(nontrivial homomorphism)}と呼ばれる.

\end{example}

\begin{proposition}\label{prop:int to nat trivial}

%Let $\mcM=(\ZZ,0,+)$ and $\mcM'=(\NN,0,+)$. The only monoid homomorphism $f\taking\mcM\to\mcM'$ sends every element $m\in\ZZ$ to $0\in\NN$.

$\mcM=(\ZZ,0,+)$および$\mcM'=(\NN,0,+)$とする. 唯一存在するモノイド準同型$f\taking\mcM\to\mcM'$は全ての要素$m\in\ZZ$を$0\in\NN$に送る.

\end{proposition}

\begin{proof}

%Let $f\taking\mcM\to\mcM'$ be a monoid homomorphism, and let $n=f(1)$ and $n'=f(-1)$ in $\NN$. Then we know that since $0=1+(-1)$ in $\ZZ$ we must have $0=f(0)=f(1+(-1))=f(1)+f(-1)=n+n'\in\NN$. But if $n\geq 1$ then this is impossible, so $n=0$. Similarly $n'=0$. Any element $m\in\ZZ$ can be written $m=1+1+\cdots+1$ or as $m=-1+-1+\cdots+-1$, and it is easy to see that $f(1)+f(1)+\cdots+f(1)=0=f(-1)+f(-1)+\cdots+f(-1)$. Therefore, $f(m)=0$ for all $m\in\ZZ$. 

$f\taking\mcM\to\mcM'$をモノイド準同型とし, $n=f(1)$と$n'=f(-1)$を$\NN$の要素とする. このとき, $0=1+(-1)$は$\ZZ$に含まれるから, $0=f(0)=f(1+(-1))=f(1)+f(-1)=n+n'\in\NN$でなければならないことが分かる. しかしもし$n\geq 1$であるならばこれは不可能であるから, よって$n=0$である. 同様にして$n'=0$である. 任意の要素$m\in\ZZ$は$m=1+1+\cdots+1$あるいは$m=-1+-1+\cdots+-1$として書くことができ, そして$f(1)+f(1)+\cdots+f(1)=0=f(-1)+f(-1)+\cdots+f(-1)$であることは簡単に見てとれる. それゆえに全ての$m\in\ZZ$に対して$f(m)=0$である.

\end{proof}

\begin{exercise}
%For any $m\in\NN$ let $i_m\taking\NN\to\ZZ$ be the function $i_m(n)=m*n$. All such functions are monoid homomorphisms $(\NN,0,+)\to(\ZZ,0,+)$. Do any monoid homomorphisms $(\NN,0,+)\to(\ZZ,0,+)$ not come in this way? For example, what about using $n\mapsto 5*n-1$ or $n\mapsto n^2$, or some other function? 
任意の$m\in\NN$に対して$i_m\taking\NN\to\ZZ$を函数$i_m(n)=m*n$とする. このような函数は全てモノイド準同型$(\NN,0,+)\to(\ZZ,0,+)$である. 任意のモノイド準同型$(\NN,0,+)\to(\ZZ,0,+)$はこの方法で出て来るものだろうか? 例えば, $n\mapsto 5*n-1$や$n\mapsto n^2$, あるいは他のなんらかの函数を使うのはどうだろうか?
\end{exercise}

\begin{exercise}
%Let $\mcM:=(\NN,0,+)$ be the additive monoid of natural numbers, let $\mcN=(\RR_{\geq0},0,+)$ be the additive monoid of nonnegative real numbers, and let $\mcP:=(\RR_{>0},1,*)$ be the multiplicitive monoid of positive real numbers. Can you think of any nontrivial monoid homomorphisms of the following sorts: $$\mcM\to\mcN,\hsp\mcM\to\mcP,\hsp\mcN\to\mcP,\hsp \mcN\to\mcM,\hsp\mcP\to\mcN?$$
$\mcM:=(\NN,0,+)$を自然数の加法モノイド, $\mcN=(\RR_{\geq0},0,+)$を非負実数の加法モノイド, $\mcN=(\RR_{\geq0},0,+)$を正実数の乗法モノイドとする. 以下のような種類の, なんらかの非自明モノイド準同型について考えることができるだろうか.  $$\mcM\to\mcN,\hsp\mcM\to\mcP,\hsp\mcN\to\mcP,\hsp \mcN\to\mcM,\hsp\mcP\to\mcN?$$
\end{exercise}


%% Subsubsection %%

%\subsubsection{Homomorphisms from free monoids}
\subsubsection{自由モノイドからの準同型}

%Recall that $(\NN,0,+)$ is the free monoid on one generator. It turns out that for any other monoid $\mcM=(M,e,\star)$, the set of monoid homomorphisms $\NN\to\mcM$ is in bijection with the set $M$. This is a special case (in which $G$ is a set with one element) of the following proposition.

$(\NN,0,+)$は一生成子上の自由モノイドであることを思い出そう. 任意の他のモノイド$\mcM=(M,e,\star)$に対して, モノイド準同型$\NN\to\mcM$の集合は, 集合$M$との間に全単射がある. これは以下のpropositionの($G$が一要素であるという)特殊な場合である.

\begin{proposition}\label{prop:free monoid}

%Let $G$ be a set, let $F(G):=(\List(G),[\;],\plpl)$ be the free monoid on $G$, and let $\mcM:=(M,e,\star)$ be any monoid. There is a natural bijection
%$$\Hom_\Mon(F(G),\mcM)\To{\iso}\Hom_\Set(G,M).$$

$G$を集合, $F(G):=(\List(G),[\;],\plpl)$を$G$上の自由モノイドとし, $\mcM:=(M,e,\star)$を任意のモノイドとする. このとき自然な全単射が存在する.
$$\Hom_\Mon(F(G),\mcM)\To{\iso}\Hom_\Set(G,M).$$

\end{proposition}

\begin{proof}

%We provide a function $\phi\taking\Hom_\Mon(F(G),\mcM)\to\Hom_\Set(G,M)$ and a function $\psi\taking\Hom_\Set(G,M)\to\Hom_\Mon(F(G),\mcM)$ and show that they are mutually inverse. Let us first construct $\phi$. Given a monoid homomorphism $f\taking F(G)\to\mcM$, we need to provide $\phi(f)\taking G\to M$. Given any $g\in G$ we define $\phi(f)(g):=f([g]).$ 

我々は函数$\phi\taking\Hom_\Mon(F(G),\mcM)\to\Hom_\Set(G,M)$および函数$\psi\taking\Hom_\Set(G,M)\to\Hom_\Mon(F(G),\mcM)$を与え, それらが互いに逆函数であることを示す. 最初に$\phi$を構成しよう. モノイド準同型$f\taking F(G)\to\mcM$が与えられたとき, $\phi(f)\taking G\to M$を与えることが必要である. 任意の$g\in G$が与えられたとき, $\phi(f)(g):=f([g])$と定義する.

%Now let us construct $\psi$. Given $p\taking G\to M$, we need to provide $\psi(p)\taking\List(G)\to\mcM$ such that $\psi(p)$ is a monoid homomorphism. For a list $L=[g_1,\ldots,g_n]\in\List(G)$, define $\psi(p)(L):=p(g_1)\star\cdots\star p(g_n)\in M$. In particular, $\psi(p)([\;])=e$. It is not hard to see that this is a monoid homomorphism. It is also easy to see that $\phi\circ\psi(p)=p$ for all $p\in\Hom_\Set(G,M)$. We show that $\psi\circ\phi(f)=f$ for all $f\in\Hom_\Mon(F(G),\mcM)$. Choose $L=[g_1,\ldots,g_n]\in\List(G)$. Then 
%$$\psi(\phi f)(L)=(\phi f)(g_1)\star\cdots\star(\phi f)(g_n)=f[g_1]\star\cdots\star f[g_n]=f([g_1,\ldots,g_n])=f(L).$$

次に$\psi$を構成しよう. $p\taking G\to M$が与えられたとき, $\psi(p)$がモノイド準同型であるところの$\psi(p)\taking\List(G)\to\mcM$を与える必要がある. リスト$L=[g_1,\ldots,g_n]\in\List(G)$に対して, $\psi(p)(L):=p(g_1)\star\cdots\star p(g_n)\in M$と定義する. 別に$\psi(p)([\;])=e$とする. これがモノイド準同型であることを見るのは難しくない. 全ての$p\in\Hom_\Set(G,M)$に対して$\phi\circ\psi(p)=p$であることを見るのも簡単である. 我々は全ての$f\in\Hom_\Mon(F(G),\mcM)$に対して$\psi\circ\phi(f)=f$を示す. $L=[g_1,\ldots,g_n]\in\List(G)$を選ぶ. このとき
$$\psi(\phi f)(L)=(\phi f)(g_1)\star\cdots\star(\phi f)(g_n)=f[g_1]\star\cdots\star f[g_n]=f([g_1,\ldots,g_n])=f(L).$$

\end{proof}

\begin{exercise}
%Let $G=\{a,b\}$, let $\mcM:=(M,e,\star)$ be any monoid, and let $f\taking G\to M$ be given by $f(a)=m$ and $f(b)=n$, where $m,n\in M$. If $\psi\taking\Hom_\Set(G,M)\to\Hom_\Mon(F(G),\mcM)$ is the function from the proof of Proposition \ref{prop:free monoid} and $L=[a,a,b,a,b]$, what is $\psi(f)(L)$ ?
$G=\{a,b\}$, $\mcM:=(M,e,\star)$を任意のモノイドとし$f\taking G\to M$を$f(a)=m$および$f(b)=n$で与えるとする ここで$m,n\in M$である. $\psi\taking\Hom_\Set(G,M)\to\Hom_\Mon(F(G),\mcM)$がProposition \ref{prop:free monoid}の証明での函数であり, $L=[a,a,b,a,b]$であるとすれば, $\psi(f)(L)$は何だろうか?
\end{exercise}

%% Subsubsection %%

%\subsubsection{Restriction of scalars}
\subsubsection{係数制限}

%A monoid homomorphism $f\taking M\to M'$ (see Definition \ref{def:monoid hom}) ensures that the elements of $M$ have a reasonable interpretation in $M'$; they act the same way over in $M'$ as they did back home in $M$. If we have such a homomorphism $f$ and we have an action $\alpha\taking M'\times S\to S$ of $M'$ on a set $S$, then we have a method for allowing $M$ to act on $S$ as well. Namely, we take an element of $M$, send it over to $M'$, and act on $S$. In terms of functions, we compose $\alpha$ with the function $f\times\id_S\taking M\times S\to M'\times S$, to get a function we'll denote $$\Delta_f(\alpha)\taking M\times S\to S.$$ After Proposition \ref{prop:restriction of scalars} we will know that $\Delta_f(\alpha)$ is indeed a monoid action, and we say that it is given by {\em restriction of scalars along $f$}.\index{restriction of scalars}

モノイド準同型$f\taking M\to M'$ (Definition \ref{def:monoid hom}を見よ)は, $M$の要素が$M'$において妥当な解釈を持つことを保証している. それらは出自の$M$で作用したのと同じように$M'$の上でも作用する. もしそのような準同型$f$が得られ, 集合$S$への$M'$の作用$\alpha\taking M'\times S\to S$ of $M'$が得られたならば, $M$が$S$へ作用することを許す手段もまた得ることになる. 函数の観点でいえば, 我々は$\alpha$を$f\times\id_S\taking M\times S\to M'\times S$と組み合わせて, $$\Delta_f(\alpha)\taking M\times S\to S$$と書かれる函数を得ることになる. Proposition \ref{prop:restriction of scalars}後には, $\Delta_f(\alpha)$が実際にモノイド作用であることが分かるだろう. これは\emph{restriction of scalars along $f$}\index{restriction of scalars}によって得られたと言う.

\begin{proposition}\label{prop:restriction of scalars}

%Let $\mcM:=(M,e,\star)$ and $\mcM':=(M',e',\star')$ be monoids, $f\taking\mcM\to\mcM'$ a monoid homomorphism, $S$ a set, and suppose that $\alpha\taking M'\times S\to S$ is an action of $\mcM'$ on $S$. Then $\Delta_f(\alpha)\taking M\times S\to S$, defined as above, is a monoid action as well.

$\mcM:=(M,e,\star)$と$\mcM':=(M',e',\star')$をモノイド, $f\taking\mcM\to\mcM'$をモノイド準同型, $S$を集合, $\alpha\taking M'\times S\to S$を$S$への$\mcM'$の作用とする. このとき, 上記で定義される$\Delta_f(\alpha)\taking M\times S\to S$もまた, モノイド作用である.

\end{proposition}

\begin{proof}

%Refer to Remark \ref{rmk:monoid action}; we assume $\alpha$ is a monoid action and want to show that $\Delta_f(\alpha)$ is too. We have $\Delta_f(\alpha)(e,s)=\alpha(f(e),s)=\alpha(e',s)=s$. We also have 
Remark \ref{rmk:monoid action}を参照せよ. $\alpha$がモノイド作用であることを仮定して, $\Delta_f(\alpha)$もモノイド作用であることを示したい. $\Delta_f(\alpha)(e,s)=\alpha(f(e),s)=\alpha(e',s)=s$が得られ, そして以下が得られる. 
\begin{align*}
\Delta_f(\alpha)(m,\Delta_f(\alpha)(n,s))=\alpha(f(m),\alpha(f(n),s))&=\alpha(f(m)\star' f(n),s)\\
&=\alpha(f(m\star n),s)\\
&=\Delta_f(\alpha)(m\star n,s)
\end{align*}

\end{proof}

\begin{example}

%Let $\NN$ and $\ZZ$ denote the additive monoids of natural numbers and integers, respectively, and let $i\taking\NN\to\ZZ$ be the inclusion, which we saw in Example \ref{ex:nat to int} is a monoid homomorphism. There is an action $\alpha\taking\ZZ\times\RR\to\RR$ of the monoid $\ZZ$ on the set $\RR$ of real numbers, given by $\alpha(n,x)=n+x$. Clearly, this action works just as well if we restrict our scalars to $\NN\ss\ZZ$, allowing ourselves only to add natural numbers to reals. The action $\Delta_i\alpha\taking\NN\times\RR\to\RR$ is given on $(n,x)\in\NN\times\RR$ by $\Delta_i\alpha(n,x)=\alpha(i(n),x)=\alpha(n,x)=n+x$, just as expected.

$\NN$と$\ZZ$がそれぞれ自然数と整数の加法モノイドを表しているとし, $i\taking\NN\to\ZZ$を包含写像とする. Example \ref{ex:nat to int}でみたこれはモノイド準同型である. $\alpha(n,x)=n+x$によって与えられる, 実数の集合$\RR$へのモノイド$\ZZ$の作用$\alpha\taking\ZZ\times\RR\to\RR$が存在する. 明かに, 実数に自然数を足すことのみを許したとし, 係数を$\NN\ss\ZZ$に制限しても, この作用はうまく働く. 予想されるように, 作用$\Delta_i\alpha\taking\NN\times\RR\to\RR$は, $(n,x)\in\NN\times\RR$に対して$\Delta_i\alpha(n,x)=\alpha(i(n),x)=\alpha(n,x)=n+x$によって与えられる.

\end{example}

\begin{example}

%Suppose that $V$ is a complex vector space. In particular, this means that the monoid $\CC$ of complex numbers (under multiplication) acts on the elements of $V$. If $i\taking\RR\to\CC$ is the inclusion of the real line inside $\CC$, then $i$ is a monoid homomorphism. Restriction of scalars in the above sense turns $V$ into a real vector space, so the name ``restriction of scalars" is apt.

$V$を複素ベクトル空間と仮定する. 特に, 複素数の(乗法の)モノイド$\CC$は$V$の要素に作用する, ということをこれは意味している. もし$i\taking\RR\to\CC$が$\CC$の中の実軸への包含写像であるならば, $i$はモノイド準同型である. 上の意味での係数制限は$V$を実ベクトル空間に変換し, よって``係数の制限(restriction of scalars)''という名前がぴったしあてはまる.

\end{example}

\begin{exercise}
%Let $\NN$ be the free monoid on one generator, let $\Sigma=\{a,b\}$, and let $S=\{\tn{State 0, State 1, State 2}\}$. Consider the map of monoids $f\taking\NN\to\List(\Sigma)$ given by sending $1\mapsto [a,b,b]$. The monoid action $\alpha\taking\List(\Sigma)\times S\to S$ given in Example \ref{ex:action table} can be transformed by restriction of scalars along $f$ to an action $\Delta_f(\alpha)$ of $\NN$ on $S$. Write down its action table.
$\NN$を一生成子上の自由モノイドとし, $\Sigma=\{a,b\}$, $S=\{\tn{State 0, State 1, State 2}\}$とする. $1\mapsto [a,b,b]$によって得られるモノイドの写像$f\taking\NN\to\List(\Sigma)$を考える. Example \ref{ex:action table}で与えられるモノイド作用$\alpha\taking\List(\Sigma)\times S\to S$はrestriction of scalars along $f$によって$S$への$\NN$の作用に変換できる. この作用表を書き下せ.
\end{exercise}



%%%%%% Section %%%%%%

%\section{Groups}\label{sec:groups}
\section{群}\label{sec:groups}

%Groups are monoids in which every element has an inverse. If we think of these structures in terms of how they act on sets, the difference between groups and monoids is that the action of every group element can be undone. One way of thinking about groups is in terms of symmetries. For example, the rotations and reflections of a square form a group. 

群は, 全ての要素がその逆元を持っているモノイドである. どのようにしてそれらが集合に作用するかという観点でその構造を考えたならば, 群とモノイドの差異は全ての群の要素による作用はなかったことにできるという点にある. 群について考える一つの道筋は対称性の観点である. 例えば, 正方形の回転と鏡映は群を形成する.

%Another way to think of the difference between monoids and groups is in terms of time. Monoids are likely useful in thinking about diffusion, in which time plays a role and things cannot be undone. Groups are more likely useful in thinking about mechanics, where actions are time-reversible. 

モノイドと群の間の差異を考えるまた別の道筋は, 時間の観点である. モノイドは拡散を考える時におそらく便利であろう. そこでは時間が役割を果たし物事はなかったことにできない. 群は力学を考える時におそらく便利であろう. そこでは作用は時間に対して可逆である.

%%%% Subsection %%%%

%\subsection{Definition and examples}
\subsection{定義と例}

\begin{definition}\label{def:group}\index{group}\index{monoid!inverse of an element in}

%Let $(M,e,\star)$ be a monoid. An element $m\in M$ is said to {\em have an inverse} if there exists an $m'\in M$ such that $mm'=e$ and $m'm=e$. A {\em group} is a monoid $(M,e,\star)$ in which every element $m\in M$ has an inverse.

$(M,e,\star)$をモノイドとする. 要素$m\in M$は$mm'=e$かつ$m'm=e$であるところの要素$m'\in M$が存在するときに\emph{逆元を持つ(have an inverse)}と言う. \emph{群(group)}は全ての要素$m\in M$が逆元を持つモノイド$(M,e,\star)$である.

\end{definition}

\begin{proposition}

%Suppose that $\mcM:=(M,e,\star)$ is a monoid and let $m\in M$ be an element. Then $m$ has at most one inverse.
%\footnote{If $\mcM$ is a group then every element $m$ has exactly one inverse.}

$\mcM:=(M,e,\star)$をモイノドと仮定し, $m\in M$をその要素とする. このとき$m$は逆元を最大でも一つしか持たない.
\footnote{$\mcM$が群であるならば全ての要素$m$はちょうど一つだけ逆元を持つ.}

\end{proposition}

\begin{proof}

%Suppose that both $m'$ and $m''$ are inverses of $m$; we want to show that $m'=m''$. This follows by the associative law for monoids:
%$$m'=m'(mm'')=(m'm)m''=m''.$$

$m'$と$m''$の両者が$m$の逆元だと仮定する. $m'=m''$であることを示したい. これはモノイドの結合則により従う.
$$m'=m'(mm'')=(m'm)m''=m''.$$

\end{proof}

\begin{example}

%The additive monoid $(\NN,0,+)$ is not a group because none of its elements are invertible, except for $0$. However, the monoid of integers $(\ZZ,0,+)$ is a group. The monoid of clock positions from Example \ref{ex:cyclic} is also a group. For example the inverse of $Q^5$ is $Q^7$ because $Q^5\star Q^7=e=Q^7\star Q^5$.

加法モノイド$(\NN,0,+)$は群ではない. なぜなら$0$を除いてそのどの要素も可逆ではないからだ. しかしながら, 整数のモノイド$(\ZZ,0,+)$は群である. Example \ref{ex:cyclic}での時計の針の位置のモノイドもまた群である. 例えば$Q^5\star Q^7=e=Q^7\star Q^5$であるから, $Q^5$の逆元は$Q^7$である.

\end{example}

\begin{example}

%Consider a square centered at the origin in $\RR^2$. It has rotational and mirror symmetries. There are eight of these, which we denote $$\{e,\rho,\rho^2,\rho^3,\phi,\phi\rho,\phi\rho^2,\phi\rho^3\},$$ where $\rho$ stands for $90^\circ$ counterclockwise rotation and $\phi$ stands for horizontal-flip (across the vertical axis). So relations include $\rho^4=e$, $\phi^2=e$, and $\rho^3\phi=\phi\rho$.

$\RR^2$の原点に中心がある正方形を考える. これは回転と鏡映の対称性を持つ. 対称性は8個あり, これを$$\{e,\rho,\rho^2,\rho^3,\phi,\phi\rho,\phi\rho^2,\phi\rho^3\},$$と表記する. ここで$\rho$は$90^\circ$の反時計回りの回転を表しており, $\phi$は(縦軸を中心とした)左右反転を表している. よって関係には$\rho^4=e$や$\phi^2=e$, $\rho^3\phi=\phi\rho$が含まれている.

\end{example}

\begin{example}\label{ex:important groups}

%The set of $3\times 3$ matrices can be given the structure of a monoid, where the identity element is the $3\times 3$ identity matrix, the multiplication is matrix multiplication. The subset of invertible matrices forms a group, called {\em the general linear group of dimension 3}\index{a group!$GL_3$} and denoted $GL_3$. Inside of $GL_3$ is the so-called {\em orthogonal group}, denoted $O_3$, of matrices $M$ such that $M^\m1=M^\top$. These matrices correspond to symmetries of the sphere centered at the origin.

$3\times 3$の行列の集合にはモノイドの構造を与えることができる. ここで単位元は$3\times 3$の単位行列であり, 積は行列の積である. 可逆行列の部分集合は群をなし, \emph{3次元の一般線形群(the general linear group of dimension 3)}\index{a group!$GL_3$}と呼ばれ$GL_3$と表記される. $GL_3$の中には$O_3$で表記されるいわゆる\emph{直交群(orthogonal group)}, $M^\m1=M^\top$であるところの行列$M$が存在する. これらの行列は, 原点に中心がある球の対称性に対応している.

%Another interesting group is the Euclidean group\index{a group!$E_3$} $E(3)$ which consists of all {\em isometries} of $\RR^3$, i.e. all functions $\RR^3\to\RR^3$ that preserve distances.  

また別の興味深い群として, $\RR^3$上の全ての\emph{等距変換(isometry)}, i.e. 距離を保つ函数$\RR^3\to\RR^3$の全てからなるユークリッド群\index{a group!$E_3$} $E(3)$がある.

\end{example}

\begin{application}\label{app:groups for symmetry}\index{symmetry}

%In \href{http://en.wikipedia.org/wiki/Crystallography}{\text crystallography} one is often concerned with the symmetries that arise in the arrangement $A$ of atoms in a molecule. To think about symmetries in terms of groups, we first define an {\em atom-arrangement} to be a finite subset $i\taking A\ss\RR^3$. A symmetry in this case is an isometry of $\RR^3$ (see Example \ref{ex:important groups}), say $f\taking\RR^3\to\RR^3$ such that there exists a dotted arrow making the diagram below commute:
%$$
%\xymatrix{A\ar@{-->}[r]\ar[d]_i&A\ar[d]^i\\\RR^3\ar[r]_f&\RR^3}
%$$
%That is, it's an isometry of $\RR^3$ such that each atom of $A$ is sent to a position currently occupied by an atom of $A$. It is not hard to show that the set of such isometries forms a group, called the \href{http://en.wikipedia.org/wiki/Space_group}{\em space group}\index{space group} of the crystal.

\href{http://en.wikipedia.org/wiki/Crystallography}{結晶学(crystallography)}では, 分子の内部の\footnote{訳注: 正確には単位胞(unit cell). 一分子の対称性を議論する際に用いられる, 並進対称性を含まない群は\href{http://en.wikipedia.org/wiki/Point_group}{\emph{点群(point group)}}\index{point group}と呼ばれる.}原子の配置$A$によって生起する対称性をしばしば考察することになる. 群の観点で対称性について考えるために, 我々は最初に$i\taking A\ss\RR^3$の有限部分集合になるように\emph{原子配置(atom-arrangement)}を定義する. この場合の対称性は$\RR^3$の等距変換(Example \ref{ex:important groups}を見よ)であり, 言い換えれば下記の図式を可換にする破線矢印が存在するところの$f\taking\RR^3\to\RR^3$である. 
$$
\xymatrix{A\ar@{-->}[r]\ar[d]_i&A\ar[d]^i\\\RR^3\ar[r]_f&\RR^3}
$$
これはすなわち, $A$のそれぞれの原子を, $A$の(同じ種類の)原子が占めている場所へと送る等距変換である. これらの等距変換による集合が, 結晶の\href{http://en.wikipedia.org/wiki/Space_group}{\emph{空間群(space group)}}\index{space group}と呼ばれる群を形成することを示すのは難しくない.
\end{application}

\begin{exercise}\label{exc:permutation}\index{set!permutation of}
%Let $S$ be a finite set. A {\em permutation of $S$}\index{permutation} is an isomorphism $f\taking S\To{\iso}S$. 
$S$を有限集合とする. \emph{$A$の置換(permutation of $S$)}\index{permutation}は同型写像$f\taking S\To{\iso}S$である.
\begin{center}
\parbox{2.3in}{
\includegraphics[height=2in]{SetPermutation}}
\end{center}
%\sexc Come up with an identity, and a  multiplication formula, such that the set of permutations of $S$ forms a monoid. 
%\next Is it a group?
%\endsexc
\sexc $S$の置換の集合がモノイドをなすような単位元と積の式を見出せ.
\next それは群か?
\endsexc
\end{exercise}

\begin{exercise}
%In Exercise \ref{exc:classify cyclic} you classified the cyclic monoids. Which of them are groups? 
Exercise \ref{exc:classify cyclic}では巡回モノイドを分類した. それらのうち群はどれか?
\end{exercise}

%\begin{definition}[Group action]\label{def:group action}\index{group!action}\index{action!of a group}
\begin{definition}[群作用]\label{def:group action}\index{group!action}\index{action!of a group}

%Let $(G,e,\star)$ be a group and $S$ a set. An {\em action} of $G$ on $S$ is a function $\acts\taking G\times S\to S$ such that for all $s\in S$ and $g,g'\in G$, we have
%\begin{itemize}
%\item $e\acts s=s$ and
%\item $g\acts(g'\acts s)=(g\star g')\acts s.$
%\end{itemize}
%In other words, considering $G$ as a monoid, it is an action in the sense of Definition \ref{def:monoid action}.

$(G,e,\star)$を群とし, $S$を集合とする. $S$への$G$の\emph{作用(action)}は函数$\acts\taking G\times S\to S$で, 全ての$s\in S$と$g,g'\in G$に対して以下をみたすものである.
\begin{itemize}
\item $e\acts s=s$
\item $g\acts(g'\acts s)=(g\star g')\acts s$
\end{itemize}
別の言い方をすれば, $G$をモノイドとみなせば, Definition \ref{def:monoid action}の意味でこれは作用である.

\end{definition}

\begin{example}\label{ex:U(1)}\index{a group!$U(1)$}

%When a group acts on a set, it has the character of \href{http://en.wikipedia.org/wiki/Symmetry}{\text symmetry}. For example, consider the group whose elements are angles $\theta$. This group may be denoted $U(1)$ and is often formalized as the unit circle in $\CC$ of complex numbers $z=a+bi$ such that $|z|=a^2+b^2=1$. The set of such points is given the structure of a group $(U(1),e,\star)$ by defining the identity element to be $e:=1+0i$ and the group law to be complex multiplication. But for those unfamiliar with complex numbers, this is simply angle addition where we understand that $360^\circ=0^\circ$. If $\theta_1=190^\circ$ and $\theta_2=278^\circ$, then $\theta_1\star\theta_2=468^\circ=108^\circ.$ In the language of complex numbers, $z=e^{i\theta}$.

群が集合に作用するとき, それは\href{http://en.wikipedia.org/wiki/Symmetry}{対称性(symmetry)}の特徴を持つ. 例えば, 角度$\theta$が要素である群を考える. この群は$U(1)$と書かることもあり, $\CC$上の単位円すなわち複素数$z=a+bi$で$|z|=a^2+b^2=1$であるものとしてしばしば定式化される. このような点の集合は, 単位元を$e:=1+0i$, 積の式を複素数の積と定義することによって, 群$(U(1),e,\star)$の構造を与えることができる. 複素数に不慣れな人のために加えると, これは単に$360^\circ=0^\circ$が分かっている上での角度の足し算である. もし$\theta_1=190^\circ$で$\theta_2=278^\circ$ならば, $\theta_1\star\theta_2=468^\circ=108^\circ$である. 複素数の言葉では, $z=e^{i\theta}$である.

%The group $U(1)$ acts on any set that we can picture as having rotational symmetry about a fixed axis, such as the earth around the north-south axis. We will define $S=\{(x,y,z)\in\RR^3\|x^2+y^2+z^2=1\}$, the unit sphere, and understand the rotational action of $U(1)$ on $S$.\index{orbit!rotating earth}

群$U(1)$は, 例えば地軸回りの地球のような, 固定された軸について回転対称性を持つように描けるような任意の集合に対して作用する. 単位球面$S=\{(x,y,z)\in\RR^3\|x^2+y^2+z^2=1\}$を定義し, $S$への$U(1)$の回転作用を理解してみよう.\index{orbit!rotating earth}

%We first show that $U(1)$ acts on $\RR^3$ by $\theta\acts(x,y,z)=(x\cos\theta+y\sin\theta, -x\sin\theta+y\cos\theta,z)$, or with matrix notation as 
%$$\theta\acts(x,y,z)
%:=(x,y,z)\left(\begin{array}{ccc}
%\cos(\theta)&-\sin(\theta)&0\\
%\sin(\theta)&\cos(\theta)&0\\
%0&0&1\end{array}\right)
%$$
%\href{http://en.wikipedia.org/wiki/List_of_trigonometric_identities#Matrix_form}{\text Trigonometric identities} ensure that this is indeed an action.

最初に$U(1)$が$\RR^3$へ作用することを示そう. これは$\theta\acts(x,y,z)=(x\cos\theta+y\sin\theta, -x\sin\theta+y\cos\theta,z)$によって, あるいは行列表示
$$\theta\acts(x,y,z)
:=(x,y,z)\left(\begin{array}{ccc}
\cos(\theta)&-\sin(\theta)&0\\
\sin(\theta)&\cos(\theta)&0\\
0&0&1\end{array}\right)
$$
による. \href{http://en.wikipedia.org/wiki/List_of_trigonometric_identities#Matrix_form}{三角函数の公式}からこれが実際に作用であることは保証されている.

%In terms of action tables, we would need infinitely many columns to express this action. Here is a sample
作用表の観点では, 我々はこの作用を表現するのに無限の行を必要とする. 下記は一つの例である.
$$
\begin{tabular}{| l || l | l | l |}
\bhline
%\multicolumn{4}{|c|}{Action of $U(1)$ on $\RR^3$}\\\bhline
\multicolumn{4}{|c|}{\text{$\RR^3$への$U(1)$の作用}}\\\bhline
{$\RR^3$}&{$\theta=45^\circ$}&{$\theta=90^\circ$}&{$\theta=100^\circ$}\\\bbhline
(0,0,0)&(0,0,0)&(0,0,0)&(0,0,0)\\\hline
(1,0,0)&(.71,.71,0)&(0,1,0)&(-.17,.98,0)\\\hline
(0,1,-4.2)&(-.71,.71,-4.2)&(-1,0,-4.2)&(-.98,-.17,-4.2)\\\hline
(3,4,2)&(4.95,.71,2)&(-4,3,2)&(3.42,-3.65,2)\\\hline
$\vdots$&$\vdots$&$\vdots$&$\vdots$\\\bhline
\end{tabular}
$$

%Finally, we are looking to see that the action preserves length so that if $(x,y,z)\in S$ then $\theta\acts(x,y,z)\in S$; this way we will have confirmed that $U(1)$ indeed acts on $S$. The calculation begins by assuming $x^2+y^2+z^2=1$ and checks 
%$$
%(x\cos\theta+y\sin\theta)^2+(-x\sin\theta+y\cos\theta)^2+z^2=x^2+y^2+z^2=1.
%$$

最後に, $(x,y,z)\in S$ならば$\theta\acts(x,y,z)\in S$であるために, この作用が長さを保存しすることを見てみよう. この方法で, 我々は$U(1)$が実際に$S$へ作用することを確かめることになる. 計算としては, 最初に$x^2+y^2+z^2=1$を仮定し, そして以下を確かめればよい.
$$
(x\cos\theta+y\sin\theta)^2+(-x\sin\theta+y\cos\theta)^2+z^2=x^2+y^2+z^2=1.
$$

\end{example}

\begin{exercise}\label{exc:permutation group}
%Let $X$ be a set and consider the group of permutations of $X$ (see Exercise \ref{exc:permutation}), which we will denote $\Sigma_X$\index{a group!$\Sigma_X$}. Find a canonical action of $\Sigma_X$ on $X$.
$X$を集合とする. $X$の置換群(Exercise \ref{exc:permutation}を見よ)を考え, これを$\Sigma_X$\index{a group!$\Sigma_X$}で示すことにする. $X$への$\Sigma_X$の標準作用(canonical action)を見出せ.
\end{exercise}

\begin{definition}

%Let $G$ be a group acting on a set $X$. For any point $x\in X$, the {\em orbit of $x$},\index{orbit}\index{action!orbit of} denoted $Gx$, is the set 
%$$Gx:=\{x'\in X\|\exists g\in G \tn{ such that }gx=x'\}.$$

$G$を$X$への群作用とする. 任意の点$x\in X$に対して, $Gx$で示される\emph{$x$の軌道(orbit of $x$)}\index{orbit}\index{action!orbit of}は以下の集合である.
$$Gx:=\{x'\in X\|\exists g\in G \tn{ such that }gx=x'\}.$$

\end{definition}

\begin{application}

%Let $S$ be the surface of the earth, understood as a sphere, and let $G=U(1)$ be the group of angles acting on $S$ as in Example \ref{ex:U(1)}. The orbit of any point $p=(x,y,z)\in S$ is the set of points on the same latitude line as $p$.

地球を球体と理解するとし, $S$をその表面とする. $G=U(1)$をExample \ref{ex:U(1)}での$S$へ作用する角度の群とする. 任意の点$p=(x,y,z)\in S$の軌道は$p$と同じ緯度の点の集合である.

%One may also consider a small band around the earth, i.e. the set $A=\{(x,y,z)\|1.0\leq x^2+y^2+z^2\leq 1.05\}$. The action of $U(1)\acts S$ extends to an action $U(1)\acts A$. The orbits are latitude-lines-at-altitude. A simplifying assumption in \href{http://en.wikipedia.org/wiki/Climatology}{\text climatology} may be given by assuming that $U(1)$ acts on all currents in the atmosphere in an appropriate sense. That way, instead of considering movement within the whole space $A$, we only allow movement that behaves the same way throughout each orbit of the group action.

地球の周囲に狭い帯状領域を考えてみたいかもしれない, i.e. 集合$A=\{(x,y,z)\|1.0\leq x^2+y^2+z^2\leq 1.05\}$. 作用$U(1)\acts S$は作用$U(1)\acts A$に拡張される. 軌道は緯度上の緯線である. \href{http://en.wikipedia.org/wiki/Climatology}{気候学(climatology)}における仮定を単純化すれば, おそらく全ての大気の流れについて適切な意味において$U(1)$が作用するという仮定を与えるということであろう. この方法では, 全空間$A$の中での動きを考慮するかわりに, 我々はそれぞれの群作用の軌道を通じて同じようにふるまう動きだけを許している.

\end{application}

\begin{exercise}~
%\sexc Consider the $U(1)$ action on $\RR^3$ given in Example \ref{ex:U(1)}. Describe the set of orbits of this action.
%\next What are the orbits of the action of the permutation group $\Sigma_{\{1,2,3\}}$ on the set $\{1,2,3\}$? (See Exercise \ref{exc:permutation group}.)
%\endsexc
\sexc Example \ref{ex:U(1)}で与えられた$\RR^3$への作用$U(1)$を考える. この作用の軌道の集合を記述せよ.
\next 集合$\{1,2,3\}$への置換群$\Sigma_{\{1,2,3\}}$の作用の軌道は何だろうか? (Exercise \ref{exc:permutation group}を見よ.)
\endsexc
\end{exercise}

\begin{exercise}
%Let $G$ be a group and $X$ a set on which $G$ acts by $\acts\taking G\times X\to X$. Is ``being in the same orbit" an equivalence relation on $X$? 
$G$を群, $X$を$\acts\taking G\times X\to X$によって$G$が作用するところの集合とする. ``同じ軌道である''ことは$X$での同値関係であるか?
\end{exercise}

\begin{definition}\label{def:group homomorphism}\index{group!homomorphism of}

%Let $G$ and $G'$ be groups. A {\em group homomorphism} $f\taking G\to G'$ is defined to be a monoid homomorphism $G\to G'$, where $G$ and $G'$ are being regarded as monoids in accordance with Definition \ref{def:group}.

$G$と$G'$を群とする. \emph{群準同型(group homomorphism)}はモノイド準同型$G\to G'$として定義される. ここで$G$ と$G'$はDefinition \ref{def:group}に一致するモノイドとみなされている.

\end{definition}

%%%%%% Section %%%%%%

%\section{Graphs}\label{sec:graphs}
\section{グラフ}\label{sec:graphs}

%In this course, unless otherwise specified, whenever we speak of graphs we are not talking about curves in the plane, such as parabolas, or pictures of functions generally. We are speaking of systems of vertices and arrows.

この教程では特に指定しない限り, 我々がグラフ(graph)と言うときには常に, 放物線のような平面上の曲線や, あるいはもっと一般の函数の図などを, 語ってはいない. 我々は頂点と矢印(辺)からなる系について語っている.

%We will take our graphs to be {\em directed}, meaning that every arrow points {\em from} a vertex {\em to} a vertex; rather than merely connecting vertices, arrows will have direction. If $a$ and $b$ are vertices, there can be many arrows from $a$ to $b$, or none at all. There can be arrows from $a$ to itself. Here is the formal definition in terms of sets and functions.

我々は\emph{有向(directed)}グラフを扱うことにする. 有向という言葉が意味するところは, 全ての矢印はある頂点\emph{から}ある頂点\emph{に向って}いるということである. 矢印は単に頂点を結ぶのではなく, 向きを持つ. もし$a$と$b$が頂点ならば, $a$から$b$への多くの矢印が存在しうるし, あるいは全く存在しない. $a$からそれ自身への矢印も存在しうる. 以下は集合と函数の観点においての形式的な定義である.

%%%% Subsection %%%%

%\subsection{Definition and examples}
\subsection{定義と例}

\begin{definition}\label{def:graph}\index{graph}

%A {\em graph} $G$ consists of a sequence $G:=(V,A,src,tgt)$ where 
%\begin{itemize}
%\item $V$ is a set, called {\em the set of vertices of $G$} (singular:{\em vertex}),\index{vertex}
%\item $A$ is a set, called {\em the set of arrows of $G$},\index{arrow}
%\item $src\taking A\to V$ is a function, called {\em the source function for $G$}, and
%\item $tgt\taking A\to V$ is a function, called {\em the target function for $G$}.
%\end{itemize}
%Given an arrow $a\in A$ we refer to $src(a)$ as the {\em source vertex} of $a$ and to $tgt(a)$ as the {\em target vertex} of $a$.

\emph{グラフ(graph)}$G$は列$G:=(V,A,\src,\tgt)$からなる. ここで
\begin{itemize}
\item $V$は集合であり, \emph{$G$の頂点集合(the set of vertices\footnote{単数形: \emph{vertex}.} of $G$)}\index{vertex}と呼ばれる.
\item $A$は集合であり, \emph{$G$の矢の集合(the set of arrows of $G$)}\index{arrow}と呼ばれる.
\item $\src\taking A\to V$は函数であり, \emph{$G$に対する始点函数(the source function for $G$)}と呼ばれる.
\item $\tgt\taking A\to V$は函数であり, \emph{$G$に対する終点函数(the target function for $G$)}と呼ばれる.
\end{itemize}
矢$a\in A$が与えられたとき, 我々は$\src(a)$を$a$の\emph{始点(source vertex)}, $\tgt(a)$を$a$の\emph{終点(target vertex)}として参照する.

\end{definition}

%To draw a graph, first draw a dot for every element of $V$. Then for every element $a\in A$, draw an arrow connecting dot $\src(a)$ to dot $\tgt(a)$.

グラフを描く際には, 最小に$V$の要素それぞれに対して点を描き, 次にそれぞれの要素$a\in A$に対して, 点$\src(a)$と$\tgt(a)$を結ぶ矢を描く.

%\begin{example}[Graph]\label{ex:graph}
\begin{example}[グラフ]\label{ex:graph}

%Here is a picture of a graph $G=(V,A,src,tgt)$:
これはグラフ$G=(V,A,\src,\tgt)$の図である.
\begin{align}\label{dia:graph}
G:=\parbox{2in}{\fbox{\xymatrix{\bullet^v\ar[r]^f&\bullet^w\ar@/_1pc/[r]_h\ar@/^1pc/[r]^g&\bullet^x\\\bullet^y\ar@(l,u)[]^i\ar@/^1pc/[r]^j&\bullet^z\ar@/^1pc/[l]^k}}}
\end{align} 
%We have $V=\{v,w,x,y,z\}$ and $A=\{f,g,h,i,j,k\}$. The source and target functions $src,tgt\taking A\to V$ can be captured in the table to the left below:
$V=\{v,w,x,y,z\}$であり$A=\{f,g,h,i,j,k\}$である. 始点函数および終点函数$\src,\tgt\taking A\to V$は左下の表によって把握することができる.
\begin{align*}
\begin{array}{| l || l | l |}\bhline
%{\bf A}&{\bf src}&{\bf tgt}\\\bbhline
{A}&{\src}&{\tgt}\\\bbhline
f&v&w\\\hline
g&w&x\\\hline
h&w&x\\\hline
i&y&y\\\hline
j&y&z\\\hline
k&z&y\\\bhline
\end{array}
\hspace{1in}
\begin{array}{| l |}\bhline
%{\bf V}\\\bbhline
{V}\\\bbhline
v\\\hline
w\\\hline
x\\\hline
y\\\hline
z\\\bhline
\end{array}
\end{align*}
%In fact, all of the data of the graph $G$ is captured in the two tables above---together they tell us the sets $A$ and $V$ and the functions $src$ and $tgt$.
実際, グラフ$G$の全てのデータは上記の二つの表によって把握可能である---二つを併せることによって集合$A$, 集合$V$, そして函数$\src$および$\tgt$を知ることができる.
\end{example}

\begin{example}

%Every olog has an underlying graph. The additional information in an olog has to do with which pairs of paths are declared equivalent, as well as text that has certain English-readability rules.\index{olog!underlying graph}

全てのologには基底となるグラフがある. ologの追加情報は, 英語で読むための法則を持つことだけでなく, どの経路の対が透過であると宣言をとり扱う.\index{olog!underlying graph}

\end{example}

\begin{exercise}
%\sexc Draw the graph corresponding to the following tables:
\sexc 以下の表に対応するグラフを描け.
\begin{align*}
\begin{array}{| l || l | l |}\bhline
%{\bf A}&{\bf src}&{\bf tgt}\\\bbhline
{A}&{\src}&{\tgt}\\\bbhline
f&v&w\\\hline
g&v&w\\\hline
h&v&w\\\hline
i&x&w\\\hline
j&z&w\\\hline
k&z&z\\\bhline
\end{array}
\hspace{1in}
\begin{array}{| l |}\bhline
%{\bf V}\\\bbhline
{V}\\\bbhline
u\\\hline
v\\\hline
w\\\hline
x\\\hline
y\\\hline
z\\\bhline
\end{array}
\end{align*}
%\next Write down two tables, as above, corresponding to the following graph:
\next 上記のように, 下記のグラフに対応する二つの表を書け.
$$\fbox{\xymatrix{
\LMO{a}\ar[r]^{1}&\LMO{b}\ar[r]^2\ar@/^1pc/[r]^3&\LMO{c}\ar@/^1pc/[l]^4\ar[r]^5&\LMO{d}\\
\LMO{e}&\LMO{f}\ar[l]^6\ar[r]_7&\LMO{g}\ar[ur]_8}}
$$
\endsexc
\end{exercise}

\begin{exercise}
%Let $A=\{1,2,3,4,5\}$ and $B=\{a,b,c\}$. Draw them and choose an arbitrary function $f\taking A\to B$ and draw it. Let $A\sqcup B$ be the coproduct of $A$ and $B$ (Definition \ref{def:coproduct}) and let $A\To{i_1}A\sqcup B\From{i_2}B$ be the two inclusions. Consider the two functions $src,tgt\taking A\to A\sqcup B$, where $src=i_1$ and $tgt$ is the composition $A\To{f}B\To{i_2}A\sqcup B$. Draw the associated graph $(A\sqcup B,A,src,tgt)$.
$A=\{1,2,3,4,5\}$で $B=\{a,b,c\}$とする. $A$と$B$を描いて任意の函数$f\taking A\to B$を選べ. $A\sqcup B$を$A$と$B$の余積(Definition \ref{def:coproduct})とし, $A\To{i_1}A\sqcup B\From{i_2}B$を二つの包含函数とする. $\src,\tgt\taking A\to A\sqcup B$を考える. ここで$\src=i_1$とし, $\tgt$を合成$A\To{f}B\To{i_2}A\sqcup B$とした. 関連するグラフ$(A\sqcup B,A,\src,\tgt)$を描け.
\end{exercise}

\begin{exercise}~
%\sexc Let $V$ be a set. Suppose we just draw the elements of $V$ as vertices and have no arrows between them. Is this a graph?
%\next Given $V$, is there any other ``canonical" or somehow automatic non-random procedure for generating a graph with those vertices? 
%\endsexc
\sexc $V$を集合とする. $V$の要素を頂点として描き, それらの間に一本も矢がないと仮定する. これはグラフか?
\next $V$が与えられたとき, これらの頂点からグラフを生成する, 他の``canonical''あるいはなんらかの自動的で乱択的ではない手続きがあるだろうか?
\endsexc
\end{exercise}

\begin{example}

%Recall from Construction \ref{const:bipartite} the notion of bipartite graph, which we defined to be a span (i.e. pair of functions, see Definition \ref{def:span}) $A\From{f}R\To{g}B$. Now that we have a formal definition of graph, we might hope that bipartite graphs fit in, and they do. Let $V=A\sqcup B$ and let $i\taking A\to V$ and $j\taking B\to V$ be the inclusions. Let $src=i\circ f\taking R\to V$ and let $tgt=j\circ g\taking R\to V$ be the composites.
%$$
%\xymatrix{&A\ar[dr]^i\\R\ar@/^1pc/[rr]_{src}\ar@/_1pc/[rr]^{tgt}\ar[ur]^f\ar[dr]_g&&V\\&B\ar[ur]_j}
%$$ 
%Then $(V,R,src,tgt)$ is a graph that would be drawn exactly as we specified the drawing of spans in Construction \ref{const:bipartite}.

Construction \ref{const:bipartite}での二部グラフの概念を思いだそう. 我々はこれをスパン(i.e. 函数の対. Definition \ref{def:span}を見よ)$A\From{f}R\To{g}B$と定義した. 今や我々はグラフの形式的な定義を手に入れた. 次は二部グラフをその概念にまたグラフをスパンの概念にあてはめたい. $V=A\sqcup B$とし, $i\taking A\to V$ and $j\taking B\to V$を包含函数とする. $\src=i\circ f\taking R\to V$とし, $\tgt=j\circ g\taking R\to V$を合成とする.
$$
\xymatrix{&A\ar[dr]^i\\R\ar@/^1pc/[rr]_{\src}\ar@/_1pc/[rr]^{\tgt}\ar[ur]^f\ar[dr]_g&&V\\&B\ar[ur]_j}
$$ 
このとき, $(V,R,\src,\tgt)$は, Construction \ref{const:bipartite}のスパンの描き方として詳細を述べたのとかっちり同じ描き方をすることになるグラフである.

\end{example}

\begin{example}\label{ex:[n] as graph}

%Let $n\in\NN$ be a natural number. The {\em chain graph of length $n$},\index{graph!chain} denoted $[n]$ is the graph depicted here:
%$$
%\xymatrix{
%\LMO{0}\ar[r]&\LMO{1}\ar[r]&\cdots\ar[r]&\LMO{n}
%}
%$$
%In general $[n]$ has $n$ arrows and $n+1$ vertices. In particular, when $n=0$ we have that $[0]$ is the graph consisting of a single vertex and no arrows. 

$n\in\NN$を自然数とする. $[n]$で示される\emph{長さ$n$の連鎖グラフ(chain graph of length $n$)}\index{graph!chain}は以下に描画されるグラフである.
$$
\xymatrix{
\LMO{0}\ar[r]&\LMO{1}\ar[r]&\cdots\ar[r]&\LMO{n}
}
$$
一般に$[n]$は$n$本の矢と$n+1$個の頂点を持つ. 特に, $n=0$のとき$[0]$は単一の頂点を持ち矢を持たないグラフである.

\end{example}

\begin{example}\label{ex:ZxG}

%Let $G=(V,A,src,tgt)$ be a graph; we want to spread it out over discrete time, so that each arrow does not occur within a given time-slice but instead over a quantum unit of time. 

$G=(V,A,\src,\tgt)$をグラフとする. 我々は離散的な時間の上をこれによって覆い尽したい. これによってそれぞれの矢は与えられた時間の断片の中に局在するかわりに, 量子的な時間単位の上に遍在することになる.

%Let $N=(\NN,\NN,n\mapsto n,n\mapsto n+1)$ be the graph depicted 
%$$\xymatrix{\LMO{0}\ar[r]^0&\LMO{1}\ar[r]^1&\LMO{2}\ar[r]^2&\cdots}$$
%When we get to limits in a category, we will understand that products can be taken in the category of graphs (see  Example \ref{ex:product of graphs}), and $N\times G$ will make sense. For now, we construct it by hand.

$N=(\NN,\NN,n\mapsto n,n\mapsto n+1)$を以下で描画されるグラフとする.
$$\xymatrix{\LMO{0}\ar[r]^0&\LMO{1}\ar[r]^1&\LMO{2}\ar[r]^2&\cdots}$$
圏論での極限の概念を知れば, 積はグラフの圏(Example \ref{ex:product of graphs}を見よ)として理解でき, $N\times G$が意味を持つことが分かる. 今のところは, 我々は手作業でこれを構成することにする.

%Let $T(G)=(V\times \NN,A\times\NN,src',tgt')$ be a new graph, where for $a\in A$ and $n\in\NN$ we have $src'(a,n):=(src(a),n)$ and $tgt'(a,n)=(tgt(a),n+1)$. This may be a bit much to swallow, so try to simply understand what is being done in the following example. 

$T(G)=(V\times \NN,A\times\NN,\src',\tgt')$を新しいグラフとする. ここで$a\in A$と$n\in\NN$に対して$\src'(a,n):=(src(a),n)$および$\tgt'(a,n)=(tgt(a),n+1)$とする. これは少々飲み込みにくいかもしれない. よって, 以下の例で何がなされたかを簡単に理解することを試みよう.

%Let $G$ be the graph drawn below 
%$$\xymatrix{\LMO{a}\ar@(ul,ur)[]^f\ar[d]_g\\\LMO{b}}$$
%Then $T(G)$ will be the graph 
%$$\xymatrix@=30pt{
%\LMO{a0}\ar[r]^{f0}\ar[rd]_{g0}&\LMO{a1}\ar[r]^{f1}\ar[rd]_{g1}&\LMO{a2}\ar[r]^{f2}\ar[rd]_{g2}&\cdots\\
%\LMO{b0}&\LMO{b1}&\LMO{b2}&\cdots
%}
%$$
%As you can see, $f$-arrows still take $a$'s to $a$'s and $g$-arrows still take $a$'s to $b$'s, but they always march forward in time.

$G$を以下に描かれたグラフとする.
$$\xymatrix{\LMO{a}\ar@(ul,ur)[]^f\ar[d]_g\\\LMO{b}}$$
このとき$T(G)$はグラフ
$$\xymatrix@=30pt{
\LMO{a0}\ar[r]^{f0}\ar[rd]_{g0}&\LMO{a1}\ar[r]^{f1}\ar[rd]_{g1}&\LMO{a2}\ar[r]^{f2}\ar[rd]_{g2}&\cdots\\
\LMO{b0}&\LMO{b1}&\LMO{b2}&\cdots
}
$$
になる. 見てとれるように, $f$の矢は依然として$a$から$a$であり$g$の矢は依然として$a$から$b$である. しかしどちらも時間に中で常に前方向へ進行している.

\end{example}

\begin{exercise}\label{exc:secret turing}
%Let $G$ be the graph depicted below:
%$$
%\xymatrix{\LMO{a}\ar@/^1pc/[rr]^w\ar@(lu,ld)[]_x&&\LMO{b}\ar@/^1pc/[ll]^y\ar@(ur,dr)[]^z}
%$$
%Draw (using ellipses ``$\cdots$" if necessary) the graph $T(G)$ defined in Example \ref{ex:ZxG}.
$G$を以下で描画されるグラフとする.
$$
\xymatrix{\LMO{a}\ar@/^1pc/[rr]^w\ar@(lu,ld)[]_x&&\LMO{b}\ar@/^1pc/[ll]^y\ar@(ur,dr)[]^z}
$$
Example \ref{ex:ZxG}で定義されるグラフ$T(G)$を(必要があれば省略記号``$\cdots$''を用いて)描け.
\end{exercise}

\begin{exercise}\label{exc:lattice}
%Consider the infinite graph $G=(V,A,src,tgt)$ depicted below,
以下で描画される無限グラフ$G=(V,A,\src,\tgt)$を考える.
$$
\xymatrix{
\vdots&\vdots&\vdots\\
(0,2)\ar[r]\ar[u]&(1,2)\ar[r]\ar[u]&(2,2)\ar[r]\ar[u]&\cdots\\
(0,1)\ar[r]\ar[u]&(1,1)\ar[r]\ar[u]&(2,1)\ar[r]\ar[u]&\cdots\\
(0,0)\ar[r]\ar[u]&(1,0)\ar[r]\ar[u]&(2,0)\ar[r]\ar[u]&\cdots}
$$
%\sexc Write down the sets $A$ and $V$. 
%\next What are the source and target function $A\to V$?  
%\endsexc
\sexc 集合$A$と$V$を書き下せ.
\next 始点函数および終点函数$A\to V$は何か?
\endsexc
\end{exercise}

\begin{exercise}\label{exc:(co)equalizer of graph}
%A graph is a pair of functions $A\tto V$. This sets up the notion of equalizer and coequalizer (see Definitions \ref{def:equalizer} and \ref{def:coequalizer}). 
%\sexc What feature of a graph is captured by the equalizer of its source and target functions? 
%\next What feature of a graph is captured by the coequalizer of its source and target functions?
%\endsexc
グラフは函数の対$A\tto V$である. ここから等価子と余等価子を作ることができる(Definitions \ref{def:equalizer}と\ref{def:coequalizer}を見よ).
\sexc 始点函数と終点函数の等価子によって, グラフのどんな性質がとらえられるだろうか?
\next 始点函数と終点函数の余等価子によって, グラフのどんな性質がとらえられるだろうか?
\endsexc
\end{exercise}

%%%% Subsection %%%%

%\subsection{Paths in a graph}\label{sec:paths in graph}\index{graph!paths}
\subsection{グラフにおける経路}\label{sec:paths in graph}\index{graph!paths}

%We all know what a path in a graph is, especially if we understand that a path must always follow the direction of arrows. The following definition makes this idea precise. In particular, one can have paths of any finite length $n\in\NN$, even length $0$ or $1$. Also, we want to be able to talk about the source vertex and target vertex of a path, as well as concatenation of paths.

我々はみなグラフの経路が何であるかを知っている. 特に経路が矢の方向に常に従わなければならないと理解しているならば. 以下の定義によってこのアイデアを正確になる. とりわけ, 任意の長さ$n\in\NN$の経路, 長さ$0$や$1$の経路すらも得ることができる. また, 我々は経路の結合だけでなく, 経路の始点および終点についても語ることを可能にしたい.

\begin{definition}\label{def:paths in graph}

%Let $G=(V,A,src,tgt)$ be a graph. A {\em path of length $n$}\index{path} in $G$, denoted $p\in\Path_G^{(n)}$\index{a symbol!$\Path$} is a head-to-tail sequence \begin{align}\label{dia:path}p=(v_0\To{a_1}v_1\To{a_2}v_2\To{a_3}\ldots\To{a_n}v_n)\end{align} of arrows in $G$, which we denote by $v_0 a_1 a_2 \ldots a_n$. In particular we have canonical isomorphisms $\Path_G^{(1)}\iso A$ and $\Path_G^{(0)}\iso V$; we refer to the path of length 0 on vertex $v$ as the {\em trivial path on $v$} and denote it simply by $v$. We denote by $\Path_G$ the set of paths in $G$, $$\Path_G:=\bigcup_{n\in\NN}\Path_G^{(n)}.$$ Every path $p\in\Path_G$ has a source vertex and a target vertex, and we may denote these by $\ol{src},\ol{tgt}\taking\Path_G\to V$. If $p$ is a path with $\ol{src}(p)=v$ and $\ol{tgt}(p)=w$, we may denote it by $p\taking v\to w$. Given two vertices $v,w\in V$, we write $\Path_G(v,w)$ to denote the set of all paths $p\taking v\to w$.

$G=(V,A,\src,\tgt)$をグラフとする. $p\in\Path_G^{(n)}$\index{a symbol!$\Path$}で示される, $G$の\emph{長さ$n$の経路(path of length $n$)}\index{path}は, 頭部から尾部への直鎖状の$G$の矢の列\begin{align}\label{dia:path}p=(v_0\To{a_1}v_1\To{a_2}v_2\To{a_3}\ldots\To{a_n}v_n)\end{align}である. 我々はこれを$v_0 a_1 a_2 \ldots a_n$と表す. 特に, 標準同型写像$\Path_G^{(1)}\iso A$および$\Path_G^{(0)}\iso V$が存在する. 頂点$v$上の長さ0の経路を\emph{$v$上の自明な経路(trivial path on $v$)}として参照し, 簡単に$v$で示す. 我々は$\Path_G$によって$G$上の経路の集合を示す. $$\Path_G:=\bigcup_{n\in\NN}\Path_G^{(n)}.$$ 全ての経路$p\in\Path_G$は始点と終点を持つ. 我々はこれらを$\ol{\src},\ol{\tgt}\taking\Path_G\to V$で示すことにしよう. $p$が経路で$\ol{\src}(p)=v$かつ$\ol{\tgt}(p)=w$であるならば, これを$p\taking v\to w$で示すことにしよう. 与えられた二つの頂点$v,w\in V$について, $\Path_G(v,w)$と書いて$p\taking v\to w$なる経路全ての集合を示すことにする.

%There is a concatenation operation on paths.\index{concatenation!of paths} Given a path $p\taking v\to w$ and $q\taking w\to x$, we define the concatenation, denoted $p q\taking v\to x$ in the obvious way. If $p=va_1,a_2\ldots a_m$ and $q= wb_1b_2\ldots b_n$ then $pq=va_1\ldots a_mb_1\ldots b_n.$ In particular, if $p$ (resp. $r$) is the trivial path on vertex $v$ (resp. vertex $w$) then for any path $q\taking v\to w$, we have $pq=q$ (resp. $qr=q$). 

経路には連結演算が存在する.\index{concatenation!of paths} 経路$p\taking v\to w$と$q\taking w\to x$が与えられたとき, 我々は明らかな方法で結合演算を定義し, $p q\taking v\to x$で示す. もし$p=va_1,a_2\ldots a_m$で$q= wb_1b_2\ldots b_n$ならば$pq=va_1\ldots a_mb_1\ldots b_n.$である. とくに, もし$q$ (あるいは$r$)が頂点$v$ (あるいは頂点$w$)上の自明な経路であれば, 任意の経路$q\taking v\to w$に対して$pq=q$ (あるいは$qr=q$)が得られる.

\end{definition}

\begin{example}

%In Diagram (\ref{dia:graph}), page \pageref{dia:graph}, there are no paths from $v$ to $y$, one path ($f$) from $v$ to $w$, two paths ($fg$ and $fh$) from $v$ to $x$, and infinitely many paths $$\{y i^{p_1}(jk)^{q_1}\cdots i^{p_n}(jk)^{q_n}\;|\;n,p_1,q_1,\ldots,p_n,q_n\in\NN\}$$ from $y$ to $y$. There are other paths as well, including the five trivial paths.

\pageref{dia:graph}ページのDiagram \eqref{dia:graph}では, $v$から$y$への経路は存在せず, $v$から$w$への1本の経路($f$)が存在し, $v$から$x$への2本の経路($fg$および$fh$)が存在し, $y$から$y$への無限の経路$$\{y i^{p_1}(jk)^{q_1}\cdots i^{p_n}(jk)^{q_n}\;|\;n,p_1,q_1,\ldots,p_n,q_n\in\NN\}$$が存在している. また経路は他にも, 5本の自明な経路を含めて存在している.

\end{example}

\begin{exercise}
%How many paths are there in the following graph? 
以下のグラフには何本の経路があるか?
$$\xymatrix{\LMO{1}\ar[r]^{f}&\LMO{2}\ar[r]^{g}&\LMO{3}}$$
\end{exercise}

\begin{exercise}
%Let $G$ be a graph and consider the set $\Path_G$ of paths in $G$. Suppose someone claimed that there is a monoid structure on the set $\Path_G$, where the multiplication formula is given by concatenation of paths. Are they correct? Why or why not? Hint: what should be the identity element?
$G$をグラフとして, $G$の経路の集合$\Path_G$について考える. 誰かが, $\Path_G$には積の式が経路の連結によって与えられるところのモノイド構造が存在していると主張したと仮定しよう. これは正しいだろうか? 正しい, あるいは正しくないのはなぜか? ヒント: 単位元は何であるべきだろうか?
\end{exercise}


%%%% Subsection %%%%

\subsection{Graph homomorphisms}

A graph $(V,A,src,tgt)$ involves two sets and two functions. For two graphs to be comparable, their two sets and their two functions should be appropriately comparable.\index{appropriate comparison}

\begin{definition}\label{def:graph homomorphism}\index{graph!homomorphism}

Let $G=(V,A,src,tgt)$ and $G'=(V',A',src',tgt')$ be graphs. A {\em graph homomorphism $f$ from $G$ to $G'$}, denoted $f\taking G\to G'$, consists of two functions $f_0\taking V\to V'$ and $f_1\taking A\to A'$ such that the two diagrams below commute:
\begin{align}\label{dia:graph hom}
\xymatrix{A\ar[r]^{f_1}\ar[d]_{src}&A'\ar[d]^{src'}\\V\ar[r]_{f_0}&V'
}\hspace{1in}
\xymatrix{A\ar[r]^{f_1}\ar[d]_{tgt}&A'\ar[d]^{tgt'}\\V\ar[r]_{f_0}&V'
}
\end{align}

\end{definition}

\begin{remark}

The above conditions (\ref{dia:graph hom}) may look abstruse at first, but they encode a very important idea, roughly stated ``arrows are bound to their vertices". Under a map of graphs $G\to G'$ , one cannot flippantly send an arrow of $G$ any old arrow of $G'$: it must still connect the vertices it connected before. Below is an example of a mapping that does not respect this condition: $a$ connects $1$ and $2$ before, but not after:
$$
\fbox{\xymatrix{\LMO{\color{red}{1}}\ar[r]^{a}&\LMO{\color{blue}{2}}}}
\xymatrix{~\ar[rr]^{1\mapsto 1',2\mapsto 2', a\mapsto a'}&\hsp&~}
\fbox{\xymatrix{\LMO{\color{red}{1'}}&\LMO{\color{blue}{2'}}\ar[r]^{a'}&\LMO{\color{ForestGreen}{3'}}}}
$$
The commutativity of the diagrams in (\ref{dia:graph hom}) is exactly what is needed to ensure that arrows are handled in the expected way by a proposed graph homomorphism.
 
\end{remark}

\begin{example}[Graph homomorphism]\label{ex:graph hom}

Let $G=(V,A,src,tgt)$ and $G'=(V',A',src',tgt')$ be the graphs drawn to the left and right (respectively) below:
\begin{align}\label{dia:graph hom example}
\parbox{1.5in}{\fbox{\xymatrix{\LMO{\color{red}{1}}\ar[r]^a\ar@/^1pc/[d]^d\ar@/_1pc/[d]_c&\LMO{\color{ForestGreen}{2}}\ar[r]^b&\LMO{\color{red}{3}}\\\LMO{4}&\LMO{\color{blue}{5}}\ar[r]^e&\LMO{\color{blue}{6}}}}}
\parbox{1in}{\xymatrix{~\ar[rr]^{\parbox{.8in}{\vspace{-.2in}\footnotesize$1\mapsto 1', 2\mapsto 2',\\ 3\mapsto 1',4\mapsto 4',\\ 5\mapsto 5',6\mapsto5'$}}&\hsp&~}}
\parbox{.8in}{\fbox{\xymatrix{\LMO{\color{red}{1'}}\ar@<.5ex>[r]^w\ar[d]_y&\LMO{\color{ForestGreen}{2'}}\ar@<.5ex>[l]^x\\\LMO{4'}&\LMO{\color{blue}{5'}}\ar@(r,u)[]_z}}}
\end{align}
The colors indicate our choice of function $f_0\taking V\to V'$. Given that choice, condition (\ref{dia:graph hom}) imposes in this case that there is a unique choice of graph homomorphism $f\taking G\to G'$. 

\end{example}

\begin{exercise}~
\sexc Where are $a,b,c,d,e$ sent under $f_1\taking A\to A'$ in Diagram (\ref{dia:graph hom example})? 
\next Choose a couple elements of $A$ and check that they behave as specified by Diagram (\ref{dia:graph hom}).
\endsexc
\end{exercise}

\begin{exercise}
Let $G$ be a graph, let $n\in\NN$ be a natural number, and let $[n]$ be the chain graph of length $n$, as in Example \ref{ex:[n] as graph}. Is a path of length $n$ in $G$ the same thing as a graph homomorphism $[n]\to G$, or are there subtle differences? More precisely, is there always an isomorphism between the set of graph homomorphisms $[n]\to G$ and the set $\Path_G^{(n)}$ of length-$n$ paths in $G$?
\end{exercise}

\begin{exercise}
Given a morphism of graphs $f\taking G\to G'$, there an induced function $\Path(f)\taking\Path(G)\to\Path(G')$. 
\sexc Is it the case that for every $n\in\NN$, the function $\Path(f)$ carries $\Path^{(n)}(G)$ to $\Path^{(n)}(G')$, or can path lengths change in this process?
\next Suppose that $f_0$ and $f_1$ are injective (meaning no two distinct vertices in $G$ are sent to the same vertex (respectively for arrows) under $f$). Does this imply that $\Path(f)$ is also injective (meaning no two distinct paths are sent to the same path under $f$)?
\next Suppose that $f_0$ and $f_1$ are surjective (meaning every vertex in $G'$ and every arrow in $G'$ is in the image of $f$). Does this imply that $\Path(f)$ is also surjective? Hint: at least one of the answers to these three questions is ``no".
\endsexc
\end{exercise}

\begin{exercise}\label{exc:single condition for graph hom}

Given a graph $(V,A,src,tgt)$, let $i\taking A\to V\times V$ be function guaranteed by the universal property for products, as applied to $src,tgt\taking A\to V$. One might hope to summarize Condition (\ref{dia:graph hom}) for graph homomorphisms by the commutativity of the single square 
\begin{align}\label{dia:equiv graph hom}
\xymatrix{A\ar[r]^{f_1}\ar[d]_{i}&A'\ar[d]^{i'}\\V\times V\ar[r]_{f_0\times f_0}&V'\times V'.}
\end{align}
Is the commutativity of the diagram in (\ref{dia:equiv graph hom}) indeed equivalent to the commutativity of the diagrams in (\ref{dia:graph hom})?
\end{exercise}

%% Subsubsection %%

\subsubsection{Binary relations and graphs}

\begin{definition}\label{def:binary relation}\index{relation!binary}

Let $X$ be a set. A {\em binary relation on $X$} is a subset $R\ss X\times X$. 

\end{definition}

If $X=\NN$ is the set of integers, then the usual $\leq$ defines a relation on $X$: given $(m,n)\in\NN\times\NN$, we put $(m,n)\in R$ iff $m\leq n$. As a table it might be written as to the left
\begin{align}\label{dia:3 relations}
\begin{tabular}{|p{.7cm}|p{.7cm}|}
\bhline
\multicolumn{2}{|c|}{$m\leq n$}\\\bhline
m&n\\\bbhline
0&0\\\hline
0&1\\\hline
1&1\\\hline
0&2\\\hline
1&2\\\hline
2&2\\\hline
0&3\\\bhline
$\vdots$&$\vdots$\\\hline
\end{tabular}
\hspace{1in}
\begin{tabular}{|p{.7cm}|p{.7cm}|}
\bhline
\multicolumn{2}{|c|}{$n=5m$}\\\bhline
m&n\\\bbhline
0&0\\\hline
1&5\\\hline
2&10\\\hline
3&15\\\hline
4&20\\\hline
5&25\\\hline
6&30\\\bhline
$\vdots$&$\vdots$\\\hline
\end{tabular}
\hspace{1in}
\begin{tabular}{|p{.7cm}|p{.7cm}|}
\bhline
\multicolumn{2}{|c|}{$|n-m|\leq 1$}\\\bhline
m&n\\\bbhline
0&0\\\hline
0&1\\\hline
1&0\\\hline
1&1\\\hline
1&2\\\hline
2&1\\\hline
2&2\\\hline
$\vdots$&$\vdots$\\\hline
\end{tabular}
\end{align}
The middle table is the relation $\{(m,n)\in\NN\times\NN\|n=5m\}\ss\NN\times\NN$ and the right-hand table is the relation $\{(m,n)\in\NN\times\NN\||n-m|\leq 1\}\ss\NN\times\NN$. 

\begin{exercise}
A relation on $\RR$ is a subset of $\RR\times\RR$, and one can indicate such a subset of the plane by shading. Choose an error bound $\epsilon>0$ and draw the relation one might refer to as ``$\epsilon$-approximation". To say it another way, draw the relation ``$x$ is within $\epsilon$ of $y$".
\end{exercise}

\begin{exercise}[Binary relations to graphs]\label{exc:rel to graph}\index{relation!graph of}

\sexc If $R\ss S\times S$ is a binary relation, find a natural way to make a graph out of it, having vertices $S$. 
\next What is the set $A$ of arrows? 
\next What are the source and target functions $src,tgt\taking A\to S$?
\next Take the left-hand table in (\ref{dia:3 relations}) and consider its first $7$ rows (i.e. forget the $\vdots$). Draw the corresponding graph (do you see a tetrahedron?). 
\next Do the same for the right-hand table.
\endsexc
\end{exercise}

\begin{exercise}[Graphs to binary relations]\label{ex:graph to rel}~
\sexc If $(V,A,src,tgt)$ is a graph, find a natural way to make a binary relation $R\ss V\times V$ out of it. 
\next Take the left-hand graph $G$ from (\ref{dia:graph hom example}) and write out the corresponding binary relation in table form.
\endsexc
\end{exercise}

\begin{exercise}[Going around the loops]
\sexc Given a binary relation $R\ss S\times S$, you know from Exercise \ref{exc:rel to graph} how to construct a graph out of it, and from Exercise \ref{ex:graph to rel} how to make a new binary relation out of that. How does the resulting relation compare with the original?
\next Given a graph $(V,A,src,tgt)$, you know from Exercise \ref{ex:graph to rel} how to make a new binary relation out of it, and from Exercise \ref{exc:rel to graph} how to construct a new graph out of that. How does the resulting graph compare with the original? 
\endsexc
\end{exercise}







%%%%%%%% Chapter %%%%%%%%

\printindex


\bibliographystyle{amsalpha}
\begin{thebibliography}{SGWB}\rr

\bibitem [Ati]{Ati} Atiyah, M. (1989) ``Topological quantum field theories.'' \textit{Publications Math\'{e}matiques de l'IHÉS} 68 (68), pp. 175--186.

\bibitem [Axl]{Axl} Axler, S. (1997) \textit{Linear algebra done right}. Springer. 

\bibitem [Awo]{Awo} Awodey, S. (2010) \textit{Category theory.} Second edition. Oxford Logic Guides, 52. Oxford University Press, Oxford (前原和寿 訳 (2015) 『圏論 原著第2版』, 共立出版, 東京).

\bibitem [Bar]{Bar} Bralow, H. (1961) ``Possible principles underlying the transformation of sensory messages.'' \textit{Sensory communication}, pp. 217 -- 234.

\bibitem [BD]{BD} Baez, J.C.; Dolan, J. (1995) ``Higher-dimensional algebra and topological quantum field theory.'' \textit{Journal of mathematical physics} vol 36, 6073.

\bibitem [BFL]{BFL} Baez, J.C.; Fritz, T.; Leinster, T. (2011) ``A characterization of entropy in terms of information loss''. \textit{Entropy} 13, no. 11.

\bibitem[BS]{BS} Baez, J.C.; Stay, M. (2011) ``Physics, topology, logic and computation: a Rosetta Stone.'' \textit{New structures for physics}, 95--172. Lecture Notes in Phys., 813, Springer, Heidelberg.

\bibitem [BP1]{BP1} Brown, R.; Porter, T. (2006) ``Category Theory: an abstract setting for
analogy and comparison'', In: \textit{What is Category Theory?} Advanced
Studies in Mathematics and Logic, Polimetrica Publisher, Italy, pp. 257--274.

\bibitem [BP2]{BP2} Brown, R.; Porter, T. (2003) ``Category theory and higher dimensional
algebra: potential descriptive tools in neuroscience'', \textit{Proceedings
of the International Conference on Theoretical Neurobiology, Delhi}, edited by Nandini Singh, National Brain Research
Centre, Conference Proceedings 1 80--92. 

\bibitem [BW]{BW} Barr, M.; Wells, C.  (1990) \textit{Category theory for computing science.} Prentice Hall International Series in Computer Science. Prentice Hall International, New York.

\bibitem [Big]{Big} Biggs, N.M. (2004) \textit{Discrete mathematics}. Oxford University Press, NY. 

\bibitem [Dia]{Dia} Diaconescu, R. (2008) \textit{Institution-independent model theory} Springer.

\bibitem[DI]{DI} D\"{o}ring, A.; Isham, C. J. ``A topos foundation for theories of physics. I. Formal languages for physics.''
\textit{Journal of mathematical physics} 49 (2008), no. 5, 053515.

\bibitem[EV]{EV} Ehresmann, A.C.; Vanbremeersch, J.P. (2007) \textit{Memory evolutive systems; hierarchy, emergence, cognition}. Elsevier.

\bibitem[Eve]{Eve} Everett III, H. (1973). ``The theory of the universal wave function.'' In \textit{The many-worlds interpretation of quantum mechanics} (Vol. 1, p. 3).

\bibitem [Gog]{Gog} Goguen, J. (1992) ``Sheaf semantics for concurrent interacting objects'' \textit{Mathematical structures in Computer Science} Vol 2, pp. 159 -- 191.

\bibitem [Gro]{Gro} Grothendieck, A. (1971). \textit{S\'eminaire de G\'eom\'etrie Alg\'ebrique du Bois Marie - 1960-61 - Revêtements \'etales et groupe fondamental - (SGA 1)} (Lecture notes in mathematics 224) (in French). Berlin; New York: Springer-Verlag.

\bibitem [Kro]{Kro} Kr\"{o}mer, R. (2007). \textit{Tool and Object: A History and Philosophy of Category Theory}, Birkhauser.

\bibitem [Lam]{Lam} Lambek, J. (1980) ``From $\lambda$-calculus to Cartesian closed categories.'' In \textit{Formalism}, Academic Press, London, pp. 375 -- 402.

\bibitem [Law]{Law} Lawvere, F.W. (2005) ``An elementary theory of the category of sets (long version) with
   commentary." (Reprinted and expanded from Proc. Nat. Acad. Sci. U.S.A. 52
   (1964)) \textit{Repr. Theory Appl. Categ.} \textbf{11}, pp. 1 -- 35.
   
\bibitem [Kho]{Kho} Khovanov, M. (2000) ``A categorificiation of the Jones polynomial'' \textit{Duke Math J.}.

\bibitem [Le1]{Le1} Leinster, T. (2004) \textit{Higher Operads, Higher Categories}. London Mathematical Society Lecture Note Series \textbf{298}, Cambridge University Press.

\bibitem [Le2]{Le2} Leinster, T. (2012) ``Rethinking set theory.'' ePrint available \url{http://arxiv.org/abs/1212.6543}.

\bibitem [Lin]{Lin} Linsker, R. (1988) ``Self-organization in a perceptual network.'' \textit{Computer} \textbf{21}, no. 3, pp. 105 -- 117.

\bibitem [LM]{LM} Landry, E.; Marquis, J-P., 2005, "Categories in Contexts: historical, foundational, and philosophical.'' \textit{Philosophia Mathematica}, (3), vol. 13, no. 1, 1 -- 43.

\bibitem [LS]{LS} Lawvere, F.W.; Schanuel, S.H. (2009) \textit{Conceptual mathematics. 
A first introduction to categories.} Second edition. Cambridge University Press, Cambridge.

\bibitem [MacK]{MacK} MacKay, D.J. (2003). \textit{Information theory, inference and learning algorithms.} Cambridge university press.

\bibitem [Mac]{Mac} Mac Lane, S. (1998) \textit{Categories for the working mathematician.} Second edition. Graduate Texts in Mathematics, 5. Springer-Verlag, New York (三好博之, 高木理 訳 (2012) 『圏論の基礎』 丸善, 東京).

\bibitem[Mar1]{Mar1} Marquis, J-P. (2009) \textit{From a Geometrical Point of View: a study in the history and philosophy of category theory}, Springer.

\bibitem [Mar2]{Mar2} Marquis, J-P. ``Category Theory'', \textit{The Stanford Encyclopedia of Philosophy} (Spring 2011 Edition), Edward N. Zalta (ed.), \url{http://plato.stanford.edu/archives/spr2011/entries/category-theory}

\bibitem[Min]{Min} Minsky, M. \textit{The Society of Mind.}  Simon and Schuster, NY 1985.

\bibitem[Mog]{Mog} Moggi, E. (1989) ``A category-theoretic account of program modules.'' \textit{Category theory and computer science (Manchester, 1989),} 101--117, Lecture Notes in Comput. Sci., 389, Springer, Berlin. 

\bibitem [nLa]{nLa} nLab authors.  \url{http://ncatlab.org/nlab/show/HomePage}

\bibitem [Pen]{Pen} Penrose, R. (2006) \textit{The road to reality}. Random house.

\bibitem [RS]{RS} Radul, A.; Sussman, G.J. (2009). ``The art of the propagator.'' \textit{MIT Computer science and artificial intelligence laboratory technical report.}

\bibitem [Sp1]{Sp1} Spivak, D.I. (2012) ``Functorial data migration.'' \textit{Information and communication} 

\bibitem [Sp2]{Sp2} Spivak, D.I. (2012) ``Queries and constraints via lifting problems.'' Submitted to \textit{Mathematical structures in computer science}. ePrint available: \url{http://arxiv.org/abs/1202.2591}

\bibitem [Sp3]{Sp3} Spivak, D.I. (2012) ``Kleisli database instances''. ePrint available: \url{http://arxiv.org/abs/1209.1011}

\bibitem [Sp4]{Sp4} Spivak, D.I. (2013) ``The operad of wiring diagrams: Formalizing a graphical language for databases, recursion, and plug-and-play circuits.'' Available online: \url{http://arxiv.org/abs/1305.0297}

\bibitem[SGWB]{SGWB} Spivak D.I.; Giesa, T.; Wood, E.; Buehler, M.J. (2011) ``Category Theoretic Analysis of Hierarchical Protein Materials and Social Networks.'' PLoS ONE 6(9): e23911. doi:10.1371/journal.pone.0023911

\bibitem[SK]{SK} Spivak, D.I.; Kent, R.E. (2012) ``Ologs: A Categorical Framework for Knowledge Representation.'' \textit{PLoS ONE} 7(1): e24274. doi:10.1371/journal.pone.0024274.

\bibitem[WeS]{WeS} Weinberger, S. (2011) ``What is... Persistent Homology?'' AMS.

\bibitem[WeA]{WeA} Weinstein, A. (1996) ``Groupoids: unifying internal and external symmetry. \textit{Notices of the AMS} Vol 43, no. 7, pp. 744 -- 752.

%\bibitem[Wik]{Wik} \href{http://www.wikipedia.org}{\text Wikipedia} (multiple authors). Various articles, all linked with a hyperreference are scattered throughout this text. All accessed December 6, 2012 -- \today.
\bibitem[Wik]{Wik} \href{http://www.wikipedia.org}{\text Wikipedia} (multiple authors). Various articles, all linked with a hyperreference are scattered throughout this text. All accessed 2012-12-06 -- \the \year-{\ifnum \month < 10 0\the\month \else \the\month \fi}-{\ifnum \day < 10 0\the\day \else \the\day \fi}.

\end{thebibliography}


\end{document}

