

%%%%%%%% Chapter %%%%%%%%

%\chapter{The category of sets}\label{chap:sets}

\chapter{集合の圏}\label{chap:sets}

%The theory of sets was invented as a foundation for all of mathematics. The notion of sets and functions serves as a basis on which to build our intuition about categories in general. In this chapter we will give examples of sets and functions and then move on to discuss commutative diagrams. At this point we can introduce ologs which will allow us to use the language of category theory to speak about real world concepts. Then we will introduce limits and colimits, and their universal properties. All of this material is basic set theory, but it can also be taken as an investigation of our first category, the {\em category of sets}, which we call $\Set$. We will end this chapter with some other interesting constructions in $\Set$ that do not fit into the previous sections.

集合論は全数学の基礎として発明された. 集合と函数についての記述は一般の圏について我々の直感を構築するための基礎となる. この章では我々は集合と函数の例を出した後に, 可換図形(commutative diagram)の議論に以降する. この時点で我々はologを導入することができるようになる. ologにより我々は現実世界の概念について語る際に圏論の言葉を使うことができるようになる. 続いて我々は極限(limit)と余極限(colimit), およびその普遍的な性質(universal properties)を導入する. これらの全ては基本的な集合論であるが, 一方で我々の最初の圏---集合の圏$\Set$---の調査ともとらえることができる.
%
それまでの各章にはうまく当てはまらない, また別の興味深い$\Set$の構成をもって, 我々の本章を終えることにする.

%%%%%% Section %%%%%%

%\section{Sets and functions}\index{set} 

\section{集合と函数}\index{set} 

%%%% Subsection %%%%

%\subsection{Sets}

\subsection{集合}

%In this course I'll assume you know what a set is. We can think of a set $X$ as a collection of things $x\in X$, each of which is recognizable as being in $X$ and such that for each pair of named elements $x,x'\in X$ we can tell if $x=x'$ or not.
%\footnote{Note that the symbol $x'$, read ``x-prime", has nothing to do with calculus or derivatives. It is simply notation that we use to name a symbol that is suggested as being somehow like $x$. This suggestion of kinship between $x$ and $x'$ is meant only as an aid for human cognition, and not as part of the mathematics.}
%The set of pendulums is the collection of things we agree to call pendulums, each of which is recognizable as being a pendulum, and for any two people pointing at pendulums we can tell if they're pointing at the same pendulum or not. 

この教程では, 著者は集合が何であるかを読者がよく知っていることを想定している. 我々は集合$X$を要素$x \in X$の集まりだと考えることができる. 要素それぞれは$X$の中にあると認識でき, また名前付きの要素の組$x,x'\in X$に対して我々は$x=x'$であるかそうでないかを答えることができる.
\footnote{記号$x'$---``xプライム''と読む---はcalculusや微分とは全く関係がない. これは単に$x$となにかしら似ていることを示唆する記号として我々が用いる記号の名前である. この$x$と$x'$との間の類似性の示唆が意味するものは人間の認知のみを目的にしており, 数学の一部ではない.}
振り子の集合は我々が振り子と呼ぶことを認めたものの集合であり, 要素のそれぞれは振り子であると認識される. また任意の二人の人間が指差した振り子に対して, 我々は彼らが同じ振り子を指差しているかどうかを答えることができる.

\begin{figure}
\begin{center}
\includegraphics[height=2in]{aSet}
\end{center}
%\caption{A set $X$ with $9$ elements and a set $Y$ with no elements, $Y=\emptyset$.}
\caption{要素が$9$つある集合$X$と要素がない集合$Y=\emptyset$.}
\end{figure}

\begin{notation}\label{not:basic math notation}

%The symbol $\emptyset$\index{a symbol!$\emptyset$} denotes the set with no elements. The symbol $\NN$\index{a symbol!$\NN$} denotes the set of natural numbers, which we can write as 
%$$\NN:=\{0,1,2,3,4,\ldots,877,\ldots\}.$$
%The symbol $\ZZ$\index{a symbol!$\ZZ$} denotes the set of integers, which contains both the natural numbers and their negatives, 
%$$\ZZ:=\{\ldots,-551,\ldots,-2,-1,0,1,2,\ldots\}.$$ 

記号$\emptyset$\index{a symbol!$\emptyset$}は要素がない集合を示す. 記号$\NN$\index{a symbol!$\NN$}は自然数の集合を示す. これは
$$\NN:=\{0,1,2,3,4,\ldots,877,\ldots\}.$$
と書くことができる.
記号$\ZZ$\index{a symbol!$\ZZ$}は整数の集合を示す. これは自然数とその負の要素を両方含む.
$$\ZZ:=\{\ldots,-551,\ldots,-2,-1,0,1,2,\ldots\}.$$ 

%If $A$ and $B$ are sets, we say that $A$ is a {\em subset}\index{subset} of $B$, and write $A\ss B$, if every element of $A$ is an element of $B$. So we have $\NN\ss\ZZ$. Checking the definition, one sees that for any set $A$, we have (perhaps uninteresting) subsets $\emptyset\ss A$ and $A\ss A$. We can use {\em set-builder notation}\index{set!set builder notation} to denote subsets. For example the set of even integers can be written $\{n\in\ZZ\|n\tn{ is even}\}$. The set of integers greater than $2$ can be written in many ways, such as $$\{n\in\ZZ\|n>2\} \hsp\tn{or}\hsp\{n\in\NN\|n>2\}\hsp\tn{or}\hsp\{n\in\NN\|n\geq 3\}.$$

$A$と$B$が集合であり, $A$の全ての要素が$B$の要素であるとき, $A$は$B$の\emph{部分集合}(subset)\index{subset}であると言い, $A\ss B$と書く. よって$\NN\ss\ZZ$が得られる. 定義を確認すれば, 任意の集合$A$について, (おそらく興味をひくことはない)部分集合$\emptyset\ss A$と$A\ss A$が得られることが分かる. 部分集合を示すのに\emph{内含表記(set-builder notation)}\index{set!set builder notation}を使うこともできる. 例えば偶数の集合は$\{n\in\ZZ\|n\tn{ は偶数}\}$と書くことができる. $2$より大きな整数の集合の書き方は数多くある. 例えば$$\{n\in\ZZ\|n>2\} \hsp\tn{or}\hsp\{n\in\NN\|n>2\}\hsp\tn{or}\hsp\{n\in\NN\|n\geq 3\}.$$

%The symbol $\exists$ means ``there exists".\index{a symbol!$\exists$} So we could write the set of even integers as $$\{n\in\ZZ\|n\tn{ is even}\}\hsp=\hsp\{n\in\ZZ\|\exists m\in\ZZ\tn{ such that } 2m=n\}.$$ The symbol $\exists!$\index{a symbol!$\exists$"!} means ``there exists a unique". So the statement ``$\exists! x\in\RR\tn{ such that } x^2=0$" means that there is one and only one number whose square is 0. Finally, the symbol $\forall$ means ``for all".\index{a symbol!$\forall$} So the statement ``$\forall m\in\NN\;\exists n\in\NN\tn{ such that } m<n$" means that for every number there is a bigger one.

記号$\exists$は``少なくとも一つ存在する''ことを意味する.\index{a symbol!$\exists$} よって偶数は$$\{n\in\ZZ\|n\tn{は偶数}\}\hsp=\hsp\{n\in\ZZ\|\exists m\in\ZZ\tn{ such that } 2m=n\}.$$と書くこともできた. 記号$\exists!$\index{a symbol!$\exists$"!}は``一意的に(唯一)存在する''ことを意味する. よって``$\exists! x\in\RR\tn{ such that } x^2=0$''という言明は, 二乗が0である数が存在し, それがただ一つだけであることを意味している. 最後に, 記号$\forall$は``全ての''を意味している.\index{a symbol!$\forall$} よって``$\forall m\in\NN\;\exists n\in\NN\tn{ such that } m<n$''という言明は全ての数に対してそれより大きい数が存在することを意味している.

%As you may have noticed, we use the colon-equals notation `` $A:=XYZ$ " to mean something like ``define $A$ to be $XYZ$".\index{a symbol!:=} That is, a colon-equals declaration is not denoting a fact of nature (like $2+2=4$), but a choice of the speaker. It just so happens that the notation above, such as $\NN:=\{0,1,2,\ldots\}$, is a widely-held choice.

既に気付いている読者もいるであろうが, 我々はコロン-等号記法を用い, `` $A:=XYZ$''が``$A$を$XYZ$と定義する''ことを意味するとしている.\index{a symbol!:=} これは, コロン-等号による宣言は自然界における事実(例えば$2+2=4$というようなもの)を示しているのではなく, 話者の選択を示しているということである. 上記の, 例えば$\NN:=\{0,1,2,\ldots\}$といった記法は, たまたま広く取られる選択である.

\end{notation}

\begin{exercise}
%Let $A=\{1,2,3\}$. What are all the subsets of $A$? Hint: there are 8.
$A=\{1,2,3\}$とする. $A$の部分集合を全て求めよ. ヒント: 8つある.
\end{exercise}

%%%% Subsection %%%%

\subsection{Functions}\label{sec:functions}

If $X$ and $Y$ are sets, then a {\em function $f$ from $X$ to $Y$},\index{function} denoted $f\taking X\to Y$, is a mapping that sends each element $x\in X$ to an element of $Y$, denoted $f(x)\in Y$. We call $X$ the {\em domain}\index{function!domain} of the function $f$ and we call $Y$ the {\em codomain}\index{function!codomain} of $f$. 

\begin{align}\label{dia:setmap}
\parbox{2.3in}{\includegraphics[height=2in]{SetMap}}
\end{align}

Note that for every element $x\in X$, there is exactly one arrow emanating from $x$, but for an element $y\in Y$, there can be several arrows pointing to $y$, or there can be no arrows pointing to $y$. 

\begin{application}\label{app:force-extension}\index{materials!force-extension curves}

In studying the mechanics of materials, one wishes to know how a material responds to tension. For example a rubber band responds to tension differently than a spring does. To each material we can associate a \href{http://en.wikipedia.org/wiki/Stressミstrain_curve}{\text force-extension curve}, recording how much force the material carries when extended to various lengths. Once we fix a methodology for performing experiments, finding a material's force-extension curve would ideally constitute a function from the set of materials to the set of curves.
\footnote{In reality, different samples of the same material, say samples of different sizes or at different temperatures, may have different force-extension curves. If we want to see this as a true function whose codomain is curves it should have as domain something like the set of material samples.}

\end{application}

\begin{exercise}

Here is a simplified account of how the \href{http://en.wikipedia.org/wiki/Retina}{\text brain receives light}. The eye contains about 100 million photoreceptor (PR) cells. Each connects to a retinal ganglion (RG) cell. No PR cell connects to two different RG cells, but usually many PR cells can attach to a single RG cell. 

Let $PR$ denote the set of photoreceptor cells and let $RG$ denote the set of retinal ganglion cells. 
\sexc According to the above account, does the connection pattern constitute a function $RG\to PR$, a function $PR\to RG$ or neither one? 
\next Would you guess that the connection pattern that exists between other areas of the brain are ``function-like"?
\endsexc
\end{exercise}

\begin{example}\label{ex:subset as function}\index{subset!as function}

Suppose that $X$ is a set and $X'\ss X$ is a subset. Then we can consider the function $X'\to X$ given by sending every element of $X'$ to ``itself" as an element of $X$. For example if $X=\{a,b,c,d,e,f\}$ and $X'=\{b,d,e\}$ then $X'\ss X$ and we turn that into the function $X'\to X$ given by $b\mapsto b, d\mapsto d, e\mapsto e$.
\footnote{This kind of arrow,\;\;$\mapsto$\;\;, is read aloud as ``maps to". A function $f\taking X\to Y$ means a rule for assigning to each element $x\in X$ an element $f(x)\in Y$. We say that ``$x$ maps to $f(x)$" and write $x\mapsto f(x)$.}\index{a symbol!$\mapsto$}

As a matter of notation, we may sometimes say something like the following: Let $X$ be a set and let $i\taking X'\ss X$ be a subset. Here we are making clear that $X'$ is a subset of $X$, but that $i$ is the name of the associated function.

\end{example}

\begin{exercise}
Let $f\taking\NN\to\NN$ be the function that sends every natural number to its square, e.g. $f(6)=36$. First fill in the blanks below, then answer a question.
\sexc $2\mapsto\ul{\hspace{.5in}}$
\next $0\mapsto\ul{\hspace{.5in}}$
\next $-2\mapsto\ul{\hspace{.5in}}$
\next $5\mapsto\ul{\hspace{.5in}}$
\next Consider the symbol $\to$ and the symbol $\mapsto$. What is the difference between how these two symbols are used in this book?
\endsexc
\end{exercise}

Given a function $f\taking X\to Y$, the elements of $Y$ that have at least one arrow pointing to them are said to be {\em in the image} of $f$; that is we have \index{image}
\begin{align}\label{dia:image}
\im(f):=\{y\in Y\| \exists x\in X \tn{ such that } f(x)=y\}.
\end{align} 

\begin{exercise}
If $f\taking X\to Y$ is depicted by (\ref{dia:setmap}) above, write its image, $\im(f)$ as a set.
\end{exercise}

Given a function $f\taking X\to Y$ and a function $g\taking Y\to Z$, where the codomain of $f$ is the same set as the domain of $g$ (namely $Y$), we say that $f$ and $g$ are composable 
$$X\Too{f}Y\Too{g}Z.$$ The {\em composition of $f$ and $g$}\label{function composition}\index{function!composition}\index{composition!of functions}\index{a symbol!$\circ$} is denoted by $g\circ f\taking X\to Z$. 

\begin{figure}[h]
\begin{center}
\includegraphics[height=2in]{composition}
\end{center}
\caption{Functions $f\taking X\to Y$ and $g\taking Y\to Z$ compose to a function $g\circ f\taking X\to Z$; just follow the arrows.}
\end{figure}

Let $X$ and $Y$ be sets. We write $\Hom_\Set(X,Y)$\index{a symbol!$\Hom_\Set$} to denote the set of functions $X\to Y$.
\footnote{The strange notation $\Hom_\Set(-,-)$ will make more sense later, when it is seen as part of a bigger story.} 
Note that two functions $f,g\taking X\to Y$ are equal\index{function!equality of} if and only if for every element $x\in X$ we have $f(x)=g(x)$. 

\begin{exercise}
Let $A=\{1,2,3,4,5\}$ and $B=\{x,y\}.$ 
\sexc How many elements does $\Hom_\Set(A,B)$ have? 
\next How many elements does $\Hom_\Set(B,A)$ have?
\endsexc
\end{exercise}

\begin{exercise}~
\sexc Find a set $A$ such that for all sets $X$ there is exactly one element in $\Hom_\Set(X,A)$. Hint: draw a picture of proposed $A$'s and $X$'s.
\next Find a set $B$ such that for all sets $X$ there is exactly one element in $\Hom_\Set(B,X)$.
\endsexc 
\end{exercise}

For any set $X$, we define the {\em identity function on $X$}\index{function!identity}, denoted $\id_X\taking X\to X$, to be the function such that for all $x\in X$ we have $\id_X(x)=x$.\index{a symbol!$\id_X$}

\begin{definition}[Isomorphism]\label{def:iso in set}

Let $X$ and $Y$ be sets. A function $f\taking X\to Y$ is called an {\em isomorphism}\index{function!isomorphism}\index{isomorphism!of sets}, denoted $f\taking X\To{\iso}Y$, if there exists a function $g\taking Y\to X$ such that $g\circ f=\id_X$ and $f\circ g=\id_Y$. We also say that $f$ is {\em invertible} and we say that $g$ is {\em the inverse}\index{function!inverse} of $f$. If there exists an isomorphism $X\To\iso Y$ we say that $X$ and $Y$ are {\em isomorphic} sets and may write $X\iso Y$. \index{a symbol!$\iso$}

\end{definition}

\begin{example}

If $X$ and $Y$ are sets and $f\taking X\to Y$ is an isomorphism then the analogue of Diagram \ref{dia:setmap} will look like a perfect matching, more often called a {\em one-to-one correspondence}\index{one-to-one correspondence}\index{correspondence!one-to-one}. That means that no two arrows will hit the same element of $Y$, and every element of $Y$ will be in the image. For example, the following depicts an isomorphism $X\To{\iso}Y$.

\begin{align}\label{dia:setmapiso}
\parbox{2.3in}{\includegraphics[height=2in]{SetMapIso}}
\end{align}

\end{example}

\begin{application}\label{app:DNA RNA}\index{RNA transcription}

There is an isomorphism between the set $\tn{Nuc}_\tn{DNA}$ of \href{http://en.wikipedia.org/wiki/Nucleotides}{\text nucleotides} found in DNA and the set $\tn{Nuc}_\tn{RNA}$ of nucleotides found in RNA. Indeed both sets have four elements, so there are 24 different isomorphisms. But only one is useful. Before we say which one it is, let us say there is also an isomorphism $\tn{Nuc}_\tn{DNA}\iso\{A,C,G,T\}$ and an isomorphism $\tn{Nuc}_\tn{RNA}\iso\{A,C,G,U\}$, and we will use the letters as abbreviations for the nucleotides. 

The convenient isomorphism $\tn{Nuc}_\tn{DNA}\To{\iso}\tn{Nuc}_\tn{RNA}$ is that given by RNA transcription; it sends 
$$A\mapsto U, C\mapsto G, G\mapsto C, T\mapsto A.$$ 
(See also Application \ref{app:polymerase}.) There is also an isomorphism $\tn{Nuc}_\tn{DNA}\To{\iso}\tn{Nuc}_\tn{DNA}$ (the matching in the double-helix) given by 
$$A\mapsto T, C\mapsto G, G\mapsto C, T\mapsto A.$$

Protein production can be modeled as a function from the set of 3-nucleotide sequences to the set of eukaryotic amino acids. However, it cannot be an isomorphism because there are $4^3=64$ triplets of RNA nucleotides, but only 21 eukaryotic amino acids. 

\end{application}

\begin{exercise}
Let $n\in\NN$ be a natural number and let $X$ be a set with exactly $n$ elements. 
\sexc How many isomorphisms are there from $X$ to itself? 
\next Does your formula from part a.) hold when $n=0$?
\endsexc
\end{exercise}

\begin{lemma}\label{lemma:isomorphic ER in Set}

The following facts hold about isomorphism.
\begin{enumerate}
\item Any set $A$ is isomorphic to itself; i.e. there exists an isomorphism $A\To{\iso} A$.
\item For any sets $A$ and $B$, if $A$ is isomorphic to $B$ then $B$ is isomorphic to $A$.
\item For any sets $A, B,$ and $C$, if $A$ is isomorphic to $B$ and $B$ is isomorphic to $C$ then $A$ is isomorphic to $C$.
\end{enumerate}

\end{lemma}

\begin{proof}

\begin{enumerate}
\item The identity function $\id_A\taking A\to A$ is invertible; its inverse is $\id_A$ because $\id_A\circ\id_A=\id_A$.
\item If $f\taking A\to B$ is invertible with inverse $g\taking B\to A$ then $g$ is an isomorphism with inverse $f$.
\item If $f\taking A\to B$ and $f'\taking B\to C$ are each invertible with inverses $g\taking B\to A$ and $g'\taking C\to B$ then the following calculations show that $f'\circ f$ is invertible with inverse $g\circ g'$: 
\begin{align*}
(f'\circ f)\circ(g\circ g')=f'\circ(f\circ g)\circ g'=f'\circ\id_B\circ g'=f'\circ g'=\id_C\\
(g\circ g')\circ(f'\circ f)=g\circ(g'\circ f')\circ f=g\circ\id_B\circ f=g\circ f=\id_A
\end{align*}
\end{enumerate}

\end{proof}

\begin{exercise}\label{exc:functions are not iso invariant}
Let $A$ and $B$ be the sets drawn below:
$$
\parbox{1.1in}{\boxtitle{A:=}\fbox{\xymatrix@=1pt{\\&\LMO{\;a\;}&&&\LMO{\;\;\;7\;\;}&\\\\\\&&&\LMO{Q}\\&}}}
\hspace{.8in}
\parbox{1.2in}{\boxtitle{B:=}\fbox{\xymatrix@=1pt{&&&\LMO{r8}&&\\\\\\\\&\LMO{``Bob"}\\&&\LMO{\clubsuit}}}}
$$
Note that the sets $A$ and $B$ are isomorphic. Supposing that $f\taking B\to\{1,2,3,4,5\}$ sends ``Bob" to $1$, sends $\clubsuit$ to $3$, and sends $r8$ to $4$, is there a canonical function $A\to\{1,2,3,4,5\}$ corresponding to $f$?
\footnote{Canonical means something like ``best choice", a choice that stands out as the only reasonable one.}\index{canonical}
\end{exercise}

\begin{exercise}\label{exc:generator for set}
Find a set $A$ such that for any set $X$ there is a isomorphism of sets $$X\iso\Hom_\Set(A,X).$$ Hint: draw a picture of proposed $A$'s and $X$'s.
\end{exercise}

For any natural number $n\in\NN$, define a set 
\begin{align}\label{dia:underline n}\index{a symbol!$\ul{n}$}
\ul{n}:=\{1,2,3,\ldots,n\}.
\end{align}
So, in particular, $\ul{2}=\{1,2\}, \ul{1}=\{1\}$, and $\ul{0}=\emptyset$. 

Let $A$ be any set. A function $f\taking\ul{n}\to A$ can be written as a sequence $$f=(f(1),f(2),\ldots,f(n)).$$

\begin{exercise}\label{exc:sequence}~
\sexc Let $A=\{a,b,c,d\}$. If $f\taking\ul{10}\to A$ is given by $(a,b,c,c,b,a,d,d,a,b)$, what is $f(4)$?
\next Let $s\taking\ul{7}\to\NN$ be given by $s(i)=i^2$. Write $s$ out as a sequence.
\endsexc
\end{exercise}

\begin{definition}Cardinality of finite sets]\label{def:cardinality}[

Let $A$ be a set and $n\in\NN$ a natural number. We say that $A$ is {\em has cardinality $n$}\index{cardinality}, denoted $$|A|=n,$$ if there exists an isomorphism of sets $A\iso\ul{n}$. If there exists some $n\in\NN$ such that $A$ has cardinality $n$ then we say that $A$ is {\em finite}. Otherwise, we say that $A$ is {\em infinite} and write $|A|\geq\infty$.

\end{definition}

\begin{exercise}~
\sexc Let $A=\{5,6,7\}$. What is $|A|$? 
\next What is $|\NN|$? 
\next What is $|\{n\in\NN\|n\leq 5\}|$?
\endsexc
\end{exercise}

\begin{lemma}

Let $A$ and $B$ be finite sets. If there is an isomorphism of sets $f\taking A\to B$ then the two sets have the same cardinality, $|A|=|B|$.

\end{lemma}

\begin{proof}

Suppose $f\taking A\to B$ is an isomorphism. If there exists natural numbers $m,n\in\NN$ and isomorphisms $a\taking\ul{m}\To\iso A$ and $b\taking\ul{n}\To\iso B$ then $\ul{m}\To{a^\m1}A\To{f}B\To{b}\ul{n}$ is an isomorphism. One can prove by induction that the sets $\ul{m}$ and $\ul{n}$ are isomorphic if and only if $m=n$. 

\end{proof}

