

%%%%%%%% Chapter %%%%%%%%

%\chapter{The category of sets}\label{chap:sets}

\chapter{集合の圏}\label{chap:sets}

%The theory of sets was invented as a foundation for all of mathematics. The notion of sets and functions serves as a basis on which to build our intuition about categories in general. In this chapter we will give examples of sets and functions and then move on to discuss commutative diagrams. At this point we can introduce ologs which will allow us to use the language of category theory to speak about real world concepts. Then we will introduce limits and colimits, and their universal properties. All of this material is basic set theory, but it can also be taken as an investigation of our first category, the {\em category of sets}, which we call $\Set$. We will end this chapter with some other interesting constructions in $\Set$ that do not fit into the previous sections.

集合論は全数学の基礎として発明された. 集合と函数についての記述は一般の圏について我々の直感を構築するための基礎となる. この章では我々は集合と函数の例を出した後に, 可換図式(commutative diagram)の議論に移行する. この時点で我々はologを導入することができるようになる. ologにより我々は現実世界の概念について語る際に圏論の言葉を使うことができるようになる. 続いて我々は極限(limit)と余極限(colimit), およびその普遍的な性質(universal properties)を導入する. これらの全ては基本的な集合論であるが, 一方で我々の最初の圏---集合の圏$\Set$---の調査ともとらえることができる.
%
それまでの各章にはうまく当てはまらない, また別の興味深い$\Set$の構成をもって, 我々の本章を終えることにする.

%%%%%% Section %%%%%%

%\section{Sets and functions}\index{set} 

\section{集合と函数}\index{set} 

%%%% Subsection %%%%

%\subsection{Sets}

\subsection{集合}

%In this course I'll assume you know what a set is. We can think of a set $X$ as a collection of things $x\in X$, each of which is recognizable as being in $X$ and such that for each pair of named elements $x,x'\in X$ we can tell if $x=x'$ or not.
%\footnote{Note that the symbol $x'$, read ``x-prime", has nothing to do with calculus or derivatives. It is simply notation that we use to name a symbol that is suggested as being somehow like $x$. This suggestion of kinship between $x$ and $x'$ is meant only as an aid for human cognition, and not as part of the mathematics.}
%The set of pendulums is the collection of things we agree to call pendulums, each of which is recognizable as being a pendulum, and for any two people pointing at pendulums we can tell if they're pointing at the same pendulum or not. 

この教程では, 著者は集合が何であるかを読者がよく知っていることを想定している. 我々は集合$X$を要素$x \in X$の集まりだと考えることができる. 要素それぞれは$X$の中にあると認識でき, また名前付きの要素の組$x,x'\in X$に対して我々は$x=x'$であるかそうでないかを答えることができる.
\footnote{記号$x'$---``xプライム''と読む---はcalculusや微分とは全く関係がない. これは単に$x$となにかしら似ていることを示唆する記号として我々が用いる記号の名前である. この$x$と$x'$との間の類似性の示唆が意味するものは人間の認知のみを目的にしており, 数学の一部ではない.}
振り子の集合は我々が振り子と呼ぶことを認めたものの集合であり, 要素のそれぞれは振り子であると認識される. また任意の二人の人間が指差した振り子に対して, 我々は彼らが同じ振り子を指差しているかどうかを答えることができる.

\begin{figure}
\begin{center}
\includegraphics[height=2in]{aSet}
\end{center}
%\caption{A set $X$ with $9$ elements and a set $Y$ with no elements, $Y=\emptyset$.}
\caption{要素が$9$つある集合$X$と要素がない集合$Y=\emptyset$.}
\end{figure}

\begin{notation}\label{not:basic math notation}

%The symbol $\emptyset$\index{a symbol!$\emptyset$} denotes the set with no elements. The symbol $\NN$\index{a symbol!$\NN$} denotes the set of natural numbers, which we can write as 
%$$\NN:=\{0,1,2,3,4,\ldots,877,\ldots\}.$$
%The symbol $\ZZ$\index{a symbol!$\ZZ$} denotes the set of integers, which contains both the natural numbers and their negatives, 
%$$\ZZ:=\{\ldots,-551,\ldots,-2,-1,0,1,2,\ldots\}.$$ 

記号$\emptyset$\index{a symbol!$\emptyset$}は要素がない集合を示す. 記号$\NN$\index{a symbol!$\NN$}は自然数の集合を示す. これは
$$\NN:=\{0,1,2,3,4,\ldots,877,\ldots\}.$$
と書くことができる.
記号$\ZZ$\index{a symbol!$\ZZ$}は整数の集合を示す. これは自然数とその負の要素を両方含む.
$$\ZZ:=\{\ldots,-551,\ldots,-2,-1,0,1,2,\ldots\}.$$ 

%If $A$ and $B$ are sets, we say that $A$ is a {\em subset}\index{subset} of $B$, and write $A\ss B$, if every element of $A$ is an element of $B$. So we have $\NN\ss\ZZ$. Checking the definition, one sees that for any set $A$, we have (perhaps uninteresting) subsets $\emptyset\ss A$ and $A\ss A$. We can use {\em set-builder notation}\index{set!set builder notation} to denote subsets. For example the set of even integers can be written $\{n\in\ZZ\|n\tn{ is even}\}$. The set of integers greater than $2$ can be written in many ways, such as $$\{n\in\ZZ\|n>2\} \hsp\tn{or}\hsp\{n\in\NN\|n>2\}\hsp\tn{or}\hsp\{n\in\NN\|n\geq 3\}.$$

$A$と$B$が集合であり, $A$の全ての要素が$B$の要素であるとき, $A$は$B$の\emph{部分集合}(subset)\index{subset}であると言い, $A\ss B$と書く. よって$\NN\ss\ZZ$が得られる. 定義を確認すれば, 任意の集合$A$について, (おそらく興味をひくことはない)部分集合$\emptyset\ss A$と$A\ss A$が得られることが分かる. 部分集合を示すのに\emph{内含表記(set-builder notation)}\index{set!set builder notation}を使うこともできる. 例えば偶数の集合は$\{n\in\ZZ\|n\tn{ は偶数}\}$と書くことができる. $2$より大きな整数の集合の書き方は数多くある. 例えば$$\{n\in\ZZ\|n>2\} \hsp\tn{or}\hsp\{n\in\NN\|n>2\}\hsp\tn{or}\hsp\{n\in\NN\|n\geq 3\}.$$

%The symbol $\exists$ means ``there exists".\index{a symbol!$\exists$} So we could write the set of even integers as $$\{n\in\ZZ\|n\tn{ is even}\}\hsp=\hsp\{n\in\ZZ\|\exists m\in\ZZ\tn{ such that } 2m=n\}.$$ The symbol $\exists!$\index{a symbol!$\exists$"!} means ``there exists a unique". So the statement ``$\exists! x\in\RR\tn{ such that } x^2=0$" means that there is one and only one number whose square is 0. Finally, the symbol $\forall$ means ``for all".\index{a symbol!$\forall$} So the statement ``$\forall m\in\NN\;\exists n\in\NN\tn{ such that } m<n$" means that for every number there is a bigger one.

記号$\exists$は``少なくとも一つ存在する''ことを意味する.\index{a symbol!$\exists$} よって偶数は$$\{n\in\ZZ\|n\tn{は偶数}\}\hsp=\hsp\{n\in\ZZ\|\exists m\in\ZZ\tn{ such that } 2m=n\}.$$と書くこともできた. 記号$\exists!$\index{a symbol!$\exists$"!}は``一意的に(唯一)存在する''ことを意味する. よって``$\exists! x\in\RR\tn{ such that } x^2=0$''という言明は, 二乗が0である数が存在し, それがただ一つだけであることを意味している. 最後に, 記号$\forall$は``全ての''を意味している.\index{a symbol!$\forall$} よって``$\forall m\in\NN\;\exists n\in\NN\tn{ such that } m<n$''という言明は全ての数に対してそれより大きい数が存在することを意味している.

%As you may have noticed, we use the colon-equals notation `` $A:=XYZ$ " to mean something like ``define $A$ to be $XYZ$".\index{a symbol!:=} That is, a colon-equals declaration is not denoting a fact of nature (like $2+2=4$), but a choice of the speaker. It just so happens that the notation above, such as $\NN:=\{0,1,2,\ldots\}$, is a widely-held choice.

既に気付いている読者もいるであろうが, 我々はコロン-等号記法を用い, `` $A:=XYZ$''が``$A$を$XYZ$と定義する''ことを意味するとしている.\index{a symbol!:=} これは, コロン-等号による宣言は自然界における事実(例えば$2+2=4$というようなもの)を示しているのではなく, 話者の選択を示しているということである. 上記の, 例えば$\NN:=\{0,1,2,\ldots\}$といった記法は, たまたま広く取られる選択である.

\end{notation}

\begin{exercise}
%Let $A=\{1,2,3\}$. What are all the subsets of $A$? Hint: there are 8.
$A=\{1,2,3\}$とする. $A$の部分集合を全て求めよ. ヒント: 8つある.
\end{exercise}

%%%% Subsection %%%%

%\subsection{Functions}\label{sec:functions}
\subsection{函数}\label{sec:functions}

%If $X$ and $Y$ are sets, then a {\em function $f$ from $X$ to $Y$},\index{function} denoted $f\taking X\to Y$, is a mapping that sends each element $x\in X$ to an element of $Y$, denoted $f(x)\in Y$. We call $X$ the {\em domain}\index{function!domain} of the function $f$ and we call $Y$ the {\em codomain}\index{function!codomain} of $f$. 

$X$と$Y$が集合であるならば, \emph{$X$から$Y$への函数$f$}\index{function}は$f\taking X\to Y$と表記され, これはそれぞれの要素$x\in X$を$f(x)\in Y$と表記される$Y$の要素へ送る写像である. $X$は函数$f$の\emph{始域(domain)}\index{function!domain}と呼ばれ, $Y$は$f$の\emph{終域(codomain)}\index{function!codomain}と呼ばれる.

\begin{align}\label{dia:setmap}
\parbox{2.3in}{\includegraphics[height=2in]{SetMap}}
\end{align}

%Note that for every element $x\in X$, there is exactly one arrow emanating from $x$, but for an element $y\in Y$, there can be several arrows pointing to $y$, or there can be no arrows pointing to $y$. 

全ての要素$x\in X$に対して, $x$から出る矢印が正確に一本存在するが, しかしある要素$y\in Y$に対しては, $y$を指す矢印が何本も存在するかもしれないし, また$y$を指す矢印が存在しないかもしれないことに注意しよう.

\begin{application}\label{app:force-extension}\index{materials!force-extension curves}

%In studying the mechanics of materials, one wishes to know how a material responds to tension. For example a rubber band responds to tension differently than a spring does. To each material we can associate a \href{http://en.wikipedia.org/wiki/Stress–strain_curve}{\text force-extension curve}, recording how much force the material carries when extended to various lengths. Once we fix a methodology for performing experiments, finding a material's force-extension curve would ideally constitute a function from the set of materials to the set of curves.
%\footnote{In reality, different samples of the same material, say samples of different sizes or at different temperatures, may have different force-extension curves. If we want to see this as a true function whose codomain is curves it should have as domain something like the set of material samples.}

物質の力学を研究する時に, 張力に対する応答を知りたくなることがある. 例えばゴムバンドは張力に対してバネとは違う応答を示す. いろいろな長さに伸ばした時に, 物質がどれだけ力を生みだしたかを記録することによって, 我々はそれぞれの物質に対して\href{http://en.wikipedia.org/wiki/Stress–strain_curve}{応力-ひずみ曲線(force-extension curve)}を関連付けることができる. 一度実験を行なう方法を固定したのならば, 物質の応力-ひずみ曲線の探索は理想的には物質の集合から曲線の集合への函数の構築となるであろう.
\footnote{実際には, 同じ物質の異なったサンプル, 例えば異なった大きさのサンプルやあるいは異なった温度でのサンプルは異なった応力-ひずみ曲線を持つだろう. もし我々がこれをその終域が曲線である本物の函数であると見なしたいのであれば, それは物質のサンプルの集合といったものを始域としているべきである.}

\end{application}

\begin{exercise}

%Here is a simplified account of how the \href{http://en.wikipedia.org/wiki/Retina}{\text brain receives light}. The eye contains about 100 million photoreceptor (PR) cells. Each connects to a retinal ganglion (RG) cell. No PR cell connects to two different RG cells, but usually many PR cells can attach to a single RG cell. 

以下の記述は, どのようにして\href{http://en.wikipedia.org/wiki/Retina}{脳が光を感じとるか}を単純化したものである. 眼には約1億個の光受容細胞(photoreceptor cell)が含まれている. それぞれの細胞は網膜神経節細胞(retinal ganglion cell)に繋がっている. 二つの異なった網膜神経節細胞に繋がっている光受容細胞は存在しないが, 通常は複数の光受容細胞が一つの網膜神経節細胞に接続することができる.

%Let $PR$ denote the set of photoreceptor cells and let $RG$ denote the set of retinal ganglion cells. 
%\sexc According to the above account, does the connection pattern constitute a function $RG\to PR$, a function $PR\to RG$ or neither one? 
%\next Would you guess that the connection pattern that exists between other areas of the brain are ``function-like"?
%\endsexc

$\mathit{PR}$は光受容細胞の集合, $\textit{RG}$は網膜神経節細胞の集合を示すものとする.
\sexc 上記の記述に従えば, 結合パターンは函数$\textit{RG}\to \textit{PR}$, あるいは函数$\textit{PR}\to \textit{RG}$を構成するだろうか? もしくはどちらも函数ではないのだろうか?
\next 脳の他の領域の間に存在する結合パターンが``函数のように''ふるまうかを推測してみよう.
\endsexc
\end{exercise}

\begin{example}\label{ex:subset as function}\index{subset!as function}

%Suppose that $X$ is a set and $X'\ss X$ is a subset. Then we can consider the function $X'\to X$ given by sending every element of $X'$ to ``itself" as an element of $X$. For example if $X=\{a,b,c,d,e,f\}$ and $X'=\{b,d,e\}$ then $X'\ss X$ and we turn that into the function $X'\to X$ given by $b\mapsto b, d\mapsto d, e\mapsto e$.
%\footnote{This kind of arrow,\;\;$\mapsto$\;\;, is read aloud as ``maps to". A function $f\taking X\to Y$ means a rule for assigning to each element $x\in X$ an element $f(x)\in Y$. We say that ``$x$ maps to $f(x)$" and write $x\mapsto f(x)$.}\index{a symbol!$\mapsto$}

$X$が集合で$X'\ss X$がその部分集合であると仮定する. そのときには$X'$の全ての要素を$X$における``要素それ自身''に対応付ける函数$X'\to X$を考えることができる. 例えば, $X=\{a,b,c,d,e,f\}$として$X'=\{b,d,e\}$としたならば$X'\ss X$
であり, そこから函数$X'\to X$を$b\mapsto b, d\mapsto d, e\mapsto e$とすることによって得ることができる.
\footnote{この\;\;$\mapsto$\;\;という種類の矢印は, ``対応付ける(maps to)''と音読される. 函数$f\taking X\to Y$はそれぞれの要素$x\in X$に要素$f(x)\in Y$を割りあてる法則を意味している. 我々はこれを``$x$を$f(x)$に対応付ける''と読み, $x\mapsto f(x)$と書く.}\index{a symbol!$\mapsto$}

%As a matter of notation, we may sometimes say something like the following: Let $X$ be a set and let $i\taking X'\ss X$ be a subset. Here we are making clear that $X'$ is a subset of $X$, but that $i$ is the name of the associated function.

記法として, 我々は時々``$X$を集合, $i\taking X'\ss X$を部分集合とする''というようなことを言うだろう. ここで$X'$は$X$の部分集合であり, $i$はそれに関連付けられた函数であることを, ここで明確に述べておく.

\end{example}

\begin{exercise}
%Let $f\taking\NN\to\NN$ be the function that sends every natural number to its square, e.g. $f(6)=36$. First fill in the blanks below, then answer a question.
%\sexc $2\mapsto\ul{\hspace{.5in}}$
%\next $0\mapsto\ul{\hspace{.5in}}$
%\next $-2\mapsto\ul{\hspace{.5in}}$
%\next $5\mapsto\ul{\hspace{.5in}}$
%\next Consider the symbol $\to$ and the symbol $\mapsto$. What is the difference between how these two symbols are used in this book?
%\endsexc
$f\taking\NN\to\NN$を, 全ての自然数をその二乗に対応付ける函数とする, e.g. $f(6)=36$. 最初に以下の空白を埋め, それから問題に答えよ.
\sexc $2\mapsto\ul{\hspace{.5in}}$
\next $0\mapsto\ul{\hspace{.5in}}$
\next $-2\mapsto\ul{\hspace{.5in}}$
\next $5\mapsto\ul{\hspace{.5in}}$
\next 記号$\to$と記号$\mapsto$について考えよ. この本の中において, 二つの記号の使用され方の間にある違いはなんだろうか?
\endsexc
\end{exercise}

%Given a function $f\taking X\to Y$, the elements of $Y$ that have at least one arrow pointing to them are said to be {\em in the image} of $f$; that is we have \index{image}
%\begin{align}\label{dia:image}
%\im(f):=\{y\in Y\| \exists x\in X \tn{ such that } f(x)=y\}.
%\end{align} 

函数$f\taking X\to Y$が与えられたとき, $Y$の要素でそれを指す矢印が一つ以上あるものの集合を$f$の\emph{像(image)}と呼ぶ. これは\index{image}
\begin{align}\label{dia:image}
\im(f):=\{y\in Y\| \exists x\in X \tn{ such that } f(x)=y\}
\end{align}
と書ける.

\begin{exercise}
%If $f\taking X\to Y$ is depicted by (\ref{dia:setmap}) above, write its image, $\im(f)$ as a set.
$f\taking X\to Y$が\eqref{dia:setmap}で描写されるとき, 像$\im(f)$を集合として書け.
\end{exercise}

%Given a function $f\taking X\to Y$ and a function $g\taking Y\to Z$, where the codomain of $f$ is the same set as the domain of $g$ (namely $Y$), we say that $f$ and $g$ are composable 
%$$X\Too{f}Y\Too{g}Z.$$ The {\em composition of $f$ and $g$}\label{function composition}\index{function!composition}\index{composition!of functions}\index{a symbol!$\circ$} is denoted by $g\circ f\taking X\to Z$. 

函数$f\taking X\to Y$と函数$g\taking Y\to Z$が与えられ, $f$の終域が$g$の始域と同じ集合($Y$)であるとき, 我々は$f$と$g$は合成可能(composable)であるという.
$$X\Too{f}Y\Too{g}Z.$$ \emph{$f$と$g$の合成(composition of $f$ and $g$)}\label{function composition}\index{function!composition}\index{composition!of functions}\index{a symbol!$\circ$}は$g\circ f\taking X\to Z$と表記される. 

\begin{figure}[h]
\begin{center}
\includegraphics[height=2in]{composition}
\end{center}
\caption{Functions $f\taking X\to Y$ and $g\taking Y\to Z$ compose to a function $g\circ f\taking X\to Z$; just follow the arrows.}
\end{figure}

%Let $X$ and $Y$ be sets. We write $\Hom_\Set(X,Y)$\index{a symbol!$\Hom_\Set$} to denote the set of functions $X\to Y$.
%\footnote{The strange notation $\Hom_\Set(-,-)$ will make more sense later, when it is seen as part of a bigger story.} 
%Note that two functions $f,g\taking X\to Y$ are equal\index{function!equality of} if and only if for every element $x\in X$ we have $f(x)=g(x)$. 

$X$と$Y$を集合とする. 我々は函数$X\to Y$の集合を$\Hom_\Set(X,Y)$\index{a symbol!$\Hom_\Set$}と書く.
\footnote{後程, より大きな筋書きの一部として見たときに, $\Hom_\Set(-,-)$という奇妙な記法がさらなる意味を持つことになる.}
二つの函数$f,g\taking X\to Y$が等しい\index{function!equality of}とは, 全ての要素$x\in X$に対して$f(x)=g(x)$が得られるときであり, かつその時に限ることに注意しよう.

\begin{exercise}
%Let $A=\{1,2,3,4,5\}$ and $B=\{x,y\}.$ 
%\sexc How many elements does $\Hom_\Set(A,B)$ have? 
%\next How many elements does $\Hom_\Set(B,A)$ have?
%\endsexc
$A=\{1,2,3,4,5\}$で$B=\{x,y\}$であるとする.
\sexc $\Hom_\Set(A,B)$の要素はいくつあるか?
\next $\Hom_\Set(B,A)$の要素はいくつあるか?
\endsexc
\end{exercise}

\begin{exercise}~
%\sexc Find a set $A$ such that for all sets $X$ there is exactly one element in $\Hom_\Set(X,A)$. Hint: draw a picture of proposed $A$'s and $X$'s.
%\next Find a set $B$ such that for all sets $X$ there is exactly one element in $\Hom_\Set(B,X)$.
%\endsexc 
\sexc 全ての集合$X$に対して$\Hom_\Set(X,A)$がただ一つだけ要素を持つ, そのような集合$A$を求めよ. ヒント: 考えた $A$と$X$の要素に関しての絵を書け.
\next 全ての集合$X$に対して$\Hom_\Set(B, X)$がただ一つだけ要素を持つ, そのような集合$B$を求めよ.
\endsexc
\end{exercise}

%For any set $X$, we define the {\em identity function on $X$}\index{function!identity}, denoted $\id_X\taking X\to X$, to be the function such that for all $x\in X$ we have $\id_X(x)=x$.\index{a symbol!$\id_X$}

任意の集合$X$に対して, \emph{$X$上の恒等写像(identity function on $X$)}\index{function!identity}を, 全ての$x\in X$に対して$\id_X(x)=x$.\index{a symbol!$\id_X$}である函数として定義し, $\id_X\taking X\to X$で示す.

%\begin{definition}[Isomorphism]\label{def:iso in set}
\begin{definition}[同型写像(Isomorphism)]\label{def:iso in set}

%Let $X$ and $Y$ be sets. A function $f\taking X\to Y$ is called an {\em isomorphism}\index{function!isomorphism}\index{isomorphism!of sets}, denoted $f\taking X\To{\iso}Y$, if there exists a function $g\taking Y\to X$ such that $g\circ f=\id_X$ and $f\circ g=\id_Y$. We also say that $f$ is {\em invertible} and we say that $g$ is {\em the inverse}\index{function!inverse} of $f$. If there exists an isomorphism $X\To\iso Y$ we say that $X$ and $Y$ are {\em isomorphic} sets and may write $X\iso Y$. \index{a symbol!$\iso$}

$X$と$Y$を集合とする. 函数$f\taking X\to Y$は, $g\circ f=\id_X$かつ$f\circ g=\id_Y$である函数$g\taking Y\to X$が存在するとき, \emph{同型写像(isomorphism)}\index{function!isomorphism}\index{isomorphism!of sets}であるといい, $f\taking X\To{\iso}Y$と書いて示す. またこのとき$f$は\emph{可逆(invertible)}であるといい, $g$は$f$の\emph{逆函数(the inverse)}\index{function!inverse}であるという. 同型$X\To\iso Y$が存在するとき, $X$と$Y$は\emph{同型(isomorphic)}な集合であるといい, $X\iso Y$と書くこともある. \index{a symbol!$\iso$}

\end{definition}

\begin{example}

%If $X$ and $Y$ are sets and $f\taking X\to Y$ is an isomorphism then the analogue of Diagram \ref{dia:setmap} will look like a perfect matching, more often called a {\em one-to-one correspondence}\index{one-to-one correspondence}\index{correspondence!one-to-one}. That means that no two arrows will hit the same element of $Y$, and every element of $Y$ will be in the image. For example, the following depicts an isomorphism $X\To{\iso}Y$.

$X$と$Y$が集合で$f\taking X\to Y$が同型写像であるならば, 図式\ref{dia:setmap}の類似物は完全対応となり, よりしばしば\emph{一対一対応(one-to-one correspondence)}\index{one-to-one correspondence}\index{correspondence!one-to-one}と呼ばれる. これが意味するところは, どの二つの矢印も$Y$の同じ要素に当たることがなく, $Y$の全ての要素がその像の中に含まれるということである. 例えば, 次の図は同型写像$X\To{\iso}Y$を描いている.

\begin{align}\label{dia:setmapiso}
\parbox{2.3in}{\includegraphics[height=2in]{SetMapIso}}
\end{align}

\end{example}

\begin{application}\label{app:DNA RNA}\index{RNA transcription}

%There is an isomorphism between the set $\tn{Nuc}_\tn{DNA}$ of \href{http://en.wikipedia.org/wiki/Nucleotides}{\text nucleotides} found in DNA and the set $\tn{Nuc}_\tn{RNA}$ of nucleotides found in RNA. Indeed both sets have four elements, so there are 24 different isomorphisms. But only one is useful. Before we say which one it is, let us say there is also an isomorphism $\tn{Nuc}_\tn{DNA}\iso\{A,C,G,T\}$ and an isomorphism $\tn{Nuc}_\tn{RNA}\iso\{A,C,G,U\}$, and we will use the letters as abbreviations for the nucleotides. 

DNAの中にある\href{http://en.wikipedia.org/wiki/Nucleotides}{ヌクレオチド(nucleotides)}の集合$\tn{Nuc}_\tn{DNA}$と, RNAの中にあるヌクレオチドの集合$\tn{Nuc}_\tn{RNA}$の間には, 同型写像が存在している. 実際にはどちらの集合も4つの要素を持つから, 24個の異なった同型写像が存在しうる. しかしその中で一つだけが有用である. どれがその一つかを述べる前に, また別の同型写像$\tn{Nuc}_\tn{DNA}\iso\{A,C,G,T\}$および同型写像$\tn{Nuc}_\tn{RNA}\iso\{A,C,G,U\}$が存在するということを述べておこう. 今後はこれらの文字をヌクレオチドの略記として用いることにする.

%The convenient isomorphism $\tn{Nuc}_\tn{DNA}\To{\iso}\tn{Nuc}_\tn{RNA}$ is that given by RNA transcription; it sends 
%$$A\mapsto U, C\mapsto G, G\mapsto C, T\mapsto A.$$ 
%(See also Application \ref{app:polymerase}.) There is also an isomorphism $\tn{Nuc}_\tn{DNA}\To{\iso}\tn{Nuc}_\tn{DNA}$ (the matching in the double-helix) given by 
%$$A\mapsto T, C\mapsto G, G\mapsto C, T\mapsto A.$$

有用な同型写像$\tn{Nuc}_\tn{DNA}\To{\iso}\tn{Nuc}_\tn{RNA}$はRNA転写によって与えられる. これは
$$A\mapsto U, C\mapsto G, G\mapsto C, T\mapsto A.$$ 
と移す. (Application \ref{app:polymerase}も見よ.) また同型写像$\tn{Nuc}_\tn{DNA}\To{\iso}\tn{Nuc}_\tn{DNA}$ (二重螺旋の対合)が存在し, 
$$A\mapsto T, C\mapsto G, G\mapsto C, T\mapsto A.$$
で与えられる.

%Protein production can be modeled as a function from the set of 3-nucleotide sequences to the set of eukaryotic amino acids. However, it cannot be an isomorphism because there are $4^3=64$ triplets of RNA nucleotides, but only 21 eukaryotic amino acids. 

たんぱく質の合成は, 3要素のヌクレオチド列から真核生物の持つアミノ酸の集合への函数として模型化される. しかしこれは同型写像ではない. なぜならばRNAの三つ組は$4^3=64$種類存在しているが, 真核生物の持つアミノ酸は21種類しかないからだ.

\end{application}

\begin{exercise}
%Let $n\in\NN$ be a natural number and let $X$ be a set with exactly $n$ elements. 
%\sexc How many isomorphisms are there from $X$ to itself? 
%\next Does your formula from part a.) hold when $n=0$?
%\endsexc

$n\in\NN$を自然数とし, $X$を正確に$n$個の要素を持つ集合とする.
\sexc $X$から$X$自身への同型写像はいくつあるか? 
\next a.) での式は$n=0$の時にも成立するか?
\endsexc
\end{exercise}

\begin{lemma}\label{lemma:isomorphic ER in Set}

%The following facts hold about isomorphism.
%\begin{enumerate}
%\item Any set $A$ is isomorphic to itself; i.e. there exists an isomorphism $A\To{\iso} A$.
%\item For any sets $A$ and $B$, if $A$ is isomorphic to $B$ then $B$ is isomorphic to $A$.
%\item For any sets $A, B,$ and $C$, if $A$ is isomorphic to $B$ and $B$ is isomorphic to $C$ then $A$ is isomorphic to $C$.
%\end{enumerate}

同型写像に関して以下が成立する.
\begin{enumerate}
\item 任意の集合$A$はそれ自身に対して同型である. i.e. 同型写像$A\To{\iso} A$が存在する.
\item 任意の集合$A$と$B$に対して, $A$が$B$に同型ならば$B$は$A$に同型である.
\item 任意の集合$A, B,$および$C$に対して, もし$A$が$B$に同型で$B$が$C$に同型ならば, $A$は$C$に同型である.
\end{enumerate}

\end{lemma}

\begin{proof}

%\begin{enumerate}
%\item The identity function $\id_A\taking A\to A$ is invertible; its inverse is $\id_A$ because $\id_A\circ\id_A=\id_A$.
%\item If $f\taking A\to B$ is invertible with inverse $g\taking B\to A$ then $g$ is an isomorphism with inverse $f$.
%\item If $f\taking A\to B$ and $f'\taking B\to C$ are each invertible with inverses $g\taking B\to A$ and $g'\taking C\to B$ then the following calculations show that $f'\circ f$ is invertible with inverse $g\circ g'$: 
%\begin{align*}
%(f'\circ f)\circ(g\circ g')=f'\circ(f\circ g)\circ g'=f'\circ\id_B\circ g'=f'\circ g'=\id_C\\
%(g\circ g')\circ(f'\circ f)=g\circ(g'\circ f')\circ f=g\circ\id_B\circ f=g\circ f=\id_A
%\end{align*}
%\end{enumerate}

\begin{enumerate}
\item 恒等写像$\id_A\taking A\to A$は可逆である. $\id_A\circ\id_A=\id_A$であるから逆函数は$\id_A$である.
\item もし$f\taking A\to B$が可逆でその逆函数が$g\taking B\to A$であるならば, $g$は同型写像でその逆函数は$f$である.
\item もし$f\taking A\to B$および$f'\taking B\to C$が可逆でそれぞれの逆函数が$g\taking B\to A$および$g'\taking C\to B$であるならば, 以下の計算によって$f'\circ f$が可逆でありその逆函数が$g\circ g'$であることが示される.
\begin{align*}
(f'\circ f)\circ(g\circ g')=f'\circ(f\circ g)\circ g'=f'\circ\id_B\circ g'=f'\circ g'=\id_C\\
(g\circ g')\circ(f'\circ f)=g\circ(g'\circ f')\circ f=g\circ\id_B\circ f=g\circ f=\id_A
\end{align*}
\end{enumerate}

\end{proof}

\begin{exercise}\label{exc:functions are not iso invariant}
%Let $A$ and $B$ be the sets drawn below:
%$$
%\parbox{1.1in}{\boxtitle{A:=}\fbox{\xymatrix@=1pt{\\&\LMO{\;a\;}&&&\LMO{\;\;\;7\;\;}&\\\\\\&&&\LMO{Q}\\&}}}
%\hspace{.8in}
%\parbox{1.2in}{\boxtitle{B:=}\fbox{\xymatrix@=1pt{&&&\LMO{r8}&&\\\\\\\\&\LMO{``Bob"}\\&&\LMO{\clubsuit}}}}
%$$
%Note that the sets $A$ and $B$ are isomorphic. Supposing that $f\taking B\to\{1,2,3,4,5\}$ sends ``Bob" to $1$, sends $\clubsuit$ to $3$, and sends $r8$ to $4$, is there a canonical function $A\to\{1,2,3,4,5\}$ corresponding to $f$?
%\footnote{Canonical means something like ``best choice", a choice that stands out as the only reasonable one.}\index{canonical}
$A$と$B$が集合であり以下のように描写されるとする.
$$
\parbox{1.1in}{\boxtitle{A:=}\fbox{\xymatrix@=1pt{\\&\LMO{\;a\;}&&&\LMO{\;\;\;7\;\;}&\\\\\\&&&\LMO{Q}\\&}}}
\hspace{.8in}
\parbox{1.2in}{\boxtitle{B:=}\fbox{\xymatrix@=1pt{&&&\LMO{r8}&&\\\\\\\\&\LMO{``Bob"}\\&&\LMO{\clubsuit}}}}
$$
$A$と$B$が同型であることに注意しよう. $f\taking B\to\{1,2,3,4,5\}$が, ``Bob''を$1$に移し, $\clubsuit$を$3$に移し, $r8$を$4$に移すと仮定したとき, $f$に対応する標準函数(canonical function)$A\to\{1,2,3,4,5\}$は存在するだろうか?
\footnote{標準(canonical)とは, 唯一の妥当なものとして際だっているような選びかたである, すなわち``もっともよい選択''であるというような意味である.}\index{canonical}
\end{exercise}

\begin{exercise}\label{exc:generator for set}
%Find a set $A$ such that for any set $X$ there is a isomorphism of sets $$X\iso\Hom_\Set(A,X).$$ Hint: draw a picture of proposed $A$'s and $X$'s.
任意の集合$X$に対して同型写像$$X\iso\Hom_\Set(A,X).$$が存在する集合$A$を求めよ. ヒント: 考えた$A$の要素と$X$の要素についての絵を描け.
\end{exercise}

%For any natural number $n\in\NN$, define a set 
%\begin{align}\label{dia:underline n}\index{a symbol!$\ul{n}$}
%\ul{n}:=\{1,2,3,\ldots,n\}.
%\end{align}
%So, in particular, $\ul{2}=\{1,2\}, \ul{1}=\{1\}$, and $\ul{0}=\emptyset$. 

任意の自然数$n\in\NN$に対して, 集合 
\begin{align}\label{dia:underline n}\index{a symbol!$\ul{n}$}
\ul{n}:=\{1,2,3,\ldots,n\}.
\end{align}
を定義する. よって, 特に$\ul{2}=\{1,2\}, \ul{1}=\{1\}$および$\ul{0}=\emptyset$である.

%Let $A$ be any set. A function $f\taking\ul{n}\to A$ can be written as a sequence $$f=(f(1),f(2),\ldots,f(n)).$$

$A$を任意の集合とする. 函数$f\taking\ul{n}\to A$は列$$f=(f(1),f(2),\ldots,f(n)).$$として書くことができる.

\begin{exercise}\label{exc:sequence}~
%\sexc Let $A=\{a,b,c,d\}$. If $f\taking\ul{10}\to A$ is given by $(a,b,c,c,b,a,d,d,a,b)$, what is $f(4)$?
%\next Let $s\taking\ul{7}\to\NN$ be given by $s(i)=i^2$. Write $s$ out as a sequence.
%\endsexc
\sexc $A=\{a,b,c,d\}$とする. もし$f\taking\ul{10}\to A$が$(a,b,c,c,b,a,d,d,a,b)$で与えられたならば, $f(4)$はどうなるか? 
\next $s\taking\ul{7}\to\NN$が$s(i)=i^2$で与えられるとする. $s$を列として書き下せ.
\endsexc
\end{exercise}

%\begin{definition}[Cardinality of finite sets]\label{def:cardinality}
\begin{definition}[有限集合の濃度(Cardinality)]\label{def:cardinality}

%Let $A$ be a set and $n\in\NN$ a natural number. We say that $A$ is {\em has cardinality $n$}\index{cardinality}, denoted $$|A|=n,$$ if there exists an isomorphism of sets $A\iso\ul{n}$. If there exists some $n\in\NN$ such that $A$ has cardinality $n$ then we say that $A$ is {\em finite}. Otherwise, we say that $A$ is {\em infinite} and write $|A|\geq\infty$.

$A$を集合とし$n\in\NN$を自然数とする. 集合間の同型写像$A\iso\ul{n}$が存在するならば, $A$の\emph{濃度(cardinality)は$n$}\index{cardinality}であると言い, $$|A|=n,$$と書く. もし$A$の濃度が$n$である$n\in\NN$が存在するならば$A$は\emph{有限である}と言う. そうでなければ, $A$は\emph{無限である}と言い, $|A|\geq\infty$と書く.

\end{definition}

\begin{exercise}~
%\sexc Let $A=\{5,6,7\}$. What is $|A|$? 
%\next What is $|\NN|$? 
%\next What is $|\{n\in\NN\|n\leq 5\}|$?
%\endsexc
\sexc $A=\{5,6,7\}$とする. $|A|$はどうなるか?
\next $|\NN|$はどうなるか? 
\next $|\{n\in\NN\|n\leq 5\}|$はどうなるか?
\endsexc
\end{exercise}

\begin{lemma}

%Let $A$ and $B$ be finite sets. If there is an isomorphism of sets $f\taking A\to B$ then the two sets have the same cardinality, $|A|=|B|$.

$A$と$B$を有限集合とする. もし集合間の同型写像$f\taking A\to B$が存在したならば, 二つの集合は同じ濃度を持つ, すなわち$|A|=|B|$である.

\end{lemma}

\begin{proof}

%Suppose $f\taking A\to B$ is an isomorphism. If there exists natural numbers $m,n\in\NN$ and isomorphisms $a\taking\ul{m}\To\iso A$ and $b\taking\ul{n}\To\iso B$ then $\ul{m}\To{a^\m1}A\To{f}B\To{b}\ul{n}$ is an isomorphism. One can prove by induction that the sets $\ul{m}$ and $\ul{n}$ are isomorphic if and only if $m=n$. 

$f\taking A\to B$が同型写像であると仮定する. もし自然数$m,n\in\NN$が存在して同型写像$a\taking\ul{m}\To\iso A$および$b\taking\ul{n}\To\iso B$が存在するならば, $\ul{m}\To{a^\m1}A\To{f}B\To{b}\ul{n}$も同型写像である. 帰納法により, 集合$\ul{m}$と$\ul{n}$が同型であるのは$m=n$である時かつその時に限られることが証明できる.

\end{proof}


%%%%%% Section %%%%%%

%\section{Commutative diagrams}\label{sec:comm diag}
\section{可換図式}\label{sec:comm diag}
%\addtocounter{subsection}{1}\setcounter{subsubsection}{0}

%At this point it is difficult to precisely define diagrams or commutative diagrams in general, but we can give the heuristic idea.
%\footnote{We will define commutative diagrams precisely in Section \ref{sec:diagrams in a category}.}
%Consider the following picture: 
%\begin{align}\label{dia:triangle}
%\xymatrix{A\ar[r]^f\ar[rd]_h&B\ar[d]^g\\&C}
%\end{align}
%We say this is a {\em diagram of sets}\index{diagram!in $\Set$} if each of $A,B,C$ is a set and each of $f,g,h$ is a function. We say this diagram {\em commutes}\index{commuting diagram}\index{diagam!commutes} if $g\circ f = h$. In this case we refer to it as a commutative triangle of sets.

この段階では, 図式(diagram)あるいは可換図式(commutative diagram)を正確に定義するのは一般には難しいが, 
heuristicな考えを与えることならばできる.
\footnote{我々は第\ref{sec:diagrams in a category}章で可換図式を正確に定義することになる.}
次の絵を考える.
\begin{align}\label{dia:triangle}
\xymatrix{A\ar[r]^f\ar[rd]_h&B\ar[d]^g\\&C}
\end{align}
$A,B,C$が集合で$f,g,h$が函数であるとき, 我々はこれを\emph{集合の図式(diagram of sets)}\index{diagram!in $\Set$}と言う. もし$g\circ f = h$であるならば, この図式は\emph{可換である(commutes)}\index{commuting diagram}\index{diagam!commutes}と言う. 可換な場合の上図を, 我々は集合の可換三角形として参照する.

\begin{application}

%\href{http://en.wikipedia.org/wiki/Central_dogma_of_molecular_biology}{\text The central dogma of molecular biology} is that ``DNA codes for RNA codes for protein". That is, there is a function from DNA triplets to RNA triplets and a function from RNA triplets to amino acids. But sometimes we just want to discuss the translation from DNA to amino acids, and this is the composite of the other two. The commutative diagram is a picture of this fact.

\href{http://en.wikipedia.org/wiki/Central_dogma_of_molecular_biology}{分子生物学のセントラルドグマ}は``DNAはRNAを符号化しRNAはたんぱく質を符号化する''. これは, DNAの三つ組からRNAの三つ組への函数と, RNAの三つ組からアミノ酸への函数が存在しているということである. しかし, DNAからアミノ酸への変換だけを議論したいこともしばしばある. そしてこれは二つの函数の合成である. 可換図式はこの事実に関する絵である.

\end{application}

%Consider the following picture:
%$$\xymatrix{A\ar[r]^f\ar[d]_h&B\ar[d]^g\\C\ar[r]_i&D}$$
%We say this is a {\em diagram of sets} if each of $A,B,C,D$ is a set and each of $f,g,h,i$ is a function. We say this diagram {\em commutes} if $g\circ f=i\circ h$. In this case we refer to it as a commutative square of sets.

次の絵を考える.
$$\xymatrix{A\ar[r]^f\ar[d]_h&B\ar[d]^g\\C\ar[r]_i&D}$$
$A,B,C,D$が集合であり$f,g,h,i$が函数であるとき, 我々はこれを\emph{集合の図式}と言う. もし$g\circ f=i\circ h$であるならば, この図式は\emph{可換である}と言う. 可換な場合の上図を, 我々は集合の可換四角形として参照する.

\begin{application}

%Given a physical system $S$, there may be two mathematical approaches $f\taking S\to A$ and $g\taking S\to B$ that can be applied to it. Either of those results in a prediction of the same sort, $f'\taking A\to P$ and $g'\taking B\to P$. For example, in \href{http://en.wikipedia.org/wiki/Hamiltonian_mechanics#As_a_reformulation_of_Lagrangian_mechanics}{\text mechanics} we can use either Lagrangian approach or the Hamiltonian approach to predict future states. To say that the diagram 
%$$
%\xymatrix{S\ar[r]\ar[d]&A\ar[d]\\B\ar[r]&P}
%$$
%commutes would say that these approaches give the same result.

物理系$S$が与えられたとき, 適用できる数学的な取り組み方に$f\taking S\to A$と$g\taking S\to B$という二種類があるかもしれない. そのどちらからも, 同じ種類の予言が得られる. すなわち$f'\taking A\to P$と$g'\taking B\to P$. 例えば
\href{http://en.wikipedia.org/wiki/Hamiltonian_mechanics#As_a_reformulation_of_Lagrangian_mechanics}{力学}において, 未来の状態を予言するために我々はLagrangian形式とHamiltonian形式のどちらかを使うことができる. 図式
$$
\xymatrix{S\ar[r]\ar[d]&A\ar[d]\\B\ar[r]&P}
$$
は可換であると言えるならば, どちらの方法をとっても同じ結果が得られることが言えるだろう.

\end{application}

%And so on. Note that diagram (\ref{dia:triangle}) is considered to be the same diagram as each of the following:
%$$
%\xymatrix{A\ar[r]^f\ar[d]_h&B\ar[dl]^g\\C}\hspace{.8in}
%\xymatrix{A\ar[r]^f\ar@/_1pc/[rr]_h&B\ar[r]^g&C}\hspace{.8in}
%\xymatrix{B\ar[rd]^g\\&C\\A\ar[ru]_h\ar[uu]^f}$$

以下も同様である. 図式\eqref{dia:triangle}は次のそれぞれと同じ図式であることに注意しよう.
$$
\xymatrix{A\ar[r]^f\ar[d]_h&B\ar[dl]^g\\C}\hspace{.8in}
\xymatrix{A\ar[r]^f\ar@/_1pc/[rr]_h&B\ar[r]^g&C}\hspace{.8in}
\xymatrix{B\ar[rd]^g\\&C\\A\ar[ru]_h\ar[uu]^f}$$


\section{Olog}\label{sec:ologs}\index{olog}

%In this course we will ground the mathematical ideas in applications whenever possible. To that end we introduce ologs, which will serve as a bridge between mathematics and various conceptual landscapes. The following material is taken from \cite{SK}, an introduction to ologs.
%\begin{align}\label{dia:arginine}\fbox{\xymatrixnocompile{\obox{D}{1in}{\rr an amino acid found in dairy}\LAL{dr}{is}&\obox{A}{.5in}{arginine}\ar@{}[dl]|(.3){\checkmark}\ar@{}[dr]|(.3){\checkmark}\LA{r}{has}\LAL{l}{is}\LA{d}{is}&\obox{E}{.9in}{\rr an electrically-charged side chain}\LA{d}{is}\\&\obox{X}{.9in}{an amino acid}\LAL{dl}{has}\LA{dr}{has}\LA{r}{has}&\smbox{R}{a side chain}\\\mebox{N}{an amine group}&&\mebox{C}{a carboxylic acid}}}\end{align}  

この教程では, 我々は数学的なアイデアを可能な限り実際の応用に基づかせようとしている. その目的に沿って, 我々はologを導入する. ologは様々な数学と概念の風景との間への架け橋となる. 以下の題材はologの入門書である\cite{SK}から取っている.
\footnote{訳注: ologは英語で記述されることを前提としているようである. 訳においては, 日本語でologを記述することは試みておらず, また日本語用に文章を置き換えたりもしていない.}
\begin{align}\label{dia:arginine}\fbox{\xymatrixnocompile{\obox{D}{1in}{\rr an amino acid found in dairy}\LAL{dr}{is}&\obox{A}{.5in}{arginine}\ar@{}[dl]|(.3){\checkmark}\ar@{}[dr]|(.3){\checkmark}\LA{r}{has}\LAL{l}{is}\LA{d}{is}&\obox{E}{.9in}{\rr an electrically-charged side chain}\LA{d}{is}\\&\obox{X}{.9in}{an amino acid}\LAL{dl}{has}\LA{dr}{has}\LA{r}{has}&\smbox{R}{a side chain}\\\mebox{N}{an amine group}&&\mebox{C}{a carboxylic acid}}}\end{align}  


%\cite{SGWB}. 
%\newpage
%\newgeometry{left=.7in,right=.4in,top=1.4in,bottom=1.4in}
%\begin{figure}
%\includegraphics[height=7in]{olog--ProteinSocial}
%\caption{This olog, taken from \cite{SGWB}, describes both a particular kind of social network and a particular kind of protein material in terms of the same arrangement of 23 sets and 44 functions. The elements of these sets should be known to the olog's author, but may not be interpretable by an arbitrary reader of the olog; for example, {\bf V}=\fakebox{a real number} should be interpretable by most readers, but it is unlikely that {\bf U}=\fakebox{a building block} is interpretable without reading that paper.}
%\label{fig:protein social olog}
%\end{figure}
%\newgeometry{left=1.6in,right=1.6in,top=1.4in,bottom=1.4in}

%%%% Subsection %%%%

%\subsection{Types}\index{olog!types}
\subsection{型}\index{olog!types}

%A type is an abstract concept, a distinction the author has made.  We represent each type as a box containing a {\em singular indefinite noun phrase.}   Each of the following four boxes is a type: \begin{align}\label{dia:types}\xymatrixnocompile{\fbox{a man}&\fbox{an automobile}\\\obox{}{1.5in}{a pair $(a,w)$, where $w$ is a woman and $a$ is an automobile}&\obox{}{1.5in}{a pair $(a,w)$ where $w$ is a woman and $a$ is a blue automobile owned by $w$}}\end{align}

型(type)は抽象的な概念であり, ologの作者が行なった区分を示す. 我々はそれぞれの型を\emph{不定な(定冠詞をともなわない)単数の名詞句}を含む箱で表現する. 次の四つの箱は型である. \begin{align}\label{dia:types}\xymatrixnocompile{\fbox{a man}&\fbox{an automobile}\\\obox{}{1.5in}{a pair $(a,w)$, where $w$ is a woman and $a$ is an automobile}&\obox{}{1.5in}{a pair $(a,w)$ where $w$ is a woman and $a$ is a blue automobile owned by $w$}}\end{align}

Each of the four boxes in (\ref{dia:types}) represents a type of thing, a whole class of things, and the label on that box is what one should call {\em each example} of that class.  Thus \fakebox{a man} does not represent a single man, but the set of men, each example of which is called ``a man".  Similarly, the bottom right box represents an abstract type of thing, which probably has more than a million examples, but the label on the box indicates the common name for each such example.  

Typographical problems emerge when writing a text-box in a line of text, e.g. the text-box \fbox{a man} seems out of place here, and the more in-line text-boxes there are, the worse it gets.  To remedy this, I will denote types which occur in a line of text with corner-symbols; e.g. I will write \fakebox{a man} instead of \fbox{a man}.

%% Subsubsection %%

\subsubsection{Types with compound structures}

Many types have compound structures; i.e. they are composed of smaller units.  Examples include \begin{align}\label{dia:compound}\xymatrixnocompile{\obox{}{.7in}{\rr a man and a woman}&\obox{}{1.3in}{\rr a food portion $f$ and a child $c$ such that $c$ ate all of $f$}&\labox{}{a triple $(p,a,j)$ where $p$ is a paper, $a$ is an author of $p$, and $j$ is a journal in which $p$ was published}}\end{align}  It is good practice to declare the variables in a ``compound type", as I did in the last two cases of (\ref{dia:compound}).  In other words, it is preferable to replace the first box above with something like $$\obox{}{.8in}{a man $m$ and a woman $w$}\hsp\tn{or}\hsp\obox{}{1.1in}{\rr a pair $(m,w)$ where $m$ is a man and $w$ is a woman}$$ so that the variables $(m,w)$ are clear.

\begin{rules}\label{rules:types}\index{olog!rules}

A type is presented as a text box.  The text in that box should 
\begin{enumerate}[(i)]
\item begin with the word ``a" or ``an";
\item refer to a distinction made and recognizable by the olog's author;
\item refer to a distinction for which instances can be documented;
\item declare all variables in a compound structure. 
\end{enumerate}

\end{rules}

The first, second, and third rules ensure that the class of things represented by each box appears to the author as a well-defined set.  The fourth rule encourages good ``readability" of arrows, as will be discussed next in Section \ref{sec:aspects}.  

I will not always follow the rules of good practice throughout this document.  I think of these rules being followed ``in the background" but that I have ``nicknamed" various boxes.  So \fakebox{Steve} may stand as a nickname for \fakebox{a thing classified as Steve} and \fakebox{arginine} as a nickname for \fakebox{a molecule of arginine}. However, when pressed, one should always be able to rename each type according to the rules of good practice.

%%%% Subsection %%%%

\subsection{Aspects}\label{sec:aspects}\index{olog!aspects}

An aspect of a thing $x$ is a way of viewing it, a particular way in which $x$ can be regarded or measured.  For example, a woman can be regarded as a person; hence ``being a person" is an aspect of a woman.  A molecule has a molecular mass (say in daltons), so ``having a molecular mass" is an aspect of a molecule.  In other words, by {\em aspect} we simply mean a function. The domain $A$ of the function $f\taking A\to B$ is the thing we are measuring, and the codomain is the set of possible ``answers" or results of the measurement. 
\begin{align}\label{dia:aspect 1}\xymatrixnocompile{\fbox{a woman}\LA{r}{is}&\fbox{a person}}\end{align}\begin{align}\label{dia:aspect 2}\xymatrixnocompile{\fbox{a molecule}\LA{rr}{has as molecular mass (Da)}&\hspace{.7in}&\fbox{a positive real number}}\end{align}

So for the arrow in (\ref{dia:aspect 1}), the domain is the set of women (a set with perhaps 3 billion elements); the codomain is the set of persons (a set with perhaps 6 billion elements).   We can imagine drawing an arrow from each dot in the ``woman" set to a unique dot in the ``person" set, just as in (\ref{dia:setmap}).  No woman points to two different people, nor to zero people --- each woman is exactly one person --- so the rules for a function are satisfied.  Let us now concentrate briefly on the arrow in (\ref{dia:aspect 2}).  The domain is the set of molecules, the codomain is the set $\RR_{>0}$ of positive real numbers.  We can imagine drawing an arrow from each dot in the ``molecule" set to a single dot in the ``positive real number" set.  No molecule points to two different masses, nor can a molecule have no mass: each molecule has exactly one mass.  Note however that two different molecules can point to the same mass.

%% Subsubsection %%

\subsubsection{Invalid aspects}\label{sec:invalid aspect}\index{olog!invalid aspects}

I tried above to clarify what it is that makes an aspect ``valid", namely that it must be a ``functional relationship."  In this subsection I will show two arrows which on their face may appear to be aspects, but which on closer inspection are not functional (and hence are not valid as aspects).  
 
Consider the following two arrows:
\begin{align}\tag{\arabic{subsection}.\arabic{equation}*}\addtocounter{equation}{1}\label{dia:invalid 1}
\xymatrixnocompile{\fbox{a person}\LA{r}{has}&\fbox{a child}}
\end{align}
\vspace{-.13in}
\begin{align}\tag{\arabic{subsection}.\arabic{equation}*}\addtocounter{equation}{1}\label{dia:invalid 2}
\xymatrixnocompile{\fbox{a mechanical pencil}\LA{r}{uses}&\fbox{a piece of lead}}
\end{align}  
A person may have no children or may have more than one child, so the first arrow is invalid: it is not a function.  Similarly, if we drew an arrow from each mechanical pencil to each piece of lead it uses, it would not be a function.

\begin{warning}\label{warn:worldview}\index{a warning!different worldviews}

The author of an olog has a world-view, some fragment of which is captured in the olog.  When person A examines the olog of person B, person A may or may not ``agree with it."  For example, person B may have the following olog $$\fbox{\xymatrix{&\fbox{a marriage}\LA{dr}{ includes}\LAL{dl}{includes }\\\fbox{a man}&&\fbox{a woman}}}$$ which associates to each marriage a man and a woman.  Person A may take the position that some marriages involve two men or two women, and thus see B's olog as ``wrong."  Such disputes are not ``problems" with either A's olog or B's olog, they are discrepancies between world-views.  Hence, throughout this paper, a reader R may see a displayed olog and notice a discrepancy between R's world-view and my own, but R should not worry that this is a problem.  This is not to say that ologs need not follow rules, but instead that the rules are enforced to ensure that an olog is structurally sound, rather than that it ``correctly reflects reality," whatever that may mean.

Consider the aspect $\fakebox{an object}\Too{\tn{has}}\fakebox{a weight}$. At some point in history, this would have been considered a valid function. Now we know that the same object would have a different weight on the moon than it has on earth. Thus as world-views change, we often need to add more information to our olog. Even the validity of $\fakebox{an object on earth}\Too{\tn{has}}\fakebox{a weight}$ is questionable. However to build a model we need to choose a level of granularity and try to stay within it, or the whole model evaporates into the nothingness of truth!

\end{warning}

\begin{remark}

In keeping with Warning \ref{warn:worldview}, the arrows (\ref{dia:invalid 1}) and (\ref{dia:invalid 2}) may not be wrong but simply reflect that the author has a strange world-view or a strange vocabulary.  Maybe the author believes that every mechanical pencil uses exactly one piece of lead.  If this is so, then $\fakebox{a mechanical pencil}\To{\tn{uses}}\fakebox{a piece of lead}$ is indeed a valid aspect!   Similarly, suppose the author meant to say that each person {\em was once} a child, or that a person has an inner child.  Since every person has one and only one inner child (according to the author), the map $\fakebox{a person}\To{\tn{has as inner child}}\fakebox{a child}$ is a valid aspect.  We cannot fault the olog if the author has a view, but note that we have changed the name of the label to make his or her intention more explicit.

\end{remark}

%% Subsubsection %%

\subsubsection{Reading aspects and paths as English phrases}

Each arrow (aspect) $X\To{f} Y$ can be read by first reading the label on its source box (domain of definition) $X$, then the label on the arrow $f$, and finally the label on its target box (set of values) $Y$.  For example, the arrow \begin{align}\label{dia:first author}\fbox{\xymatrixnocompile{\smbox{}{a book}\LA{rrr}{has as first author}&&&\smbox{}{a person}}}\end{align} is read ``a book has as first author a person".  

\begin{remark}

Note that the map in (\ref{dia:first author}) is a valid aspect, but that a similarly benign-looking map $\fakebox{a book}\To{\tn{has as author}}\fakebox{a person}$ would not be valid, because it is not functional.  The authors of an olog must be vigilant about this type of mistake because it is easy to miss and it can corrupt the olog.

\end{remark}

Sometimes the label on an arrow can be shortened or dropped altogether if it is obvious from context.  We will discuss this more in Section \ref{sec:facts} but here is a common example from the way I write ologs. \begin{align}\label{dia:pair of integers}\fbox{\xymatrixnocompile{&\obox{A}{1.2in}{\rr a pair $(x,y)$ where $x$ and $y$ are integers}\ar[dl]_x\ar[dr]^y\\\smbox{B}{an integer}&&\smbox{B}{an integer}}}\end{align}  Neither arrow is readable by the protocol given above (e.g. ``a pair $(x,y)$ where $x$ and $y$ are integers $x$ an integer" is not an English sentence), and yet it is obvious what each map means.  For example, given $(8,11)$ in $A$, arrow $x$ would yield $8$ and arrow $y$ would yield $11$.  The label $x$ can be thought of as a nickname for the full name ``yields, via the value of $x$," and similarly for $y$.  I do not generally use the full name for fear that the olog would become cluttered with text.

One can also read paths through an olog by inserting the word ``which" after each intermediate box.
\footnote{If the intended elements of an intermediate box are humans, it is polite to use ``who" rather than ``which", and other such conventions may be upheld if one so desires.}
For example the following olog has two paths of length 3 (counting arrows in a chain): \small\begin{align}\label{olog:paths}\fbox{\xymatrixnocompile{\fbox{a child}\LA{r}{is}&\fbox{a person}\LA{rr}{has as parents}\LAL{dr}{has, as birthday}&&\obox{}{.8in}{\rr a pair $(w,m)$ where $w$ is a woman and $m$ is a man}\LA{r}{$w$}&\fbox{a woman}\\&&\fbox{a date}\LA{r}{includes}&\fbox{a year}}}\end{align}  \normalsize The top path is read ``a child is a person, who has as parents a pair $(w,m)$ where $w$ is a woman and $m$ is a man, which yields, via the value of $w$, a woman."  The reader should read and understand the content of the bottom path, which associates to every child a year.  


%% Subsubsection %%

\subsubsection{Converting non-functional relationships to aspects}\label{sec:relations}

There are many relationships that are not functional, and these cannot be considered aspects.  Often the word ``has" indicates a relationship --- sometimes it is functional as in $\fakebox{a person}\To{\tn{ has }}\fakebox{a stomach}$, and sometimes it is not, as in $\fakebox{a father}\To{\tn{has}}\fakebox{a child}$. Obviously, a father may have more than one child. This one is easily fixed by realizing that the arrow should go the other way: there is a function $\fakebox{a child}\To{\tn{has}}\fakebox{a father}$. 

What about $\fakebox{a person}\To{\tn{owns}}\fakebox{a car}$. Again, a person may own no cars or more than one car, but this time a car can be owned by more than one person too. A quick fix would be to replace it by $\fakebox{a person}\To{\tn{owns}}\fakebox{a set of cars}$.   This is ok, but the relationship between \fakebox{a car} and \fakebox{a set of cars} then becomes an issue to deal with later.  There is another way to indicate such ``non-functional" relationships. In this case it would look like this:
$$
\fbox{\xymatrix{&\obox{}{1.15in}{a pair $(p,c)$ where $p$ is a person, $c$ is a car, and $p$ owns $c$.}\ar[ddl]_p\ar[ddr]^c\\\\
\obox{}{.5in}{a person}&&\obox{}{.3in}{a car}}}
$$
This setup will ensure that everything is properly organized. In general, relationships can involve more than two types, and the general situation looks like this $$\fbox{\xymatrixnocompile{&&\fbox{$R$}\ar[ddll]\ar[ddl]\ar[ddr]\\\\\fbox{$A_1$}&\fbox{$A_2$}&\cdots&\fbox{$A_n$}}}$$  For example, $$\fbox{\xymatrixnocompile{&\labox{R}{a sequence $(p,a,j)$ where $p$ is a paper, $a$ is an author of $p$, and $j$ is a journal in which $p$ was published}\ar[ddl]_p\ar[dd]_a\ar[ddr]^j\\\\\smbox{A_1}{a paper}&\smbox{A_2}{an author}&\smbox{A_3}{a journal}}}$$ 

\begin{exercise}
On page \pageref{dia:invalid 1} we indicate a so-called invalid aspect, namely 
\begin{align}\tag{\ref{dia:invalid 1}}\xymatrixnocompile{\fbox{a person}\LA{r}{has}&\fbox{a child}}
\end{align}
Create a (valid) olog that captures the parent-child relationship; your olog should still have boxes \fakebox{a person} and \fakebox{a child} but may have an additional box.
\end{exercise}

\begin{rules}\label{rules:aspects}\index{olog!rules}

An aspect is presented as a labeled arrow, pointing from a source box to a target box.  The arrow text should

\begin{enumerate}[(i)]
\item begin with a verb;
\item yield an English sentence, when the source-box text followed by the arrow text followed by the target-box text is read; and
\item refer to a functional relationship: each instance of the source type should give rise to a specific instance of the target type.
\end{enumerate}

\end{rules}

%%%% Subsection %%%%

\subsection{Facts}\label{sec:facts}\index{olog!facts}

In this section I will discuss facts, which are simply ``path equivalences" in an olog. It is the notion of path equivalences that make category theory so powerful. 

A {\em path}\index{olog!path in} in an olog is a head-to-tail sequence of arrows. That is, any path starts at some box $B_0$, then follows an arrow emanating from $B_0$ (moving in the appropriate direction), at which point it lands at another box $B_1$, then follows any arrow emanating from $B_1$, etc, eventually landing at a box $B_n$ and stopping there. The number of arrows is the {\em length} of the path. So a path of length 1 is just an arrow, and a path of length 0 is just a box. We call $B_0$ the {\em source} and $B_n$ the {\em target} of the path.

Given an olog, the author may want to declare that two paths are equivalent.  For example consider the two paths from $A$ to $C$ in the olog 
\begin{align}\label{olog:commute}\fbox{\xymatrixnocompile{\smbox{A}{a person}\LA{rr}{has as parents}\LAL{drr}{\parbox{.8in}{has as mother}}&&\obox{B}{.8in}{\rr a pair $(w,m)$ where $w$ is a woman and $m$ is a man}\ar@{}[dll]|(.4){\checkmark}\LA{d}{yields as $w$}\\&&\smbox{C}{a woman}}}\end{align}  We know as English speakers that a woman parent is called a mother, so these two paths $A\to C$ should be equivalent.  A more mathematical way to say this is that the triangle in Olog (\ref{olog:commute}) {\em commutes}. That is, path equivalences are simply commutative diagrams as in Section \ref{sec:comm diag}. In the example above we concisely say ``a woman parent is equivalent to a mother."  We declare this by defining the diagonal map in (\ref{olog:commute}) to be {\em the composition} of the horizontal map and the vertical map. 

I generally prefer to indicate a commutative diagram by drawing a check-mark, $\checkmark$, in the region bounded by the two paths, as in Olog (\ref{olog:commute}).  Sometimes, however, one cannot do this unambiguously on the 2-dimensional page.  In such a case I will indicate the commutative diagrams (fact) by writing an equation.  For example to say that the diagram $$\xymatrix{A\ar[r]^f\ar[d]_h&B\ar[d]^g\\C\ar[r]_i&D}$$ commutes, we could either draw a checkmark inside the square or write the equation $A\;f\;g\simeq A\;h\;i$ above it\index{a symbol!$\simeq$}.
\footnote{We defined function composition on page \ref{function composition}, but here we're using a different notation.\index{a warning!notation for composition} There we would have said $g\circ f = i\circ h$, which is in the backwards-seeming {\em classical order}.\index{composition!classical order} Category theorists and others often prefer the {\em diagrammatic order}\index{composition!diagrammatic order} for writing compositions, which is $f;g = h;i$. For ologs, we follow the latter because it makes for better English sentences, and for the same reason we add the source object to the equation, writing $A f g \simeq A h i$.}
  Either way, it means that ``$f$ then $g$" is equivalent to ``$h$ then $i$".  

Here is another, more scientific example:
\begin{align*}
\fbox{\xymatrix{
\obox{}{1in}{a DNA sequence}\LA{rr}{is transcribed to}\LAL{drr}{codes for}&\hspace{.1in}&\obox{}{1.1in}{an RNA sequence}\ar@{}[dll]|(.35){\checkmark}\LA{d}{is translated to}\\
&&\obox{}{.6in}{a protein}}}
\end{align*}
Note how this diagram gives us the established terminology for the various ways in which DNA, RNA, and protein are related in this context.

\begin{exercise}\label{exc:family olog}

Create an olog for human nuclear biological families that includes the concept of person, man, woman, parent, father, mother, and child. Make sure to label all the arrows, and make sure each arrow indicates a valid aspect in the sense of Section \ref{sec:invalid aspect}. Indicate with check-marks ($\checkmark$) the diagrams that are intended to commute. If the 2-dimensionality of the page prevents a check-mark from being unambiguous, indicate the intended commutativity with an equation.
\end{exercise}

\begin{example}[Non-commuting diagram]

In my conception of the world, the following diagram does not commute:
\begin{align}\label{dia:non-commuting}
\xymatrixnocompile@=50pt{\obox{}{.5in}{a person}\LA{r}{has as father}\LAL{dr}{lives in}&\obox{}{.4in}{a man}\LA{d}{lives in}\\&\obox{}{.4in}{a city}}
\end{align}
The non-commutativity of Diagram (\ref{dia:non-commuting}) does not imply that, in my conception, no person lives in the same city as his or her father. Rather it implies that, in my conception, it is not the case that {\em every} person lives in the same city as his or her father.

\end{example}

\begin{exercise}
Create an olog about a scientific subject, preferably one you think about often. The olog should have at least five boxes, five arrows, and one commutative diagram. 
\end{exercise}

%% Subsubsection %%

\subsubsection{A formula for writing facts as English}\index{olog!facts in English}

Every fact consists of two paths, say $P$ and $Q$, that are to be declared equivalent. The paths $P$ and $Q$ will necessarily have the same source, say $s$, and target, say $t$, but their lengths may be different, say $m$ and $n$ respectively.
\footnote{If the source equals the target, $s=t$, then it is possible  to have $m=0$ or $n=0$, and the ideas below still make sense.} 
We draw these paths as 
\begin{align}\label{dia:two paths for equivalence}
P:&\hsp\xymatrix@=22pt{\LMO{a_0=s}\ar[r]^{f_1}&\LMO{a_1}\ar[r]^{f_2}&\LMO{a_2}\ar[r]^{f_3}&\cdots\ar[r]^{f_{m-1}}&\LMO{a_{m-1}}\ar[r]^{f_m}&\LMO{a_m=t}}\\\nonumber
Q:&\hsp\xymatrix@=23pt{\LMO{b_0=s}\ar[r]^{g_1}&\LMO{b_1}\ar[r]^{g_2}&\LMO{b_2}\ar[r]^{g_3}&\cdots\ar[r]^{g_{n-1}}&\LMO{b_{n-1}}\ar[r]^{g_n}&\LMO{b_n=t}}
\end{align}
Every part $\ell$ of an olog (i.e. every box and every arrow) has an associated English phrase, which we write as $\qt{\ell}$. Using a dummy variable $x$ we can convert a fact into English too. The following general formula is a bit difficult to understand, see Example \ref{ex:English fact}, but here goes. The fact $P\simeq Q$ from (\ref{dia:two paths for equivalence}) can be Englishified as follows:

\begin{align}\label{dia:Englishification}\index{Englishification}
&\tn{Given }x,\qt{s},\tn{ consider the following. We know that }x\tn{ is }\qt{s}, \\
\nonumber&\tn {which } \qt{f_1}\;\qt{a_1}, \tn{ which } \qt{f_2}\;\qt{a_2}, \tn { which }\ldots \; \qt{f_{m-1}}\;\qt{a_{m-1}}, \tn { which } \qt{f_m}\;\qt{t}\\
\nonumber&\tn{that we'll call } P(x).\\
\nonumber&\tn{We also know that }x\tn{ is } \qt{s},\\
\nonumber&\tn {which } \qt{g_1}\;\qt{b_1}, \tn{ which }\qt{g_2}\;\qt{b_2}, \tn { which }\ldots\;\qt{g_{n-1}}\;\qt{b_{n-1}}, \tn { which } \qt{g_n}\;\qt{t}\\
\nonumber&\tn{that we'll call } Q(x).\\
\nonumber&\tn{Fact: whenever }x\tn{ is }``s",\tn{ we will have }P(x)=Q(x).
\end{align}

\begin{example}\label{ex:English fact}

Consider the olog
\begin{align}\label{olog:commute2}\fbox{\xymatrixnocompile{\smbox{A}{a person}\LA{rr}{has}\LAL{drr}{\parbox{.8in}{lives in}}&&\obox{B}{.7in}{\rr an address}\ar@{}[dll]|(.4){\checkmark}\LA{d}{is in}\\&&\smbox{C}{a city}}}
\end{align}
To put the fact that Diagram \ref{olog:commute2} commutes into English, we first Englishify the two paths: $F$=``a person has an address which is in a city" and $G$=``a person lives in a city". The source of both is $s$=``a person" and the target of both is $t$=``a city".
write:
\begin{align*}
&\tn{Given }x,\tn{a person, consider the following. We know that } x\tn{ is a person,}\\
&\tn{which has an address, which is in a city}\\
&\tn{that we'll call } P(x).\\
&\tn{We also know that }x\tn{ is a person,}\\
&\tn{which lives in a city}\\
&\tn{that we'll call } Q(x).\\
&\tn{Fact: whenever }x\tn{ is a person, we will have }P(x)=Q(x).
\end{align*}

\end{example}

\begin{exercise}
This olog was taken from \cite{Sp1}.
\begin{align}\label{dia:phone paths}\xymatrix{&\obox{N}{1in}{a phone number}\LA{rr}{has}&&\obox{C}{.8in}{an area code}\ar@{}[dll]|{\checkmark}\LA{d}{corresponds to}\\\obox{OLP}{1.2in}{an operational landline phone}\LA{ru}{is assigned}\LAL{r}{is}&\obox{P}{1in}{a physical phone}\LAL{rr}{\parbox{.55in}{\scriptsize is currently located in}}&&\obox{R}{.5in}{a region}}
\end{align} 
It says that a landline phone is physically located in the region that its phone number is assigned. Translate this fact into English using the formula from \ref{dia:Englishification}.
\end{exercise}

\begin{exercise}
In the above olog (\ref{dia:phone paths}), suppose that the box \fakebox{an operational landline phone} is replaced with the box \fakebox{an operational mobile phone}. Would the diagram still commute?
\end{exercise}

%% Subsubsection %%

\subsubsection{Images}\label{sec:images}\index{olog!images}\index{image!in olog}

In this section we discuss a specific kind of fact, generated by any aspect. Recall that every function has an image, meaning the subset of elements in the codomain that are ``hit" by the function. For example the function $f(x)=2*x\taking \ZZ\to\ZZ$ has as image the set of all even numbers.

Similarly the set of mothers arises as is the image of the ``has as mother" function, as shown below 
$$
\xymatrix{\obox{P}{.5in}{a person}\LAL{rd}{has}\LA{rr}{$\stackrel{f\taking P\to P}{\tn{has as mother}}$}&&\obox{P}{.5in}{a person}\\
&\obox{M=\im(f)}{.6in}{a mother}\LAL{ur}{is}\ar@{}[u]|(.6){\checkmark}
}$$

\begin{exercise}
For each of the following types, write down a function for which it is the image, or say ``not clearly an image type" 
\sexc \fakebox{a book}
\next \fakebox{a material that has been fabricated by a process of type $T$}
\next \fakebox{a bicycle owner}
\next \fakebox{a child}
\next \fakebox{a used book}
\next \fakebox{an inhabited residence}
\endsexc
\end{exercise}


