

%%%%%%%% Chapter %%%%%%%%

\chapter{緒論}

%The title page of this book contains a graphic that we reproduce here. 
%\begin{align}\label{dia:scientific method}
%\dashbox{\includegraphics[width=.7\textwidth]{ScientificMethod}}
%\end{align}
%It is intended to evoke thoughts of the scientific method. 
%\begin{quote}
%A hypothesis analyzed by a person produces a prediction, which motivates the specification of an experiment, which when executed results in an observation, which analyzed by a person yields a hypothesis.
%\end{quote}
%This sounds valid, and a good graphic can be exceptionally useful for leading a reader through the story that the author wishes to tell. 

この本の題扉には, 次に再掲する図が含まれている.
\begin{align}\label{dia:scientific method}
\dashbox{\includegraphics[width=.7\textwidth]{ScientificMethod}}
\end{align}
この図は科学的手法の思想を呼び起こすことを意図している.
\begin{quote}
仮説の分析によって予言が可能となり, それによって個々の実験の動機が生じ, 実験によって観測結果が得られ, それを分析することにより仮説が生じる.
\end{quote}
これは妥当であろう. そしてこのよき図は, この後に著者が語ろうとすることを理解しようとする際に並外れて便利でありうる.

%Interestingly, a graphic has the power to evoke feelings of understanding, without really meaning much. The same is true for text: it is possible to use a language such as English to express ideas that are never made rigorous or clear. When someone says ``I believe in free will," what does she believe in? We may all have some concept of what she's saying---something we can conceptually work with and discuss or argue about. But to what extent are we all discussing the same thing, the thing she intended to convey?
興味深いことに, 図というものは, 実際にはそれほど意味を持ってなくとも, 理解したという気持ちを呼び起こす力を持っている. これと同じことは文章にも言える. 厳密あるいは明快でない考えを, 例えば英語という言語を使って表現することが可能である. ある人が「私は自由意志を信じている」と言った時, その人はいったい何を信じているのだろうか? おそらく我々みなが, その人が言ったことについてなんらかの概念を持っていることだろう. そしてその何かに対して議論や主義主張などの概念的な行為を行なうことができる. しかし, その人が伝達しようと意図したそのことに関して, 我々がいかほど議論できているだろうか?

%Science is about agreement. When we supply a convincing argument, the result of this convincing is agreement. When, in an experiment, the observation matches the hypothesis---success!---that is agreement. When my methods make sense to you, that is agreement. When practice does not agree with theory, that is disagreement. Agreement is the good stuff in science; it's the high fives.
科学とは一致(agreement)である. 我々が納得した議論を提供するとき, その納得の結果は一致する. 実験において, 観測が仮説と整合した時---成功したぞ!---それは一致である.  実際が理論と合わないとき, それは不一致(disagreement)である. 一致は科学のよき素質である. それはまさにハイタッチだ.

%But it is easy to think we're in agreement, when really we're not. Modeling our thoughts on heuristics and pictures may be convenient for quick travel down the road, but we're liable to miss our turnoff at the first mile. The danger is in mistaking our convenient conceptualizations for what's actually there. It is imperative that we have the ability at any time to ground out in reality. What does that mean?
しかし, 実際にはそうでないのに我々がin agreementである場合を考えることもまた容易である. 道筋を駆け足で簡単にたどるにはheuristicsや図は便利であるかもしれないけれども, 一里目にある分かれ道で間違えることも受けいれなければならなくなる.
危険なのは, 実際にそこにあるものに対する簡易な概念化における間違いである. 現実においては我々は常に内野ゴロでアウトにされてしまう. これは何を意味しているのか?

%Data. Hard evidence. The physical world. It is here that science touches down and heuristics evaporate. So let's look again at the diagram on the cover. It is intended to evoke an idea of how science is performed. Is there hard evidence and data to back this theory up? Can we set up an experiment to find out whether science is actually performed according to such a protocol? To do so we have to shake off the stupor evoked by the diagram and ask the question: ``what does this diagram intend to communicate?"
データ. 確実な証拠. 物理世界. そここそが科学の領域でありそこではheuristicsは霧散する. それでは表紙の図をもう一度見てみよう. 
この図は科学がどのように動いているのかについての発想を呼び起こすことを意図している. この理論を支える確実な証拠やデータはどこにあるのだろうか? 科学が実際にこのようなprotocolに従って動いているかどうかを調べるための実験を, 我々は設計できるだろうか? それを行なうために, 我々は図によって呼び起こされる無感覚な状態を振り払って問いを投げかけなければならない. 「この図は何を伝えることを意図しているのだろうか?」
%In this course I will use a mathematical tool called {\em ologs}, or ontology logs, to give some structure to the kinds of ideas that are often communicated in pictures like the one on the cover. Each olog inherently offers a framework in which to record data about the subject. More precisely it encompasses a {\em database schema}, which means a system of interconnected tables that are initially empty but into which data can be entered. For example consider the olog below
%$$\xymatrix{
%\obox{}{.5in}{a mass}&&\obox{}{1.1in}{an object of mass $m$ held at height $h$ above the ground}\LAL{ll}{\footnotesize has as mass}\LA{rrdd}{\hspace{.4in}\parbox{1in}{\singlespace \footnotesize when dropped has as number of seconds till hitting the ground}}\LAL{dd}{\parbox{.7in}{\singlespace\footnotesize has as height in meters}}&&\\\\
%&&\obox{}{1in}{a real number $h$}\ar@{}[uurr]|(.35){?}\ar[rr]_-{\sqrt{2h\div 9.8}}&\hspace{.3in}&\obox{}{.9in}{a real number}
%}
%$$
%This olog represents a framework in which to record data about objects held above the ground, their mass, their height, and a comparison (the ?-mark in the middle) between the number of seconds till they hit the ground and a certain real-valued function of their height. We will discuss ologs in detail throughout this course. 
この教程では, 著者は\emph{olog}あるいはontology logと呼ばれる数学的な道具を使うことになる. ologを用いることによって, 表紙にあるような絵によってしばしば伝達されるアイデアの類に構造を与えることができる. それぞれのologは本質的には題材についてのrecord dataにおける枠組みを提供する. もっと正確には, \emph{database schema}を含む. これは最初は空のそれぞれの結ばれたテーブルにデータを挿入することができることを意味している. 例として以下のologを考える.
\[
\xymatrix{
\obox{}{.5in}{a mass}&&\obox{}{1.1in}{an object of mass $m$ held at height $h$ above the ground}\LAL{ll}{\footnotesize has as mass}\LA{rrdd}{\hspace{.4in}\parbox{1in}{\singlespace \footnotesize when dropped has as number of seconds till hitting the ground}}\LAL{dd}{\parbox{.7in}{\singlespace\footnotesize has as height in meters}}&&\\\\
&&\obox{}{1in}{a real number $h$}\ar@{}[uurr]|(.35){?}\ar[rr]_-{\sqrt{2h\div 9.8}}&\hspace{.3in}&\obox{}{.9in}{a real number}
}
\]
このologは地上から持ち上げた物体に関してのデータにおける枠組みを表現している. すなわち質量, 高さ, そして地上に落ちるまでの秒数の比較(中心の?マーク)となんらかの高さに関する実函数である.
我々はこの教程を通してologに関する議論を行なう.

%The picture in (\ref{dia:scientific method}) looks like an olog, but it does not conform to the rules that we lay out for ologs in Section \ref{sec:ologs}. In an olog, every arrow is intended to represent a mathematical function. It is difficult to imagine a function that takes in predictions and outputs experiments, but such a function is necessary in order for the arrow
%$$\fbox{a prediction}\To{\tn{motivates the specification of}}\fbox{an experiment}
%$$
%in (\ref{dia:scientific method}) to make sense. To produce an experiment design from a prediction probably requires an expert, and even then the expert may be motivated to specify a different experiment on Tuesday than he is on Monday. But perhaps our criticism has led to a way forward: if we say that every arrow represents a function {\em when in the context of a specific expert who is actually doing the science at a specific time}, then Figure (\ref{dia:scientific method}) begins to make sense. In fact, we will return to the figure in Section \ref{sec:monads} (specifically Example \ref{ex:scientific method}), where background methodological context is discussed in earnest.

\eqref{dia:scientific method}の図はologのように見えるが, \ref{sec:ologs}でologのために容易した設計には整合していない. ologでは, それぞれの矢印は数学での函数を表現することが意図されている. 予言を受け取って実験を出力する函数を想像することは困難であるが, \eqref{dia:scientific method}で矢印
\[
\fbox{a prediction}\To{\tn{motivates the specification of}}\fbox{an experiment}
\]
が意味を持つためにはその函数が必須である. 予言から実験を設計するのはおそらく専門家が必要とされるし, 専門家でさえ火曜日には月曜日と異なった実験をしたくなることがあるだろう. しかし, おそらく我々の批判はさらに先へ進むことになる. \emph{ある文脈の下で特定の専門家が特定の時に実際に科学を行なう}とし時に, もし我々が全ての矢印が函数であると主張するのであるならば, その時には図\eqref{dia:scientific method}は意味を持ち始める. 実際に, 我々は第\ref{sec:monads}章で(具体的には例\ref{ex:scientific method})この図に戻り, そこでまじめにbackground methdological contextを議論することになる.

%This course is an attempt to extol the virtues of a new branch of mathematics, called {\em category theory}, which was invented for powerful communication of ideas between different fields and subfields within mathematics. By powerful communication of ideas I actually mean something precise. Different branches of mathematics can be formalized into categories. These categories can then be connected together by functors. And the sense in which these functors provide powerful communication of ideas is that facts and theorems proven in one category can be transferred through a connecting functor to yield proofs of analogous theorems in another category. A functor is like a conductor of mathematical truth.

この教程は\emph{圏論(category theory)}と呼ばれる数学の新しい分野の価値を褒め称えることを目的としている. 圏論は異なった分野と数学の中の分野でのアイデアの間の強力な情報伝達手段として開発された. 様々な数学の分野が圏(category)によって定式化できる. これらの圏は函手(functor)によって結びついている. そしてこれら函手が強力な思考伝達の手段となる理由は, ある圏の中で事実や証明された定理は, 結びついた函手を通して他の圏でのよく似た定理の証明の導出となる. 函手は数学的事実の導管のようなものだ.

%I believe that the language and toolset of category theory can be useful throughout science. We build scientific understanding by developing models, and category theory is the study of basic conceptual building blocks and how they cleanly fit together to make such models. Certain structures and conceptual frameworks show up again and again in our understanding of reality. No one would dispute that vector spaces are ubiquitous. But so are hierarchies, symmetries, actions of agents on objects, data models, global behavior emerging as the aggregate of local behavior, self-similarity, and the effect of methodological context. 

著者は圏論の言葉と道具一式は科学の全域において有用だと信じている. 我々は模型を開発することにより科学的な理解を構築する. そして圏論とは基本的な概念の構成要素と, それらがどのように模型にきれいに適用できるかの研究である. ある構造と概念の枠組みは我々の現実の理解において繰り返し繰り返し出現する. ベクトル空間が普遍的であるかどうかを議論する人間はいないであろう. しかしながら, 階層性, 対称性, 物質に対するなんらかの作用, データ模型, 局所的なふるまいの集合によって創発する大域的なふるまい, 自己相似性, そして方法論の文脈による効果.

%Some ideas are so common that our use of them goes virtually undetected, such as set-theoretic intersections. For example, when we speak of a material that is both lightweight and ductile, we are intersecting two sets. But what is the use of even mentioning this set-theoretic fact? The answer is that when we formalize our ideas, our understanding is almost always clarified. Our ability to communicate with others is enhanced, and the possibility for developing new insights expands. And if we are ever to get to the point that we can input our ideas into computers, we will need to be able to formalize these ideas first.

いくつかのアイデア, 例えは集合の共通部分などは, あまりにもありふれているので, それらの使用は実質的には検知できない. 例えば, 我々がある物質について軽くて延性があると言うとき我々は二つの集合の共通部分を取っている. しかしこの淡々とした集合論的事実の記述は何を意味しているのか? その答は, 我々がアイデアを形式化したとき, 我々の理解はほとんど常に明確である. 我々の他者との情報伝達能力は増強され, 新しい洞察を生みだす可能性がひろがっていく. そしてもし我々がアイデアを計算機に入力できるようになった時点では常に, 我々は最初にアイデアを形式化できるようになっている必要がある.

%It is my hope that this course will offer scientists a new vocabulary in which to think and communicate, and a new pipeline to the vast array of theorems that exist and are considered immensely powerful within mathematics. These theorems have not made their way out into the world of science, but they are directly applicable there. Hierarchies are partial orders, symmetries are group elements, data models are categories, agent actions are monoid actions, local-to-global\index{local-to-global} principles are sheaves, self-similarity is modeled by operads, context can be modeled by monads.

著者は, この教程が, 科学者に思考や情報伝達に用いる新しい語彙と, 数学の中に存在していて大いに強力だと考えられている大量の整備された定理への経路をもたらすことを期待している. これらの定理は科学の世界ではこれまでのところは真価を発揮しているとは言えないが, そこに直接適用することができる. 階層性はpartial orderであり, 対称性は群の要素であり, データ模型は圏であり, agent actionはモノイドの作用であり, local-to-global\index{local-to-global}原理は層(sheve)であり, 自己相似性はoperandによってモデル化され, 文脈はモナドによってモデル化される.

%%%%%% Section %%%%%%

%\section{A brief history of category theory}

\section{圏論の簡潔な歴史}

%The paradigm shift brought on by Einstein's theory of relativity brought on the realization that there is no single perspective from which to view the world. There is no background framework that we need to find; there are infinitely many different frameworks and perspectives, and the real power lies in being able to translate between them. It is in this historical context that category theory got its start.
%\footnote{The following history of category theory is far too brief, and perhaps reflects more of the author's aesthetic than any kind of objective truth, whatever that may mean. Here are some much better references: \cite{Kro}, \cite{Mar1}, \cite{LM}.}

Einsteinの相対性理論によってもたらされたパラダイムシフトは, 世界を観るにあたって唯一の観点というものは存在しないという認識をもたらした. 我々が探し求めなければならない背景の枠組みというものは存在しない. 無限に多い異なった枠組みと観点が存在しており, 真なる力はそれらの間を翻訳するというところに存在している. 歴史的には, 圏論はこの文脈から開始した.
\footnote{以下の圏論の歴史は簡潔に過ぎ, それが何を意味するかは別にして, おそらく客観的な真実の類よりも著者の美的感覚を反映していることだろう. はるかによい参考文献を次に挙げる: \cite{Kro}, \cite{Mar1}, \cite{LM}.}. 

%Category theory was invented in the early 1940s by Samuel Eilenberg\index{Eilenberg, Samuel} and Saunders Mac Lane.\index{Mac Lane, Saunders} It was specifically designed to bridge what may appear to be two quite different fields: topology and algebra. Topology is the study of abstract shapes such as 7-dimensional spheres; algebra is the study of abstract equations such as $y^2z=x^3-xz^2$. People had already created important and useful links (e.g. cohomology theory) between these fields, but Eilenberg and Mac Lane needed to precisely compare different links with one another. To do so they first needed to boil down and extract the fundamental nature of these two fields. But the ideas they worked out amounted to a framework that fit not only topology and algebra, but many other mathematical disciplines as well.

圏論は1940年代始めにSamuel Eilenberg\index{Eilenberg, Samuel}とSaunders Mac Lane\index{Mac Lane, Saunders}によって発明された. 圏論は, 具体的には, トポロジーと代数という非常に異なったように見える二つの分野の間に橋を架けるために設計された. トポロジーは, 例えば7次元球のような抽象的な形の研究であり, 代数は$y^2z=x^3-xz^2$のような抽象的な方程式の研究である. これらの分野の間には, 重要で便利な結び付き(e.g. コホモロジー理論)が既に発見され存在していた. しかしEilenbergとMac Laneはそれぞれの結び付きを正確に比較する必要があった. これを遂行するために, 彼等は最初にこれら二つの分野から基本的な性質を煮詰めて抽出する必要があった. ところが, 彼等が考え出した枠組みはトポロジーと代数だけでなく, 他の多くの数学分野にも適用できるものになっていた.

%At first category theory was little more than a deeply clarifying language for existing difficult mathematical ideas. However, in 1957 Alexander Grothendieck\index{Grothendieck!in history} used category theory to build new mathematical machinery (new cohomology theories) that granted unprecedented insight into the behavior of algebraic equations. Since that time, categories have been built specifically to zoom in on particular features of mathematical subjects and study them with a level of acuity that is simply unavailable elsewhere.

圏論は, 最初は既に存在する数学的アイデアを非常に明確するする言語という以上のものではなかった. しかし, 1957年に, Alexander Grothendieck\index{Grothendieck!in history}は圏論を用いて, 代数方程式のふるまいに対してこれまで存在しなかった洞察をもたらす新しい数学的手続き(新しいコホモロジー理論)を構築した. その瞬間から, 圏はとりわけ数学的対象の詳しい特徴を拡大するために, また他では簡単には得られないような鋭敏さのレベルでもって研究を行うために, 構築されてきた.

%Bill Lawvere\index{Lawvere, William} saw category theory as a new foundation for all mathematical thought. Mathematicians had been searching for foundations in the 19th century and were reasonably satisfied with set theory as {\em the foundation}. But Lawvere showed that the category of sets is simply a category with certain nice properties, not necessarily the center of the mathematical universe. He explained how whole algebraic theories can be viewed as examples of a single system. He and others went on to show that higher order logic was beautifully captured in the setting of category theory (more specifically toposes). It is here also that Grothendieck and his school worked out major results in algebraic geometry. 

Bill Lawvere\index{Lawvere, William}は圏論を全ての数学的思考対象の新しい基礎とみなした. 19世紀に数学者は数学の基礎を探し求め, そして\emph{基礎(foundation)}として納得がいく形で集合論に満足した. しかしLawvereは, 集合の圏は単にいくつかのよい性質を持った圏の一つであって, 必ずしも数学の中心にいなければならないものではないということを示した. 彼は, どのようにして代数的理論全体を単一の系の例の一つとしてみることができるかを説明した. 彼と仲間たちは, 高階論理を圏論の設定(より具体的にはトポス)に美しく当てはめることに邁進した. Grothendieckとその学派の代数幾何における重要な結果もまたこの文脈にある.

%In 1980 Joachim Lambek\index{Lambek, Joachim} showed that the types and programs used in computer science form a specific kind of category. This provided a new semantics for talking about programs, allowing people to investigate how programs combine and compose to create other programs, without caring about the specifics of implementation. Eugenio Moggi\index{Moggi, Eugenio} brought the category theoretic notion of monads into computer science to encapsulate ideas that up to that point were considered outside the realm of such theory..

1980年, Joachim Lambek\index{Lambek, Joachim}は型とプログラムが特定の種の圏をなしていることを示した. これによってプログラムについて語るための新しい意味論が得られ, 実装の詳細について考慮することなく, プログラムを結合・合成して他のプログラムを生成することがどのようなことであるかを調べることができるようになった. Eugenio Moggi\index{Moggi, Eugenio}は, 彼が提案するまではこのような理論の適用範囲外だと考えられていた観念をカプセル化するため, 計算機科学にモナドという圏論の概念を導入した.

%It is difficult to explain the clarity and beauty brought to category theory by people like Daniel Kan\index{Kan, Daniel} and Andr\'{e} Joyal\index{Joyal, Andr\'{e}}. They have each repeatedly extracted the essence of a whole mathematical subject to reveal and formalize a stunningly simple yet extremely powerful pattern of thinking, revolutionizing how mathematics is done. 

Daniel Kan\index{Kan, Daniel}やAndr\'{e} Joyal\index{Joyal, Andr\'{e}}といった人々が圏論にもたらした明確さと美しさを説明するのは難しい. 彼らは, 華麗なほどに単純にもかかわらずとてつもなく強力な, 数学的が機能するところの思考と革新の様式を形式化して明らかにするために, それぞれ繰り返し数学的対象の全体からその本質を抽出した.

%All this time, however, category theory was consistently seen by much of the mathematical community as ridiculously abstract. But in the 21st century it has finally come to find healthy respect within the larger community of pure mathematics. It is the language of choice for graduate-level algebra and topology courses, and in my opinion will continue to establish itself as the basic framework in which mathematics is done.

ここまでの全ての時代において, 数学のほとんどのコミュニティにおいて圏論は笑ってしまうくらい抽象的だと一貫してみなされてきた. しかし21世紀になって遂に圏論は純粋数学者の多くのコミュニティにおいて健全であると見いだされることとなった. 圏論は大学院水準の代数とトポロジーの教程のために選択する言語であり, そして著者の意見では, 圏論はその中で数学が機能するところの基本的な枠組みとして, それ自身が確立され続けている. 

%As mentioned above category theory has branched out into certain areas of science as well. Baez\index{Baez, John} and Dolan\index{Dolan, James} have shown its value in making sense of quantum physics, it is well established in computer science, and it has found proponents in several other fields as well. But to my mind, we are the very beginning of its venture into scientific methodology. Category theory was invented as a bridge and it will continue to serve in that role. 

上で述べたように, 圏論は科学の特定の分野にも広がっている. Baez\index{Baez, John}とDolan\index{Dolan, James}は量子物理学の意味づけにおいてその価値を示した. 圏論は計算機科学でよく確立しており, いくつかの他の分野でも支持者が見いだされる. しかし著者の考えでは, まだ人類は科学的方法論への冒険の出発地点にいる. 圏論は異分野の架け橋として発明され, 今後もその役割を果たし続けるだろう.


%%%%%% Section %%%%%%

%\section{Intention of this book}
\section{この本の目的}

%The world of {\em applied mathematics} is much smaller than the world of {\em applicable mathematics}. As alluded to above, this course is intended to create a bridge between the vast array of mathematical concepts that are used daily by mathematicians to describe all manner of phenomena that arise in our studies, and the models and frameworks of scientific disciplines such as physics, computation, and neuroscience. 

\emph{応用数学(applied mathematics)}の世界は\emph{応用可能な数学(applicable mathematics)}の世界よりもはるかに小さい. これまでにほめのかしているように, この教程は研究の中で生じる現象全てを記述するために数学者たちが日々使っている莫大な数学的概念と, 例えば物理学, 計算機科学, 神経科学といった科学的分野の模型や枠組みの間に橋をわたすことを目的としている.

%To the pure mathematician I'll try to prove that concepts such as categories, functors, natural transformations, limits, colimits, functor categories, sheaves, monads, and operads---concepts that are often considered too abstract for even math majors---can be communicated to scientists with no math background beyond linear algebra. If this material is as teachable as I think, it means that category theory is not esoteric but somehow well-aligned with ideas that already make sense to the scientific mind. Note, however, that this book is example-based rather than proof-based, so it may not be suitable as a reference for students of pure mathematics.

純粋数学者に対しては, 圏, 函手, 自然変換, 極限, 余極限, 函手圏, 層, モナド, オペラドのような概念---しばしば数学専攻でさえも抽象的すぎるとみなされるような概念---を, 線形代数以上の数学的背景を持たない科学者に届けられることを, 著者は証明していきたいと思っている. もしこのような概念が著者が考えているように教示可能であるならば, それはすなわち, 圏論は難解なものではなくて, 科学者の脳内において既に意味をなしているアイデアになんらかの形でよくあてはまっているということを意味している. しかしながら, この本は証明中心ではなく例を中心としているため, 純粋数学の学生に対しては参考書として向いていないだろうことには, 注意して欲しい.

%To the scientist I'll try to prove the claim that category theory includes a formal treatment of conceptual structures that the scientist sees often, perhaps without realizing that there is well-oiled mathematical machinery to be employed. We will work on the structure of information; how data is made meaningful by its connections, both internal and outreaching, to other data. Note, however, that this book should most certainly not be taken as a reference on scientific matters themselves. One should assume that any account of physics, materials science, chemistry, etc. has been oversimplified.\index{a warning!oversimplified science} The intention is to give a flavor of how category theory may help us model scientific ideas, not to explain these ideas in a serious way. 

科学者に対しては, 科学者がよく見るところの概念構造, おそらくよく油がさされた数学的機構が供えつけられていることに気付いていないであろうところのその構造の, 形式的な取り扱いを圏論は包含しているという主張を, 著者は証明していきたいと思っている. 我々は情報の構造について取りくむことになる. すなわち, データはその内部および外部の他のデータとの間の結びつきによって, どのように意味を持つことになるのか, ということについてである. しかしながら, おそらくほとんどの場合において, 科学的な対象それ自身に関しての参考文献としてはこの本を使うべきでないことに注意して欲しい. 物理学, 物質科学, 化学, etc. の記述はどれも過度に単純化しすぎていると認識するべきである.\index{a warning!oversimplified science} 本書の目的は, どのようにして圏論の持ち味が我々の科学的思考模型の力になるかを示すことであって, これらのアイデアに真剣に取りくむための説明をするつもりはない.

%Data gathering is ubiquitous in science. Giant databases are currently being mined for unknown patterns, but in fact there are many (many) known patterns that simply have not been catalogued. Consider the well-known case of medical records. A patient's medical history is often known by various individual doctor-offices but quite inadequately shared between them. Sharing medical records often means faxing a hand-written note or a filled-in house-created form between offices. 

データ収集は科学において普遍的である. 現在, 未知のパターンを求めて巨大なデータベースの解析が行なわれているが, 実際のところそこには既知だが未だ名付けられていないとても(とても)多くのパターンが存在している. よく知られている例として医療記録を考えてみよう. 患者の治療履歴はしばしば様々な個々の医院において知られているが, しかしそれらの間では非常に不適切な形でしか共有されていない. 医療記録の共有がは, しばしば手書きのノートあるいは自家製の書式で書き込んだものをファックスでやりとりすることを意味している. 

%Similarly, in science there exists substantial expertise making brilliant connections between concepts, but it is being conveyed in silos of English prose known as journal articles. Every scientific journal article has a methods section, but it is almost impossible to read a methods section and subsequently repeat the experiment---the English language is inadequate to precisely and concisely convey what is being done.

同様に, 科学においては概念の間に光輝く橋をわたす本物の専門知識が存在している. しかしそれらは学術雑誌の記事として知られるそびえたつ散文英語の山によって伝達されている. 科学雑誌の各々の記事には手法(method)と呼ばれる章がある. しかし手法の章を読むのはほとんど不可能だしそれゆえに実験を繰り返すのもほとんど不可能である---なにが行なわれたかを正確かつ簡潔に伝達するにあたって, 英語は適切ではない.\footnote{訳注:もちろん英語以外の自然言語も適切ではない.}

%The first thing to understand in this course is that reusable methodologies can be formalized, and that doing so is inherently valuable. Consider the following analogy. Suppose you want to add up the area of a region in space (or the area under a curve). You break the region down into small squares, each of which you know has area $A$; then you count the number of squares, say $n$, and the result is that the region has an area of about $nA$. If you want a more precise and accurate result you repeat the process with half-size squares. This methodology can be used for any area-finding problem (of which there are more than a first-year calculus student generally realizes) and thus it deserves to be formalized. But once we have formalized this methodology, it can be taken to its limit and out comes integration by Riemann sums. 

この教程で最初に理解するべきことは, 再利用可能な方法論は形式化することができ, そして形式化することは本質的に価値を持つということである. 類推として以下を見てみよう. 空間上の領域の面積(あるいは曲線下面積)を足し上げたいとする. 領域を小さな正方形に分割すれば, そのそれぞれは面積$A$を持つことが分かっている. このとき, 正方形の数$n$を数えれば, 領域は約$nA$の面積を持つというのが, その結果になる. より正確かつ精緻な結果を求めたいのであれば, この半分の大きさの正方形でこの過程を繰り返すことになる. この方法論は(大学一年の解析学の講義の学生が一般に理解できる以上のものまで含む)任意の求積問題に使うことができ, よって形式化する価値がある. この方法を一度形式化してしまえば, その極限をとることができ, そこからRiemann和が出現する.

%I intend to show that category theory is incredibly efficient as a language for experimental design patterns, introducing formality while remaining flexible. It forms a rich and tightly woven conceptual fabric that will allow the scientist to maneuver between different perspectives whenever the need arises. Once one builds that fabric for oneself, he or she has an ability to think about models in a way that simply would not occur without it.  Moreover, putting ideas into the language of category theory forces a person to clarify their assumptions. This is highly valuable both for the researcher and for his or her audience.

この本で著者は, 実験を設計するパターンを記述する言語として圏論が信じられないくらい強力であることを示そうと思っている. 圏論によって形式化を導入できる一方で柔軟性を残すことができる. 圏論は豊富かつ堅固に編み込まれた概念の構造体を形成する. この構造体を使えば, 必要が生じたときはいつでも科学者は異なった観点の間を飛び移ることができるようになる. 一度この構造体を自分のものとして構築できれば, その人はそれなしでは生じえなかったであろうくらいに簡潔な方法で模型について考える能力を得たことになる. 加えて, 圏論の言葉の中にアイデアを投入することによってその前提条件が明瞭化される. これは研究者とその聴衆の両者にとって非常に価値あることだ. 

%What must be recognized in order to find value in this course is that conceptual chaos is a major problem. Creativity demands clarity of thinking, and to think clearly about a subject requires an organized understanding of how its pieces fit together. Organization and clarity also lead to better communication with others. Academics often say they are paid to think and understand, but that is not true. They are paid to think, understand, and {\em communicate their findings}. Universal languages for science---languages such as calculus and differential equations, matrices, or simply graphs and pie-charts---already exist, and they grant us a cohesiveness that makes scientific research worthwhile. In this book I will attempt to show that category theory can be similarly useful in describing complex scientific understandings.

この教程に価値を見いだすために認識しなければならないことは, 概念の混沌は巨大な困難だということである. 創造性には思考の明瞭さが必要であって, 主題を明瞭に思考するためには, その断片々々が互いにどのように組み合っているかを組織化して理解することが必要である. また組織化と明瞭性は他者とのよい情報伝達に繋がる. アカデミックでは, しばしば組織化と明瞭化は思考と理解の対価と語られているが, しかしこれは本当ではない. 組織化と明瞭化は, 思考と理解と\emph{発見したことを伝達する}のための対価である. 科学の普遍的な言語---解析学と微分方程式, 行列, あるいは単純なグラフに円グラフ---は既に存在している. そしてこれらによって我々は科学研究を価値あるものにする結合力を得ることができる. この本では, 複雑な科学的理解を記述する際に, 圏論は上記の言語と同じくらい有用でありうるということを示したいと思う.


